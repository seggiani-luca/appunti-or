
\documentclass[a4paper,11pt]{article}
\usepackage[a4paper, margin=8em]{geometry}

% usa i pacchetti per la scrittura in italiano
\usepackage[french,italian]{babel}
\usepackage[T1]{fontenc}
\usepackage[utf8]{inputenc}
\frenchspacing 

% usa i pacchetti per la formattazione matematica
\usepackage{amsmath, amssymb, amsthm, amsfonts}

% usa altri pacchetti
\usepackage{gensymb}
\usepackage{hyperref}
\usepackage{standalone}

% imposta il titolo
\title{Appunti Ricerca Operativa}
\author{Luca Seggiani}
\date{2024}

% disegni
\usepackage{pgfplots}
\pgfplotsset{width=10cm,compat=1.9}

% imposta lo stile
% usa helvetica
\usepackage[scaled]{helvet}
% usa palatino
\usepackage{palatino}
% usa un font monospazio guardabile
\usepackage{lmodern}

\renewcommand{\rmdefault}{ppl}
\renewcommand{\sfdefault}{phv}
\renewcommand{\ttdefault}{lmtt}

% disponi il titolo
\makeatletter
\renewcommand{\maketitle} {
	\begin{center} 
		\begin{minipage}[t]{.8\textwidth}
			\textsf{\huge\bfseries \@title} 
		\end{minipage}%
		\begin{minipage}[t]{.2\textwidth}
			\raggedleft \vspace{-1.65em}
			\textsf{\small \@author} \vfill
			\textsf{\small \@date}
		\end{minipage}
		\par
	\end{center}

	\thispagestyle{empty}
	\pagestyle{fancy}
}
\makeatother

% disponi teoremi
\usepackage{tcolorbox}
\newtcolorbox[auto counter, number within=section]{theorem}[2][]{%
	colback=blue!10, 
	colframe=blue!40!black, 
	sharp corners=northwest,
	fonttitle=\sffamily\bfseries, 
	title=Teorema~\thetcbcounter: #2, 
	#1
}

% disponi definizioni
\newtcolorbox[auto counter, number within=section]{definition}[2][]{%
	colback=red!10,
	colframe=red!40!black,
	sharp corners=northwest,
	fonttitle=\sffamily\bfseries,
	title=Definizione~\thetcbcounter: #2,
	#1
}

% disponi problemi
\newtcolorbox[auto counter, number within=section]{problem}[2][]{%
	colback=green!10,
	colframe=green!40!black,
	sharp corners=northwest,
	fonttitle=\sffamily\bfseries,
	title=Problema~\thetcbcounter: #2,
	#1
}

% disponi codice
\usepackage{listings}
\usepackage[table]{xcolor}

\lstdefinestyle{codestyle}{
		backgroundcolor=\color{black!5}, 
		commentstyle=\color{codegreen},
		keywordstyle=\bfseries\color{magenta},
		numberstyle=\sffamily\tiny\color{black!60},
		stringstyle=\color{green!50!black},
		basicstyle=\ttfamily\footnotesize,
		breakatwhitespace=false,         
		breaklines=true,                 
		captionpos=b,                    
		keepspaces=true,                 
		numbers=left,                    
		numbersep=5pt,                  
		showspaces=false,                
		showstringspaces=false,
		showtabs=false,                  
		tabsize=2
}

\lstdefinestyle{shellstyle}{
		backgroundcolor=\color{black!5}, 
		basicstyle=\ttfamily\footnotesize\color{black}, 
		commentstyle=\color{black}, 
		keywordstyle=\color{black},
		numberstyle=\color{black!5},
		stringstyle=\color{black}, 
		showspaces=false,
		showstringspaces=false, 
		showtabs=false, 
		tabsize=2, 
		numbers=none, 
		breaklines=true
}

\lstdefinelanguage{javascript}{
	keywords={typeof, new, true, false, catch, function, return, null, catch, switch, var, if, in, while, do, else, case, break},
	keywordstyle=\color{blue}\bfseries,
	ndkeywords={class, export, boolean, throw, implements, import, this},
	ndkeywordstyle=\color{darkgray}\bfseries,
	identifierstyle=\color{black},
	sensitive=false,
	comment=[l]{//},
	morecomment=[s]{/*}{*/},
	commentstyle=\color{purple}\ttfamily,
	stringstyle=\color{red}\ttfamily,
	morestring=[b]',
	morestring=[b]"
}

% disponi sezioni
\usepackage{titlesec}

\titleformat{\section}
	{\sffamily\Large\bfseries} 
	{\thesection}{1em}{} 
\titleformat{\subsection}
	{\sffamily\large\bfseries}   
	{\thesubsection}{1em}{} 
\titleformat{\subsubsection}
	{\sffamily\normalsize\bfseries} 
	{\thesubsubsection}{1em}{}

% disponi alberi
\usepackage{forest}

\forestset{
	rectstyle/.style={
		for tree={rectangle,draw,font=\large\sffamily}
	},
	roundstyle/.style={
		for tree={circle,draw,font=\large}
	}
}

% disponi algoritmi
\usepackage{algorithm}
\usepackage{algorithmic}
\makeatletter
\renewcommand{\ALG@name}{Algoritmo}
\makeatother

% disponi numeri di pagina
\usepackage{fancyhdr}
\fancyhf{} 
\fancyfoot[L]{\sffamily{\thepage}}

\makeatletter
\fancyhead[L]{\raisebox{1ex}[0pt][0pt]{\sffamily{\@title \ \@date}}} 
\fancyhead[R]{\raisebox{1ex}[0pt][0pt]{\sffamily{\@author}}}
\makeatother

\begin{document}

% sezione (data)
\section{Lezione del 14-10-24}

% stili pagina
\thispagestyle{empty}
\pagestyle{fancy}

% testo
\subsection{Algoritmi dei rendimenti}
Riassumiamo i quattro algoritmi presentati finora per le valutazioni inferiori e superiori di problemi di PLI (con vincolo $x \in \mathbb{Z}^n$). 
Prendiamo in esempio lo "zaino":\begin{table}[h!]
	\center 
	\begin{tabular} { c | c c c c }
		$v$ & $10$ & $17$ & $22$ & $21$ \\
		\hline 
		$p$ & $4$ & $5$ & $6$ & $2$ \\
	\end{tabular}
\end{table}
con $P = 7$.
Il vettore dei rendimenti sarà quindi:
$$
r = \left( \frac{10}{4}, \frac{17}{5}, \frac{22}{6}, \frac{21}{2} \right) \approx (2.5, 3.4, 3.66, 10.5)
$$

\begin{itemize}
	\item \textbf{\textsf{Problema booleano}} \\
	\[
		\begin{cases}
			\max(10 x_1 + 17 x_2 + 22 x_33 + 21 x_4) \\ 
			4 x_1 + 5 x_2 + 6 x_3 + 2 x_4 \leq 7 \\ 
			x \in \{ 0, 1 \}^n
		\end{cases}
	\]
	si hanno gli algoritmi di valutazione:
	\begin{itemize}
		\item \textbf{Valutazione inferiore:} si prendono le variabili con rendimenti migliori, una volta sola, finché non si satura.
			Nel caso la variabile esca da $P$, si prende quella dopo, con il caso limite di non prendere nulla.
			Si ha quindi:
			$$ x = (0, 1, 0, 1), \quad V_I = 21 + 17 = 38 $$
		\item \textbf{Valutazione superiore:} si prende il rilassato continuo:
			\[
				\begin{cases}
					\max (v^T x) \\ 
					p^T x \leq \\ 
					0 \leq x \leq 1
				\end{cases}
			\]
			e si riempie con le variabili dai rendimenti migliori, saturando l'ultima:
			$$ x = \left( 0, 0, \frac{5}{6}, 1 \right), \quad V_\alpha = 21 + \frac{5}{6} 22 = 39.\overline{3}, \quad V_S = \lfloor V_\alpha \rfloor = \lfloor 39.\overline{3} \rfloor = 39 $$
	\end{itemize}
	Si ha quindi $V_I = 38$ e $V_S = 39$, con errore $\epsilon = \frac{39 - 38}{38} = \frac{1}{38} = 2.6\%$.
\item \textbf{\textsf{Problema intero}} \\
	\[
		\begin{cases}
			\max(10 x_1 + 17 x_2 + 22 x_33 + 21 x_4) \\ 
			4 x_1 + 5 x_2 + 6 x_3 + 2 x_4 \leq 7 \\ 
			x \leq 0
		\end{cases}
	\]
	si hanno gli algoritmi di valutazione:
	\begin{itemize}
		\item \textbf{Valutazione inferiore:} si prendono le variabili con rendimenti migliori, con coefficienti maggiori possibili, finché non si satura.
			Nel caso la variabile esca da $P$, si prende quella dopo, con il caso limite di non prendere nulla.
			Si ha quindi:
			$$ x = (0, 0, 0, 3), \quad V_I = 3 \cdot 21 = 63 $$
		\item \textbf{Valutazione superiore:} si prende il rilassato continuo:
			\[
				\begin{cases}
					\max (v^T x) \\ 
					p^T x \leq \\ 
					0 \leq 0
				\end{cases}
			\]
			e si riempie con le variabili dai rendimenti migliori, saturando dalla prima:
			$$ x = \left( 0, 0, 0, \frac{7}{2} \right), \quad V_\alpha = \frac{7}{2} 22 = 73.5, \quad V_S = \lfloor V_\alpha \rfloor = \lfloor 73.5 \rfloor = 73.5 $$
	\end{itemize}
	Si ha quindi $V_I = 63$ e $V_S = 73.5$, con errore $\epsilon = \frac{73.5 - 63}{63} = \frac{1}{6} = 16\%$.
	Notiamo come sul problema intero si accumuli molto più errore.
\end{itemize}

\end{document}
