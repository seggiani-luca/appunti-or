
\documentclass[a4paper,11pt]{article}
\usepackage[a4paper, margin=8em]{geometry}

% usa i pacchetti per la scrittura in italiano
\usepackage[french,italian]{babel}
\usepackage[T1]{fontenc}
\usepackage[utf8]{inputenc}
\frenchspacing 

% usa i pacchetti per la formattazione matematica
\usepackage{amsmath, amssymb, amsthm, amsfonts}

% usa altri pacchetti
\usepackage{gensymb}
\usepackage{hyperref}
\usepackage{standalone}

% imposta il titolo
\title{Appunti Ricerca Operativa}
\author{Luca Seggiani}
\date{2024}

% disegni
\usepackage{pgfplots}
\pgfplotsset{width=10cm,compat=1.9}

% imposta lo stile
% usa helvetica
\usepackage[scaled]{helvet}
% usa palatino
\usepackage{palatino}
% usa un font monospazio guardabile
\usepackage{lmodern}

\renewcommand{\rmdefault}{ppl}
\renewcommand{\sfdefault}{phv}
\renewcommand{\ttdefault}{lmtt}

% disponi il titolo
\makeatletter
\renewcommand{\maketitle} {
	\begin{center} 
		\begin{minipage}[t]{.8\textwidth}
			\textsf{\huge\bfseries \@title} 
		\end{minipage}%
		\begin{minipage}[t]{.2\textwidth}
			\raggedleft \vspace{-1.65em}
			\textsf{\small \@author} \vfill
			\textsf{\small \@date}
		\end{minipage}
		\par
	\end{center}

	\thispagestyle{empty}
	\pagestyle{fancy}
}
\makeatother

% disponi teoremi
\usepackage{tcolorbox}
\newtcolorbox[auto counter, number within=section]{theorem}[2][]{%
	colback=blue!10, 
	colframe=blue!40!black, 
	sharp corners=northwest,
	fonttitle=\sffamily\bfseries, 
	title=Teorema~\thetcbcounter: #2, 
	#1
}

% disponi definizioni
\newtcolorbox[auto counter, number within=section]{definition}[2][]{%
	colback=red!10,
	colframe=red!40!black,
	sharp corners=northwest,
	fonttitle=\sffamily\bfseries,
	title=Definizione~\thetcbcounter: #2,
	#1
}

% disponi problemi
\newtcolorbox[auto counter, number within=section]{problem}[2][]{%
	colback=green!10,
	colframe=green!40!black,
	sharp corners=northwest,
	fonttitle=\sffamily\bfseries,
	title=Problema~\thetcbcounter: #2,
	#1
}

% disponi codice
\usepackage{listings}
\usepackage[table]{xcolor}

\lstdefinestyle{codestyle}{
		backgroundcolor=\color{black!5}, 
		commentstyle=\color{codegreen},
		keywordstyle=\bfseries\color{magenta},
		numberstyle=\sffamily\tiny\color{black!60},
		stringstyle=\color{green!50!black},
		basicstyle=\ttfamily\footnotesize,
		breakatwhitespace=false,         
		breaklines=true,                 
		captionpos=b,                    
		keepspaces=true,                 
		numbers=left,                    
		numbersep=5pt,                  
		showspaces=false,                
		showstringspaces=false,
		showtabs=false,                  
		tabsize=2
}

\lstdefinestyle{shellstyle}{
		backgroundcolor=\color{black!5}, 
		basicstyle=\ttfamily\footnotesize\color{black}, 
		commentstyle=\color{black}, 
		keywordstyle=\color{black},
		numberstyle=\color{black!5},
		stringstyle=\color{black}, 
		showspaces=false,
		showstringspaces=false, 
		showtabs=false, 
		tabsize=2, 
		numbers=none, 
		breaklines=true
}

\lstdefinelanguage{javascript}{
	keywords={typeof, new, true, false, catch, function, return, null, catch, switch, var, if, in, while, do, else, case, break},
	keywordstyle=\color{blue}\bfseries,
	ndkeywords={class, export, boolean, throw, implements, import, this},
	ndkeywordstyle=\color{darkgray}\bfseries,
	identifierstyle=\color{black},
	sensitive=false,
	comment=[l]{//},
	morecomment=[s]{/*}{*/},
	commentstyle=\color{purple}\ttfamily,
	stringstyle=\color{red}\ttfamily,
	morestring=[b]',
	morestring=[b]"
}

% disponi sezioni
\usepackage{titlesec}

\titleformat{\section}
	{\sffamily\Large\bfseries} 
	{\thesection}{1em}{} 
\titleformat{\subsection}
	{\sffamily\large\bfseries}   
	{\thesubsection}{1em}{} 
\titleformat{\subsubsection}
	{\sffamily\normalsize\bfseries} 
	{\thesubsubsection}{1em}{}

% disponi alberi
\usepackage{forest}

\forestset{
	rectstyle/.style={
		for tree={rectangle,draw,font=\large\sffamily}
	},
	roundstyle/.style={
		for tree={circle,draw,font=\large}
	}
}

% disponi algoritmi
\usepackage{algorithm}
\usepackage{algorithmic}
\makeatletter
\renewcommand{\ALG@name}{Algoritmo}
\makeatother

% disponi numeri di pagina
\usepackage{fancyhdr}
\fancyhf{} 
\fancyfoot[L]{\sffamily{\thepage}}

\makeatletter
\fancyhead[L]{\raisebox{1ex}[0pt][0pt]{\sffamily{\@title \ \@date}}} 
\fancyhead[R]{\raisebox{1ex}[0pt][0pt]{\sffamily{\@author}}}
\makeatother

\begin{document}

% sezione (data)
\section{Lezione del 04-12-24}

% stili pagina
\thispagestyle{empty}
\pagestyle{fancy}

% testo
\subsection{Insiemi di livello}
Per la ricerca di massimi e minimi vincolati può essere utile la definizione di insiemi di livello:
\begin{definition}{Insieme di livello}
	Definiamo l'insieme di livello di una funzione $f$ per una certa costante $k$ come:
	$$
	\mathrm{lev}_{\leq k} \, f(x) = \left\{ x \in \mathbb{R}^n: f(x) \leq k \right\}
	$$
	o alternativamente:
	$$
	\mathrm{lev}_{\geq k} \, f(x) = \left\{ x \in \mathbb{R}^n: f(x) \geq k \right\}
	$$
\end{definition}

Si ha che, se l'insieme $\mathrm{lev}_{\leq k} \, f(x)$ di $f$ esiste ed è compatto per ogni $k$, allora $f$ ammette minimo, mentre se l'insieme $\mathrm{lev}_{\geq k} \, f(x)$  esiste ed è compatto, sempre per ogni $k$, allora $f$ ammette massimo.

\subsection{Definizione generale di un algoritmo di PNL}
Abbiamo quindi che vorremmo che un algoritmo di PNL, generalmente, fosse costituito da:
\begin{itemize}
	\item Una \textbf{regola} per la scelta delle direzioni di crescita $d_k$;
	\item Una \textbf{regola} per la scelta del passo $t$;
	\item Un \textbf{teorema} che dimsotra che le regole di scelta portano sempre a un massimo.
\end{itemize}

Purtroppo, se non in alcuni casi particolari, questo teorema non esiste.
Di base, arriviamo sicuramente ad un \textit{punto stazionario}, cioè un punto $\bar{x}$ dove $\nabla \bar{x} = 0$, cioè vale il teorema:
\begin{theorem}{sui punti di accumulazione di successioni crescenti}
	Supponiamo $\mathrm{lev}_{\geq x}$ compatti.
	Allora la successione $x_k \in \mathbb{R}^n$ ha punti di accumulazione, ognuno dei quali è stazionario.
\end{theorem}

Notiamo che, nel caso di funzioni convesse, significa che il massimo effettivamente viene sempre raggiunto (e l'unico punto a gradiente nullo), come di contro nel caso di funzioni concave il minimo viene sempre raggiunto.

\subsubsection{Criterio di stop}
Visto che non è assicurato che la serie raggiunga il massimo in un numero finito di passi, introduciamo un \textbf{criterio di stop}, cioè che decide quando il risultato trovato finora dall'algoritmo e valido e so può estrarre la soluzione approssimata che ha trovato:
\begin{enumerate}
	\item Innanzitutto possiamo limitare le iterazioni: prendi $x_k$ con $k < M$ elevato a piacere; \\ 
	\item Possiamo scegliere il punto di scomparsa del gradiente sotto un certo $\varepsilon$, cioè $|\nabla f(x_k)| \leq \varepsilon$;
	\item Posssiamo pensare in termini di \textit{guadagni} veri e propri su ogni passo, cioè $\Delta f = f(x_{k+1}) - f(x_k)$, e porre $\Delta f < \varepsilon$ per qualche $\varepsilon$.
\end{enumerate}

\par\medskip

Vediamo quindi alcuni algoritmi di PNL di questo tipo:

\begin{itemize}
	\item \textsf{\textbf{Metodo del gradiente a passo costante}} \\
		Si pone come regola:
		\[
			\begin{cases}
				a_k = \nabla f (x_k) \\ 
				t_k = \eta 
			\end{cases}
		\]
		con $\eta \in 0 < \eta < \frac{2}{L}$, dove $L$, detto \textit{numero di Lipschitz}, è il masssimo delle derivate seconde.
		
		Si ha quindi la relazione di ricorrenza:
		$$
			x_{k+1} = x_k + \eta \nabla f(x_k)
		$$

		Il teorema di convergenza sarà allora, semplicemente, che $x_k$ ha punto di accumulazione su $f$ e ogni punto di accumulazione $\bar{x}$ è tale che $\nabla f(\bar{x}) = 0$, $\forall x_0$ posizione di partenza:

	\item \textsf{\textbf{Metodo del gradiente a passo ideale}} \\ 
		Si pone come regola:
		\[
			\begin{cases}
				d_k = \nabla f(x_k) \\ 
				t_k \in \mathrm{argmax}_{t \geq 0} f(x_k + t \, d_k)
			\end{cases}
		\]
		dove si deve avere $H$ invertibile.
		La relazione di ricorrenza è:
		$$
			x_{k+1} = x_k + t_k \nabla f(x_k)
		$$

		Il teorema di convergenza è come sopra;
		
	\item \textsf{\textbf{Metodo di Newton}} \\
		Si pone come regola:
		\[
			\begin{cases}
				d_k = H f(x_k)^{-1} \nabla f(x_k) \\ 
				t_k = 1
			\end{cases}
		\]
		da cui la relazione di ricorrenza:
		$$
		x_{k+1} = x_k - H f(x_k)^{-1} \nabla f(x_k)
		$$
		
		Il teorema di convergenza è come sopra.

\end{itemize}

\end{document}
