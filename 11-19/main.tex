
\documentclass[a4paper,11pt]{article}
\usepackage[a4paper, margin=8em]{geometry}

% usa i pacchetti per la scrittura in italiano
\usepackage[french,italian]{babel}
\usepackage[T1]{fontenc}
\usepackage[utf8]{inputenc}
\frenchspacing 

% usa i pacchetti per la formattazione matematica
\usepackage{amsmath, amssymb, amsthm, amsfonts}

% usa altri pacchetti
\usepackage{gensymb}
\usepackage{hyperref}
\usepackage{standalone}

% imposta il titolo
\title{Appunti Ricerca Operativa}
\author{Luca Seggiani}
\date{2024}

% disegni
\usepackage{pgfplots}
\pgfplotsset{width=10cm,compat=1.9}

% imposta lo stile
% usa helvetica
\usepackage[scaled]{helvet}
% usa palatino
\usepackage{palatino}
% usa un font monospazio guardabile
\usepackage{lmodern}

\renewcommand{\rmdefault}{ppl}
\renewcommand{\sfdefault}{phv}
\renewcommand{\ttdefault}{lmtt}

% disponi il titolo
\makeatletter
\renewcommand{\maketitle} {
	\begin{center} 
		\begin{minipage}[t]{.8\textwidth}
			\textsf{\huge\bfseries \@title} 
		\end{minipage}%
		\begin{minipage}[t]{.2\textwidth}
			\raggedleft \vspace{-1.65em}
			\textsf{\small \@author} \vfill
			\textsf{\small \@date}
		\end{minipage}
		\par
	\end{center}

	\thispagestyle{empty}
	\pagestyle{fancy}
}
\makeatother

% disponi teoremi
\usepackage{tcolorbox}
\newtcolorbox[auto counter, number within=section]{theorem}[2][]{%
	colback=blue!10, 
	colframe=blue!40!black, 
	sharp corners=northwest,
	fonttitle=\sffamily\bfseries, 
	title=Teorema~\thetcbcounter: #2, 
	#1
}

% disponi definizioni
\newtcolorbox[auto counter, number within=section]{definition}[2][]{%
	colback=red!10,
	colframe=red!40!black,
	sharp corners=northwest,
	fonttitle=\sffamily\bfseries,
	title=Definizione~\thetcbcounter: #2,
	#1
}

% disponi problemi
\newtcolorbox[auto counter, number within=section]{problem}[2][]{%
	colback=green!10,
	colframe=green!40!black,
	sharp corners=northwest,
	fonttitle=\sffamily\bfseries,
	title=Problema~\thetcbcounter: #2,
	#1
}

% disponi codice
\usepackage{listings}
\usepackage[table]{xcolor}

\lstdefinestyle{codestyle}{
		backgroundcolor=\color{black!5}, 
		commentstyle=\color{codegreen},
		keywordstyle=\bfseries\color{magenta},
		numberstyle=\sffamily\tiny\color{black!60},
		stringstyle=\color{green!50!black},
		basicstyle=\ttfamily\footnotesize,
		breakatwhitespace=false,         
		breaklines=true,                 
		captionpos=b,                    
		keepspaces=true,                 
		numbers=left,                    
		numbersep=5pt,                  
		showspaces=false,                
		showstringspaces=false,
		showtabs=false,                  
		tabsize=2
}

\lstdefinestyle{shellstyle}{
		backgroundcolor=\color{black!5}, 
		basicstyle=\ttfamily\footnotesize\color{black}, 
		commentstyle=\color{black}, 
		keywordstyle=\color{black},
		numberstyle=\color{black!5},
		stringstyle=\color{black}, 
		showspaces=false,
		showstringspaces=false, 
		showtabs=false, 
		tabsize=2, 
		numbers=none, 
		breaklines=true
}

\lstdefinelanguage{javascript}{
	keywords={typeof, new, true, false, catch, function, return, null, catch, switch, var, if, in, while, do, else, case, break},
	keywordstyle=\color{blue}\bfseries,
	ndkeywords={class, export, boolean, throw, implements, import, this},
	ndkeywordstyle=\color{darkgray}\bfseries,
	identifierstyle=\color{black},
	sensitive=false,
	comment=[l]{//},
	morecomment=[s]{/*}{*/},
	commentstyle=\color{purple}\ttfamily,
	stringstyle=\color{red}\ttfamily,
	morestring=[b]',
	morestring=[b]"
}

% disponi sezioni
\usepackage{titlesec}

\titleformat{\section}
	{\sffamily\Large\bfseries} 
	{\thesection}{1em}{} 
\titleformat{\subsection}
	{\sffamily\large\bfseries}   
	{\thesubsection}{1em}{} 
\titleformat{\subsubsection}
	{\sffamily\normalsize\bfseries} 
	{\thesubsubsection}{1em}{}

% disponi alberi
\usepackage{forest}

\forestset{
	rectstyle/.style={
		for tree={rectangle,draw,font=\large\sffamily}
	},
	roundstyle/.style={
		for tree={circle,draw,font=\large}
	}
}

% disponi algoritmi
\usepackage{algorithm}
\usepackage{algorithmic}
\makeatletter
\renewcommand{\ALG@name}{Algoritmo}
\makeatother

% disponi numeri di pagina
\usepackage{fancyhdr}
\fancyhf{} 
\fancyfoot[L]{\sffamily{\thepage}}

\makeatletter
\fancyhead[L]{\raisebox{1ex}[0pt][0pt]{\sffamily{\@title \ \@date}}} 
\fancyhead[R]{\raisebox{1ex}[0pt][0pt]{\sffamily{\@author}}}
\makeatother

\begin{document}

% sezione (data)
\section{Lezione del 19-11-24}

% stili pagina
\thispagestyle{empty}
\pagestyle{fancy}

% testo
\subsubsection{Algoritmo di Ford-Fulkerson}
L'algoritmo di \textbf{Ford-Fulkerson} (precisamente nella variante di \textit{Edmonds-Karp}) è un algoritmo per il calcolo del flusso massimo su reti.

Iniziamo con alcune definizioni, prima fra tutte quella di \textbf{taglio di rete}:
\begin{definition}{Taglio di rete}
	Dato un problema di flusso massimo, un taglio di rete è una partizione dei nodi in due sottoinsiemi $N_s$ e $N_t$, quindi con $N_s \cup N_t = N$ e $N_s \cap N_t = \emptyset$, con il vincolo aggiunto che la sorgente di flusso $s$ deve trovarsi $\in N_s$, e la destinazione del flusso $t$ deve trovarsi $\in N_t$.
\end{definition}

Possiamo quindi dare la definizione di \textbf{arco diretto del taglio}:
\begin{definition}{Arco diretto del taglio}
	Dato un taglio, si dicono archi diretti del taglio tutti gli archi $(i,j)$ che hanno $i \in N_s$ e $j \in N_t$.
\end{definition}
con enfasi sul fatto che un arco diretto del taglio va da $N_s$ a $N_t$, e \textbf{non} viceversa.

Infine, su questi archi diretti del taglio definiamo la \textbf{portata}:
\begin{definition}{Portata del taglio}
	Dato un taglio $(N_s, N_t)$, la portata $\mu(N_s, N_t)$ viene definita come:
	$$
		\mu(N_s, N_t) = \sum_{(i,j) \in A^\intercal} u_{ij}
	$$
\end{definition}

Notiamo come ogni taglio di rete rappresenta un limite superiore sul flusso massimo:
	$$
		\overline{x} \leq \mu(N_s, N_t)
	$$
Possiamo quindi dimostrare il teorema:
\begin{theorem}{Massimo flussimo - minimo taglio}
	Il flusso massimo di un problema di flusso massimo con soluzione $\neq -\infty$ è uguale al taglio di portata minima:
	$$
		\overline{v} = \overline{u}(N_s, N_t) = \min_{s, t} \mu(N_s, N_t)
	$$
\end{theorem}

Notiamo come le ultime due equazioni ricalcano rispettivamente la \textbf{dualità debole} e la \textbf{dualità forte}: si può effettivamente dimostrare che il problema del taglio minimo è il \textbf{duale} del problema di flusso massimo (effettivamente vogliamo trovare un \textit{limite superiore} del flusso, che è lo stesso procedimento che avevamo adottato,, in quel caso adattando le diseguaglianze, nel ricavare il duale nei problemi di LP).

Chiaramente è improbabile calcolare il taglio minimo per enumerazione completa: togliendo il fatto che le partizioni dei tagli lavorano su $n-2$ anziché $n$ nodi (in quanto ci sono i vincoli $s \in N_s$ e $t \in N_t$), il numero di partizioni va comunque come $\sim 2^n$, che è quindi a complessità \textbf{esponenziale}.

Vediamo allora quale algoritmo possiamo usare per ricavare i tagli minimi (e quindi i flussi massimi) in maniera efficiente.

\subsubsection{Definizione dell'algoritmo}
Prendiamo in considerazione un grafo e il suo \textbf{grafo residuo}, cioè il grafo che complementa tutti gli archi (diretti) con un arco (diretto) identico ma di verso opposto. 
Chiameremo \textit{archi reali} gli archi che fanno effettivamente parte del grafo, e \textit{archi fittizi} gli archi che introduciamo sul grafo residuo. 

Per ogni arco $(i,j)$, reale o fittizio, definiamo la \textbf{capacità residua} come:
\begin{itemize}
	\item Archi reali: $r_{ij} = u_{ij} - \overline{x}_{ij}$
	\item Archi fittizi: $r_{ji} = \overline{x}_{ij}$
\end{itemize}
assunto nelle definizioni che $(i, j)$ sia l'arco reale e di conseguenza $(j, i)$ l'arco fittizio.

A parole, la capacità residua dell'arco \textit{reale} è \textit{quanto posso ancora spedire sull'arco}, mentre la capacità residua dell'arco \textit{fittizio} è \textit{quanto ho già spedito sull'arco vero}, da cui:
$$
r_{ij} + r_{ji} = u_{ij} - \overline{x}_{ij} + \overline{x}_{ij} = u_{ij}
$$
cioè la somma delle capacità residue su entrambi gli archi nelle due direzioni del grafo residuo dà sempre la capacità massima sull'arco reale nella stessa posizione del grafo originale.

Definiamo allora il concetto di \textbf{cammino aumentante} sul grafo:
\begin{definition}{Cammino aumentante}
Dato un problema di flusso massimo, chiamiamo cammino aumentante $C_{aum}$ sul grafo residuo un qualsiasi cammino orientato da $s$ a $t$ formato da archi con capacità residue $r_{ij} > 0$.
\end{definition}

Di un dato cammino aumentante $C_{aum}$, il dato interessante è la \textbf{portata} di $C_{aum}$:
$$
\delta = \min_{i,j \in C_{aum}} \{ r_{ij} \}
$$

Il primo passo dell'algoritmo di Ford-Fulkerson sarà allora selezionare un cammino aumentante e calcolarne la portata $\delta$.
In seguito vorremo spedire queste $\delta$ unità di flusso lungo il cammino, quindi applicando la regola:
\[
	\overline{x}' = 	
	\begin{cases}
		\overline{x}_{ij} + \delta, \quad \forall (i, j) \in C_{aum} \\ 
		\overline{x}_{ij}, \quad \forall (i, j) \notin C_{aum}
	\end{cases}
\]

A questo punto basta ricalcolare le varie capacità residue $r_{ij}$ e ripetere il passaggio.
Chiaramente, quando non si riuscirà più a trovare cammini aumentanti, cioè archi con capacità residue $> 0$, significherà che avremo trovato la soluzione ottima. 

\begin{algorithm}[H]
\caption{di Ford-Fulkerson}
\begin{algorithmic}
	\STATE \textbf{Input:} un problema di flusso massimo
	\STATE \textbf{Output:} la soluzione ottima
	% body
	\STATE \textsf{ciclo:}
	\STATE Stabilisci un cammino aumentante $C_{aum}$ e calcola la portata $\delta$.
	\IF{$C_{aum} = \emptyset$} 
		\STATE Fermati, sei all'ottimo
	\ENDIF
	\STATE Aggiorna il flusso $\overline{x}$ secondo la regola:
	\[
		\overline{x}' = 	
		\begin{cases}
			\overline{x}_{ij} + \delta, \quad \forall (i, j) \in C_{aum} \\ 
			\overline{x}_{ij}, \quad \forall (i, j) \notin C_{aum}
		\end{cases}
	\]
	\STATE Ricalcola le capacità residue $r_{ij}$
	\STATE Torna a \textsf{ciclo}
\end{algorithmic}
\end{algorithm}

Si presenta un algoritmo per il calcolo dei cammini aumentanti, detto \textbf{algoritmo della croce}:
\begin{algorithm}[H]
\caption{della croce}
\begin{algorithmic}
	\STATE \textbf{Input:} il grafo residuo di un problema di flusso massimo % input
	\STATE \textbf{Output:} un cammino aumentante $C_{aum}$ % output
	% body
	\STATE Inizializza due vettori, \lstinline|closed| e \lstinline|open|
	\STATE Inserisci $s$, il nodo di partenza, in \lstinline|closed|
	\STATE Inserisci tutti i nodi raggiungibili da \lstinline|closed|, cioè quelli che sono collegati con un arco diretto dal nodo 1 con costo residuo $> 0$, in \lstinline|open|
	\STATE \textsf{ciclo:}
	\STATE Prendi il primo elemento di \lstinline|open| e mettilo in \lstinline|closed|
	\STATE Inserisci tutti i nodi raggiungibili da \lstinline|closed|, cioè quelli che sono collegati con un arco diretto dal nodo in \lstinline|closed| con costo residuo $> 0$
	\IF{non hai raggiunto $t$; cioè il nodo di arrivo}
		\STATE Torna a \textsf{ciclo}
	\ENDIF
	\STATE Ripercorrendo i nodi in \lstinline|closed| al contrario, partendo da quello di arrivo, hai il cammino aumentate $C_{aum}$
\end{algorithmic}
\end{algorithm}

Notiamo come, nell'applicazione dell'algoritmo della croce, è necessaria la presenza degli archi residui in entrambe le direzioni: potrebbe infatti succedere che un arco abbia flusso aggiuntivo $\delta$ nella direzione opposta a quella canonica.
Questo significherà che il flusso che spingiamo nella direzione opposta andrà a sottrarsi al flusso già presente sull'arco (in altre parole, l'algoritmo di Ford-Fulkerson può "cambiare idea" nell'esplorazione dello spazio delle soluzioni).

\end{document}
