
\documentclass[a4paper,11pt]{article}
\usepackage[a4paper, margin=8em]{geometry}

% usa i pacchetti per la scrittura in italiano
\usepackage[french,italian]{babel}
\usepackage[T1]{fontenc}
\usepackage[utf8]{inputenc}
\frenchspacing 

% usa i pacchetti per la formattazione matematica
\usepackage{amsmath, amssymb, amsthm, amsfonts}

% usa altri pacchetti
\usepackage{gensymb}
\usepackage{hyperref}
\usepackage{standalone}

% imposta il titolo
\title{Appunti Ricerca Operativa}
\author{Luca Seggiani}
\date{2024}

% disegni
\usepackage{pgfplots}
\pgfplotsset{width=10cm,compat=1.9}

% imposta lo stile
% usa helvetica
\usepackage[scaled]{helvet}
% usa palatino
\usepackage{palatino}
% usa un font monospazio guardabile
\usepackage{lmodern}

\renewcommand{\rmdefault}{ppl}
\renewcommand{\sfdefault}{phv}
\renewcommand{\ttdefault}{lmtt}

% disponi il titolo
\makeatletter
\renewcommand{\maketitle} {
	\begin{center} 
		\begin{minipage}[t]{.8\textwidth}
			\textsf{\huge\bfseries \@title} 
		\end{minipage}%
		\begin{minipage}[t]{.2\textwidth}
			\raggedleft \vspace{-1.65em}
			\textsf{\small \@author} \vfill
			\textsf{\small \@date}
		\end{minipage}
		\par
	\end{center}

	\thispagestyle{empty}
	\pagestyle{fancy}
}
\makeatother

% disponi teoremi
\usepackage{tcolorbox}
\newtcolorbox[auto counter, number within=section]{theorem}[2][]{%
	colback=blue!10, 
	colframe=blue!40!black, 
	sharp corners=northwest,
	fonttitle=\sffamily\bfseries, 
	title=Teorema~\thetcbcounter: #2, 
	#1
}

% disponi definizioni
\newtcolorbox[auto counter, number within=section]{definition}[2][]{%
	colback=red!10,
	colframe=red!40!black,
	sharp corners=northwest,
	fonttitle=\sffamily\bfseries,
	title=Definizione~\thetcbcounter: #2,
	#1
}

% disponi problemi
\newtcolorbox[auto counter, number within=section]{problem}[2][]{%
	colback=green!10,
	colframe=green!40!black,
	sharp corners=northwest,
	fonttitle=\sffamily\bfseries,
	title=Problema~\thetcbcounter: #2,
	#1
}

% disponi codice
\usepackage{listings}
\usepackage[table]{xcolor}

\lstdefinestyle{codestyle}{
		backgroundcolor=\color{black!5}, 
		commentstyle=\color{codegreen},
		keywordstyle=\bfseries\color{magenta},
		numberstyle=\sffamily\tiny\color{black!60},
		stringstyle=\color{green!50!black},
		basicstyle=\ttfamily\footnotesize,
		breakatwhitespace=false,         
		breaklines=true,                 
		captionpos=b,                    
		keepspaces=true,                 
		numbers=left,                    
		numbersep=5pt,                  
		showspaces=false,                
		showstringspaces=false,
		showtabs=false,                  
		tabsize=2
}

\lstdefinestyle{shellstyle}{
		backgroundcolor=\color{black!5}, 
		basicstyle=\ttfamily\footnotesize\color{black}, 
		commentstyle=\color{black}, 
		keywordstyle=\color{black},
		numberstyle=\color{black!5},
		stringstyle=\color{black}, 
		showspaces=false,
		showstringspaces=false, 
		showtabs=false, 
		tabsize=2, 
		numbers=none, 
		breaklines=true
}

\lstdefinelanguage{javascript}{
	keywords={typeof, new, true, false, catch, function, return, null, catch, switch, var, if, in, while, do, else, case, break},
	keywordstyle=\color{blue}\bfseries,
	ndkeywords={class, export, boolean, throw, implements, import, this},
	ndkeywordstyle=\color{darkgray}\bfseries,
	identifierstyle=\color{black},
	sensitive=false,
	comment=[l]{//},
	morecomment=[s]{/*}{*/},
	commentstyle=\color{purple}\ttfamily,
	stringstyle=\color{red}\ttfamily,
	morestring=[b]',
	morestring=[b]"
}

% disponi sezioni
\usepackage{titlesec}

\titleformat{\section}
	{\sffamily\Large\bfseries} 
	{\thesection}{1em}{} 
\titleformat{\subsection}
	{\sffamily\large\bfseries}   
	{\thesubsection}{1em}{} 
\titleformat{\subsubsection}
	{\sffamily\normalsize\bfseries} 
	{\thesubsubsection}{1em}{}

% disponi alberi
\usepackage{forest}

\forestset{
	rectstyle/.style={
		for tree={rectangle,draw,font=\large\sffamily}
	},
	roundstyle/.style={
		for tree={circle,draw,font=\large}
	}
}

% disponi algoritmi
\usepackage{algorithm}
\usepackage{algorithmic}
\makeatletter
\renewcommand{\ALG@name}{Algoritmo}
\makeatother

% disponi numeri di pagina
\usepackage{fancyhdr}
\fancyhf{} 
\fancyfoot[L]{\sffamily{\thepage}}

\makeatletter
\fancyhead[L]{\raisebox{1ex}[0pt][0pt]{\sffamily{\@title \ \@date}}} 
\fancyhead[R]{\raisebox{1ex}[0pt][0pt]{\sffamily{\@author}}}
\makeatother

\begin{document}

% sezione (data)
\section{Lezione del 24-10-24}

% stili pagina
\thispagestyle{empty}
\pagestyle{fancy}

% testo
\subsubsection{Riassunto TSP simmetrico}
Avevamo quindi posto problemi con vincoli  in forma:
\[
	\begin{cases}
			\min c^T \cdot x \\ 
			\sum\limits_{x < j} x_{ij} + \sum\limits_{j < y} x_{ij} = 2, \quad \forall j \\ 
			\sum\limits_{i \in S \, j \notin S} x_{ij} + \sum\limits_{i \notin S \, j \in S} x_{ij} \geq 1, \quad \forall S \subset N, \quad 2 \leq |S| \leq \left\lceil \frac{|N|}{2} \right\rceil  
	\end{cases}
\]

cioè dove si poneva la somma dei nodi entranti e uscenti (i vincoli \textit{di grado}) da $j$ come $=2$.
Visto che $j$ era l'unico nodo su cui era imposto il vincolo, si aveva che la soluzione del problema era il $j$-albero a costo minimo.

Inoltre, si aveva che i vincoli sulla terza erano quelli \textit{di connessione}, che si cercavano solo su cardinalità dei sottoinsiemi $= \left\lceil \frac{|N|}{2} \right\rceil $, in quanto i vincoli per $S$ valevano anche per $N \setminus S$.

\subsection{Branch and bound}
Avevamo visto l'algoritmo dei piani di taglio di Gomory per il calcolo di soluzioni approssimate di problemi ILP.
Presentiamo adesso un altro metodo, detto \textbf{branch and bound}.
Il metodo più naive che possiamo adottare per risolvere un problema in forma:
\[
	\begin{cases}
		\max c^T x \\ 
		A x\leq b \\ 
		x \in \{0, 1\}^n
	\end{cases}
\]
è quello di enumerare tutti le possibili soluzioni ammissibili $\in \{0,1\}^n$, costruendo il cosiddetto \textbf{albero di enumerazione}.
Scegliamo quindi una variabile, $x_1$, e costruiamo l'albero:

\begin{center}
	\begin{forest}
		[$x_1$, roundstyle 
			[$x_2$, name=1, label={$x_0 = 0$}
				[,phantom
				[$x_n$, name=1a 
						[{$\scriptstyle (0,..., 0,0)$}, rectstyle]
						[{$\scriptstyle (0,..., 0,1)$}, rectstyle]
					]
					[$x_n$, name=1b
						[{$\scriptstyle (0,,..., 1,0)$}, rectstyle]
						[{$\scriptstyle (0,..., 1,1)$}, rectstyle]
					]
				]
			]
			[$x_2$, name=2, label={$x_0 = 1$}
				[,phantom
					[$x_n$, name=2a
						[{$\scriptstyle (1,,..., 0,0)$}, rectstyle]
						[{$\scriptstyle (1,,..., 0,1)$}, rectstyle]
					]
					[$x_n$,name=2b
						[{$\scriptstyle (1,,..., 1,0)$}, rectstyle]
						[{$\scriptstyle (1,..., 1,1)$}, rectstyle]
					]
				]
			]
		]	
		\path(1)--node[sloped]{\Large\dots}(1a);
		\path(1)--node[sloped]{\Large\dots}(1b);
		\path(2)--node[sloped]{\Large\dots}(2a);
		\path(2)--node[sloped]{\Large\dots}(2b);
	\end{forest}
\end{center}

Abbiamo che il calcolo di ogni nodo dell'albero non è effettivamente necessario.
Chiamiamo quindi \textbf{problemi} $P_{ij}$ ogni nodo dell albero, con $i$ che seleziona il livello e $j$ il fratello, da sinstra verso destra.

Stabilita una valutazione superiore e inferiore di ogni problema, che chiameremo $v_S(P)$ e $v_I(P)$, abbiamo:
$$
v_I(P) \leq v(P) \leq v_S(P) 
$$

Possiamo usare queste valutazioni per stabilire \textbf{regole di taglio} che ci permettano di tagliare (si dice anche \textit{visitare implicitamente}) un'intero sottoalbero a partire da un certo nodo $P_{ij}$.
Queste regole di taglio assicurano, essenzialmente, che ogni sottoproblema $P_{{i+k}, j'}$ figlio di $P_{ij}$ non contiene l'ottimo, e quindi si può saltare.

Vediamo quindi le regole di taglio che possiamo adottare.
Mantendendo un'ottimo corrente $x$, abbiamo che calcolato un nuovo $P_{ij}$ per enumerazione:
\begin{enumerate}
	\item $P_{ij} = \emptyset$ significa che per un certo nodo $P_{ij}$ possiamo tagliare tutti i problemi che istanziano le $i$ variabili di $P_{ij}$, ergo tutti i suoi nodi figli;
	\item Calcolo di $v_S(P_{ij})$. Se $v_S(P_{ij}) < v_I(P)$ del problema, allora posso scartare il sottoalbero: non troverò soluzioni migliori scendendovi;
	\item $v_S(P_{ij}) > v_I(P)$, e l'$\bar{x}$ dove si ha tale $v_S$ è ammissibile per $P$, allora si prende $\bar{x}$ come nuovo $x$ e si fa una visita implicità di $P_{ij}$: questo perchè un suo sottoproblema non potrà darci di meglio (più scendiamo nell'albero, più stringiamo i vincoli, ergo $P_{i+k, j'}$ figlio di $P_{ij}$ ha $v_S(P_{i+k, j'}) \leq v_S(P_{ij})$).
\end{enumerate}

Ricordiamo che quanto detto finora vale su problemi \textbf{massimizzanti}: su problemi \textbf{minimizzanti} dovremo invertire l'ordine delle disugaglianze, e prendere vincoli superiori anziché inferiori e viceversa.

\end{document}
