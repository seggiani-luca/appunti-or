
\documentclass[a4paper,11pt]{article}
\usepackage[a4paper, margin=8em]{geometry}

% usa i pacchetti per la scrittura in italiano
\usepackage[french,italian]{babel}
\usepackage[T1]{fontenc}
\usepackage[utf8]{inputenc}
\frenchspacing 

% usa i pacchetti per la formattazione matematica
\usepackage{amsmath, amssymb, amsthm, amsfonts}

% usa altri pacchetti
\usepackage{gensymb}
\usepackage{hyperref}
\usepackage{standalone}

% imposta il titolo
\title{Appunti Ricerca Operativa}
\author{Luca Seggiani}
\date{2024}

% disegni
\usepackage{pgfplots}
\pgfplotsset{width=10cm,compat=1.9}

% imposta lo stile
% usa helvetica
\usepackage[scaled]{helvet}
% usa palatino
\usepackage{palatino}
% usa un font monospazio guardabile
\usepackage{lmodern}

\renewcommand{\rmdefault}{ppl}
\renewcommand{\sfdefault}{phv}
\renewcommand{\ttdefault}{lmtt}

% disponi il titolo
\makeatletter
\renewcommand{\maketitle} {
	\begin{center} 
		\begin{minipage}[t]{.8\textwidth}
			\textsf{\huge\bfseries \@title} 
		\end{minipage}%
		\begin{minipage}[t]{.2\textwidth}
			\raggedleft \vspace{-1.65em}
			\textsf{\small \@author} \vfill
			\textsf{\small \@date}
		\end{minipage}
		\par
	\end{center}

	\thispagestyle{empty}
	\pagestyle{fancy}
}
\makeatother

% disponi teoremi
\usepackage{tcolorbox}
\newtcolorbox[auto counter, number within=section]{theorem}[2][]{%
	colback=blue!10, 
	colframe=blue!40!black, 
	sharp corners=northwest,
	fonttitle=\sffamily\bfseries, 
	title=Teorema~\thetcbcounter: #2, 
	#1
}

% disponi definizioni
\newtcolorbox[auto counter, number within=section]{definition}[2][]{%
	colback=red!10,
	colframe=red!40!black,
	sharp corners=northwest,
	fonttitle=\sffamily\bfseries,
	title=Definizione~\thetcbcounter: #2,
	#1
}

% disponi problemi
\newtcolorbox[auto counter, number within=section]{problem}[2][]{%
	colback=green!10,
	colframe=green!40!black,
	sharp corners=northwest,
	fonttitle=\sffamily\bfseries,
	title=Problema~\thetcbcounter: #2,
	#1
}

% disponi codice
\usepackage{listings}
\usepackage[table]{xcolor}

\lstdefinestyle{codestyle}{
		backgroundcolor=\color{black!5}, 
		commentstyle=\color{codegreen},
		keywordstyle=\bfseries\color{magenta},
		numberstyle=\sffamily\tiny\color{black!60},
		stringstyle=\color{green!50!black},
		basicstyle=\ttfamily\footnotesize,
		breakatwhitespace=false,         
		breaklines=true,                 
		captionpos=b,                    
		keepspaces=true,                 
		numbers=left,                    
		numbersep=5pt,                  
		showspaces=false,                
		showstringspaces=false,
		showtabs=false,                  
		tabsize=2
}

\lstdefinestyle{shellstyle}{
		backgroundcolor=\color{black!5}, 
		basicstyle=\ttfamily\footnotesize\color{black}, 
		commentstyle=\color{black}, 
		keywordstyle=\color{black},
		numberstyle=\color{black!5},
		stringstyle=\color{black}, 
		showspaces=false,
		showstringspaces=false, 
		showtabs=false, 
		tabsize=2, 
		numbers=none, 
		breaklines=true
}

\lstdefinelanguage{javascript}{
	keywords={typeof, new, true, false, catch, function, return, null, catch, switch, var, if, in, while, do, else, case, break},
	keywordstyle=\color{blue}\bfseries,
	ndkeywords={class, export, boolean, throw, implements, import, this},
	ndkeywordstyle=\color{darkgray}\bfseries,
	identifierstyle=\color{black},
	sensitive=false,
	comment=[l]{//},
	morecomment=[s]{/*}{*/},
	commentstyle=\color{purple}\ttfamily,
	stringstyle=\color{red}\ttfamily,
	morestring=[b]',
	morestring=[b]"
}

% disponi sezioni
\usepackage{titlesec}

\titleformat{\section}
	{\sffamily\Large\bfseries} 
	{\thesection}{1em}{} 
\titleformat{\subsection}
	{\sffamily\large\bfseries}   
	{\thesubsection}{1em}{} 
\titleformat{\subsubsection}
	{\sffamily\normalsize\bfseries} 
	{\thesubsubsection}{1em}{}

% disponi alberi
\usepackage{forest}

\forestset{
	rectstyle/.style={
		for tree={rectangle,draw,font=\large\sffamily}
	},
	roundstyle/.style={
		for tree={circle,draw,font=\large}
	}
}

% disponi algoritmi
\usepackage{algorithm}
\usepackage{algorithmic}
\makeatletter
\renewcommand{\ALG@name}{Algoritmo}
\makeatother

% disponi numeri di pagina
\usepackage{fancyhdr}
\fancyhf{} 
\fancyfoot[L]{\sffamily{\thepage}}

\makeatletter
\fancyhead[L]{\raisebox{1ex}[0pt][0pt]{\sffamily{\@title \ \@date}}} 
\fancyhead[R]{\raisebox{1ex}[0pt][0pt]{\sffamily{\@author}}}
\makeatother

\begin{document}

% sezione (data)
\section{Lezione del 28-11-24}

% stili pagina
\thispagestyle{empty}
\pagestyle{fancy}

% testo
\subsection{Teorema di Lagrange-Kuhn-Tucker}
Abbiamo assunto come ipotesi per i domini di $f$ che prenderemo in considerazione, cioè:
$$
\Omega = 
	\begin{cases}
		g(x) \leq 0 \\ 
		h(x) = 0
	\end{cases}
$$
che $f, g, h \in C^1$, e $\Omega$ regolare secondo una qualsiasi delle forme viste in precedenza (ci basta la \textit{condizione di Slater}).

Vediamo allora il teorema, detto \textit{di Lagrange-Kuhn-Tucker}, o di \textit{Lagrange-Karush-Kuhn-Tucker}, o più semplicemente \textbf{LKT} o \textbf{LKKT}:
\begin{theorem}{Teorema di Lagrange-Kuhn-Tucker}
	Sia $\bar{x} \in D$ minimo locale.
	Allora $\exists \bar{\lambda} \in \mathbb{R}^m$, $\bar{\lambda} \geq 0$, $\exists \bar{\mu} \in \mathbb{R}^p$ tali che:
	\[
		\begin{cases}
			\nabla f(\bar{x}) + \sum\limits_{i=1}^m \bar{\lambda}_i \nabla g_i(\bar{x}) + \sum\limits_{i=1}^m \bar{\mu}_i \nabla h_j(\bar{x}) = 0 \\ 
			\bar{\lambda}_i g_i(\bar{x}) = 0, \quad \forall i = 1, ..., m \\ 
			h_j(\bar{x}) = 0, \quad \forall j = 1, ..., p	
		\end{cases}
	\]
\end{theorem}
Il sistema ha $n + m + p$ equazioni su $n + m + p$ variabili e viene detto \textbf{sistema LKT}.
Notiamo che nel caso di minimo, vorremo che i moltiplicatori $\bar{\lambda}$ siano $\leq 0$ anziché $\geq 0$.

La funzione:
$$
\mathcal{L}(x, \lambda, \mu) = f(x) + \sum\limits_{i=1}^m \lambda_i g_i(x) + \sum\limits_{j=1}^p \mu_j h_j(x)
$$
viene detta \textbf{Lagrangiana} $\mathcal{L}(x, \lambda, \mu)$.
Allora la prima equazione dell'LKT non sarà altro che:
$$
\nabla f(\bar{x}) + \sum\limits_{i=1}^m \bar{\lambda}_i \nabla g_i(\bar{x}) + \sum\limits_{i=1}^m \bar{\mu}_i \nabla h_j(\bar{x}) = \nabla_x \mathcal{L}(\bar{x}, \bar{\lambda}, \bar{\mu}) = 0 \\ 
$$

In generale, essere soluzione del sistema LKT è per un punto condizione \textbf{necessaria} ad essere massimo (minimo).
Questo significa che tutti i massimi (minimi) della funzione $f$ sottoposta ai vincoli $\Omega$ sono soluzioni dell'LKT.

\end{document}
