
\documentclass[a4paper,11pt]{article}
\usepackage[a4paper, margin=8em]{geometry}

% usa i pacchetti per la scrittura in italiano
\usepackage[french,italian]{babel}
\usepackage[T1]{fontenc}
\usepackage[utf8]{inputenc}
\frenchspacing 

% usa i pacchetti per la formattazione matematica
\usepackage{amsmath, amssymb, amsthm, amsfonts}

% usa altri pacchetti
\usepackage{gensymb}
\usepackage{hyperref}
\usepackage{standalone}

% imposta il titolo
\title{Appunti Ricerca Operativa}
\author{Luca Seggiani}
\date{2024}

% disegni
\usepackage{pgfplots}
\pgfplotsset{width=10cm,compat=1.9}

% imposta lo stile
% usa helvetica
\usepackage[scaled]{helvet}
% usa palatino
\usepackage{palatino}
% usa un font monospazio guardabile
\usepackage{lmodern}

\renewcommand{\rmdefault}{ppl}
\renewcommand{\sfdefault}{phv}
\renewcommand{\ttdefault}{lmtt}

% disponi il titolo
\makeatletter
\renewcommand{\maketitle} {
	\begin{center} 
		\begin{minipage}[t]{.8\textwidth}
			\textsf{\huge\bfseries \@title} 
		\end{minipage}%
		\begin{minipage}[t]{.2\textwidth}
			\raggedleft \vspace{-1.65em}
			\textsf{\small \@author} \vfill
			\textsf{\small \@date}
		\end{minipage}
		\par
	\end{center}

	\thispagestyle{empty}
	\pagestyle{fancy}
}
\makeatother

% disponi teoremi
\usepackage{tcolorbox}
\newtcolorbox[auto counter, number within=section]{theorem}[2][]{%
	colback=blue!10, 
	colframe=blue!40!black, 
	sharp corners=northwest,
	fonttitle=\sffamily\bfseries, 
	title=Teorema~\thetcbcounter: #2, 
	#1
}

% disponi definizioni
\newtcolorbox[auto counter, number within=section]{definition}[2][]{%
	colback=red!10,
	colframe=red!40!black,
	sharp corners=northwest,
	fonttitle=\sffamily\bfseries,
	title=Definizione~\thetcbcounter: #2,
	#1
}

% disponi problemi
\newtcolorbox[auto counter, number within=section]{problem}[2][]{%
	colback=green!10,
	colframe=green!40!black,
	sharp corners=northwest,
	fonttitle=\sffamily\bfseries,
	title=Problema~\thetcbcounter: #2,
	#1
}

% disponi codice
\usepackage{listings}
\usepackage[table]{xcolor}

\lstdefinestyle{codestyle}{
		backgroundcolor=\color{black!5}, 
		commentstyle=\color{codegreen},
		keywordstyle=\bfseries\color{magenta},
		numberstyle=\sffamily\tiny\color{black!60},
		stringstyle=\color{green!50!black},
		basicstyle=\ttfamily\footnotesize,
		breakatwhitespace=false,         
		breaklines=true,                 
		captionpos=b,                    
		keepspaces=true,                 
		numbers=left,                    
		numbersep=5pt,                  
		showspaces=false,                
		showstringspaces=false,
		showtabs=false,                  
		tabsize=2
}

\lstdefinestyle{shellstyle}{
		backgroundcolor=\color{black!5}, 
		basicstyle=\ttfamily\footnotesize\color{black}, 
		commentstyle=\color{black}, 
		keywordstyle=\color{black},
		numberstyle=\color{black!5},
		stringstyle=\color{black}, 
		showspaces=false,
		showstringspaces=false, 
		showtabs=false, 
		tabsize=2, 
		numbers=none, 
		breaklines=true
}

\lstdefinelanguage{javascript}{
	keywords={typeof, new, true, false, catch, function, return, null, catch, switch, var, if, in, while, do, else, case, break},
	keywordstyle=\color{blue}\bfseries,
	ndkeywords={class, export, boolean, throw, implements, import, this},
	ndkeywordstyle=\color{darkgray}\bfseries,
	identifierstyle=\color{black},
	sensitive=false,
	comment=[l]{//},
	morecomment=[s]{/*}{*/},
	commentstyle=\color{purple}\ttfamily,
	stringstyle=\color{red}\ttfamily,
	morestring=[b]',
	morestring=[b]"
}

% disponi sezioni
\usepackage{titlesec}

\titleformat{\section}
	{\sffamily\Large\bfseries} 
	{\thesection}{1em}{} 
\titleformat{\subsection}
	{\sffamily\large\bfseries}   
	{\thesubsection}{1em}{} 
\titleformat{\subsubsection}
	{\sffamily\normalsize\bfseries} 
	{\thesubsubsection}{1em}{}

% disponi alberi
\usepackage{forest}

\forestset{
	rectstyle/.style={
		for tree={rectangle,draw,font=\large\sffamily}
	},
	roundstyle/.style={
		for tree={circle,draw,font=\large}
	}
}

% disponi algoritmi
\usepackage{algorithm}
\usepackage{algorithmic}
\makeatletter
\renewcommand{\ALG@name}{Algoritmo}
\makeatother

% disponi numeri di pagina
\usepackage{fancyhdr}
\fancyhf{} 
\fancyfoot[L]{\sffamily{\thepage}}

\makeatletter
\fancyhead[L]{\raisebox{1ex}[0pt][0pt]{\sffamily{\@title \ \@date}}} 
\fancyhead[R]{\raisebox{1ex}[0pt][0pt]{\sffamily{\@author}}}
\makeatother

\begin{document}

% sezione (data)
\section{Lezione del 21-10-24}

% stili pagina
\thispagestyle{empty}
\pagestyle{fancy}

% testo
\subsection{Applicazione dell'algoritmo di riduzione del gap}
Applichiamo il metodo dei piani di taglio di Gomory ad alcuni problemi di ILP.

\subsubsection{Zaino intero}
Prendiamo il problema:
\[
	\begin{cases}
		\max 30 x_1 + 36 x_2 + 27 x_3 + 20 x_4 + 24 x_5 + 22 x_6 \\ 
		13 x_1 + 16 x_2 + 14 x_3 + 15 x_4 +17 x_5 + 14 x_6 \leq 57 \\ 
		x \in \mathbb{Z}_+
	\end{cases}
\]
con il vettore dei rendimenti:
$$
r = \left( 2.3 , 2.25 , 1.93 , 1.33 , 1.41 , 1.57 \right)
$$

Saturiamo per ottenere una soluzione $x_{RC}$, e quindi una valutazione superiore:
$$
x_{RC} = \left( \frac{57}{13}, 0, 0, 0 ,0 , 0 \right), \quad v_S = 131  
$$
e saturiamo l'intero per avere una valutazione inferiore:
$$
x_{I} = \left( 4, 0, 0, 0, 0, 0 \right), \quad v_I = 120
$$

Adesso applichiamo Gomory:
\begin{enumerate}
	\item Si porta il rilassato continuo in formato duale standard:
	\[
		\begin{cases}		
		\max 30 x_1 + 36 x_2 + 27 x_3 + 20 x_4 + 24 x_5 + 22 x_6 \\ 
		13 x_1 + 16 x_2 + 14 x_3 + 15 x_4 +17 x_5 + 14 x_6 + x_7 = 57 \\ 
		x_i \geq 0
		\end{cases}
	\]
\item Si individua la base ottima:
	$$ x_{RC} = \left( \frac{57}{13}, 0, 0, 0, 0, 0 \right) \Rightarrow B= \{1\}$$
\item Si ricavano $A_B$, $A_N$, $x_B$, $x_N$ e $r$:
	$$
	A_B = 
	\begin{pmatrix}
		13
	\end{pmatrix}, \quad 
	A_N = 
	\begin{pmatrix}
		16 & 14 & 15 & 17 & 14 & 1
	\end{pmatrix}, \quad 
	x_B =
	\left(
		\frac{57}{13}
	\right), \quad 
	x_N = 
	\left(
		0 , 0 , 0 , 0 , 0
	\right)
	$$
	e chiaramente $r=1$;
\item Si ricava $\tilde{A}$:
	$$
	\tilde{A} = A_B^{-1} A_N = \left( \frac{16}{13}, \frac{14}{13}, \frac{15}{13}, \frac{17}{13}, \frac{14}{13}, \frac{1}{13} \right)
	$$
\item Si trova il nuovo vincolo:
$$
\left\{ \frac{16}{13} \right\} x_2 + \left\{ \frac{14}{13} \right\} x_3 + \left\{ \frac{15}{13} \right\} x_4 + \left\{ \frac{17}{13} \right\} x_5 + \left\{ \frac{14}{13} \right\} x_6 + \left\{ \frac{1}{13} \right\} x_7 \geq \left\{ \frac{57}{13} \right\}
$$
$$
\frac{3}{13} x_2 + \frac{1}{13} x_3 + \frac{2}{13} x_4 + \frac{4}{13} x_5 + \frac{1}{13} x_6 + \frac{1}{13} x_7 \geq \frac{5}{13}
$$
Che espresso sostituendo $x_7 = 57 - 13 x_1 - 16 x_2 -14 x_3 - 15 x_4 - 17 x_5 - 14 x_6$ e unito a quanto già trovato dà il problema:
$$
	\begin{cases}
		\max 30 x_1 + 36 x_2 + 27 x_3 + 20 x_4 + 24 x_5 + 22 x_6 \\ 
		13 x_1 + 16 x_2 + 14 x_3 + 15 x_4 +17 x_5 + 14 x_6 \leq 57 \\ 
		x_1 + x_2 + x_3 + x_4 + x_5 + x_ 6 \leq 4 \\
		x \in \mathbb{Z}_+
	\end{cases}
$$
da cui $x_{RC} = \left( \frac{7}{3}, \frac{5}{3}, 0, 0, 0, 0 \right)$, che è già più vicino all'ottimo $\bar{x} = \left( 3, 1, 0, 0, 0, 0 \right)$.
\end{enumerate}

\subsubsection{Zaino booleano}
Prendiamo a questo punto:
\[
	\begin{cases}
		\max 30 x_1 + 36 x_2 + 27 x_3 + 20 x_4 + 24 x_5 + 22 x_6 \\ 
		13 x_1 + 16 x_2 + 14 x_3 + 15 x_4 +17 x_5 + 15 x_6 \leq 57 \\ 
		x \in \mathbb{Z}_+ \\
		0 \leq x \leq 1
	\end{cases}
\]

Attraverso la saturazione, troviamo:
$$
x_{RC} = \left( 1, 1, 1, 0, 0, \frac{14}{15}, 0 \right), \quad v_S = 115.4
$$
di valutazione superiore, e 
$$x_I = (1, 1, 1, 0, 0, 0), \quad v_I = 93 $$
di valutazione inferiore.

Riapplichiamo Gomory.
\begin{enumerate}
	\item L'ottimo è noto:
$$
x_{RC} = \left( 1, 1, 1, 0, 0, \frac{14}{15}, 0 \right)
$$
	\item Convertiamo in formato duale standard, ricordando i vincoli $0 \leq x_i \leq 1$:
\[
	\begin{cases}	
		\max 30 x_1 + 36 x_2 + 27 x_3 + 20 x_4 + 24 x_5 + 22 x_6 \\ 
		13 x_1 + 16 x_2 + 14 x_3 + 15 x_4 +17 x_5 + 15 x_6 + x_7 = 57 \\ 
		x_1 + x_8 = 1 \\
		x_2 + x_9 = 1 \\
		
		x_3 + x_{10} = 1 \\
		x_4 + x_{11} = 1 \\
		x_5 + x_{12} = 1 \\
		x_6 + x_{13} = 1 \\
		x \geq 0
	\end{cases}
\]
Dobbiamo ricalcolare l'ottimo:
$$ 
x_{RC}' = \left( 1, 1, 1, 0, 0, \frac{14}{15}, 0, 0, 0, 0, 1, 1, \frac{1}{15} \right)
$$
\item Si individua la base ottima:
$$
x_{RC}' = \left( 1, 1, 1, 0, 0, \frac{14}{15}, 0, 0, 0, 0, 1, 1, \frac{1}{15} \right) \Rightarrow B = \{1, 2, 3, 6, 11, 12, 13\}
$$
\item Si ricavano $A_B$, $A_N$, $x_B$, $x_N$ e $r$.
	Conviene prima scrivere $A$:
	\setcounter{MaxMatrixCols}{200}
	$$
	A = 
	\begin{pmatrix}
		13 & 16 & 14 & 15 & 17 & 15 & 1 & 0 & 0 & 0 & 0 & 0 & 0 \\
		1 & 0 & 0 & 0 & 0 & 0 & 0 & 1 & 0 & 0 & 0 & 0 & 0 \\
		0 & 1 & 0 & 0 & 0 & 0 & 0 & 0 & 1 & 0 & 0 & 0 & 0 \\
		0 & 0 & 1 & 0 & 0 & 0 & 0 & 0 & 0 & 1 & 0 & 0 & 0 \\
		0 & 0 & 0 & 1 & 0 & 0 & 0 & 0 & 0 & 0 & 1 & 0 & 0 \\
		0 & 0 & 0 & 0 & 1 & 0 & 0 & 0 & 0 & 0 & 0 & 1 & 0 \\
		0 & 0 & 0 & 0 & 0 & 1 & 0 & 0 & 0 & 0 & 0 & 0 & 1 \\
	\end{pmatrix}
	$$
	e quindi:
	$$
	A_B = 
	\begin{pmatrix}
		13 & 16 & 14 & 15 & 0 & 0 & 0 \\
		1 & 0 & 0 & 0 & 0 & 0 & 0 \\
		0 & 1 & 0 & 0 & 0 & 0 & 0 \\
		0 & 0 & 1 & 0 & 0 & 0 & 0 \\
		0 & 0 & 0 & 0 & 1 & 0 & 0 \\
		0 & 0 & 0 & 0 & 0 & 1 & 0 \\
		0 & 0 & 0 & 1 & 0 & 0 & 1 \\
	\end{pmatrix}, \quad 
	A_N = 
	\begin{pmatrix}
		15 & 17 & 1 & 0 & 0 & 0 \\ 
		0 & 0 & 0 & 1 & 0 & 0 \\ 
		0 & 0 & 0 & 0 & 1 & 0 \\ 
		0 & 0 & 0 & 0 & 0 & 1 \\ 
		1 & 0 & 0 & 0 & 0 & 0 \\ 
		0 & 1 & 0 & 0 & 0 & 0 \\
		0 & 0 & 0 & 0 & 0 & 0
	\end{pmatrix}
	$$
	$$
	x_B =
	\left(
		1, 1, 1, \frac{14}{15}, 1, 1, \frac{1}{15}
	\right), \quad 
	x_N = 
	\left(
		0, 0, 0, 0, 0, 0
	\right)
	$$
	e $r= 4, 7$;
\item Si ricava $\tilde{A}$:
	$$
	\tilde{A} = A_B^{-1} A_N = 
	\begin{pmatrix}
		0 & 0 & 0 & 1 & 0 & 0 \\ 
		0 & 0 & 0 & 0 & 1 & 0 \\ 
		0 & 0 & 0 & 0 & 0 & 1 \\ 
		1 & \frac{17}{15} & \frac{1}{15} & -\frac{13}{15} & -\frac{16}{15} & -\frac{14}{15} \\ 
		1 & 0 & 0 & 0 & 0 & 0 \\ 
		0 & 1 & 0 & 0 & 0 & 0 \\ 
		-1 & -\frac{17}{15} & -\frac{1}{15} & \frac{13}{15} & \frac{16}{15} & \frac{14}{15}
	\end{pmatrix} 
	$$
\item Si trovano due nuovi vincoli, notando che $N = \{ 4, 5, 7, 8, 9, 10 \}$:
$$
r=4 \Rightarrow x_1 + 2x_2 + x_3 + x_4 + x_5 + x_6 \leq 4 
$$
$$
r=7 \Rightarrow 13 x_1 + 15 x_2 + 14 x_3 + 14 x_4 + 15 x_5 + 14 x_6 \leq 55
$$
\end{enumerate}


\end{document}
