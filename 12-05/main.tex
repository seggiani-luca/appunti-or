
\documentclass[a4paper,11pt]{article}
\usepackage[a4paper, margin=8em]{geometry}

% usa i pacchetti per la scrittura in italiano
\usepackage[french,italian]{babel}
\usepackage[T1]{fontenc}
\usepackage[utf8]{inputenc}
\frenchspacing 

% usa i pacchetti per la formattazione matematica
\usepackage{amsmath, amssymb, amsthm, amsfonts}

% usa altri pacchetti
\usepackage{gensymb}
\usepackage{hyperref}
\usepackage{standalone}

% imposta il titolo
\title{Appunti Ricerca Operativa}
\author{Luca Seggiani}
\date{2024}

% disegni
\usepackage{pgfplots}
\pgfplotsset{width=10cm,compat=1.9}

% imposta lo stile
% usa helvetica
\usepackage[scaled]{helvet}
% usa palatino
\usepackage{palatino}
% usa un font monospazio guardabile
\usepackage{lmodern}

\renewcommand{\rmdefault}{ppl}
\renewcommand{\sfdefault}{phv}
\renewcommand{\ttdefault}{lmtt}

% disponi il titolo
\makeatletter
\renewcommand{\maketitle} {
	\begin{center} 
		\begin{minipage}[t]{.8\textwidth}
			\textsf{\huge\bfseries \@title} 
		\end{minipage}%
		\begin{minipage}[t]{.2\textwidth}
			\raggedleft \vspace{-1.65em}
			\textsf{\small \@author} \vfill
			\textsf{\small \@date}
		\end{minipage}
		\par
	\end{center}

	\thispagestyle{empty}
	\pagestyle{fancy}
}
\makeatother

% disponi teoremi
\usepackage{tcolorbox}
\newtcolorbox[auto counter, number within=section]{theorem}[2][]{%
	colback=blue!10, 
	colframe=blue!40!black, 
	sharp corners=northwest,
	fonttitle=\sffamily\bfseries, 
	title=Teorema~\thetcbcounter: #2, 
	#1
}

% disponi definizioni
\newtcolorbox[auto counter, number within=section]{definition}[2][]{%
	colback=red!10,
	colframe=red!40!black,
	sharp corners=northwest,
	fonttitle=\sffamily\bfseries,
	title=Definizione~\thetcbcounter: #2,
	#1
}

% disponi problemi
\newtcolorbox[auto counter, number within=section]{problem}[2][]{%
	colback=green!10,
	colframe=green!40!black,
	sharp corners=northwest,
	fonttitle=\sffamily\bfseries,
	title=Problema~\thetcbcounter: #2,
	#1
}

% disponi codice
\usepackage{listings}
\usepackage[table]{xcolor}

\lstdefinestyle{codestyle}{
		backgroundcolor=\color{black!5}, 
		commentstyle=\color{codegreen},
		keywordstyle=\bfseries\color{magenta},
		numberstyle=\sffamily\tiny\color{black!60},
		stringstyle=\color{green!50!black},
		basicstyle=\ttfamily\footnotesize,
		breakatwhitespace=false,         
		breaklines=true,                 
		captionpos=b,                    
		keepspaces=true,                 
		numbers=left,                    
		numbersep=5pt,                  
		showspaces=false,                
		showstringspaces=false,
		showtabs=false,                  
		tabsize=2
}

\lstdefinestyle{shellstyle}{
		backgroundcolor=\color{black!5}, 
		basicstyle=\ttfamily\footnotesize\color{black}, 
		commentstyle=\color{black}, 
		keywordstyle=\color{black},
		numberstyle=\color{black!5},
		stringstyle=\color{black}, 
		showspaces=false,
		showstringspaces=false, 
		showtabs=false, 
		tabsize=2, 
		numbers=none, 
		breaklines=true
}

\lstdefinelanguage{javascript}{
	keywords={typeof, new, true, false, catch, function, return, null, catch, switch, var, if, in, while, do, else, case, break},
	keywordstyle=\color{blue}\bfseries,
	ndkeywords={class, export, boolean, throw, implements, import, this},
	ndkeywordstyle=\color{darkgray}\bfseries,
	identifierstyle=\color{black},
	sensitive=false,
	comment=[l]{//},
	morecomment=[s]{/*}{*/},
	commentstyle=\color{purple}\ttfamily,
	stringstyle=\color{red}\ttfamily,
	morestring=[b]',
	morestring=[b]"
}

% disponi sezioni
\usepackage{titlesec}

\titleformat{\section}
	{\sffamily\Large\bfseries} 
	{\thesection}{1em}{} 
\titleformat{\subsection}
	{\sffamily\large\bfseries}   
	{\thesubsection}{1em}{} 
\titleformat{\subsubsection}
	{\sffamily\normalsize\bfseries} 
	{\thesubsubsection}{1em}{}

% disponi alberi
\usepackage{forest}

\forestset{
	rectstyle/.style={
		for tree={rectangle,draw,font=\large\sffamily}
	},
	roundstyle/.style={
		for tree={circle,draw,font=\large}
	}
}

% disponi algoritmi
\usepackage{algorithm}
\usepackage{algorithmic}
\makeatletter
\renewcommand{\ALG@name}{Algoritmo}
\makeatother

% disponi numeri di pagina
\usepackage{fancyhdr}
\fancyhf{} 
\fancyfoot[L]{\sffamily{\thepage}}

\makeatletter
\fancyhead[L]{\raisebox{1ex}[0pt][0pt]{\sffamily{\@title \ \@date}}} 
\fancyhead[R]{\raisebox{1ex}[0pt][0pt]{\sffamily{\@author}}}
\makeatother

\begin{document}

% sezione (data)
\section{Lezione del 05-12-24}

% stili pagina
\thispagestyle{empty}
\pagestyle{fancy}

% testo

\subsection{Regole di Armijo-Goldstein-Wolfe}
Abbiamo visto alcuni metodi per ricavare passi $t_k$ e direzioni di crescita $d_k$ per algoritmi iterativi di PNL.
Vediamo adesso un altro metodo, dato dalle condizioni di \textbf{Armijo-Goldstein-Wolfe}:
$$
d_k = \nabla f(x_k)
$$
$$
t_k =
\begin{cases}
	\phi(t_k) \leq \phi(0) + \alpha \, \phi'(0) \, t_k \\ 
	\phi'(t_k) \geq \beta \phi'(0)
\end{cases}
$$
dove $0 < \alpha < \beta < 1$ sono parametri dell'algoritmo.

Sostanzialmente la prima condizione va a verificare che $t_k$ stia fra uno dei punti (non diventi \textit{troppo grande}) dove la funzione è minore della retta passante per $\phi(0)$ con coefficiente angolare $\alpha \, \phi'(0)$, cioè che ad $\alpha = 0$ ha coefficiente nullo (parallela all'asse delle $x$) e ad $\alpha = 1$ ha coefficiente $\phi'(0)$ (cioè tangente alla $\phi$, non prenderà nessun punto).

La seconda condizione verifica invece che $t_k$ sia maggiore del punto $t^*$ (non diventi \textit{troppo piccola}), dove $t^*$ è il punto dove la derivata $\phi'(t^*)$ è uguale a $\beta \, \phi'(0)$, che per $\beta = 0$ è nuovamente la retta parallela all'asse $x$, mentre per $\beta = 1$ è tangente a $\phi'(0)$. 

\begin{center}
\begin{tikzpicture}
	\begin{axis}[
			axis lines = center,
			xlabel = $x$, ylabel = $y$,
			width=14cm,
			height=8cm,
			xmin = -1.9,
			xmax = 2.9,
			ymin = -0.9,
			ymax = 2.4,
			grid = minor
	]
	\addplot[
		samples=100
		]{x^2 - x + 1};

	\addplot[
		color=purple,
		thick
		]{1-0.1*x};

	\addplot[
		color=yellow,
		thick
		]{0.8775-0.3*x};

	\addplot[
		only marks,
		mark=*,
		mark size = 2pt,
		] coordinates{(0.35,0.7725)};

	\node[anchor=north, xshift=-6px] at (axis cs:0.35,0.7725) {$x^*$};

	\addplot[
		thick,
		color=blue,
		domain=0.35:0.9
		]{0};

	\addplot[
		color=yellow
	](0.35, x);

	\addplot[
		color=red
	](0.9, x);
	\end{axis}
\end{tikzpicture}

\end{center}

Nel grafico, il vincolo in rosso è il primo (relativo ad $\alpha$), mentre il vincolo in giallo è il secondo (relativo a $\beta$).

\subsection{Metodo di Frank-Wolfe}
Restringiamoci a funzioni $f(x)$ su domini $\Omega$ regolari limitati, di cui siamo interessati a trovare il massimo $\max_{x \in \Omega} f(x)$.
L'idea è quella di definire il \textbf{problema linearizzato} (detto \textit{PL}) in $x_k$, cioè $\mathrm{PL}(x_k)$.
Prendiamo allora l'approssimazione di primo ordine dato dall'espansione di Taylor al secondo termine:
$$
f(x) = f(x_k) + \nabla f(x_k) (x - x_k) + o(x-x_k)
$$
Scartando i termini costanti, vogliamo trovare:
$$
y_k = \max_{x \in \Omega} \nabla f(x_k) x
$$
Sarà allora:
$$
d_k = y_k - x_k
$$
$$
t_k \in \mathrm{argmax}_{0 \leq t \leq 1} \, f(x_k + t(y_k - x_k))
$$

Si può dimostrare che il metodo fi Frank-Wolfe, se la funzione ne ammette, trova un punto stazionario dopo un numero finito di iterazioni per qualsiasi punto iniziale $x_0$. 
\end{document}
