
\documentclass[a4paper,11pt]{article}
\usepackage[a4paper, margin=8em]{geometry}

% usa i pacchetti per la scrittura in italiano
\usepackage[french,italian]{babel}
\usepackage[T1]{fontenc}
\usepackage[utf8]{inputenc}
\frenchspacing 

% usa i pacchetti per la formattazione matematica
\usepackage{amsmath, amssymb, amsthm, amsfonts}

% usa altri pacchetti
\usepackage{gensymb}
\usepackage{hyperref}
\usepackage{standalone}

% imposta il titolo
\title{Appunti Ricerca Operativa}
\author{Luca Seggiani}
\date{2024}

% disegni
\usepackage{pgfplots}
\pgfplotsset{width=10cm,compat=1.9}

% imposta lo stile
% usa helvetica
\usepackage[scaled]{helvet}
% usa palatino
\usepackage{palatino}
% usa un font monospazio guardabile
\usepackage{lmodern}

\renewcommand{\rmdefault}{ppl}
\renewcommand{\sfdefault}{phv}
\renewcommand{\ttdefault}{lmtt}

% disponi il titolo
\makeatletter
\renewcommand{\maketitle} {
	\begin{center} 
		\begin{minipage}[t]{.8\textwidth}
			\textsf{\huge\bfseries \@title} 
		\end{minipage}%
		\begin{minipage}[t]{.2\textwidth}
			\raggedleft \vspace{-1.65em}
			\textsf{\small \@author} \vfill
			\textsf{\small \@date}
		\end{minipage}
		\par
	\end{center}

	\thispagestyle{empty}
	\pagestyle{fancy}
}
\makeatother

% disponi teoremi
\usepackage{tcolorbox}
\newtcolorbox[auto counter, number within=section]{theorem}[2][]{%
	colback=blue!10, 
	colframe=blue!40!black, 
	sharp corners=northwest,
	fonttitle=\sffamily\bfseries, 
	title=Teorema~\thetcbcounter: #2, 
	#1
}

% disponi definizioni
\newtcolorbox[auto counter, number within=section]{definition}[2][]{%
	colback=red!10,
	colframe=red!40!black,
	sharp corners=northwest,
	fonttitle=\sffamily\bfseries,
	title=Definizione~\thetcbcounter: #2,
	#1
}

% disponi problemi
\newtcolorbox[auto counter, number within=section]{problem}[2][]{%
	colback=green!10,
	colframe=green!40!black,
	sharp corners=northwest,
	fonttitle=\sffamily\bfseries,
	title=Problema~\thetcbcounter: #2,
	#1
}

% disponi codice
\usepackage{listings}
\usepackage[table]{xcolor}

\lstdefinestyle{codestyle}{
		backgroundcolor=\color{black!5}, 
		commentstyle=\color{codegreen},
		keywordstyle=\bfseries\color{magenta},
		numberstyle=\sffamily\tiny\color{black!60},
		stringstyle=\color{green!50!black},
		basicstyle=\ttfamily\footnotesize,
		breakatwhitespace=false,         
		breaklines=true,                 
		captionpos=b,                    
		keepspaces=true,                 
		numbers=left,                    
		numbersep=5pt,                  
		showspaces=false,                
		showstringspaces=false,
		showtabs=false,                  
		tabsize=2
}

\lstdefinestyle{shellstyle}{
		backgroundcolor=\color{black!5}, 
		basicstyle=\ttfamily\footnotesize\color{black}, 
		commentstyle=\color{black}, 
		keywordstyle=\color{black},
		numberstyle=\color{black!5},
		stringstyle=\color{black}, 
		showspaces=false,
		showstringspaces=false, 
		showtabs=false, 
		tabsize=2, 
		numbers=none, 
		breaklines=true
}

\lstdefinelanguage{javascript}{
	keywords={typeof, new, true, false, catch, function, return, null, catch, switch, var, if, in, while, do, else, case, break},
	keywordstyle=\color{blue}\bfseries,
	ndkeywords={class, export, boolean, throw, implements, import, this},
	ndkeywordstyle=\color{darkgray}\bfseries,
	identifierstyle=\color{black},
	sensitive=false,
	comment=[l]{//},
	morecomment=[s]{/*}{*/},
	commentstyle=\color{purple}\ttfamily,
	stringstyle=\color{red}\ttfamily,
	morestring=[b]',
	morestring=[b]"
}

% disponi sezioni
\usepackage{titlesec}

\titleformat{\section}
	{\sffamily\Large\bfseries} 
	{\thesection}{1em}{} 
\titleformat{\subsection}
	{\sffamily\large\bfseries}   
	{\thesubsection}{1em}{} 
\titleformat{\subsubsection}
	{\sffamily\normalsize\bfseries} 
	{\thesubsubsection}{1em}{}

% disponi alberi
\usepackage{forest}

\forestset{
	rectstyle/.style={
		for tree={rectangle,draw,font=\large\sffamily}
	},
	roundstyle/.style={
		for tree={circle,draw,font=\large}
	}
}

% disponi algoritmi
\usepackage{algorithm}
\usepackage{algorithmic}
\makeatletter
\renewcommand{\ALG@name}{Algoritmo}
\makeatother

% disponi numeri di pagina
\usepackage{fancyhdr}
\fancyhf{} 
\fancyfoot[L]{\sffamily{\thepage}}

\makeatletter
\fancyhead[L]{\raisebox{1ex}[0pt][0pt]{\sffamily{\@title \ \@date}}} 
\fancyhead[R]{\raisebox{1ex}[0pt][0pt]{\sffamily{\@author}}}
\makeatother

\begin{document}

% sezione (data)
\section{Lezione del 03-10-24}

% stili pagina
\thispagestyle{empty}
\pagestyle{fancy}

% testo
\subsection{Teoria della dualità}
Introduciamo adesso uno dei concetti più importanti della programmazione lineare.
Avevamo posto problemi LP in forma primale standard come:

\[
	\begin{cases}
		\min(c^\intercal \cdot x) \\
		Ax \leq b
	\end{cases}
\]

Ottimizzare questo problema significa partire dal basso e avvicinarsi verso un punto di massimo.
Potremmo scegliere di seguire il percorso opposto: cercare di estrapolare un limite superiore per la soluzione dai vincoli, e minimizzarlo.

Per fare ciò introduciamo $m$ variabili, una per ogni disequazione, che denoteremo come $y_1, ..., y_m$.
Moltiplichiamo ogni disequazione per la $y_i$ corrispondente a destra e a sinistra.
Su un semplice problema $n, m = 2$, questo darà una forma del tipo:
\[
	\begin{cases}
		a_{11} x_1 + a_{12} x_2 \leq b_1 \\
		a_{21} x_1 + a_{22} x_2 \leq b_2
	\end{cases}
	\rightarrow
	\begin{cases}
	y_1 \cdot \left(	a_{11} x_1 + a_{12} x_2 \right) \leq b_1 y_1 \\
	y_2 \cdot \left( a_{21} x_1 + a_{22} x_2 \right) \leq b_2 y_2
	\end{cases}
\]

Se vincoliamo gli $y_i$ in modo che ogni variabile decisionale $x_i$ del sistema abbia un coefficiente del costo $\geq c_i$ corrispondente, otterremo una disequazione che ha a sinistra una situazione di valore uguale o addirittura migliore di quella data dalla funzione costo, e a destra un massimo (che era ciò che stavamo cercando, un limite superiore).
Abbiamo quindi una serie di variabili vincolate:
\[
	\begin{cases}			
		y_1 a_11 + y_2 a_21 \geq c_1 \\ 
		y_1 a_12 + y_2 a_22 \geq c_2 
	\end{cases}
\]
e una funzione da minimizzare:
$$
\min(b_1 y_1 + b_2 y)
$$

Cioè, ci siamo ricondotti ad un altro problema LP.
Possiamo formalizzare questo risultato:
\begin{definition}{Duale di un problema LP}
	Per un qualsiasi problema LP $\mathcal{P}$, detto primale, con $m \geq n$, possiamo definire il duale $\mathcal{D}$:
	\[
		P:
		\begin{cases}
			\max(c^\intercal \cdot x) \\ 
			Ax \leq b
		\end{cases}
	\rightarrow
		D:
		\begin{cases}
			\min(b^\intercal \cdot y) \\ 
			A^\intercal y = c \\
			y \geq 0
		\end{cases}
	\]
	dove si nota $x \in \mathbb{R}^n$ e $y \in \mathbb{R}^m$.
\end{definition}

Il duale viene posto in forma duale standard in quanto ciò che ci interessa è \textit{stringere} il limite superiore fino al suo minimo, in un modo che fa combaciare perfettamente le variabili con il loro vettore costo, da cui le uguaglianze.

Si può dimostrare che l'operazione del calcolo del duale è involutoria: il duale del duale è nuovamente il primale, e così via.

\subsubsection{Dualità debole}
Visto che abbiamo costruito il duale per avere un limite superiore dei valori ottenuti dalla funzione obiettivo del primale, potremo dimostrare facilmente:
\begin{theorem}{Dualità debole}
	Se i poliedri $P$ e il suo duale $D$ non sono vuoti, allora:
	$$ c^\intercal x \leq y^\intercal b \quad \forall x \in P, \forall y \in D$$
\end{theorem}
Cioè il duale è sempre maggiore del primale.

\subsubsection{Dualità forte}
Idealmente, ciò che vorremmo è che primale e duale convergessero verso un punto comune, ergo l'ottimo di entrambi.
Effettivamente, questo risultato è verificato:
\begin{theorem}{Dualità forte}
	Se i poliedri $P$ e il suo duale $D$ non sono vuoti, allora:
	$$ -\infty \leq \min_{y \in D} b^\intercal y = \max_{x \in P} c^\intercal x \leq +\infty $$
\end{theorem}

Il teorema della dualità forte afferma che, se entrambi i poliedri (primale e duale) sono non vuoti, allora condividono l'ottimo, e anzi, che due soluzioni nel primale e nel duale sono ottime solo se hanno lo stesso valore.
Se invece solo il primale (solo il duale) è vuoto, si ha che entrambi condividono ottimo $-\infty$ ($\infty$).
Quando entrambi sono vuoti non si ha soluzione condivisa.

\par\medskip

Si noti che la dualità è univocamente determinata solo nel caso di soluzioni non degeneri.
Nel caso di soluzioni degeneri si può avere che a una soluzione del primale ne corrispondono multiple del duale, e viceversa.
In questo caso, è possibile che si raggiunga un massimo (o un minimo) nel primale (nel duale), a partire da più vertici complementari, o addirittura senza che la soluzione di base complementare sia ammissibile.
Più formalmente, se l'esistenza di una soluzione complementare ammissibile è \textbf{condizione necessaria e sufficiente} per problemi con soluzioni non degeneri, è solo \textbf{condizione sufficiente} per problemi con soluzioni degeneri.

Ad esempio, si noti il problema duale:
\[
	\begin{cases}
		\min 3y_1 - 7y_2 + 5y_3 + 22y_4 + 14y_5 + 15y_6 \\ 
-y_1 - y_2 +3 y_4 + 2y_5 + 2y_6 = 2 \\
y_1 - 4y_2 + y_3 + 2y_4 + y_5 - 2 y_6 = 1 \\e
y_i \geq 0	
	\end{cases}
\]
con il primale associato:
\[
	\begin{cases}
			
\max 2x_1 + x_2 \\ 
-x_1 + x_1 \leq 3 \\ 
-x_1 -4x_2 \leq -7 \\ 
x_2 \leq 5 \\ 
3 x_1 + 2 x_2 \leq 22 \\ 
2 x_1 + x_2 \leq 14 \\ 
2 x_1 - 2 x_2 \leq 5
	\end{cases}
\]

Con la base $B = \{ 1, 5 \}$, si hanno le soluzioni $\bar{x} = (4, 5)$ e $\bar{y} = (0, 0, 0,  0, 1, 0)$.
Si dimostra che $\bar{y}$ è ottima a $D$, mentre $\bar{x}$ è non ammissibile su $P$.
Ciò deriva dal fatto che $\bar{y}$ in $D$ è degenere: descrive un vincolo perpendicolare al vettore costo (di indice $i = 5$), e quindi soddisfatto da infinite soluzioni del primale.
Il punto $\bar{x}$ nel primale ha come complementare l'ottimo del duale perchè sta proprio su questo vincolo. 

\subsubsection{Scarti complementari}
Si può dimostrare il seguente teorema:
\begin{theorem}{Scarti complementari}
	Se le soluzioni $x$ e $y$ dei problemi primale e duale $\mathcal{P}$ e $\mathcal{D}$ sono entrambe ottime, allora vale:
	$$ y^\intercal (b - Ax) = 0 $$
\end{theorem}
Questo si ricava dal fatto che, per la dualità forte, si ha che:
$$
c^\intercal x = y^\intercal Ax = y^\intercal b \Rightarrow y^\intercal(b - Ax) = 0
$$

Il significato del teorema è che, se una disequazione nel primale è \textit{stretta}, allora la corrispondente variabile nel duale è $\neq 0$, e viceversa.

\subsubsection{Soluzioni di base}
Avevamo dato una definizione di soluzione di base per problemi LP in forma sia primale che duale.
Possiamo dimostrare che non solo questa nozione esiste su entrambe le formule, ma è analoga su coppie primale / duale.

Avevamo posto che la formazione di una certa base $B \in \{ 1, ..., m \}$ per ricavare soluzioni di base.
Per il primale, questo significa partizionare la matrice e i termini noti:
$$
A = \left( A_B \over A_N \right), \quad b = \left( b_A \over b_N \right)
$$
mentre per il duale, significherà partizionare le variabili introdotte:
$$
y = \left( y_B \over y_N \right)
$$
noto il numero di $y_1, ..., y_m$ uguale a $m$.

Questo significa che possiamo trovare due soluzioni di base corrispondenti per un'unica base su primale e duale.
Queste sono:
\begin{itemize}
	\item Soluzione di base primale: $ x = A_B^{-1} b_B $;
	\item Soluzione di base duale: $ y_B^\intercal = c^\intercal A_B^-1, \quad y_N = 0 $;
\end{itemize}
Si dice che le soluzioni di base sono \textbf{complementari}.

\par\smallskip
\noindent 
\textbf{\textsf{Dimostrazione}}
Vogliamo che $y^\intercal(b - Ax)$ sia $=0$ soddisfatte le condizioni di base.
Applichiamo quindi la base:
$$
y^\intercal(b - Ax) = \left( y_B^\intercal, y_N^\intercal \right) \binom{b_B - A_B x}{b_N - A_N x} = \left( c^\intercal A_B^{-1}, 0 \right) \binom{b_B - A_B A_B^{-1} b_B}{b_N - A_N A_B^{-1} b_B}
$$
$$
= \left( c^\intercal A_B^{-1}, 0 \right) \binom{0}{b_N - A_N A_B^{-1} b_B} = 0
$$

Questo nome non è a caso, in quanto si può dimostrare le due soluzioni sono in scarti complementari.
Da questo risultato, si ha che se entrambe le soluzioni sono ammissibili, cioé:
\begin{itemize}
	\item La primale è ammissibile: 
		$$ \forall i \in N \ \text{si ha} \ A_i x \leq b_i $$
		ergo i vincoli sono soddisfatti;
	\item La duale è ammissibile:
		$$ y \geq 0$$
\end{itemize}
questo è condizione sufficiente perche la soluzione sia ottima, e dagli scorsi corollari, sia l'ottima sia del primale che del duale.

Formalizziamo quanto detto in un teorema:
\begin{theorem}{Condizioni di ottimalità di soluzione di base}
	Dato un vertice del primale, ottenuto da una certa base, si può costruire il complemento duale sulla stessa base.
	Se entrambi i vertici ottenuti sono ammissibili, allora sono uguali e ottimi dei rispettivi problemi.
\end{theorem}

\end{document}
