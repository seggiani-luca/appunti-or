
\documentclass[a4paper,11pt]{article}
\usepackage[a4paper, margin=8em]{geometry}

% usa i pacchetti per la scrittura in italiano
\usepackage[french,italian]{babel}
\usepackage[T1]{fontenc}
\usepackage[utf8]{inputenc}
\frenchspacing 

% usa i pacchetti per la formattazione matematica
\usepackage{amsmath, amssymb, amsthm, amsfonts}

% usa altri pacchetti
\usepackage{gensymb}
\usepackage{hyperref}
\usepackage{standalone}

% imposta il titolo
\title{Appunti Ricerca Operativa}
\author{Luca Seggiani}
\date{2024}

% disegni
\usepackage{pgfplots}
\pgfplotsset{width=10cm,compat=1.9}

% imposta lo stile
% usa helvetica
\usepackage[scaled]{helvet}
% usa palatino
\usepackage{palatino}
% usa un font monospazio guardabile
\usepackage{lmodern}

\renewcommand{\rmdefault}{ppl}
\renewcommand{\sfdefault}{phv}
\renewcommand{\ttdefault}{lmtt}

% disponi il titolo
\makeatletter
\renewcommand{\maketitle} {
	\begin{center} 
		\begin{minipage}[t]{.8\textwidth}
			\textsf{\huge\bfseries \@title} 
		\end{minipage}%
		\begin{minipage}[t]{.2\textwidth}
			\raggedleft \vspace{-1.65em}
			\textsf{\small \@author} \vfill
			\textsf{\small \@date}
		\end{minipage}
		\par
	\end{center}

	\thispagestyle{empty}
	\pagestyle{fancy}
}
\makeatother

% disponi teoremi
\usepackage{tcolorbox}
\newtcolorbox[auto counter, number within=section]{theorem}[2][]{%
	colback=blue!10, 
	colframe=blue!40!black, 
	sharp corners=northwest,
	fonttitle=\sffamily\bfseries, 
	title=Teorema~\thetcbcounter: #2, 
	#1
}

% disponi definizioni
\newtcolorbox[auto counter, number within=section]{definition}[2][]{%
	colback=red!10,
	colframe=red!40!black,
	sharp corners=northwest,
	fonttitle=\sffamily\bfseries,
	title=Definizione~\thetcbcounter: #2,
	#1
}

% disponi problemi
\newtcolorbox[auto counter, number within=section]{problem}[2][]{%
	colback=green!10,
	colframe=green!40!black,
	sharp corners=northwest,
	fonttitle=\sffamily\bfseries,
	title=Problema~\thetcbcounter: #2,
	#1
}

% disponi codice
\usepackage{listings}
\usepackage[table]{xcolor}

\lstdefinestyle{codestyle}{
		backgroundcolor=\color{black!5}, 
		commentstyle=\color{codegreen},
		keywordstyle=\bfseries\color{magenta},
		numberstyle=\sffamily\tiny\color{black!60},
		stringstyle=\color{green!50!black},
		basicstyle=\ttfamily\footnotesize,
		breakatwhitespace=false,         
		breaklines=true,                 
		captionpos=b,                    
		keepspaces=true,                 
		numbers=left,                    
		numbersep=5pt,                  
		showspaces=false,                
		showstringspaces=false,
		showtabs=false,                  
		tabsize=2
}

\lstdefinestyle{shellstyle}{
		backgroundcolor=\color{black!5}, 
		basicstyle=\ttfamily\footnotesize\color{black}, 
		commentstyle=\color{black}, 
		keywordstyle=\color{black},
		numberstyle=\color{black!5},
		stringstyle=\color{black}, 
		showspaces=false,
		showstringspaces=false, 
		showtabs=false, 
		tabsize=2, 
		numbers=none, 
		breaklines=true
}

\lstdefinelanguage{javascript}{
	keywords={typeof, new, true, false, catch, function, return, null, catch, switch, var, if, in, while, do, else, case, break},
	keywordstyle=\color{blue}\bfseries,
	ndkeywords={class, export, boolean, throw, implements, import, this},
	ndkeywordstyle=\color{darkgray}\bfseries,
	identifierstyle=\color{black},
	sensitive=false,
	comment=[l]{//},
	morecomment=[s]{/*}{*/},
	commentstyle=\color{purple}\ttfamily,
	stringstyle=\color{red}\ttfamily,
	morestring=[b]',
	morestring=[b]"
}

% disponi sezioni
\usepackage{titlesec}

\titleformat{\section}
	{\sffamily\Large\bfseries} 
	{\thesection}{1em}{} 
\titleformat{\subsection}
	{\sffamily\large\bfseries}   
	{\thesubsection}{1em}{} 
\titleformat{\subsubsection}
	{\sffamily\normalsize\bfseries} 
	{\thesubsubsection}{1em}{}

% disponi alberi
\usepackage{forest}

\forestset{
	rectstyle/.style={
		for tree={rectangle,draw,font=\large\sffamily}
	},
	roundstyle/.style={
		for tree={circle,draw,font=\large}
	}
}

% disponi algoritmi
\usepackage{algorithm}
\usepackage{algorithmic}
\makeatletter
\renewcommand{\ALG@name}{Algoritmo}
\makeatother

% disponi numeri di pagina
\usepackage{fancyhdr}
\fancyhf{} 
\fancyfoot[L]{\sffamily{\thepage}}

\makeatletter
\fancyhead[L]{\raisebox{1ex}[0pt][0pt]{\sffamily{\@title \ \@date}}} 
\fancyhead[R]{\raisebox{1ex}[0pt][0pt]{\sffamily{\@author}}}
\makeatother

\begin{document}

% sezione (data)
\section{Lezione del 20-11-24}

% stili pagina
\thispagestyle{empty}
\pagestyle{fancy}

% testo
\subsection{Algoritmo di Dijkstra}
Torniamo sul problema dei \textbf{cammini di costo minimo}.
Prendiamo il grafo:
\begin{center}
	\begin{tikzpicture}
		\node[circle, draw=black] (1) at (0,0) {1};
		\node[circle, draw=black] (2) at (2,1) {2};
		\node[circle, draw=black] (3) at (2,-1) {3};
		\node[circle, draw=black] (4) at (4,1) {4};
		\node[circle, draw=black] (5) at (4,-1) {5};
		\node[circle, draw=black] (6) at (6, 0) {6};
		\draw[->, to path={-| (\tikztotarget)}] (1) -- (2);
		\draw[->, to path={-| (\tikztotarget)}] (1) -- (3);
		\draw[->, to path={-| (\tikztotarget)}] (2) -- (3);
		\draw[->, to path={-| (\tikztotarget)}] (2) -- (4);
		\draw[->, to path={-| (\tikztotarget)}] (3) -- (5);
		\draw[->, to path={-| (\tikztotarget)}] (5) -- (4);
		\draw[->, to path={-| (\tikztotarget)}] (3) -- (4);
		\draw[->, to path={-| (\tikztotarget)}] (4) -- (6);
		\draw[->, to path={-| (\tikztotarget)}] (5) -- (6);

		\node at (1, 1)  {$9$};
		\node at (1, -1)  {$3$};

		\node at (3, 1.5)  {$7$};
		\node at (3, -1.5)  {$6$};
		\node at (3, 0.5)  {$2$};

		\node at (1.5, 0)  {$4$};
		\node at (4.5, 0)  {$8$};

		\node at (5, 1)  {$1$};
		\node at (5, -1)  {$9$};
	\end{tikzpicture}
\end{center}

Un algoritmo celebre per il problema è quello di \textbf{Dijkstra}.
\begin{algorithm}
\caption{di Dijkstra}
\begin{algorithmic}
	\STATE \textbf{Input:} % input
	\STATE \textbf{Output:} % output
	% body
	\STATE Etichetta ogni nodo come $\pi = \left( 0, +\infty, ..., +\infty \right)$, assunto il primo nodo come partenza
	\STATE Associa un predecessore ad ogni nodo come $P = (1, 0, ..., 0)$
	\STATE Prendi il nodo $n \in N$ di etichetta $\pi$ minima, rimuovilo da $N$, e prendi la stella uscente ($\mathrm{FS}(n)$) di $n$
	\FOR{$\forall j \in \mathrm{FS}(n)$}
		\IF{$\pi_j \geq \pi_i + c_{ij}$}
			\STATE Poni il predecessore $P_j = i$
			\STATE Poni l'etichetta $\pi_j = \pi_i + c_{ij}$
		\ENDIF
	\ENDFOR
\end{algorithmic}
\end{algorithm}

Le etichette $\pi$ rappresenteranno la \textbf{distanza minima} dal nodo di partenza a ogni nodo.
Ad algoritmo compiuto (quando $N$ è vuoto) la catena dei predecessori da ogni nodo a quello di partenza forma i cammini minimi.

Sull'esempio riportato sopra, il primo passaggio è:
$$ i = 1, \quad N = \{ 2, 3, 4, 5, 6 \} $$
Da cui si calcola quindi la stella uscente $FS(1) = \{ 2, 3\}$ 
Prendiamo entrambi gli indici:
\begin{itemize}
	\item $j=2$: $\pi_2 \geq \pi_1 + c_{12} \rightarrow
		\begin{cases}
			\pi_2 = 8 \\ 
			p_2 = 1
		\end{cases}
		$
	\item $j=3$: $\pi_3 \geq \pi_1 + c_{13} \rightarrow
		\begin{cases}
			\pi_3 = 3 \\ 
			p_2 = 1
		\end{cases}
		$
\end{itemize}

Da cui, all'inizio del secondo passaggio, etichette e predecessori sono:
$$
\pi = (0, 8, 3, +\infty, + \infty, +\infty)
$$
$$
p = (1, 1, 1, 0, 0, 0)
$$

L'algoritmo, con complessità linearitmica $O(n \log{n} + m \log{n})$ su $n$ nodi a ramificazione $m$, risulta più efficiente della risoluzione diretta dei problemi dei cammini minimi come problemi di flusso.

\end{document}
