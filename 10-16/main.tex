
\documentclass[a4paper,11pt]{article}
\usepackage[a4paper, margin=8em]{geometry}

% usa i pacchetti per la scrittura in italiano
\usepackage[french,italian]{babel}
\usepackage[T1]{fontenc}
\usepackage[utf8]{inputenc}
\frenchspacing 

% usa i pacchetti per la formattazione matematica
\usepackage{amsmath, amssymb, amsthm, amsfonts}

% usa altri pacchetti
\usepackage{gensymb}
\usepackage{hyperref}
\usepackage{standalone}

% imposta il titolo
\title{Appunti Ricerca Operativa}
\author{Luca Seggiani}
\date{2024}

% disegni
\usepackage{pgfplots}
\pgfplotsset{width=10cm,compat=1.9}

% imposta lo stile
% usa helvetica
\usepackage[scaled]{helvet}
% usa palatino
\usepackage{palatino}
% usa un font monospazio guardabile
\usepackage{lmodern}

\renewcommand{\rmdefault}{ppl}
\renewcommand{\sfdefault}{phv}
\renewcommand{\ttdefault}{lmtt}

% disponi il titolo
\makeatletter
\renewcommand{\maketitle} {
	\begin{center} 
		\begin{minipage}[t]{.8\textwidth}
			\textsf{\huge\bfseries \@title} 
		\end{minipage}%
		\begin{minipage}[t]{.2\textwidth}
			\raggedleft \vspace{-1.65em}
			\textsf{\small \@author} \vfill
			\textsf{\small \@date}
		\end{minipage}
		\par
	\end{center}

	\thispagestyle{empty}
	\pagestyle{fancy}
}
\makeatother

% disponi teoremi
\usepackage{tcolorbox}
\newtcolorbox[auto counter, number within=section]{theorem}[2][]{%
	colback=blue!10, 
	colframe=blue!40!black, 
	sharp corners=northwest,
	fonttitle=\sffamily\bfseries, 
	title=Teorema~\thetcbcounter: #2, 
	#1
}

% disponi definizioni
\newtcolorbox[auto counter, number within=section]{definition}[2][]{%
	colback=red!10,
	colframe=red!40!black,
	sharp corners=northwest,
	fonttitle=\sffamily\bfseries,
	title=Definizione~\thetcbcounter: #2,
	#1
}

% disponi problemi
\newtcolorbox[auto counter, number within=section]{problem}[2][]{%
	colback=green!10,
	colframe=green!40!black,
	sharp corners=northwest,
	fonttitle=\sffamily\bfseries,
	title=Problema~\thetcbcounter: #2,
	#1
}

% disponi codice
\usepackage{listings}
\usepackage[table]{xcolor}

\lstdefinestyle{codestyle}{
		backgroundcolor=\color{black!5}, 
		commentstyle=\color{codegreen},
		keywordstyle=\bfseries\color{magenta},
		numberstyle=\sffamily\tiny\color{black!60},
		stringstyle=\color{green!50!black},
		basicstyle=\ttfamily\footnotesize,
		breakatwhitespace=false,         
		breaklines=true,                 
		captionpos=b,                    
		keepspaces=true,                 
		numbers=left,                    
		numbersep=5pt,                  
		showspaces=false,                
		showstringspaces=false,
		showtabs=false,                  
		tabsize=2
}

\lstdefinestyle{shellstyle}{
		backgroundcolor=\color{black!5}, 
		basicstyle=\ttfamily\footnotesize\color{black}, 
		commentstyle=\color{black}, 
		keywordstyle=\color{black},
		numberstyle=\color{black!5},
		stringstyle=\color{black}, 
		showspaces=false,
		showstringspaces=false, 
		showtabs=false, 
		tabsize=2, 
		numbers=none, 
		breaklines=true
}

\lstdefinelanguage{javascript}{
	keywords={typeof, new, true, false, catch, function, return, null, catch, switch, var, if, in, while, do, else, case, break},
	keywordstyle=\color{blue}\bfseries,
	ndkeywords={class, export, boolean, throw, implements, import, this},
	ndkeywordstyle=\color{darkgray}\bfseries,
	identifierstyle=\color{black},
	sensitive=false,
	comment=[l]{//},
	morecomment=[s]{/*}{*/},
	commentstyle=\color{purple}\ttfamily,
	stringstyle=\color{red}\ttfamily,
	morestring=[b]',
	morestring=[b]"
}

% disponi sezioni
\usepackage{titlesec}

\titleformat{\section}
	{\sffamily\Large\bfseries} 
	{\thesection}{1em}{} 
\titleformat{\subsection}
	{\sffamily\large\bfseries}   
	{\thesubsection}{1em}{} 
\titleformat{\subsubsection}
	{\sffamily\normalsize\bfseries} 
	{\thesubsubsection}{1em}{}

% disponi alberi
\usepackage{forest}

\forestset{
	rectstyle/.style={
		for tree={rectangle,draw,font=\large\sffamily}
	},
	roundstyle/.style={
		for tree={circle,draw,font=\large}
	}
}

% disponi algoritmi
\usepackage{algorithm}
\usepackage{algorithmic}
\makeatletter
\renewcommand{\ALG@name}{Algoritmo}
\makeatother

% disponi numeri di pagina
\usepackage{fancyhdr}
\fancyhf{} 
\fancyfoot[L]{\sffamily{\thepage}}

\makeatletter
\fancyhead[L]{\raisebox{1ex}[0pt][0pt]{\sffamily{\@title \ \@date}}} 
\fancyhead[R]{\raisebox{1ex}[0pt][0pt]{\sffamily{\@author}}}
\makeatother

\begin{document}

% sezione (data)
\section{Lezione del 16-10-24}

% stili pagina
\thispagestyle{empty}
\pagestyle{fancy}

% testo
\subsection{Problema di impacchettamento}
\begin{problem}{Impacchettamento}
	Poniamo di avere 6 file, contenenti registrazioni di tutta la musica di Mozart, con il seguente ingombro in gigabyte:

	\center \rowcolors{2}{green!10}{green!40!black!20}
	\begin{tabular} { | c || c | c | c | c | c | c | }
		\hline 
		& \bfseries Arie & \bfseries Opere & \bfseries Concerti & \bfseries Sinfonie & \bfseries Sonate & \bfseries Messe \\
		\hline
		$p$ & 3 & 6 & 5 & 4 & 4 & 8 \\
		\hline
	\end{tabular}
	
	\par\bigskip

	\raggedright
	Vogliamo trovare il numero minimo di dischi rigidi di dimensione $P = 10$ per archiviare tutti di questi file.

\end{problem}

Chiamiamo questi problemi anche problemi di \textit{bin-packing}.
Rappresentiamo la soluzione come una matrice di adiacenza:
$$
x_{ij} =	
	\begin{cases}
			0 \\ 1
	\end{cases}
$$

dove $i \in \{ 1, ..., p \}$ rappresenta il contenitore e $j \in \{1, ..., 6\}$ l'oggetto che vi inseriamo.
Conviene trovare prima una stima superiore per il numero di contenitori $p$, attraverso un'opportuno algoritmo greedy.

Un'algoritmo banale può essere quello di riempire finché è possibile il primo contenitore, cioè finche si hanno oggetti che entrano nello spazio libero (detto \textit{sfrido}) del contenitore.
Una volta che questa ipotesi è violata, si prende un'altro contenitore, e così via.

Si ricava quindi questa valutazione superiore $V_S$.
A questo punto si può porre:
\[
	\begin{cases}
		x_{11} + x_{21} + x_{31} + x_{41} = 1 \\	
		x_{12} + x_{22} + x_{32} + x_{42} = 1 \\	
		... \\ 
		x_{16} + x_{26} + x_{36} + x_{46} = 1 \\	
	\end{cases}
\]
cioè vogliamo prendere uno e uno solo di tutti gli oggetti, disposti fra i $V_S = 4$ contenitori di valutazione superiore.

Visto che non siamo sicuri di dover prendere tutti i contenitori, dovremo introdurre una variabile $y_i$ per ognuno di essi:
$$
y_i =
	\begin{cases}
			0 \\ 1
	\end{cases}
$$

Infine vogliamo inserire la dimensione di ogni contenitore, $P = 10$, nel problema, sulla base dei pesi $p_i$ di ogni oggetto:
\[
	\begin{cases}
		p_1 x_{11} + p_2 x_{12} + p_3 x_{13} + p_4 x_{14} + p_5 x_{15} + p_6 x_{16} \leq 10 y_1 \\ 
		p_1 x_{21} + p_2 x_{22} + p_3 x_{23} + p_4 x_{24} + p_5 x_{25} + p_6 x_{26} \leq 10 y_2 \\ 
		... \\
		p_1 x_{41} + p_2 x_{42} + p_3 x_{43} + p_4 x_{44} + p_5 x_{45} + p_6 x_{46} \leq 10 y_4 \\ 
	\end{cases}
\]

Combinando quanto posto finora, otteniamo il problema completo:
\[
	\begin{cases}
			\min(y_1 + y_2 + y_3 + y_4) \\

		x_{11} + x_{21} + x_{31} + x_{41} = 1 \\	
		x_{12} + x_{22} + x_{32} + x_{42} = 1 \\	
		... \\ 
		x_{16} + x_{26} + x_{36} + x_{46} = 1 \\	
		
		p_1 x_{11} + p_2 x_{12} + p_3 x_{13} + p_4 x_{14} + p_5 x_{15} + p_6 x_{16} \leq 10 y_1 \\ 
		p_1 x_{21} + p_2 x_{22} + p_3 x_{23} + p_4 x_{24} + p_5 x_{25} + p_6 x_{26} \leq 10 y_2 \\ 
		... \\
		p_1 x_{41} + p_2 x_{42} + p_3 x_{43} + p_4 x_{44} + p_5 x_{45} + p_6 x_{46} \leq 10 y_4 \\ 

		x_i \in \{ 0, 1 \} \\
		y_i \in \{ 0, 1 \} \\
	\end{cases}
\]
dove minimizziamo gli $y_1$ cercando di usare meno contenitori possibile.
Questo è un problema di ILP.
Cerchiamo quindi la valutazione inferiore $V_I$ e la superiore $V_S$.

Possiamo dare una stima inferiore attraverso la formula:
$$
V_I = \left\lceil \frac{\sum_{j=1}^6 p_i}{P} \right\rceil
$$
cioè il peso di tutti gli oggetti fratto le dimensioni dei contenitori, arrotondato per eccesso, che è il minimo numero di contenitori possibile per contenere tutti gli oggetti.

Per il calcolo della stima superiore, invece, avevamo presentato un algoritmo greedy.
In verità sono ci altre (e più intelligenti) strade che possiamo prendere:

\begin{itemize}
	\item \textbf{Next-fit decreasing:} essenzialmente l'algoritmo presentato, dove si ha:
\begin{algorithm}[H]
\caption{next-fit decreasing per impachettamento}
\begin{algorithmic}
	\STATE \textbf{Input:} un problema di impacchettamento % input
	\STATE \textbf{Output:} una soluzione ammissibile % output
	% body
	\WHILE{ci sono ancora oggetti}
		\STATE Prendi il prossimo oggetto
		\IF{entra nel contenitore} 
			\STATE Inseriscilo nel contenitore
		\ELSE 
			\STATE Prendi un'altro contenitore e inseriscici l'oggetto
		\ENDIF
	\ENDWHILE
\end{algorithmic}
\end{algorithm}
	\item \textbf{First-fit decreasing:} analogo al next-fit, ma con la differenza che per ogni oggetto si considerano tutti i contenitori:
\begin{algorithm}[H]
\caption{first-fit decreasing per impachettamento}
\begin{algorithmic}
	\STATE \textbf{Input:} un problema di impacchettamento % input
	\STATE \textbf{Output:} una soluzione ammissibile % output
	% body
	\WHILE{ci sono ancora oggetti}
		\STATE Prendi il prossimo oggetto
		\IF{l'oggetto entra in uno dei contenitori presi finora}
			\STATE Inseriscilo nel contenitore
		\ELSE 
			\STATE Prendi un'altro contenitore e inseriscici l'oggetto
		\ENDIF
	\ENDWHILE
\end{algorithmic}
\end{algorithm}
	\item \textbf{Best-fit decreasing:} è una variante del first-fit che prende sempre i contenitori con sfrido massimo, cioè cerca di riempire i contenitori con meno spazio disponibile (cioè di trovare l'"incastro" migliore per l'oggetto):
\begin{algorithm}[H]
\caption{best-fit decreasing per impachettamento}
\begin{algorithmic}
	\STATE \textbf{Input:} un problema di impacchettamento % input
	\STATE \textbf{Output:} una soluzione ammissibile % output
	% body
	\WHILE{ci sono ancora oggetti}
		\STATE Prendi il prossimo oggetto
		\IF{l'oggetto entra in uno dei contenitori, ordinati per sfrido decrescente, presi finora}
			\STATE Inseriscilo nel contenitore
		\ELSE 
			\STATE Prendi un'altro contenitore e inseriscici l'oggetto
		\ENDIF
	\ENDWHILE
\end{algorithmic}
\end{algorithm}
\end{itemize}

\end{document}
