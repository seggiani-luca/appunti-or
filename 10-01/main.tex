
\documentclass[a4paper,11pt]{article}
\usepackage[a4paper, margin=8em]{geometry}

% usa i pacchetti per la scrittura in italiano
\usepackage[french,italian]{babel}
\usepackage[T1]{fontenc}
\usepackage[utf8]{inputenc}
\frenchspacing 

% usa i pacchetti per la formattazione matematica
\usepackage{amsmath, amssymb, amsthm, amsfonts}

% usa altri pacchetti
\usepackage{gensymb}
\usepackage{hyperref}
\usepackage{standalone}

% imposta il titolo
\title{Appunti Ricerca Operativa}
\author{Luca Seggiani}
\date{2024}

% disegni
\usepackage{pgfplots}
\pgfplotsset{width=10cm,compat=1.9}

% imposta lo stile
% usa helvetica
\usepackage[scaled]{helvet}
% usa palatino
\usepackage{palatino}
% usa un font monospazio guardabile
\usepackage{lmodern}

\renewcommand{\rmdefault}{ppl}
\renewcommand{\sfdefault}{phv}
\renewcommand{\ttdefault}{lmtt}

% disponi il titolo
\makeatletter
\renewcommand{\maketitle} {
	\begin{center} 
		\begin{minipage}[t]{.8\textwidth}
			\textsf{\huge\bfseries \@title} 
		\end{minipage}%
		\begin{minipage}[t]{.2\textwidth}
			\raggedleft \vspace{-1.65em}
			\textsf{\small \@author} \vfill
			\textsf{\small \@date}
		\end{minipage}
		\par
	\end{center}

	\thispagestyle{empty}
	\pagestyle{fancy}
}
\makeatother

% disponi teoremi
\usepackage{tcolorbox}
\newtcolorbox[auto counter, number within=section]{theorem}[2][]{%
	colback=blue!10, 
	colframe=blue!40!black, 
	sharp corners=northwest,
	fonttitle=\sffamily\bfseries, 
	title=Teorema~\thetcbcounter: #2, 
	#1
}

% disponi definizioni
\newtcolorbox[auto counter, number within=section]{definition}[2][]{%
	colback=red!10,
	colframe=red!40!black,
	sharp corners=northwest,
	fonttitle=\sffamily\bfseries,
	title=Definizione~\thetcbcounter: #2,
	#1
}

% disponi problemi
\newtcolorbox[auto counter, number within=section]{problem}[2][]{%
	colback=green!10,
	colframe=green!40!black,
	sharp corners=northwest,
	fonttitle=\sffamily\bfseries,
	title=Problema~\thetcbcounter: #2,
	#1
}

% disponi codice
\usepackage{listings}
\usepackage[table]{xcolor}

\lstdefinestyle{codestyle}{
		backgroundcolor=\color{black!5}, 
		commentstyle=\color{codegreen},
		keywordstyle=\bfseries\color{magenta},
		numberstyle=\sffamily\tiny\color{black!60},
		stringstyle=\color{green!50!black},
		basicstyle=\ttfamily\footnotesize,
		breakatwhitespace=false,         
		breaklines=true,                 
		captionpos=b,                    
		keepspaces=true,                 
		numbers=left,                    
		numbersep=5pt,                  
		showspaces=false,                
		showstringspaces=false,
		showtabs=false,                  
		tabsize=2
}

\lstdefinestyle{shellstyle}{
		backgroundcolor=\color{black!5}, 
		basicstyle=\ttfamily\footnotesize\color{black}, 
		commentstyle=\color{black}, 
		keywordstyle=\color{black},
		numberstyle=\color{black!5},
		stringstyle=\color{black}, 
		showspaces=false,
		showstringspaces=false, 
		showtabs=false, 
		tabsize=2, 
		numbers=none, 
		breaklines=true
}

\lstdefinelanguage{javascript}{
	keywords={typeof, new, true, false, catch, function, return, null, catch, switch, var, if, in, while, do, else, case, break},
	keywordstyle=\color{blue}\bfseries,
	ndkeywords={class, export, boolean, throw, implements, import, this},
	ndkeywordstyle=\color{darkgray}\bfseries,
	identifierstyle=\color{black},
	sensitive=false,
	comment=[l]{//},
	morecomment=[s]{/*}{*/},
	commentstyle=\color{purple}\ttfamily,
	stringstyle=\color{red}\ttfamily,
	morestring=[b]',
	morestring=[b]"
}

% disponi sezioni
\usepackage{titlesec}

\titleformat{\section}
	{\sffamily\Large\bfseries} 
	{\thesection}{1em}{} 
\titleformat{\subsection}
	{\sffamily\large\bfseries}   
	{\thesubsection}{1em}{} 
\titleformat{\subsubsection}
	{\sffamily\normalsize\bfseries} 
	{\thesubsubsection}{1em}{}

% disponi alberi
\usepackage{forest}

\forestset{
	rectstyle/.style={
		for tree={rectangle,draw,font=\large\sffamily}
	},
	roundstyle/.style={
		for tree={circle,draw,font=\large}
	}
}

% disponi algoritmi
\usepackage{algorithm}
\usepackage{algorithmic}
\makeatletter
\renewcommand{\ALG@name}{Algoritmo}
\makeatother

% disponi numeri di pagina
\usepackage{fancyhdr}
\fancyhf{} 
\fancyfoot[L]{\sffamily{\thepage}}

\makeatletter
\fancyhead[L]{\raisebox{1ex}[0pt][0pt]{\sffamily{\@title \ \@date}}} 
\fancyhead[R]{\raisebox{1ex}[0pt][0pt]{\sffamily{\@author}}}
\makeatother

\begin{document}

% sezione (data)
\section{Lezione del 01-10-24}

% stili pagina
\thispagestyle{empty}
\pagestyle{fancy}

% testo
\subsection{Soluzioni di base primali degeneri}
Abbiamo dato un teorema di caratterizzazione dei vertici primali.
Questo teorema si basava sulla nozione di \textbf{soluzione di base primale}.
Possiamo fare una distinzione fra soluzione di base degeneri e non degeneri:
\begin{definition}{Soluzione di base degenere}
Quando una soluzione di base è soluzione di più combinazioni delle disequazioni del problema, essa si dice degenere.
\end{definition}

Questa definizione è esatta ma non particolarmente utile.
Sostanzialmente, ci dice soltanto che una soluzione degenere è \textbf{ridondante} su più combinazioni di disequazioni (cioè risolve $A_Bx = b_B$ su più permutazioni degli $1,...,m$ elementi in classi $n$ in $B$).
Si noti che ridondante non significa \textbf{eliminabile}: questa affermazione purtroppo è vera soltanto in $R^2$, dove effettivamente si può rimuovere una delle disequazioni ridondanti ed avere sempre lo stesso risultato.

Diamo quindi una caratterizzazione delle soluzioni di base primali degeneri appoggiandoci al teorema di caratterizzazione dei vertici, ergo al concetto di soluzione di base:
\begin{theorem}{Caratterizzazione delle soluzioni di base primali degeneri}
	Se una soluzione è di base, ergo scelto $B = \{ 1, ..., m \}$ con $\mathrm{card}(B) = n$ è data da $A_Bx = b_B$, possiamo dire che è pure degenere quando $\exists i \in N$ t.c. $A_i x = b_i$, con $I$ = $\{1, ..., m\} - B$. 
\end{theorem}
Quindi, una soluzione di base è degenere quando almeno una variabile di base si annulla per almeno una delle disequazioni non di base indicate dagli indici $I$, che sono tutti gli indici fra $\{1,...,m\}$ non contenuti in $B$. 

Sulla stessa linea di pensiero, possiamo dimostrare un'altro teorema, stavolta sul concetto piuttosto intuitivo di ammissibilità.
Potremmo infatti dire che una soluzione di base ammissibile, cioè che rientra all'interno della regione ammissibile, è tale se:
\begin{theorem}{Caratterizzazione delle soluzioni di base primali ammissibili}
	Se una soluzione è di base, ergo scelto $B = \{ 1, ..., m \}$ con $\mathrm{card}(B) = n$ è data da $A_Bx = b_B$, possiamo dire che è ammissible quando $\forall i \in N$ si ha $A_i x \leq b_i$, con $I$ = $\{1, ..., m\} - B$. 
\end{theorem}
cioè banalmente rispetta tutte le disequazioni.

\subsubsection{Considerazioni numeriche sui numeri di soluzioni base}
Solitamente un problema con $n$ variabili decisionali a $m \geq n$ vincoli.
Posti questi vincoli, visto che per calcolare $\mathrm{vert}(P)$ prendiamo effettivamente tutte le combinazioni degli $m$ vincoli classe $n$ variabili decisionali, possiamo usare il coefficiente binomiale per calcolare il numero massimo di potenziali vertici: 
$$ \mathrm{card}(\mathrm{vert}(P)) \sim \binom{m}{n} = \frac{m!}{n!(m-n)!} $$

In verità, i vertici sono solitamente meno, in quanto possiamo rimuovere le soluzioni non ammissibili.
Inoltre, le soluzioni degeneri non contribuiscono al risultato, ergo anche quelle non sono rilevanti.

\subsection{Riassunto delle trasformazioni equivalenti}
Riassumiamo adesso le trasformazioni equivalenti che abbiamo individuato finora per le disequazioni di problemi LP:
\begin{itemize}
	\item $\mathrm{min}(C^\intercal \cdot x) \leftrightarrow \mathrm{max}(C^\intercal \cdot x)$: trasformiamo problemi di massimo in problemi di minimo invertendo i segni;
	\item $Ax \geq b \leftrightarrow -Ax \leq -b$: invertiamo il verso della diseguaglianza moltiplicando per $-1$;
	\item $Ax = b \rightarrow Ax \leq b \wedge Ax \geq b$: convertiamo un'uguaglianza in una coppia di diseguaglianze;
	\item $Ax \leq b \rightarrow Ax + s = b$: convertiamo una diseguaglianza in un'uguaglianza introducendo una variabile di surplus. Si nota che la variabile di surplus può essere riconosciuta per essere rimossa, da:
		\begin{itemize}
			\item $s > 0$;
			\item Compare in un solo vincolo, che è di uguaglianza;
			\item Ha coefficiente $0$ nella funzione obiettivo, e $1$ nell'equazione dove compare;
		\end{itemize}
	\item $ x \mathrel{\text{\tikz[baseline]{
    \draw (0,1.1ex)--(1.1em,0.1ex);
    \node[scale=1] at (0.6em,0.3em) {$\geq$};
  }}} 0 \rightarrow x = x^+ - x^-, \quad x^+ \geq 0, \quad x^- \geq 0$: aggiriamo il vincolo di positività introducendo parti positive e negative delle variabili decisionali, con rispettivi vincoli di positività.
\end{itemize}

Usiamo queste trasformazioni per portare i problemi LP in forme standard.
Esistono molteplici forme standard, ma in questo corso ci riguardano: il formato \textit{linprog}, usato dal pacchetto software \textit{MATLAB}, le forme standard primale (già vista) e duale (che vedremo fra poco). 

\end{document}
