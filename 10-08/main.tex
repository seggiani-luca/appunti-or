
\documentclass[a4paper,11pt]{article}
\usepackage[a4paper, margin=8em]{geometry}

% usa i pacchetti per la scrittura in italiano
\usepackage[french,italian]{babel}
\usepackage[T1]{fontenc}
\usepackage[utf8]{inputenc}
\frenchspacing 

% usa i pacchetti per la formattazione matematica
\usepackage{amsmath, amssymb, amsthm, amsfonts}

% usa altri pacchetti
\usepackage{gensymb}
\usepackage{hyperref}
\usepackage{standalone}

% imposta il titolo
\title{Appunti Ricerca Operativa}
\author{Luca Seggiani}
\date{2024}

% disegni
\usepackage{pgfplots}
\pgfplotsset{width=10cm,compat=1.9}

% imposta lo stile
% usa helvetica
\usepackage[scaled]{helvet}
% usa palatino
\usepackage{palatino}
% usa un font monospazio guardabile
\usepackage{lmodern}

\renewcommand{\rmdefault}{ppl}
\renewcommand{\sfdefault}{phv}
\renewcommand{\ttdefault}{lmtt}

% disponi il titolo
\makeatletter
\renewcommand{\maketitle} {
	\begin{center} 
		\begin{minipage}[t]{.8\textwidth}
			\textsf{\huge\bfseries \@title} 
		\end{minipage}%
		\begin{minipage}[t]{.2\textwidth}
			\raggedleft \vspace{-1.65em}
			\textsf{\small \@author} \vfill
			\textsf{\small \@date}
		\end{minipage}
		\par
	\end{center}

	\thispagestyle{empty}
	\pagestyle{fancy}
}
\makeatother

% disponi teoremi
\usepackage{tcolorbox}
\newtcolorbox[auto counter, number within=section]{theorem}[2][]{%
	colback=blue!10, 
	colframe=blue!40!black, 
	sharp corners=northwest,
	fonttitle=\sffamily\bfseries, 
	title=Teorema~\thetcbcounter: #2, 
	#1
}

% disponi definizioni
\newtcolorbox[auto counter, number within=section]{definition}[2][]{%
	colback=red!10,
	colframe=red!40!black,
	sharp corners=northwest,
	fonttitle=\sffamily\bfseries,
	title=Definizione~\thetcbcounter: #2,
	#1
}

% disponi problemi
\newtcolorbox[auto counter, number within=section]{problem}[2][]{%
	colback=green!10,
	colframe=green!40!black,
	sharp corners=northwest,
	fonttitle=\sffamily\bfseries,
	title=Problema~\thetcbcounter: #2,
	#1
}

% disponi codice
\usepackage{listings}
\usepackage[table]{xcolor}

\lstdefinestyle{codestyle}{
		backgroundcolor=\color{black!5}, 
		commentstyle=\color{codegreen},
		keywordstyle=\bfseries\color{magenta},
		numberstyle=\sffamily\tiny\color{black!60},
		stringstyle=\color{green!50!black},
		basicstyle=\ttfamily\footnotesize,
		breakatwhitespace=false,         
		breaklines=true,                 
		captionpos=b,                    
		keepspaces=true,                 
		numbers=left,                    
		numbersep=5pt,                  
		showspaces=false,                
		showstringspaces=false,
		showtabs=false,                  
		tabsize=2
}

\lstdefinestyle{shellstyle}{
		backgroundcolor=\color{black!5}, 
		basicstyle=\ttfamily\footnotesize\color{black}, 
		commentstyle=\color{black}, 
		keywordstyle=\color{black},
		numberstyle=\color{black!5},
		stringstyle=\color{black}, 
		showspaces=false,
		showstringspaces=false, 
		showtabs=false, 
		tabsize=2, 
		numbers=none, 
		breaklines=true
}

\lstdefinelanguage{javascript}{
	keywords={typeof, new, true, false, catch, function, return, null, catch, switch, var, if, in, while, do, else, case, break},
	keywordstyle=\color{blue}\bfseries,
	ndkeywords={class, export, boolean, throw, implements, import, this},
	ndkeywordstyle=\color{darkgray}\bfseries,
	identifierstyle=\color{black},
	sensitive=false,
	comment=[l]{//},
	morecomment=[s]{/*}{*/},
	commentstyle=\color{purple}\ttfamily,
	stringstyle=\color{red}\ttfamily,
	morestring=[b]',
	morestring=[b]"
}

% disponi sezioni
\usepackage{titlesec}

\titleformat{\section}
	{\sffamily\Large\bfseries} 
	{\thesection}{1em}{} 
\titleformat{\subsection}
	{\sffamily\large\bfseries}   
	{\thesubsection}{1em}{} 
\titleformat{\subsubsection}
	{\sffamily\normalsize\bfseries} 
	{\thesubsubsection}{1em}{}

% disponi alberi
\usepackage{forest}

\forestset{
	rectstyle/.style={
		for tree={rectangle,draw,font=\large\sffamily}
	},
	roundstyle/.style={
		for tree={circle,draw,font=\large}
	}
}

% disponi algoritmi
\usepackage{algorithm}
\usepackage{algorithmic}
\makeatletter
\renewcommand{\ALG@name}{Algoritmo}
\makeatother

% disponi numeri di pagina
\usepackage{fancyhdr}
\fancyhf{} 
\fancyfoot[L]{\sffamily{\thepage}}

\makeatletter
\fancyhead[L]{\raisebox{1ex}[0pt][0pt]{\sffamily{\@title \ \@date}}} 
\fancyhead[R]{\raisebox{1ex}[0pt][0pt]{\sffamily{\@author}}}
\makeatother

\begin{document}

% sezione (data)
\section{Lezione del 08-10-24}

% stili pagina
\thispagestyle{empty}
\pagestyle{fancy}

% testo
Abbiamo trovato un'algoritmo per ricavare una soluzione ottimale partendo da una soluzione di base primale ammissibile.
Nel fare ciò, si è dato per scontato che l'algoritmo di partenza ci avrebbe fornito una soluzione di base primale ammissibile, e che qualsiasi successivo passo del simplesso ci avrebbe restituito altre soluzioni di base primali ammissibili.
In verita, le soluzioni di base trovate possono avere le seguenti combinazioni di ammissibilità su $P$ primale e $D$ duale:
\begin{itemize}
	\item \textbf{Ammissibile su $P$ e $D$:} in questo caso siamo all'ottimo dalla dualità forte;
	\item \textbf{Ammissibile su $P$ ma non su $D$:} in questo caso siamo su un comune punto ammissibile non ottimo;
	\item \textbf{Non ammissibile su $P$ o $D$:} in questo caso si scarta la soluzione;
	\item \textbf{Ammissibile su $D$ ma non su $P$:} effettivamente, ancora non abbiamo un modo per gestire questa situazione.
\end{itemize}

Per questo motivo estendiamo l'algoritmo del simplesso al duale.

\subsection{Algoritmo del simplesso duale}
Intendiamo mantenere, ad ogni passo, la soluzione di base duale come ammissibile e controllare l'ammissibilità di quella primale.
Se la soluzione di base primale è ammissibile (l'inverso di come avevamo visto per il simplesso primale), allora siamo sull'ottimo.
Altrimenti, si cambia base, cercando di minimizzare la funzione obiettivo su una nuova soluzione di base duale.

Abbiamo quindi un vertice del poliedro duale, cioè una soluzione di base ammissibile del duale. 
Ci chiediamo se questo vertice $\bar{y}$ è ottimo.
Visto che è vertice del duale, abbiamo che per una matrice $A_B$ e un vettore costo $c$ del primale:
$$
\bar{y} = (cA_B^{-1}, 0), \quad \text{con} \ cA_B^{-1} \geq 0 
$$
dove ricordiamo questa notazione significa impostare le variabili non di base $\in N$ a $0$ e risolvere il sistema in $m$ variabili rimasto.

Adesso possiamo costruire il complementare primale $\bar{x}$, ponendo:
$$
\bar{x} = A_B^{-1}b_b 
$$
e applicare il test di ammissibilità, cioè vedere se:
$$
A_N(A_B^{-1} b_B) \leq b_N
$$
ergo $\bar{x} \in P$, quindi il complementare duale esiste e il vertice è ottimo.
Come prima, se questa condizione risulta verificata possiamo fermarci, in quanto abbiamo trovato la soluzione ottimale.

In caso contrario, avremo $\exists i \in N$ tale che $b_i - A_i (A_B^{-1} - b_B) < 0$.
Questo equivale a ciò che avevamo trovato per il primale per quanto riguardava gli indici uscenti, con una sola differenza: per risolvere il duale, si trova \textbf{prima} l'\textbf{indice entrante}, e \textbf{poi} l'\textbf{indice uscente}.
Vediamo quindi i passaggi in ordine:

\begin{itemize}
	\item \textbf{\textsf{Indice entrante}} \\
Abbiamo che, se il vertice non è ottimo, vale:
$$
\exists i \in N \ \text{t.c.} \ b_i - A_i (A_B^{-1} - b_B) < 0
$$
Vogliamo che l'indice entrante sia quell che raggiunge questo valore negativo, quindi:
\begin{definition}{Indice entrante duale}
Chiamiamo indice entrante $k$, da una certa soluzione della base $B$:
$$k := \min\{ i \in N \ \text{t.c.} \ b_i - A_i \bar{x} < 0 \}$$
\end{definition}
Come sempre, $\min$ significa che in caso di più $i$ negativi, si adotta la regola anticiclo (di Bland) di scegliere il primo.
In caso di nessun $i$ negativo, la complementare duale esiste e siamo sull'ottimo.

	\item \textbf{\textsf{Indice uscente}} \\
Cerchiamo quindi l'indice uscente.
Dovremo prima definire la matrice $W$ come $ W = -A_B^{-1}$, e  calcolare il prodotto (perlopiù analogo al primale, ma si noti il segno della diseguaglianza capovolto):
$$ 
A_k W^i < 0
$$

Questo ci fornisce una regola, come nel simplesso primale, per l'esistenza di una soluzione: nel caso non sia verificato, si ha che il duale $\rightarrow -\infty$ e che il primale è vuoto.
In caso contrario, si possono calcolare i rapporti, come:
$$
r_i = \frac{-\bar{y}_i}{A_kW^i}, \quad i \in B, A_k W^i < 0
$$
e scegliere l'indice che da $\vartheta = \min{r_i}$, cioè:
\begin{definition}{Indice uscente duale}
	Chiamiamo indice uscente $h$, da una certa soluzione della base $B$ e un certo indice entrante $k$:
	$$
	h := \min\{ i \in B \ \text{t.c.} \ A_k W^i < 0, \quad \frac{-\bar{y}_i}{A_kW^i} = \vartheta \}	
	$$
\end{definition}
Anche qui, il $\min$ serve a selezionare il primo indice valido, ed è una regola anticiclo.
\end{itemize}

Abbiamo quindi revisionato tutti gli strumenti necessari alla formulazione dell'algoritmo del simplesso, stavolta duale:
\begin{algorithm}[H]
\caption{del simplesso duale}
\begin{algorithmic}
	\STATE \textbf{Input:} un problema LP in forma primale standard
	\STATE \textbf{Output:} la soluzione ottima 
	\STATE Trova una base B che genera una soluzione di base duale ammissibile.
	\STATE \textsf{ciclo:}
	\STATE Calcola la soluzione di base duale $\bar{y} = (cA_B^{-1}, 0)$ e la soluzione di base primale $\bar{x} = A_B^{-1}b_B$
	\IF{$A_N(A_B^{-1} b_B) \leq b_N$}
		\STATE Fermati, $\bar{y}$ è ottima per $D$ e $\bar{x}$ è ottima per $P$
	\ELSE
		\STATE Calcola l'indice entrante: 
		$$
		k := \min\{ i \in N \ \text{t.c.} \ b_i - A_i \bar{x} < 0 \}
		$$
		poni $W := -A_B^{-1}$ e indica con $W^i$ la $i$-esima colonna di $W$
	\ENDIF
	\IF{$A_k W^i \geq 0 \quad \forall i \in B$}
		\STATE Fermati, $D \rightarrow -\infty$ e $P$ non ha soluzione ottima
	\ELSE
		\STATE Calcola:
		$$
		\vartheta = \min\{ \frac{-\bar{y}_i}{A_kW^i} \text{t.c.} \quad i \in B, \quad A_k W^i < 0 \}
		$$
		e trova l'indice uscente: 
		$$ 
		h := \min\{ i \in B \ \text{t.c.} \ A_k W^i < 0, \quad \frac{-\bar{y}_i}{A_kW^i} = \vartheta \}	
		$$
	\ENDIF
	\STATE Aggiorna la base come:
	$$
	B := B \setminus \{h\} \cup \{k\}
	$$
	\STATE Torna a \textsf{ciclo}
\end{algorithmic}
\end{algorithm}

\end{document}

