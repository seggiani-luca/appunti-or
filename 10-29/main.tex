
\documentclass[a4paper,11pt]{article}
\usepackage[a4paper, margin=8em]{geometry}

% usa i pacchetti per la scrittura in italiano
\usepackage[french,italian]{babel}
\usepackage[T1]{fontenc}
\usepackage[utf8]{inputenc}
\frenchspacing 

% usa i pacchetti per la formattazione matematica
\usepackage{amsmath, amssymb, amsthm, amsfonts}

% usa altri pacchetti
\usepackage{gensymb}
\usepackage{hyperref}
\usepackage{standalone}

% imposta il titolo
\title{Appunti Ricerca Operativa}
\author{Luca Seggiani}
\date{2024}

% disegni
\usepackage{pgfplots}
\pgfplotsset{width=10cm,compat=1.9}

% imposta lo stile
% usa helvetica
\usepackage[scaled]{helvet}
% usa palatino
\usepackage{palatino}
% usa un font monospazio guardabile
\usepackage{lmodern}

\renewcommand{\rmdefault}{ppl}
\renewcommand{\sfdefault}{phv}
\renewcommand{\ttdefault}{lmtt}

% disponi il titolo
\makeatletter
\renewcommand{\maketitle} {
	\begin{center} 
		\begin{minipage}[t]{.8\textwidth}
			\textsf{\huge\bfseries \@title} 
		\end{minipage}%
		\begin{minipage}[t]{.2\textwidth}
			\raggedleft \vspace{-1.65em}
			\textsf{\small \@author} \vfill
			\textsf{\small \@date}
		\end{minipage}
		\par
	\end{center}

	\thispagestyle{empty}
	\pagestyle{fancy}
}
\makeatother

% disponi teoremi
\usepackage{tcolorbox}
\newtcolorbox[auto counter, number within=section]{theorem}[2][]{%
	colback=blue!10, 
	colframe=blue!40!black, 
	sharp corners=northwest,
	fonttitle=\sffamily\bfseries, 
	title=Teorema~\thetcbcounter: #2, 
	#1
}

% disponi definizioni
\newtcolorbox[auto counter, number within=section]{definition}[2][]{%
	colback=red!10,
	colframe=red!40!black,
	sharp corners=northwest,
	fonttitle=\sffamily\bfseries,
	title=Definizione~\thetcbcounter: #2,
	#1
}

% disponi problemi
\newtcolorbox[auto counter, number within=section]{problem}[2][]{%
	colback=green!10,
	colframe=green!40!black,
	sharp corners=northwest,
	fonttitle=\sffamily\bfseries,
	title=Problema~\thetcbcounter: #2,
	#1
}

% disponi codice
\usepackage{listings}
\usepackage[table]{xcolor}

\lstdefinestyle{codestyle}{
		backgroundcolor=\color{black!5}, 
		commentstyle=\color{codegreen},
		keywordstyle=\bfseries\color{magenta},
		numberstyle=\sffamily\tiny\color{black!60},
		stringstyle=\color{green!50!black},
		basicstyle=\ttfamily\footnotesize,
		breakatwhitespace=false,         
		breaklines=true,                 
		captionpos=b,                    
		keepspaces=true,                 
		numbers=left,                    
		numbersep=5pt,                  
		showspaces=false,                
		showstringspaces=false,
		showtabs=false,                  
		tabsize=2
}

\lstdefinestyle{shellstyle}{
		backgroundcolor=\color{black!5}, 
		basicstyle=\ttfamily\footnotesize\color{black}, 
		commentstyle=\color{black}, 
		keywordstyle=\color{black},
		numberstyle=\color{black!5},
		stringstyle=\color{black}, 
		showspaces=false,
		showstringspaces=false, 
		showtabs=false, 
		tabsize=2, 
		numbers=none, 
		breaklines=true
}

\lstdefinelanguage{javascript}{
	keywords={typeof, new, true, false, catch, function, return, null, catch, switch, var, if, in, while, do, else, case, break},
	keywordstyle=\color{blue}\bfseries,
	ndkeywords={class, export, boolean, throw, implements, import, this},
	ndkeywordstyle=\color{darkgray}\bfseries,
	identifierstyle=\color{black},
	sensitive=false,
	comment=[l]{//},
	morecomment=[s]{/*}{*/},
	commentstyle=\color{purple}\ttfamily,
	stringstyle=\color{red}\ttfamily,
	morestring=[b]',
	morestring=[b]"
}

% disponi sezioni
\usepackage{titlesec}

\titleformat{\section}
	{\sffamily\Large\bfseries} 
	{\thesection}{1em}{} 
\titleformat{\subsection}
	{\sffamily\large\bfseries}   
	{\thesubsection}{1em}{} 
\titleformat{\subsubsection}
	{\sffamily\normalsize\bfseries} 
	{\thesubsubsection}{1em}{}

% disponi alberi
\usepackage{forest}

\forestset{
	rectstyle/.style={
		for tree={rectangle,draw,font=\large\sffamily}
	},
	roundstyle/.style={
		for tree={circle,draw,font=\large}
	}
}

% disponi algoritmi
\usepackage{algorithm}
\usepackage{algorithmic}
\makeatletter
\renewcommand{\ALG@name}{Algoritmo}
\makeatother

% disponi numeri di pagina
\usepackage{fancyhdr}
\fancyhf{} 
\fancyfoot[L]{\sffamily{\thepage}}

\makeatletter
\fancyhead[L]{\raisebox{1ex}[0pt][0pt]{\sffamily{\@title \ \@date}}} 
\fancyhead[R]{\raisebox{1ex}[0pt][0pt]{\sffamily{\@author}}}
\makeatother

\begin{document}

% sezione (data)
\section{Lezione del 29-10-24}

% stili pagina
\thispagestyle{empty}
\pagestyle{fancy}

% testo
\subsection{Regole di branch}
Vediamo quindi nel dettaglio le regole di ramificazione da usare nell'applicazione del branch and bound, su problemi di minimo e di massimo.
Consideriamo un problema TSP simmetrico minimizzante:
\subsubsection{Regole di branch minimizzanti}
Partiamo dal problema $P$, e istanziamo variabili per ottenere una serie di $P_{i,j}$, dove la $i$ rappresenta il livello di profondità nell'albero e $j$ il sottoproblema corrente a quel livello.

Ogni volta che si istanzia la prima cosa da controllare è l'ammissibilità delle soluzioni ottenute istanziando le variabili.
Se si ha infatti che se $P_{i,j} = \emptyset$, quel sottoalbero può essere tagliato.

Si controlla poi se vale $v_I(P_{i,j}) \geq v_S(P)$, cioè se il minimo possibile ottenuto dal sottoproblema $P_{i,j}$ è più piccolo della valutazione superiore del problema $P$. In questo caso si taglia il sottoalbero.

Si controlla infine se vale $v_I(P_{i,j}) < v_S(P)$ e questa $v_I$ è ammissibile per $P$. Se sì, allora si aggionra il $v_S$ corrente di $P$ a $v_I$.
Inoltre, si può applicare anche la regola precedente, e quindi tagliare il sottoalbero.

Riassumendo si hanno quindi le regole:
\begin{itemize}
	\item $P_{i,j} = \emptyset \implies $ taglio;
	\item $v_I(P_{i,j}) \geq v_S(P) \implies $ taglio;
	\item $v_I(P_{i,j}) < v_S(P)$ e $v_I \in P \implies $ $v_S(P) \leftarrow v_I(P_{i,j})$, taglio.
\end{itemize}

\par\smallskip 
Consideriamo poi un problema di zaino booleano massimizzante:
\subsubsection{Regole di branch massimizzanti}
In un problema massimizzante si segue un approccio simile.
Dal problema $P$ si istanzia successivamente per ottenere $P_{i,j}$ con le stesse caratteristiche di prima.

Istanziando, se si va a svuotare la regione ammissbilie, cioè si ottiene $P_{i,j} = \emptyset$, si scarta quel sottoalbero.

Si controlla poi se vale $v_S(P_{i, j}) \leq v_I (P)$, cioè se il massimo possibile ottenuto dal sottoproblema $P_{i,j}$ è più grande della valutazione inferiore del problema $P$. In questo caso, chiaramente, si taglia il sottoalbero.

Infine si controlla se vale $v_S(P_{i,j}) > v_I(P)$, e questa $v_S$ è ammissibile per $P$ (cioè, in questo caso, è a componenti intere). Se sì, allora si aggiorna il $v_I$ corrente di $P$ a $v_S$. 
Inoltre, si taglia il sottoalbero.

Anche qui, riassumendo, si hanno le regole:
\begin{itemize}
	\item $P_{i,j} = \emptyset \implies $ taglio;
	\item $v_S(P_{i,j}) \leq v_I(P) \implies $ taglio;
	\item $v_S(P_{i,j}) > v_I(P)$ e $v_S \in P \implies $ $v_I(P) \leftarrow v_S(P_{i,j})$, taglio.
\end{itemize}


\end{document}
