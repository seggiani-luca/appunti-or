
\documentclass[a4paper,11pt]{article}
\usepackage[a4paper, margin=8em]{geometry}

% usa i pacchetti per la scrittura in italiano
\usepackage[french,italian]{babel}
\usepackage[T1]{fontenc}
\usepackage[utf8]{inputenc}
\frenchspacing 

% usa i pacchetti per la formattazione matematica
\usepackage{amsmath, amssymb, amsthm, amsfonts}

% usa altri pacchetti
\usepackage{gensymb}
\usepackage{hyperref}
\usepackage{standalone}

% imposta il titolo
\title{Appunti Ricerca Operativa}
\author{Luca Seggiani}
\date{2024}

% disegni
\usepackage{pgfplots}
\pgfplotsset{width=10cm,compat=1.9}

% imposta lo stile
% usa helvetica
\usepackage[scaled]{helvet}
% usa palatino
\usepackage{palatino}
% usa un font monospazio guardabile
\usepackage{lmodern}

\renewcommand{\rmdefault}{ppl}
\renewcommand{\sfdefault}{phv}
\renewcommand{\ttdefault}{lmtt}

% disponi il titolo
\makeatletter
\renewcommand{\maketitle} {
	\begin{center} 
		\begin{minipage}[t]{.8\textwidth}
			\textsf{\huge\bfseries \@title} 
		\end{minipage}%
		\begin{minipage}[t]{.2\textwidth}
			\raggedleft \vspace{-1.65em}
			\textsf{\small \@author} \vfill
			\textsf{\small \@date}
		\end{minipage}
		\par
	\end{center}

	\thispagestyle{empty}
	\pagestyle{fancy}
}
\makeatother

% disponi teoremi
\usepackage{tcolorbox}
\newtcolorbox[auto counter, number within=section]{theorem}[2][]{%
	colback=blue!10, 
	colframe=blue!40!black, 
	sharp corners=northwest,
	fonttitle=\sffamily\bfseries, 
	title=Teorema~\thetcbcounter: #2, 
	#1
}

% disponi definizioni
\newtcolorbox[auto counter, number within=section]{definition}[2][]{%
	colback=red!10,
	colframe=red!40!black,
	sharp corners=northwest,
	fonttitle=\sffamily\bfseries,
	title=Definizione~\thetcbcounter: #2,
	#1
}

% disponi problemi
\newtcolorbox[auto counter, number within=section]{problem}[2][]{%
	colback=green!10,
	colframe=green!40!black,
	sharp corners=northwest,
	fonttitle=\sffamily\bfseries,
	title=Problema~\thetcbcounter: #2,
	#1
}

% disponi codice
\usepackage{listings}
\usepackage[table]{xcolor}

\lstdefinestyle{codestyle}{
		backgroundcolor=\color{black!5}, 
		commentstyle=\color{codegreen},
		keywordstyle=\bfseries\color{magenta},
		numberstyle=\sffamily\tiny\color{black!60},
		stringstyle=\color{green!50!black},
		basicstyle=\ttfamily\footnotesize,
		breakatwhitespace=false,         
		breaklines=true,                 
		captionpos=b,                    
		keepspaces=true,                 
		numbers=left,                    
		numbersep=5pt,                  
		showspaces=false,                
		showstringspaces=false,
		showtabs=false,                  
		tabsize=2
}

\lstdefinestyle{shellstyle}{
		backgroundcolor=\color{black!5}, 
		basicstyle=\ttfamily\footnotesize\color{black}, 
		commentstyle=\color{black}, 
		keywordstyle=\color{black},
		numberstyle=\color{black!5},
		stringstyle=\color{black}, 
		showspaces=false,
		showstringspaces=false, 
		showtabs=false, 
		tabsize=2, 
		numbers=none, 
		breaklines=true
}

\lstdefinelanguage{javascript}{
	keywords={typeof, new, true, false, catch, function, return, null, catch, switch, var, if, in, while, do, else, case, break},
	keywordstyle=\color{blue}\bfseries,
	ndkeywords={class, export, boolean, throw, implements, import, this},
	ndkeywordstyle=\color{darkgray}\bfseries,
	identifierstyle=\color{black},
	sensitive=false,
	comment=[l]{//},
	morecomment=[s]{/*}{*/},
	commentstyle=\color{purple}\ttfamily,
	stringstyle=\color{red}\ttfamily,
	morestring=[b]',
	morestring=[b]"
}

% disponi sezioni
\usepackage{titlesec}

\titleformat{\section}
	{\sffamily\Large\bfseries} 
	{\thesection}{1em}{} 
\titleformat{\subsection}
	{\sffamily\large\bfseries}   
	{\thesubsection}{1em}{} 
\titleformat{\subsubsection}
	{\sffamily\normalsize\bfseries} 
	{\thesubsubsection}{1em}{}

% disponi alberi
\usepackage{forest}

\forestset{
	rectstyle/.style={
		for tree={rectangle,draw,font=\large\sffamily}
	},
	roundstyle/.style={
		for tree={circle,draw,font=\large}
	}
}

% disponi algoritmi
\usepackage{algorithm}
\usepackage{algorithmic}
\makeatletter
\renewcommand{\ALG@name}{Algoritmo}
\makeatother

% disponi numeri di pagina
\usepackage{fancyhdr}
\fancyhf{} 
\fancyfoot[L]{\sffamily{\thepage}}

\makeatletter
\fancyhead[L]{\raisebox{1ex}[0pt][0pt]{\sffamily{\@title \ \@date}}} 
\fancyhead[R]{\raisebox{1ex}[0pt][0pt]{\sffamily{\@author}}}
\makeatother

\begin{document}

% sezione (data)
\section{Lezione del 15-10-24}

% stili pagina
\thispagestyle{empty}
\pagestyle{fancy}

% testo

\subsection{Relazioni tra LP e ILP}
Vediamo di approfondire il legame fra un problema ILP e i problemi LP che possiamo ricavarne.
Avevamo posto un problema ILP in forma:
\[
	\begin{cases}
			\max (c^T x) \\ 
			Ax \leq b \\ 
			x \in \mathbb{Z}^n
	\end{cases}
\]

E avevamo visto che si può trovare un limite inferiore $V_I$ e un limite superiore $V_S$ a partire dagli algoritmi dei rendimenti, quindi prendendo il rilassato continuo del problema, cioè l'associato che rimuove il vincolo $x \in \mathbb{Z}^n$:
\[
	\begin{cases}
			\max (c^T x) \\ 
			Ax \leq b
	\end{cases}
\]

Chiamiamo $P$ il poliedro del rilassato continuo.
Si ha, in generale, che la soluzione di un problema di ILP è uno dei punti $\in \Omega = P \cap \mathbb{Z}^n$, cioè dei punti $\in \mathbb{Z}^n$ che stanno all'interno del poliedro del rilassato continuo.
Poniamo ad esempio il problema:

\[
	\begin{cases}
		\max(0.3 x_1 + 0.4 x_2) \\
		3 x_1 + 5 x_2 \leq 15 \\ 
		4 x_1 +  4 x_2 \leq 16 \\ 
		x \in \mathbb{Z}^2
	\end{cases}
\]

Graficamente, si ha:

\begin{center}

\begin{tikzpicture}
\begin{axis}[
    axis lines = middle,
		grid=both,
    xlabel = {$x_1$},
    ylabel = {$x_2$},
    xmin=0, xmax=6.9,
    ymin=0, ymax=3.9,
    domain=0:10,
    samples=100,
    width=15cm, height=9cm,
    legend pos=north east
  ]

% regione ammissibile

\addplot[fill=gray, opacity=0.4, forget plot] 
	coordinates {
		(0,0)
		(0,3)
		(2.5, 1.5)
		(4, 0)
	};

% rette

\addplot[domain=0:2.5, thick, red] {3 - 0.6*x};
\addlegendentry{$ 3 x_1 + 5 x_2 \leq 15 $}

\addplot[domain=2.5:4, thick, blue] {4 - x};
\addlegendentry{$ 4 x_1 + 4 x_2 \leq 16 $}

\addplot[
    only marks,  % Only marks the points without connecting lines
    mark=*,
    color=violet,
    mark size=2pt
] coordinates {
	(0,0)
	(0,1)
	(0,2)
	(0,3)
	(1,0)
	(1,1)
	(1,2)
	(2,0)
	(2,1)
	(3,0)
	(3,1)
	(4,0)
};
\addlegendentry{$\Omega$}

\addplot[
	only marks,
	mark=*,
	mark size=2pt
] coordinates {
	(2.5, 1.5)
};

\node at (axis cs:2.5, 1.5) [above right] {$x_{V_S}$};
 
\end{axis}
\end{tikzpicture}

\end{center}
dove si è riportata la soluzione ottima del rilassato $x_{V_S}$.

Notiamo quindi che esiste un'insieme:
$$
\mathrm{conv} (\Omega) = \{ x \in \mathbb{R}^n : \exists x_1, ..., x_p \in \Omega, \quad \lambda_1, ..., \lambda_p \geq 0 \quad \text{t.c.} \ \ \ x = \sum_{i=1}^p  \lambda_i x_i, \quad \sum \lambda_{i=1}^p  \lambda_i = 1 \}
$$
cioè l'involucro convesso di $\Omega$.
Visto che i punti di $\Omega$ hanno componenti intere, si può dimostrare il seguente teorema:
\begin{theorem}{Caratterizzazione della regione ammissibile di un problema ILP}
	Dato un problema ILP con regione ammissibile $\Omega$, esiste un insieme finito di punti $\{ q_l \}_{l \in L} = \{ q_1, ..., q_{|L|} \}$ di $\Omega$, e un insieme finito di direzioni di recessione $\{ r_j \}_{j \in J} = \{ r_1, ..., r_|J| \}$ di $P$, tali che:

$$ 
\Omega = \{ x \in R^n_+: x = \sum_{l\in L} \alpha_l q_l + \sum_{j\in J} \beta_j q_j, \quad \sum_{l \in L} \alpha_l = 1, \quad \alpha \in \mathbb{Z}^{|L|}_+, \quad \beta \in \mathbb{Z}^{|J|}_+ \}
$$
cioè si puo ricavare $\Omega$ attraverso una forma simile alla $P = \mathrm{conv}(V) + \mathrm{cono}(E)$ del teorema di Minkowski-Weyl, sugli insiemi $\{ q_l \}_{l \in L}$ di vertici a componenti intere e $\{ r_j \}_{j \in J}$ di direzioni di recessione.
\end{theorem}

A partire da questa caratterizzazione di $\Omega$, vogliamo caratterizzare $\mathrm{conv} (\Omega)$:

\begin{theorem}{Caratterizzazione dell'involucro convesso della regione ammissibile di un problema ILP}
	Dato un problema ILP con regione ammissibile $\Omega$, si ha che $\mathrm{conv}(\Omega)$ è un \textbf{poliedro razionale}, cioè esistono due insiemi finiti di vettori, $\{ q_l \}_{l \in L}$ e $\{ r_j \}_{j \in J}$, a \textbf{componenti razionali}, tali che:

$$
\mathrm{conv}(\Omega) = \mathrm{conv} \{ q_l \}_{l \in L} + \mathrm{cono} \{ r_j \}_{j \in J}
$$

\end{theorem}

Addirittura, normalizzando si può supporre che gli $r_j$ siano a componenti intere.

Nell'esempio precedente, $\mathrm{conv} (\Omega)$ sarebbe rappresentato dagli insiemi: 
$$
	\{ q_l \}_{l \in L} = \{ (0,0), (0,3), (3,1), (4,0) \}, \quad \{ r_j \}_{j \in J} = \emptyset 
$$
quindi:
$$
\mathrm{conv}(\Omega) = \mathrm{conv} \{ q_l \}_{l \in L} + \mathrm{cono} \{ r_j \}_{j \in J} = (x_1, x_2) \in{R^2} \quad  \text{t.c.} \quad
	\begin{cases}
		\frac{2}{3}x_1 + x_2 \leq 3 \\ 
		x_1 +x_2 \leq 4
	\end{cases}
$$

cioè sul grafico:

\begin{center}

\begin{tikzpicture}
\begin{axis}[
    axis lines = middle,
		grid=both,
    xlabel = {$x_1$},
    ylabel = {$x_2$},
    xmin=0, xmax=6.9,
    ymin=0, ymax=3.9,
    domain=0:10,
    samples=100,
    width=15cm, height=9cm,
    legend pos=north east
  ]

% regione ammissibile

\addplot[fill=gray, opacity=0.4, forget plot] 
	coordinates {
		(0,0)
		(0,3)
		(3, 1)
		(4, 0)
	};

% rette

\addplot[domain=0:3, thick, red] {3 - 0.6666666666666666666*x};
\addlegendentry{$ \frac{2}{3} x_1 +  x_2 \leq 3 $}

\addplot[domain=3:4, thick, blue] {4 - x};
\addlegendentry{$ x_1 + 4 x_2 \leq 4 $}

\addplot[
    only marks,  % Only marks the points without connecting lines
    mark=*,
    color=violet,
    mark size=2pt
] coordinates {
	(0,0)
	(0,1)
	(0,2)
	(0,3)
	(1,0)
	(1,1)
	(1,2)
	(2,0)
	(2,1)
	(3,0)
	(3,1)
	(4,0)
};
\addlegendentry{$\Omega$}

\addplot[
	only marks,
	mark=*,
	mark size=2pt
] coordinates {
	(3, 1)
};

\node at (axis cs:3, 1) [above right] {$x_{\mathrm{conv}(\Omega)}$};
 
\end{axis}
\end{tikzpicture}

\end{center}

dove è stata evidenziata la soluzione del primale sull'insieme $\mathrm{conv}(\Omega)$, $x_{\mathrm{conv}(\Omega)}$.

Possiamo quindi dire, visto che $\Omega \subset \mathrm{conv}(\Omega)$ e che $P$ è un'estensione di $\Omega$ in quanto poliedro del rilassato continuo, che è vera la catena di diseguaglianze:
$$
\max_{x \in \Omega} c^T x \leq \max_{x \in \mathrm{conv}(\Omega)} c^T x \leq \max_{x \in P} c^T x
$$
e non solo: si può stringere la diseguaglianza sul lato sinistro, per affermare che:
\begin{theorem}{Equivalenza fra problemi LP e ILP}
	Si prenda un problema di ILP, e il problema di LP associato costruito su:
$$
\mathrm{conv}(\Omega) = \mathrm{conv} \{ q_l \}_{l \in L} + \mathrm{cono} \{ r_j \}_{j \in J}
$$
con, posto $P$ come il poliedro del rilassato continuo, $q$ ricavato dai vertici $P \cap \mathbb{Z}^n$, e $r$ ricavato dalle direzioni di recessione di $P$. 

Se si prendono le soluzioni:
$$ v_\Omega = \max_{x \in \Omega} c^T x, \quad v_{\mathrm{conv}(\Omega)} \max_{x \in \mathrm{conv}(\Omega)} c^T x$$
si ha che $v_\Omega = v_{\mathrm{conv}(\Omega)}$, e che se $v_{\mathrm{conv}(\Omega)}$ è finito, allora esiste $x_\Omega \in \Omega$ tale che $c^T x_\Omega = v_\Omega = v_{\mathrm{conv}(\Omega)}$
\end{theorem}

Siamo quindi arrivati a formulare il teorema secondo cui, per ogni problema ILP, possamo costruire un problema LP associato che ha la stessa soluzione, semplicemente riformulando i vincoli in modo che descrivano l'involucro convesso dei punti in $\Omega = P \cap \mathbb{Z}^n$, ed eventuali direzioni di recessione di $P$, dove $P$ è il poliedro del rilassato continuo.
Il problema sorge dal fatto che è \textit{difficile} ricavare questo problema associato.
Esistono però alcuni casi particolari dove pouò essere conveniente applicare questo teorema, cioè quando si è in presenza di \textbf{matrici unimodulari}.

\subsubsection{Matrici unimodulari}
Definiamo innanzitutto:
\begin{definition}{Matrice unimodulare}
	Si chiama \textbf{modulare} ogni matrice quadrata intera con determinante $\det{A_{m}} \in \{ 1, -1\}$.
\end{definition}
e, sulla base di questo:
\begin{definition}{Matrice totalmente unimodulare}
	Si chiama \textbf{totalmente unimodulare} ogni matrice per cui ogni sottomatrice quadrata invertibile è unimodulare, cioè ogni sottomatrice quadrata ha determinante $\det(A_{m}) \in \{0, 1, -1\}$.
\end{definition}

Si ha che se una matrice è unimodulare, allora i vertici della regione ammissibile appartengono a $\mathbb{Z}^n$, infatti:
\begin{theorem}{Soluzioni di base di matrici unimodulari}
	Dato $Ax \leq b$, se $A$ e $b$ sono a componenti intere, e $A$ è totalmente unimodulare, allora tutte le soluzioni di base del poliedro $P$:
	$$
		P = \{ x \in \mathbb{R}^n : Ax \leq b \}
	$$
	sono a componenti intere.
\end{theorem}
dove per componenti intere intendiamo anche razionali sotto normalizzazione.

Abbiamo che le matrici $A$ dei problemi di \textbf{trasporto} e \textbf{assegnamento di costo minimo} sono totalmente unimodulari, ergo possiamo risolvere le versioni ILP di quei problemi semplicemente rimuovendo il vincolo di interezza $x \in \mathbb{Z}^n$.

\end{document}
