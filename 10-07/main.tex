
\documentclass[a4paper,11pt]{article}
\usepackage[a4paper, margin=8em]{geometry}

% usa i pacchetti per la scrittura in italiano
\usepackage[french,italian]{babel}
\usepackage[T1]{fontenc}
\usepackage[utf8]{inputenc}
\frenchspacing 

% usa i pacchetti per la formattazione matematica
\usepackage{amsmath, amssymb, amsthm, amsfonts}

% usa altri pacchetti
\usepackage{gensymb}
\usepackage{hyperref}
\usepackage{standalone}

% cose fluttuanti
\usepackage{float}

% imposta il titolo
\title{Appunti Ricerca Operativa}
\author{Luca Seggiani}
\date{2024}

% disegni
\usepackage{pgfplots}
\pgfplotsset{width=10cm,compat=1.9}

% imposta lo stile
% usa helvetica
\usepackage[scaled]{helvet}
% usa palatino
\usepackage{palatino}
% usa un font monospazio guardabile
\usepackage{lmodern}

\renewcommand{\rmdefault}{ppl}
\renewcommand{\sfdefault}{phv}
\renewcommand{\ttdefault}{lmtt}

% disponi il titolo
\makeatletter
\renewcommand{\maketitle} {
	\begin{center} 
		\begin{minipage}[t]{.8\textwidth}
			\textsf{\huge\bfseries \@title} 
		\end{minipage}%
		\begin{minipage}[t]{.2\textwidth}
			\raggedleft \vspace{-1.65em}
			\textsf{\small \@author} \vfill
			\textsf{\small \@date}
		\end{minipage}
		\par
	\end{center}

	\thispagestyle{empty}
	\pagestyle{fancy}
}
\makeatother

% disponi teoremi
\usepackage{tcolorbox}
\newtcolorbox[auto counter, number within=section]{theorem}[2][]{%
	colback=blue!10, 
	colframe=blue!40!black, 
	sharp corners=northwest,
	fonttitle=\sffamily\bfseries, 
	title=Teorema~\thetcbcounter: #2, 
	#1
}

% disponi definizioni
\newtcolorbox[auto counter, number within=section]{definition}[2][]{%
	colback=red!10,
	colframe=red!40!black,
	sharp corners=northwest,
	fonttitle=\sffamily\bfseries,
	title=Definizione~\thetcbcounter: #2,
	#1
}

% disponi problemi
\newtcolorbox[auto counter, number within=section]{problem}[2][]{%
	colback=green!10,
	colframe=green!40!black,
	sharp corners=northwest,
	fonttitle=\sffamily\bfseries,
	title=Problema~\thetcbcounter: #2,
	#1
}

% disponi codice
\usepackage{listings}
\usepackage[table]{xcolor}

\lstdefinestyle{codestyle}{
		backgroundcolor=\color{black!5}, 
		commentstyle=\color{codegreen},
		keywordstyle=\bfseries\color{magenta},
		numberstyle=\sffamily\tiny\color{black!60},
		stringstyle=\color{green!50!black},
		basicstyle=\ttfamily\footnotesize,
		breakatwhitespace=false,         
		breaklines=true,                 
		captionpos=b,                    
		keepspaces=true,                 
		numbers=left,                    
		numbersep=5pt,                  
		showspaces=false,                
		showstringspaces=false,
		showtabs=false,                  
		tabsize=2
}

\lstdefinestyle{shellstyle}{
		backgroundcolor=\color{black!5}, 
		basicstyle=\ttfamily\footnotesize\color{black}, 
		commentstyle=\color{black}, 
		keywordstyle=\color{black},
		numberstyle=\color{black!5},
		stringstyle=\color{black}, 
		showspaces=false,
		showstringspaces=false, 
		showtabs=false, 
		tabsize=2, 
		numbers=none, 
		breaklines=true
}

\lstdefinelanguage{javascript}{
	keywords={typeof, new, true, false, catch, function, return, null, catch, switch, var, if, in, while, do, else, case, break},
	keywordstyle=\color{blue}\bfseries,
	ndkeywords={class, export, boolean, throw, implements, import, this},
	ndkeywordstyle=\color{darkgray}\bfseries,
	identifierstyle=\color{black},
	sensitive=false,
	comment=[l]{//},
	morecomment=[s]{/*}{*/},
	commentstyle=\color{purple}\ttfamily,
	stringstyle=\color{red}\ttfamily,
	morestring=[b]',
	morestring=[b]"
}

% disponi sezioni
\usepackage{titlesec}

\titleformat{\section}
	{\sffamily\Large\bfseries} 
	{\thesection}{1em}{} 
\titleformat{\subsection}
	{\sffamily\large\bfseries}   
	{\thesubsection}{1em}{} 
\titleformat{\subsubsection}
	{\sffamily\normalsize\bfseries} 
	{\thesubsubsection}{1em}{}

% disponi alberi
\usepackage{forest}

\forestset{
	rectstyle/.style={
		for tree={rectangle,draw,font=\large\sffamily}
	},
	roundstyle/.style={
		for tree={circle,draw,font=\large}
	}
}

% disponi algoritmi
\usepackage{algorithm}
\usepackage{algorithmic}
\makeatletter
\renewcommand{\ALG@name}{Algoritmo}
\makeatother

% disponi numeri di pagina
\usepackage{fancyhdr}
\fancyhf{} 
\fancyfoot[L]{\sffamily{\thepage}}

\makeatletter
\fancyhead[L]{\raisebox{1ex}[0pt][0pt]{\sffamily{\@title \ \@date}}} 
\fancyhead[R]{\raisebox{1ex}[0pt][0pt]{\sffamily{\@author}}}
\makeatother

\begin{document}

% sezione (data)
\section{Lezione del 07-10-24}

% stili pagina
\thispagestyle{empty}
\pagestyle{fancy}

% testo
\subsection{Algoritmo del simplesso primale}
Supponiamo di avere un problema LP in formato primale standard con $n = 8$ vincoli, espresso come:

\[
	\begin{cases}
		\max(c^\intercal \cdot x) \\
		Ax \leq b
	\end{cases}
\]

e con poliedro:

\begin{center}
	\begin{tikzpicture}
	\begin{axis}[
			axis lines = middle,
			xlabel = {$x_A$},
			ylabel = {$x_B$},
			xmin=0, xmax=7.9,
			ymin=0, ymax=3.9,
			samples=100,
			width=13cm, height=7cm,
			legend pos=north east
		]

	\addplot[blue, thick] coordinates {(0,1) (1,0)};  
	\addplot[blue, thick] coordinates {(1,0) (2,0)};  
	\addplot[blue, thick] coordinates {(0,1) (0,2)};  
	\addplot[blue, thick] coordinates {(0,2) (1,3)};  
	\addplot[blue, thick] coordinates {(1,3) (2,3)};  
	\addplot[blue, thick] coordinates {(1,3) (2,3)};  
	\addplot[blue, thick] coordinates {(2,3) (3,2)};  
	\addplot[blue, thick] coordinates {(3,2) (3,1)};  
	\addplot[blue, thick] coordinates {(3,1) (2,0)}; 

	\node at (axis cs:0.7,0.7) [anchor=center] {1};
	\node at (axis cs:1.5,0.3) [anchor=center] {2};
	\node at (axis cs:2.3,0.7) [anchor=center] {3};
	\node at (axis cs:2.7,1.5) [anchor=center] {4};
	\node at (axis cs:2.3,2.3) [anchor=center] {5};
	\node at (axis cs:1.5,2.7) [anchor=center] {6};
	\node at (axis cs:0.7,2.3) [anchor=center] {7};
	\node at (axis cs:0.3,1.5) [anchor=center] {8};
		
	\end{axis}
	\end{tikzpicture}
\end{center}

Scegliamo un vertice di partenza, per adesso ad arbitrio: diciamo $\bar{x} = (0, 1)$ (vedremo in seguito un'algoritmo particolare per ricavare un vertice, che ci permetterà anche di determinare se il poliedro è vuoto o meno).
Ci chiediamo se questo vertice $\bar{x}$ è ottimo.
Visto che è vertice, abbiamo che per una matrice $A_B$ e un vettore $b_B$ di base:
$$
\bar{x} = A_B^{-1} b_B
$$
e che possiamo costruire il complementare duale $\bar{y}$, impostando a zero le variabili fuori base e risolvendo il sistema:
$$
\bar{y} = (cA_B^{-1}, 0)
$$
e applicare il test di ottimalità, cioè vedere se:
$$
cA_B^{-1} \geq 0
$$
ergo $\bar{y} \in D$, quindi il complementare duale esiste e il vertice è ottimo.
Se questa condizione risulta verificata, possiamo fermarci, in quanto abbiamo trovato la soluzione ottimale.

In caso contrario, avremo $\exists k \in B$ tale che $\bar{y}_k < 0$.
Dovremo quindi spostarci verso un'altro vertice, magari \textit{adiacente}, che dal punto di vista delle basi, significa cambiare un solo indice di base, conservando gli altri.
Possiamo formalizzare questa affermazione definendo un \textbf{indice uscente} $h$ ed un \textbf{indice entrante} $k$.
Sostituire un indice di base significa effettuare il cambio di base:
$$
B := B \setminus \{h\} \cup \{k\}
$$

Resta la domanda di \textit{quale} spigolo scegliere: in uno spazio vettoriale $\mathbb{R}^n$, ho a disposizione $n$ spigoli che si staccano dallo stesso vertice.
Ovviamente, vorrei scegliere uno spigolo che accresce la funzione obiettivo, e si può dimostrare che ne esiste almeno uno: altrimenti sarei già all'ottimo.
Inoltre, avendo un metodo per scegliere sempre lo spigolo di crescita maggiore potrei dire 2 cose: l'algoritmo tende all'ottimo (il vertice da cui non si staccano spigoli che accrescono la funzione obiettivo), e termina in un numero finito di passi (prima o poi raggiungerà inevitabilmente un vertice che massimizza la funzione).

Prima però dobbiamo chiarire una questione: scegliere un nuovo spigolo significa trovare 2 indici base, uno da eliminare e uno da inserire.
Si può dire che il primo indice, quello uscente, indica anche la direzione di spostamento: allentando un vincolo ci spostiamo sulla semiretta del prossimo.
Allo stesso tempo, scegliere un indice da rimuovere non basta: dobbiamo scegliere quale introdurre, che geometricamente significa capire \textit{quanto} ci possiamo spostare lungo la semiretta prima di uscire dalla regione di ammissibilità.
Vediamo quindi questi due passaggi in ordine.

\begin{itemize}
	\item \textbf{\textsf{Indice uscente}} \\
Diciamo:
$$
W = \left( -A_B^{-1} \right)
$$
e prendiamo le colonne $W^i$ corrispondenti agli indici di base scelti.

Possiamo allora dire che l'equazione degli spigoli dati dalle disequazioni all'indice $i$ sono:
$$
\bar{x} + \lambda W^i
$$

Mettiamo questa equazione nella funzione costo:
$$
c\left( \bar{x} + \lambda W^i \right) = c \bar{x} + \lambda c W^{i}
$$

Qui abbiamo $c\bar{x}$, che è il valore nel vertice, e un'altro termine scalato da $\lambda$.
Ricordiamo poi che $cA_B^{-1} = \bar{y}_B$, e che $W = \left( -A_B^{-1} \right)$, ergo $c W^{i} = -\bar{y}_B$:
$$
c \bar{x} + \lambda c W^{i} = c \bar{x} - \lambda \bar{y}_B
$$
Vogliamo quindi "allentare" l'indice (e il corrispettivo vertice) che ci dà $\bar{y}_i < 0$, in quanto è quello che restituisce un $c W^{i} > 0$, e quindi un accrescimento della funzione. 
Definiamo allora questo indice:
\begin{definition}{Indice uscente primale}
Chiamiamo indice uscente $h$, da una certa soluzione della base $B$:
$$h := \min\{ i \in B \ \text{t.c.} \ \bar{y_i} < 0  \}$$
\end{definition}
Il $\min$ significa che in caso di più $i$ negativi, si adotta la regola anticiclo (di Bland) di scegliere il primo.
In caso di nessun $i$ negativo, la complementare duale esiste e siamo sull'ottimo.

	\item \textbf{\textsf{Indice entrante}} \\
Adesso cerchiamo per quali $\lambda$ lo spigolo $\bar{x} + \lambda W^h$ resta ammissibile, ergo soddisfa:
$$
A_i \left( \bar{x} + \lambda W^h \right) \leq b_i, \quad i \in N 
$$
Questo significa effettivamente vedere qual'è il primo vincolo che "stringiamo", o che incontriamo, spostandoci lungo la semiretta ottenuta allentando il vincolo dato dall'indice uscente.

Possiamo dire:
$$
A_i \left( \bar{x} + \lambda W^h \right) = A_i \bar{x} + \lambda A_i W^h \leq b_i
$$
da cui si ricava (e si risolve) la disequazione di primo grado:
$$
\lambda A_i W^h \leq b_i - A_i \bar{x} \Rightarrow \lambda \leq \frac{b_i - A_i \bar{x}}{A_i W^h}
$$
Notiamo che se fosse $A_i W^h \leq 0, \ \forall i \in N$, avremmo che l'indice rappresenta una direzione di regressione, in quanto $\lambda \rightarrow +\infty$.
Si ha quindi che il duale non ha soluzione, e il primale $\rightarrow +\infty$.
In caso contrario, noi vogliamo trovare il primo vincolo che si va a stringere, quindi dovremo calcolare tutti gli $r_i$:
$$
r_i = \frac{b_i - A_i \bar{x}}{A_i W^h}, \quad i \in N, \quad A_i W^h > 0
$$
e scegliere l'indice che dà $\vartheta = \min(r_i)$.
Definiamo allora anche questo indice:
\begin{definition}{Indice entrante primale}
	Chiamiamo indice entrante $k$, da una certa soluzione della base $B$ e un certo indice uscente $h$:
	$$
	k := \min\{ i \in N \ \text{t.c.} \ A_i W^h > 0, \quad \frac{b_i - A_i \bar{x}}{A_i W^h} = \vartheta \}	
	$$
\end{definition}
Anche qui, il $\min$ serve a selezionare il primo indice valido, ed è una regola anticiclo (di Bland).
Notiamo due possibili situazioni:
\begin{itemize}
	\item Si potrebbero avere più $r_i$ uguali: questi rappresentano soluzioni di base degenere \textit{in arrivo}, in quanto sono più modi di arrivare allo stesso vertice stringendo vincoli diversi;
	\item Si potrebbe avere un $r_i$ nullo: questo significa che il vertice è sullo stesso vertice da dove siamo partiti, ergo rappresenta una soluzione di base degenere \textit{in partenza}.
\end{itemize}
Come prima, le regole anticiclo di Bland assicurano anche che l'algoritmo non si blocchi a ciclare su queste soluzioni degeneri.
\end{itemize}

Abbiamo quindi tutti gli strumenti necessari alla formulazione dell'algoritmo del simplesso:
\begin{algorithm}[H]
\caption{del simplesso primale}
\begin{algorithmic}
	\STATE \textbf{Input:} un problema LP in forma primale standard
	\STATE \textbf{Output:} la soluzione ottima 
	\STATE Trova una base B che genera una soluzione di base primale ammissibile
	\STATE \textsf{ciclo:}
	\STATE Calcola la soluzione di base primale $\bar{x} = A_b^{-1} b_B$ e la soluzione di base duale $\bar{y} = (cA_b^{-1}, 0)$
	\IF{$\bar{y_B} \geq 0$}
		\STATE Fermati, $\bar{x}$ è ottima per $P$ e $\bar{y}$ è ottima per $D$
\ELSE
		\STATE Calcola l'indice uscente: 
		$$
		h := \min\{ i \in B \ \text{t.c.} \ \bar{y_i} < 0 \}
		$$
		poni $W := -A_B^{-1}$ e indica con $W^h$ la $h$-esima colonna di $W$
	\ENDIF
	\IF{$A_i W^h \leq 0 \quad \forall i \in N$}
		\STATE Fermati, $P \rightarrow +\infty$ e $D$ non ha soluzione ottima
	\ELSE
		\STATE Calcola:
		$$
		\vartheta = \min\{ \frac{b_i - A_i \bar{x}}{A_i W^h} \text{t.c.} \quad i \in N, \quad A_i W^h > 0 \}
		$$
		e trova l'indice entrante: 
		$$ 
		k := \min\{ i \in N \ \text{t.c.} \ A_i W^h > 0, \quad \frac{b_i - A_i \bar{x}}{A_i W^h} = \vartheta \} 
		$$
	\ENDIF
	\STATE Aggiorna la base come:
	$$
	B := B \setminus \{h\} \cup \{k\}
	$$
	\STATE Torna a \textsf{ciclo}
\end{algorithmic}
\end{algorithm}

\end{document}
