
\documentclass[a4paper,11pt]{article}
\usepackage[a4paper, margin=8em]{geometry}

% usa i pacchetti per la scrittura in italiano
\usepackage[french,italian]{babel}
\usepackage[T1]{fontenc}
\usepackage[utf8]{inputenc}
\frenchspacing 

% usa i pacchetti per la formattazione matematica
\usepackage{amsmath, amssymb, amsthm, amsfonts}

% usa altri pacchetti
\usepackage{gensymb}
\usepackage{hyperref}
\usepackage{standalone}

% imposta il titolo
\title{Appunti Ricerca Operativa}
\author{Luca Seggiani}
\date{2024}

% disegni
\usepackage{pgfplots}
\pgfplotsset{width=10cm,compat=1.9}

% imposta lo stile
% usa helvetica
\usepackage[scaled]{helvet}
% usa palatino
\usepackage{palatino}
% usa un font monospazio guardabile
\usepackage{lmodern}

\renewcommand{\rmdefault}{ppl}
\renewcommand{\sfdefault}{phv}
\renewcommand{\ttdefault}{lmtt}

% disponi il titolo
\makeatletter
\renewcommand{\maketitle} {
	\begin{center} 
		\begin{minipage}[t]{.8\textwidth}
			\textsf{\huge\bfseries \@title} 
		\end{minipage}%
		\begin{minipage}[t]{.2\textwidth}
			\raggedleft \vspace{-1.65em}
			\textsf{\small \@author} \vfill
			\textsf{\small \@date}
		\end{minipage}
		\par
	\end{center}

	\thispagestyle{empty}
	\pagestyle{fancy}
}
\makeatother

% disponi teoremi
\usepackage{tcolorbox}
\newtcolorbox[auto counter, number within=section]{theorem}[2][]{%
	colback=blue!10, 
	colframe=blue!40!black, 
	sharp corners=northwest,
	fonttitle=\sffamily\bfseries, 
	title=Teorema~\thetcbcounter: #2, 
	#1
}

% disponi definizioni
\newtcolorbox[auto counter, number within=section]{definition}[2][]{%
	colback=red!10,
	colframe=red!40!black,
	sharp corners=northwest,
	fonttitle=\sffamily\bfseries,
	title=Definizione~\thetcbcounter: #2,
	#1
}

% disponi problemi
\newtcolorbox[auto counter, number within=section]{problem}[2][]{%
	colback=green!10,
	colframe=green!40!black,
	sharp corners=northwest,
	fonttitle=\sffamily\bfseries,
	title=Problema~\thetcbcounter: #2,
	#1
}

% disponi codice
\usepackage{listings}
\usepackage[table]{xcolor}

\lstdefinestyle{codestyle}{
		backgroundcolor=\color{black!5}, 
		commentstyle=\color{codegreen},
		keywordstyle=\bfseries\color{magenta},
		numberstyle=\sffamily\tiny\color{black!60},
		stringstyle=\color{green!50!black},
		basicstyle=\ttfamily\footnotesize,
		breakatwhitespace=false,         
		breaklines=true,                 
		captionpos=b,                    
		keepspaces=true,                 
		numbers=left,                    
		numbersep=5pt,                  
		showspaces=false,                
		showstringspaces=false,
		showtabs=false,                  
		tabsize=2
}

\lstdefinestyle{shellstyle}{
		backgroundcolor=\color{black!5}, 
		basicstyle=\ttfamily\footnotesize\color{black}, 
		commentstyle=\color{black}, 
		keywordstyle=\color{black},
		numberstyle=\color{black!5},
		stringstyle=\color{black}, 
		showspaces=false,
		showstringspaces=false, 
		showtabs=false, 
		tabsize=2, 
		numbers=none, 
		breaklines=true
}

\lstdefinelanguage{javascript}{
	keywords={typeof, new, true, false, catch, function, return, null, catch, switch, var, if, in, while, do, else, case, break},
	keywordstyle=\color{blue}\bfseries,
	ndkeywords={class, export, boolean, throw, implements, import, this},
	ndkeywordstyle=\color{darkgray}\bfseries,
	identifierstyle=\color{black},
	sensitive=false,
	comment=[l]{//},
	morecomment=[s]{/*}{*/},
	commentstyle=\color{purple}\ttfamily,
	stringstyle=\color{red}\ttfamily,
	morestring=[b]',
	morestring=[b]"
}

% disponi sezioni
\usepackage{titlesec}

\titleformat{\section}
	{\sffamily\Large\bfseries} 
	{\thesection}{1em}{} 
\titleformat{\subsection}
	{\sffamily\large\bfseries}   
	{\thesubsection}{1em}{} 
\titleformat{\subsubsection}
	{\sffamily\normalsize\bfseries} 
	{\thesubsubsection}{1em}{}

% disponi alberi
\usepackage{forest}

\forestset{
	rectstyle/.style={
		for tree={rectangle,draw,font=\large\sffamily}
	},
	roundstyle/.style={
		for tree={circle,draw,font=\large}
	}
}

% disponi algoritmi
\usepackage{algorithm}
\usepackage{algorithmic}
\makeatletter
\renewcommand{\ALG@name}{Algoritmo}
\makeatother

% disponi numeri di pagina
\usepackage{fancyhdr}
\fancyhf{} 
\fancyfoot[L]{\sffamily{\thepage}}

\makeatletter
\fancyhead[L]{\raisebox{1ex}[0pt][0pt]{\sffamily{\@title \ \@date}}} 
\fancyhead[R]{\raisebox{1ex}[0pt][0pt]{\sffamily{\@author}}}
\makeatother

\begin{document}

% sezione (data)
\section{Lezione del 14-11-24}

% stili pagina
\thispagestyle{empty}
\pagestyle{fancy}

% testo
\subsection{Simplesso per flussi capacitati}
Vediamo quindi come applicare l'algoritmo del simplesso ai problemi di flusso minimo capacitato.
Si ha che, dal teorema di Bellman, gli archi entranti saranno quelli che violano i vincoli:
\[
	\begin{cases}
		c_{ij}^\pi \geq 0, \quad \forall (i, j) \in L \\ 
		c_{ij}^\pi \leq 0, \quad \forall (i, j) \in U
	\end{cases}
\]

Il fatto che un arco entrante può appartenere sia a $L$ che a $U$ significherà che dovremo fare delle considerazioni diverse in fase  di elminazione dei cicli e in fase di inserzione dell'arco fra gli insiemi $T, L, U$ 

Troviamo quindi questo arco entrante $(p, q) \in L \cup U$, discriminando attraverso la regola anticiclo di Bland sull'ordine lessicografico degli archi. 
Vorremo eliminare il ciclo formato introducendovi $\vartheta \geq 0$ unità di flusso.

\begin{itemize}
	\item 
Nel caso $(p,q) \in L$, introdurremo le unità $\vartheta$ nella direzione \textbf{concorde} al flusso per cercare di \textit{riempire} l'arco.
	\item 
In caso contrario, con $(p,q) \in U$, introdurremo le unità $\vartheta$ nella direzione \textbf{discorde} al flusso, per cercare di \textit{svuotare} l'arco (ricordiamo che gli archi in $U$ sono quelli forzati pieni).
\end{itemize}

Scelta quindi una direzione per le unità $\vartheta$, avremo che i flussi sul ciclo si evolvono come:
$$
x_\vartheta =
	\begin{cases}
		\overline{x}_{ij} + \vartheta, \quad \forall (i, j) \in \mathcal{C}^+ \\ 
		\overline{x}_{ij} - \vartheta, \quad \forall (i, j) \in \mathcal{C}^- \\
		\overline{x}_{ij}, \quad \forall (i, j) \notin C
	\end{cases}
$$
stabilite le partizioni concordi e discordi sul ciclo $\mathcal{C}^+$ e $\mathcal{C}^-$.

Notiamo che il flusso trovato aggiungendo o rimuovendo $\vartheta$ unità di flusso, che chiamiamo $x_\vartheta$, rispetta Bellman, in quanto per un nodo posto in qualsiasi posizione fra archi concordi e discordi del ciclo, modificando i flussi secondo la regola riportata sopra non modificheremo mai il bilancio complessivo.

Questo si può dimostrare per enumerazione completa: preso un nodo $n_c$ ad arbitrio su $\mathcal{C}$, si hanno i casi:
\begin{itemize}
	\item $n_c$ si trova fra due archi discordi: uno sarà discorde e l'altro concorde alla direzione di percorrenza, ergo avremo un unità di flusso in meno da un lato e una in più dall'altro.
	\item $n_c$ si trova fra due archi concordi: questi cresceranno o diminuiranno di uno, ma in entrambi i casi l'unità in più (o in meno) di un arco dovrà essere fornita (ottenuta) dall'altro arco.
\end{itemize}

Ciò che cambierà invece saranno i costi, in quanto assunto che $c_{pq}^\pi$ è uguale al costo del ciclo, si avrà che il percorso fatto sul ciclo senza $(p,q)$ costerà la differenza di potenziale $c_{cic} = \pi_{p} - \pi_{q}$, e da $c_{pq} < c_{cic}$ sarà chiaramente più vantaggioso passare per $(p,q)$.

Avremo quindi due condizioni di arresto per la variabile $\vartheta$:
\begin{itemize}
	\item Il primo caso è che un arco in $\mathcal{C}^-$ si annulli, cioè quello che avevamo visto sul flusso non capacitato, che si applica all'arco $(r,s) \in \mathcal{C}^-$ con $x_{rs} = \vartheta$
	\item Il secondo caso è che un arco in $\mathcal{C}^+$ arrivi a capienza totale, e si applica all'arco $(r,s) \in \mathcal{C}^-$ con $u_{rs} - x_{rs} = \vartheta$.
\end{itemize}

Notiamo che dai due casi non è possibile dare una definizione univoca di $\vartheta$. Diciamo allora:
$$
\vartheta^+ = \min\{ u_{ij} - x_{ij}, \, (i, j) \in \mathcal{C}^+ \}
$$
$$
\vartheta^- = \min\{ x_{\ij}, \, (i, j) \in \mathcal{C}^- \}
$$
da cui:
$$
\vartheta = \min\{\vartheta^+, \vartheta^-\}
$$
cioè definiamo separatamente il $\vartheta^+$ del primo arco saturo su $\mathcal{C}^+$, e il $\vartheta^-$ del primo arco vuoto su $\mathcal{C}^-$.
Di conseguenza, $\vartheta$ sarà il minimo fra $\vartheta^+$ e $\vartheta^-$, cioè il primo che porta a una violazione dei vincoli.

Notiamo infine come $\vartheta$ può tendere a $\infty$: questo è il caso già visto dei cicli a costo negativo, e porta la soluzione del problema a $-\infty$.

Decideremo quindi di scegliere come arco uscente il primo arco $(r,s)$, dalla regola anticiclo di Bland sull'ordine lessicografico degli archi, che ha $u_{rs} - x_{rs} = \vartheta$ per $(r,s) \in \mathcal{C}^+$ o $x_{rs} = \vartheta$ per $(r,s) \in \mathcal{C}^-$.

Resta quindi il problema di aggiornare le partizioni $T, L, U$ sulla base degli archi $(p,q)$ entrante e $(r,s)$ uscente scelti.
Dovremo effettivamente porci due domande:
\begin{itemize}
	\item L'arco entrante $(p,q)$ è vuoto ($\in L$) o saturo ($\in U$)?
	\item L'arco uscente $(r,s)$ esce dopo essere svuotato ($\in \mathcal{C}^-$) o saturato ($\in \mathcal{C}^+$)?
\end{itemize}

Iniziamo a discriminare dalla prima domanda.
\begin{itemize}
	\item $(p, q) \in L$: 
		\begin{itemize}
			\item $(r,s) \in \mathcal{C}^-$: l'arco entrante è vuoto e l'uscente esce svuotato. Vorremo spostare l'arco entrante in $T$ e l'arco uscente in in $L$, cioè:
				$$ T = T \setminus (r,s) \cup (p,q), \quad L = L \setminus (p, q) \cup (r,s) $$
			\item $(r,s) \in \mathcal{C}^+$: l'arco entrante è vuoto e l'uscente esce saturato. Vorremo spostare l'arco entrante in $T$ e l'arco uscente in in $U$, cioè:
				$$ T = T \setminus (r,s) \cup (p,q), \quad L = L \setminus (p, q), \quad U = U \cup (r, s) $$
				Notiamo che in questa situazione può presentarsi il caso dove $(p,q) = (r,s)$, cioè l'arco $(p,q)$ entra da vuoto e esce saturato, cioè semplicemente passa da $L$ a $U$.
		\end{itemize}
	\item $(p,q) \in U$:
		\begin{itemize}
			\item $(r,s) \in \mathcal{C}^-$: l'arco entrante è pieno e l'uscente esce svuotato. Vorremo spostare l'arco entrante in $T$ e l'arco uscente in in $L$, cioè:
				$$ T = T \setminus (r,s) \cup (p,q), \quad L = L \cup (r,s), \quad U = U \setminus (p,q) $$
				Notiamo che in questa situazione può presentarsi il caso dove $(p,q) = (r,s)$, cioè l'arco $(p,q)$ entra da pieno e esce svuotato, cioè semplicemente passa da $U$ a $l$.
			\item $(r,s) \in \mathcal{C}^+$: l'arco entrante è pieno e l'uscente esce saturato. Vorremo spostare l'arco entrante in $T$ e l'arco uscente in in $U$, cioè:
				$$ T = T \setminus (r,s) \cup (p,q), \quad U = U \setminus (p, q) \cup (r,s) $$
		\end{itemize}
\end{itemize}

Possiamo quindi formulare l'algoritmo:
\begin{algorithm}[H]
\caption{del simplesso per flussi capacitati}
\begin{algorithmic}
	\STATE \textbf{Input:} un problema di flusso di costo minimo capacitato
	\STATE \textbf{Output:} la soluzione ottima 
	\STATE Trova una partizione degli archi $(T,L,U)$ con $T$ albero di copertura che genera un flusso ammissibile
	\STATE \textsf{ciclo:}
	\STATE Calcola il flusso di base capacitato:
	$$
	\bar{x} = \left(E_T^{-1} (b - E_U u_U), 0, u_U, u_T - x_T, u_L, 0\right) $$
	e il potenziale di base capacitato:
	$$
	\bar{\pi} = \left( c_T^\intercal E_T^{-1}, 0, 0, c_U^\intercal - \pi^\intercal E_U \right)
	$$
	\STATE Calcola i costi ridotti $c_{ij}^{\bar{\pi}} = c_{ij} + \bar{\pi}_{ij} - \bar{\pi}_{ij}$ per ogni arco
	\IF{Bellman è soddisfatto, cioè:
	$$
c_{ij}^{\overline{\pi}} \geq 0 \ \ \forall (i, j) \in L
$$
$$
c_{ij}^{\overline{\pi}} \leq 0 \ \ \forall (i, j) \in U
$$}
	\STATE Fermati, $\bar{x}$ è un flusso ottimo e $\bar{\pi}$ è un potenziale ottimo
\ELSE
		\STATE Calcola l'arco entrante: 
		$$
		(p, q) = \min \left\{ \{ (i, j) \in L : c_{ij}^{\bar{\pi}} < 0 \} \cup \{ (i, j) \in U : c_{ij}^{\bar{\pi}} > 0 \} \right\}
		$$
		stabilito l'ordinamento lessicografico degli archi
		\STATE Chiama $\mathcal{C}$ il ciclo che l'arco $(p, q)$ forma con gli archi in $T$
		\STATE Fissa un orientamento concorde a $(p,q)$ su $\mathcal{C}$ se $(p,q) \in L$, e discorde se $(p,q) \in U$ e partiziona $\mathcal{C}$ in $\mathcal{C^+}$ archi concordi e $\mathcal{C^-}$ archi discordi a tale orientamento
	\ENDIF
		\STATE Calcola:
		$$
		\vartheta^+ = \min\{ u_{ij} - \bar{x}_{ij} : (i, j) \in \mathcal{C}^+ \}
		$$
		$$
		\vartheta^- = \min\{ \bar{x}_{ij} : (i, j) \in \mathcal{C}^- \}
		$$
		$$
		\vartheta = \min\{\vartheta^+, \vartheta^-\}
		$$

\end{algorithmic}
\end{algorithm}
\begin{algorithm}
\begin{algorithmic}
	\IF{$\vartheta = \infty$}
		\STATE Fermati, il flusso di costo minimo ha valore $-\infty$ 
	\ELSE
		\STATE Trova l'arco uscente: 
		$$ 
		(r, s) = \min \left\{ \{(i, j) \in \mathcal{C}^+ : u_{ij} - \bar{x}_{ij} = \vartheta \} \cup \{(i, j) \in \mathcal{C}^- : \bar{x}_{ij} = \vartheta \} \right\} 
		$$
		stabilito un ordinamento lessicografico degli archi
	\ENDIF
	\STATE Aggiorna le partizioni come:
	
	\IF{$(p,q) \in L$}
		\IF{$(r,s) \in \mathcal{C}^-$}
			\STATE $ T = T \setminus (r,s) \cup (p,q), \, L = L \setminus (p, q) \cup (r,s) $
		\ELSE
			\IF{$(p,q) = (r,s)$}
				\STATE $ L = L \setminus (p, q), \, U = U \cup (p, q) $
			\ELSE
				\STATE $ T = T \setminus (r,s) \cup (p,q), \, L = L \setminus (p, q), \, U = U \cup (r, s) $
			\ENDIF
		\ENDIF
	\ELSE
		\IF{$(r,s) \in \mathcal{C}^-$}
			\IF{$(p,q) = (r,s)$}
				\STATE $ L = L \cup (p, q), \, U = U \setminus (p, q) $
			\ELSE
				\STATE $ T = T \setminus (r,s) \cup (p,q), \, L = L \cup (r,s), \, U = U \setminus (p,q) $
			\ENDIF
		\ELSE
				\STATE $ T = T \setminus (r,s) \cup (p,q), \, U = U \setminus (p, q) \cup (r,s) $
		\ENDIF
	\ENDIF
	
	\STATE Torna a \textsf{ciclo}
\end{algorithmic}
\end{algorithm}

Notiamo che le stesse ottimizzazioni che avevamo visto sul simplesso per i flussi valgono per il simplesso per i flussi capacitati.


\end{document}
