
\documentclass[a4paper,11pt]{article}
\usepackage[a4paper, margin=8em]{geometry}

% usa i pacchetti per la scrittura in italiano
\usepackage[french,italian]{babel}
\usepackage[T1]{fontenc}
\usepackage[utf8]{inputenc}
\frenchspacing 

% usa i pacchetti per la formattazione matematica
\usepackage{amsmath, amssymb, amsthm, amsfonts}

% usa altri pacchetti
\usepackage{gensymb}
\usepackage{hyperref}
\usepackage{standalone}

% imposta il titolo
\title{Appunti Ricerca Operativa}
\author{Luca Seggiani}
\date{2024}

% disegni
\usepackage{pgfplots}
\pgfplotsset{width=10cm,compat=1.9}

% imposta lo stile
% usa helvetica
\usepackage[scaled]{helvet}
% usa palatino
\usepackage{palatino}
% usa un font monospazio guardabile
\usepackage{lmodern}

\renewcommand{\rmdefault}{ppl}
\renewcommand{\sfdefault}{phv}
\renewcommand{\ttdefault}{lmtt}

% disponi il titolo
\makeatletter
\renewcommand{\maketitle} {
	\begin{center} 
		\begin{minipage}[t]{.8\textwidth}
			\textsf{\huge\bfseries \@title} 
		\end{minipage}%
		\begin{minipage}[t]{.2\textwidth}
			\raggedleft \vspace{-1.65em}
			\textsf{\small \@author} \vfill
			\textsf{\small \@date}
		\end{minipage}
		\par
	\end{center}

	\thispagestyle{empty}
	\pagestyle{fancy}
}
\makeatother

% disponi teoremi
\usepackage{tcolorbox}
\newtcolorbox[auto counter, number within=section]{theorem}[2][]{%
	colback=blue!10, 
	colframe=blue!40!black, 
	sharp corners=northwest,
	fonttitle=\sffamily\bfseries, 
	title=Teorema~\thetcbcounter: #2, 
	#1
}

% disponi definizioni
\newtcolorbox[auto counter, number within=section]{definition}[2][]{%
	colback=red!10,
	colframe=red!40!black,
	sharp corners=northwest,
	fonttitle=\sffamily\bfseries,
	title=Definizione~\thetcbcounter: #2,
	#1
}

% disponi problemi
\newtcolorbox[auto counter, number within=section]{problem}[2][]{%
	colback=green!10,
	colframe=green!40!black,
	sharp corners=northwest,
	fonttitle=\sffamily\bfseries,
	title=Problema~\thetcbcounter: #2,
	#1
}

% disponi codice
\usepackage{listings}
\usepackage[table]{xcolor}

\lstdefinestyle{codestyle}{
		backgroundcolor=\color{black!5}, 
		commentstyle=\color{codegreen},
		keywordstyle=\bfseries\color{magenta},
		numberstyle=\sffamily\tiny\color{black!60},
		stringstyle=\color{green!50!black},
		basicstyle=\ttfamily\footnotesize,
		breakatwhitespace=false,         
		breaklines=true,                 
		captionpos=b,                    
		keepspaces=true,                 
		numbers=left,                    
		numbersep=5pt,                  
		showspaces=false,                
		showstringspaces=false,
		showtabs=false,                  
		tabsize=2
}

\lstdefinestyle{shellstyle}{
		backgroundcolor=\color{black!5}, 
		basicstyle=\ttfamily\footnotesize\color{black}, 
		commentstyle=\color{black}, 
		keywordstyle=\color{black},
		numberstyle=\color{black!5},
		stringstyle=\color{black}, 
		showspaces=false,
		showstringspaces=false, 
		showtabs=false, 
		tabsize=2, 
		numbers=none, 
		breaklines=true
}

\lstdefinelanguage{javascript}{
	keywords={typeof, new, true, false, catch, function, return, null, catch, switch, var, if, in, while, do, else, case, break},
	keywordstyle=\color{blue}\bfseries,
	ndkeywords={class, export, boolean, throw, implements, import, this},
	ndkeywordstyle=\color{darkgray}\bfseries,
	identifierstyle=\color{black},
	sensitive=false,
	comment=[l]{//},
	morecomment=[s]{/*}{*/},
	commentstyle=\color{purple}\ttfamily,
	stringstyle=\color{red}\ttfamily,
	morestring=[b]',
	morestring=[b]"
}

% disponi sezioni
\usepackage{titlesec}

\titleformat{\section}
	{\sffamily\Large\bfseries} 
	{\thesection}{1em}{} 
\titleformat{\subsection}
	{\sffamily\large\bfseries}   
	{\thesubsection}{1em}{} 
\titleformat{\subsubsection}
	{\sffamily\normalsize\bfseries} 
	{\thesubsubsection}{1em}{}

% disponi alberi
\usepackage{forest}

\forestset{
	rectstyle/.style={
		for tree={rectangle,draw,font=\large\sffamily}
	},
	roundstyle/.style={
		for tree={circle,draw,font=\large}
	}
}

% disponi algoritmi
\usepackage{algorithm}
\usepackage{algorithmic}
\makeatletter
\renewcommand{\ALG@name}{Algoritmo}
\makeatother

% disponi numeri di pagina
\usepackage{fancyhdr}
\fancyhf{} 
\fancyfoot[L]{\sffamily{\thepage}}

\makeatletter
\fancyhead[L]{\raisebox{1ex}[0pt][0pt]{\sffamily{\@title \ \@date}}} 
\fancyhead[R]{\raisebox{1ex}[0pt][0pt]{\sffamily{\@author}}}
\makeatother

\begin{document}

% sezione (data)
\section{Lezione del 14-11-24}

% stili pagina
\thispagestyle{empty}
\pagestyle{fancy}

% testo
\subsection{Simplesso per flussi capacitati}
Vediamo quindi come applicare l'algoritmo del simplesso ai problemi di flusso minimo capacitato.
Si ha che, dal teorema di Bellman, gli archi entranti saranno quelli che violano i vincoli:
\[
	\begin{cases}
		c_{ij}^\pi \geq 0, \quad \forall (i, j) \in L \\ 
		c_{ij}^\pi \leq 0, \quad \forall (i, j) \in U
	\end{cases}
\]
applicando la regola anticiclo di Bland per cui prendiamo il primo fra gli archi che viola i vincoli.

Distinguiamo qui due casi: quello dove l'arco uscente è in $L$, e quello dove l'arco uscente è in $U$:
\begin{itemize}
	\item $\exists (i, j) \in L : c_{ij}^\pi > 0$: troviamo l'arco entrante $(p, q) \in L$. 
		Vorremo eliminare il ciclo formato introducendo, in direzione concorde a $(p, q)$, $\vartheta \geq 0$ unità di flusso, che come avevamo visto porta i costi degli archi sul ciclo a:
$$
x_\vartheta =
	\begin{cases}
		\overline{x}_{ij} + \vartheta, \quad \forall (i, j) \in \mathcal{C}^+ \\ 
		\overline{x}_{ij} - \vartheta, \quad \forall (i, j) \in \mathcal{C}^- \\
		\overline{x}_{ij}, \quad \forall (i, j) \notin C
	\end{cases}
$$
stabilite le partizioni concordi e discordi sul ciclo $\mathcal{C}^+$ e $\mathcal{C}^-$. 
Questo dà la solita regola sull'arco uscente: 
$$ \vartheta^- \leq \min_{(i, j) \in \mathcal{C}^-} \{ \overline{x}_{ij} \} $$

A questo punto vogliamo verificare la validità del flusso trovato, stabilendo che:
\begin{itemize}
	\item $x_\vartheta$ rispetta Bellman, in quanto, per un nodo posto in qualsiasi posizione fra archi, concordi e discordi, del ciclo, modificando i flussi secondo la regola riportata sopra non modificheremo mai il bilancio complessivo (la dimostrazione si fa per enumerazione completa # falla);
	\item I vincoli $0 \leq x_\vartheta \leq u$ sono dimostrati se scegliamo:
		$$
		\vartheta^+ = \min_{(i, j) \in C^+} \{ u_{ij} - \overline{x}_{ij} \}
		$$
		e prendiamo l'arco uscente, nel caso $\vartheta^+ < \vartheta^-$, come # boh

		Questo significa che, anzichè avere la singola regola sull'arco uscente vista per il flusso semplice, nel flusso capacitato dovremo imporre:
		$$
		\vartheta = \min \{ \vartheta^+, \vartheta^- \}, \quad \vartheta^- \leq \min_{(i, j) \in \mathcal{C}^-} \{ \overline{x}_{ij} \}, \quad \vartheta^+ = \min_{(i, j) \in C^+} \{ u_{ij} - \overline{x}_{ij} \}
		$$
		Nel caso $\vartheta = \vartheta^-$, prenderemo l'arco che $\vartheta^-$ svuota.
		In caso contrario, prenderemo l'arco che $\vartheta^+$ svuota.
\end{itemize}
	\item $\exists (i, j) \in U : c_{ij}^\pi < 0$:
\end{itemize}



\end{document}
