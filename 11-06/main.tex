
\documentclass[a4paper,11pt]{article}
\usepackage[a4paper, margin=8em]{geometry}

% usa i pacchetti per la scrittura in italiano
\usepackage[french,italian]{babel}
\usepackage[T1]{fontenc}
\usepackage[utf8]{inputenc}
\frenchspacing 

% usa i pacchetti per la formattazione matematica
\usepackage{amsmath, amssymb, amsthm, amsfonts}

% usa altri pacchetti
\usepackage{gensymb}
\usepackage{hyperref}
\usepackage{standalone}

% imposta il titolo
\title{Appunti Ricerca Operativa}
\author{Luca Seggiani}
\date{2024}

% disegni
\usepackage{pgfplots}
\pgfplotsset{width=10cm,compat=1.9}

% imposta lo stile
% usa helvetica
\usepackage[scaled]{helvet}
% usa palatino
\usepackage{palatino}
% usa un font monospazio guardabile
\usepackage{lmodern}

\renewcommand{\rmdefault}{ppl}
\renewcommand{\sfdefault}{phv}
\renewcommand{\ttdefault}{lmtt}

% disponi il titolo
\makeatletter
\renewcommand{\maketitle} {
	\begin{center} 
		\begin{minipage}[t]{.8\textwidth}
			\textsf{\huge\bfseries \@title} 
		\end{minipage}%
		\begin{minipage}[t]{.2\textwidth}
			\raggedleft \vspace{-1.65em}
			\textsf{\small \@author} \vfill
			\textsf{\small \@date}
		\end{minipage}
		\par
	\end{center}

	\thispagestyle{empty}
	\pagestyle{fancy}
}
\makeatother

% disponi teoremi
\usepackage{tcolorbox}
\newtcolorbox[auto counter, number within=section]{theorem}[2][]{%
	colback=blue!10, 
	colframe=blue!40!black, 
	sharp corners=northwest,
	fonttitle=\sffamily\bfseries, 
	title=Teorema~\thetcbcounter: #2, 
	#1
}

% disponi definizioni
\newtcolorbox[auto counter, number within=section]{definition}[2][]{%
	colback=red!10,
	colframe=red!40!black,
	sharp corners=northwest,
	fonttitle=\sffamily\bfseries,
	title=Definizione~\thetcbcounter: #2,
	#1
}

% disponi problemi
\newtcolorbox[auto counter, number within=section]{problem}[2][]{%
	colback=green!10,
	colframe=green!40!black,
	sharp corners=northwest,
	fonttitle=\sffamily\bfseries,
	title=Problema~\thetcbcounter: #2,
	#1
}

% disponi codice
\usepackage{listings}
\usepackage[table]{xcolor}

\lstdefinestyle{codestyle}{
		backgroundcolor=\color{black!5}, 
		commentstyle=\color{codegreen},
		keywordstyle=\bfseries\color{magenta},
		numberstyle=\sffamily\tiny\color{black!60},
		stringstyle=\color{green!50!black},
		basicstyle=\ttfamily\footnotesize,
		breakatwhitespace=false,         
		breaklines=true,                 
		captionpos=b,                    
		keepspaces=true,                 
		numbers=left,                    
		numbersep=5pt,                  
		showspaces=false,                
		showstringspaces=false,
		showtabs=false,                  
		tabsize=2
}

\lstdefinestyle{shellstyle}{
		backgroundcolor=\color{black!5}, 
		basicstyle=\ttfamily\footnotesize\color{black}, 
		commentstyle=\color{black}, 
		keywordstyle=\color{black},
		numberstyle=\color{black!5},
		stringstyle=\color{black}, 
		showspaces=false,
		showstringspaces=false, 
		showtabs=false, 
		tabsize=2, 
		numbers=none, 
		breaklines=true
}

\lstdefinelanguage{javascript}{
	keywords={typeof, new, true, false, catch, function, return, null, catch, switch, var, if, in, while, do, else, case, break},
	keywordstyle=\color{blue}\bfseries,
	ndkeywords={class, export, boolean, throw, implements, import, this},
	ndkeywordstyle=\color{darkgray}\bfseries,
	identifierstyle=\color{black},
	sensitive=false,
	comment=[l]{//},
	morecomment=[s]{/*}{*/},
	commentstyle=\color{purple}\ttfamily,
	stringstyle=\color{red}\ttfamily,
	morestring=[b]',
	morestring=[b]"
}

% disponi sezioni
\usepackage{titlesec}

\titleformat{\section}
	{\sffamily\Large\bfseries} 
	{\thesection}{1em}{} 
\titleformat{\subsection}
	{\sffamily\large\bfseries}   
	{\thesubsection}{1em}{} 
\titleformat{\subsubsection}
	{\sffamily\normalsize\bfseries} 
	{\thesubsubsection}{1em}{}

% disponi alberi
\usepackage{forest}

\forestset{
	rectstyle/.style={
		for tree={rectangle,draw,font=\large\sffamily}
	},
	roundstyle/.style={
		for tree={circle,draw,font=\large}
	}
}

% disponi algoritmi
\usepackage{algorithm}
\usepackage{algorithmic}
\makeatletter
\renewcommand{\ALG@name}{Algoritmo}
\makeatother

% disponi numeri di pagina
\usepackage{fancyhdr}
\fancyhf{} 
\fancyfoot[L]{\sffamily{\thepage}}

\makeatletter
\fancyhead[L]{\raisebox{1ex}[0pt][0pt]{\sffamily{\@title \ \@date}}} 
\fancyhead[R]{\raisebox{1ex}[0pt][0pt]{\sffamily{\@author}}}
\makeatother

\begin{document}

% sezione (data)
\section{Lezione del 06-11-24}

% stili pagina
\thispagestyle{empty}
\pagestyle{fancy}

% testo
\subsection{Dualità nei problemi di ottimizzazione su grafi}
Abbiamo visto come l'algoritmo del simplesso si basava sulla dualità fra un problema, espresso in \textit{forma primale standard}, e un suo \textbf{duale}, espresso in \textit{forma duale standard}.
Abbiamo inoltre detto che l'operazione duale è \textbf{involutoria}: il duale del duale è nuovamente il primale, e così via.
Si ha che questo è vero anche sui problemi di ottimizzazione su grafi, con l'unica differenza che il problema da cui partiamo è solitamente espresso in \textit{forma duale standard} (come abbiamo visto negli esempi precedenti), e che vi applicheremo l'operazione duale per ricavarne, sì un duale, ma che verrà espresso in \textit{forma primale standard}. 
Chiameremo comunque \textbf{primale} il primo problema, e \textbf{duale} il secondo.

\subsubsection{Forma della matrice dei vincoli}
Facciamo innanzitutto una precisazione sulla forma in cui si esprime la matrice $A$ (o nei problemi su grafi, $E$ di incidenza).
Sono equivalenti le forme:
$$
Ax = b \Leftrightarrow x^\intercal A^\intercal = b
$$
e quindi sui grafi:
$$
Ex = b \Leftrightarrow x^\intercal E^\intercal = b
$$

La seconda forma ha un vantaggio: la matrice così espressa risulta identica sia nel primale che nel duale, in quanto avevamo visto i vincoli duali si esprimono, almeno nei problemi di ottimizzazione visti finora, come:
$$
A^\intercal y = c
$$

\subsection{Duali di problemi di ottimizzazione sui grafi}
Veniamo quindi al calcolo vero e proprio dei problemi associati di problemi di ottimizzazione sui grafi.
A partire dal problema:
\[
	\begin{cases}
		\min c^\intercal \cdot x \\ 
		Ex = b \\ 
		x \geq 0
	\end{cases}
\]
potremo ricavare:
\[
	\begin{cases}
			\max b^\intercal \cdot \pi \\ 
			\pi^\intercal E \leq c \Leftrightarrow E^\intercal\pi \leq c
	\end{cases}
\]
dove si nota, come prima, l'equivalenza delle due forme di espressione dei vincoli.

Chiamiamo questo problema \textbf{problema dei potenziali} ai nodi del grafo.
:qa!
:qa!
Notiamo che, se nel duale avevamo rimosso una riga, dal teorema del rango $n-1$, qui potremo rimuovere una \textbf{variabile}, cioè una colonna della matrice $E^\intercal$.

Presa quindi una base $T$ (cioè un albero di copertura), avremo che il \textbf{potenziale di base} relativo a $T$ è la soluzione del sistema $ \pi^\intercal E_T =c_T^\intercal $, cioè:
$$
\pi^\intercal = c^\intercal E_T^{-1}
$$

Il potenziale di base, come il flusso di base, ha un algoritmo greedy di calcolo piuttosto intuitivo: partendo dalla \textbf{radice} dell'albero $T$, si calcola il costo complessivo sugli archi necessario a spostarsi da questa al nodo di cui vogliamo trovare il potenziale.

Dopo il calcolo dei potenziali, potremo calcolare i cosiddetti \textbf{costi ridotti}, come la differenza fra il costo di ogni arco e la \textbf{differenza di potenziale} dei nodi che collega:
$$
c^\pi_{ij} = c_{ij} + \pi_i - \pi_j
$$

Notiamo come l'unico valore che ci interessa riguardo ai potenziali non sono i potenziali stessi, ma le loro differenze. 

Una volta stabilito il vettore dei potenziali, e quindi dei costi ridotti (sul duale), si può dare la seguente caratterizzazione di vertici ammissibili:

\begin{definition}{Caratterizzazione dei potenziali di base ammissibili}
	Un potenziale di base, posto $T$ albero di copertura, è ammissibile se $\forall (i, j) \in L$ si ha $c^\pi_{ij} \geq 0$. 
\end{definition}

e degeneri, ricordando l'ortogonalità di queste due caratteristiche:

\begin{definition}{Caratterizzazione dei potenziali di base degeneri}
	Un potenziale di base, posto $T$ albero di copertura, è degenere se $\exists (i, j) \in L$ tale per cui $c^\pi_{ij} = 0$.
\end{definition}

Infine, possiamo riformulare il concetto di \textbf{scarti complementari} per soluzioni di base complementari nei problemi di ottimizzazione su grafi:
\begin{itemize}
	\item Un flusso ammissibile $x$ è ottimo se e solo se esiste un potenziale $\pi$ tale che:
		\[
			\begin{cases}
				x_{ij} = 0 \implies c^\pi_{ij} \geq 0 \\ 	
				x_{ij} > 0 \implies c^\pi_{ij} = 0 \\ 	
			\end{cases}
		\]
	\item Un potenziale ammissibile $\pi$ è ottimo se e solo se esiste un flusso $x$ tale che:
		\[
			\begin{cases}
				c^\pi_{ij} = 0 \implies x_{ij} \geq 0 \\ 
				c^\pi_{ij} > 0 \implies x_{ij} = 0
			\end{cases}
		\]
\end{itemize}

Applicando la dualità forte e le condizioni di ammissibilità riportate prima, possiamo formulare il seguente teorema:
\begin{theorem}{di Bellman}
	Una partizione $(T, L)$ di archi con $T$ albero di copertura che genera un flusso di base $x$ ammissibile è ottima se:
	$$c^\pi_{i,j} \geq 0 \quad \forall (i, j) \in L$$
\end{theorem}

Questo teorema non rivela altro che un risultato già noto: se una soluzione è ammissibile sia nel primale che nel duale, allora è ottima.

\end{document}
