
\documentclass[a4paper,11pt]{article}
\usepackage[a4paper, margin=8em]{geometry}

% usa i pacchetti per la scrittura in italiano
\usepackage[french,italian]{babel}
\usepackage[T1]{fontenc}
\usepackage[utf8]{inputenc}
\frenchspacing 

% usa i pacchetti per la formattazione matematica
\usepackage{amsmath, amssymb, amsthm, amsfonts}

% usa altri pacchetti
\usepackage{gensymb}
\usepackage{hyperref}
\usepackage{standalone}

% imposta il titolo
\title{Appunti Ricerca Operativa}
\author{Luca Seggiani}
\date{2024}

% disegni
\usepackage{pgfplots}
\pgfplotsset{width=10cm,compat=1.9}

% imposta lo stile
% usa helvetica
\usepackage[scaled]{helvet}
% usa palatino
\usepackage{palatino}
% usa un font monospazio guardabile
\usepackage{lmodern}

\renewcommand{\rmdefault}{ppl}
\renewcommand{\sfdefault}{phv}
\renewcommand{\ttdefault}{lmtt}

% disponi il titolo
\makeatletter
\renewcommand{\maketitle} {
	\begin{center} 
		\begin{minipage}[t]{.8\textwidth}
			\textsf{\huge\bfseries \@title} 
		\end{minipage}%
		\begin{minipage}[t]{.2\textwidth}
			\raggedleft \vspace{-1.65em}
			\textsf{\small \@author} \vfill
			\textsf{\small \@date}
		\end{minipage}
		\par
	\end{center}

	\thispagestyle{empty}
	\pagestyle{fancy}
}
\makeatother

% disponi teoremi
\usepackage{tcolorbox}
\newtcolorbox[auto counter, number within=section]{theorem}[2][]{%
	colback=blue!10, 
	colframe=blue!40!black, 
	sharp corners=northwest,
	fonttitle=\sffamily\bfseries, 
	title=Teorema~\thetcbcounter: #2, 
	#1
}

% disponi definizioni
\newtcolorbox[auto counter, number within=section]{definition}[2][]{%
	colback=red!10,
	colframe=red!40!black,
	sharp corners=northwest,
	fonttitle=\sffamily\bfseries,
	title=Definizione~\thetcbcounter: #2,
	#1
}

% disponi problemi
\newtcolorbox[auto counter, number within=section]{problem}[2][]{%
	colback=green!10,
	colframe=green!40!black,
	sharp corners=northwest,
	fonttitle=\sffamily\bfseries,
	title=Problema~\thetcbcounter: #2,
	#1
}

% disponi codice
\usepackage{listings}
\usepackage[table]{xcolor}

\lstdefinestyle{codestyle}{
		backgroundcolor=\color{black!5}, 
		commentstyle=\color{codegreen},
		keywordstyle=\bfseries\color{magenta},
		numberstyle=\sffamily\tiny\color{black!60},
		stringstyle=\color{green!50!black},
		basicstyle=\ttfamily\footnotesize,
		breakatwhitespace=false,         
		breaklines=true,                 
		captionpos=b,                    
		keepspaces=true,                 
		numbers=left,                    
		numbersep=5pt,                  
		showspaces=false,                
		showstringspaces=false,
		showtabs=false,                  
		tabsize=2
}

\lstdefinestyle{shellstyle}{
		backgroundcolor=\color{black!5}, 
		basicstyle=\ttfamily\footnotesize\color{black}, 
		commentstyle=\color{black}, 
		keywordstyle=\color{black},
		numberstyle=\color{black!5},
		stringstyle=\color{black}, 
		showspaces=false,
		showstringspaces=false, 
		showtabs=false, 
		tabsize=2, 
		numbers=none, 
		breaklines=true
}

\lstdefinelanguage{javascript}{
	keywords={typeof, new, true, false, catch, function, return, null, catch, switch, var, if, in, while, do, else, case, break},
	keywordstyle=\color{blue}\bfseries,
	ndkeywords={class, export, boolean, throw, implements, import, this},
	ndkeywordstyle=\color{darkgray}\bfseries,
	identifierstyle=\color{black},
	sensitive=false,
	comment=[l]{//},
	morecomment=[s]{/*}{*/},
	commentstyle=\color{purple}\ttfamily,
	stringstyle=\color{red}\ttfamily,
	morestring=[b]',
	morestring=[b]"
}

% disponi sezioni
\usepackage{titlesec}

\titleformat{\section}
	{\sffamily\Large\bfseries} 
	{\thesection}{1em}{} 
\titleformat{\subsection}
	{\sffamily\large\bfseries}   
	{\thesubsection}{1em}{} 
\titleformat{\subsubsection}
	{\sffamily\normalsize\bfseries} 
	{\thesubsubsection}{1em}{}

% disponi alberi
\usepackage{forest}

\forestset{
	rectstyle/.style={
		for tree={rectangle,draw,font=\large\sffamily}
	},
	roundstyle/.style={
		for tree={circle,draw,font=\large}
	}
}

% disponi algoritmi
\usepackage{algorithm}
\usepackage{algorithmic}
\makeatletter
\renewcommand{\ALG@name}{Algoritmo}
\makeatother

% disponi numeri di pagina
\usepackage{fancyhdr}
\fancyhf{} 
\fancyfoot[L]{\sffamily{\thepage}}

\makeatletter
\fancyhead[L]{\raisebox{1ex}[0pt][0pt]{\sffamily{\@title \ \@date}}} 
\fancyhead[R]{\raisebox{1ex}[0pt][0pt]{\sffamily{\@author}}}
\makeatother

\begin{document}

% sezione (data)
\section{Lezione del 13-11-24}

% stili pagina
\thispagestyle{empty}
\pagestyle{fancy}

% testo
\subsection{Potenziali di base capacitati}
Abbiamo visto come calcolare i flussi di base, data un'opportuna tripartizione, sulla base $T, U, T', L'$ di un problema di flusso minimo capacitato.
Vediamo adesso come calcolare i potenziali di base.
Si imposta innanzitutto il duale, cioè il \textbf{problema dei potenziali}: 
\[
	\begin{cases}
		\max b^\intercal \pi + u^\intercal \mu \\ 
		E^\intercal \pi + \mu \leq c \\ 
		\mu \leq 0
	\end{cases}
\]
dove $\mu$ rappresenta gli \textbf{scarti} ai potenziali. 

Notiamo che avremmo potuto usare la stessa matrice dei vincoli nel primale seguendo le stesse uguaglianze già viste sulle forme di matrici primali e duali:
\[
	\begin{cases}
		\min
		\begin{pmatrix}
			x & w
		\end{pmatrix}^\intercal
		\begin{pmatrix}
			c \\ 0
		\end{pmatrix} \\ 
		\begin{pmatrix}
			x & w
		\end{pmatrix}^\intercal
		\begin{pmatrix}
			E^\intercal & I \\ 
			0 & I 
		\end{pmatrix}
		=
		\begin{pmatrix}
			b & u
		\end{pmatrix}^\intercal \\ 
		\begin{pmatrix}
			x & w
		\end{pmatrix}
		\geq 0
	\end{cases}
\]

Lo stesso problema potrebbe essere stato espresso nella forma a blocchi, come avevamo visto sul primale:
\[
	\begin{cases}
		\max 
		\begin{pmatrix}
			b^\intercal & u^\intercal
		\end{pmatrix}
		\begin{pmatrix}
			\pi \\ \mu
		\end{pmatrix}\\
		\begin{pmatrix}
			E^\intercal & I \\ 
			0 & I
		\end{pmatrix}
		\begin{pmatrix}
			\pi \\ \mu
		\end{pmatrix}
			\leq
		\begin{pmatrix}
			c \\ 0
		\end{pmatrix}
	\end{cases}
\]

Vediamo quindi il calcolo vero e proprio.
Si tratta il problema come qualsiasi altro problema in formato primale, cioè si rendono valide le equazioni date dalla base $T, U, T', L'$.
Avremo quindi che $\mu_T = 0$ e $\mu_L = 0$.
A questo punto troviamo i $\pi$ dati da $E_T^\intercal \mu = c_T$, su $T$, $L$ e $U$.
Infine, abbiamo i $\mu$ su $U$, dati da $\pi_T E_U + \mu_U^\intercal = c^\intercal$.
Cioè, riassumendo, secondo la stessa notazione usata per i flussi di base (si noti che il flusso in $T$ è soluzione di un unico sistema):
$$
(\pi(T), \mu(T'. L', U')) = \left(c_T^\intercal E_T^{-1}, 0, 0, c_U^\intercal - \pi^\intercal E_U \right)
$$

Dai potenziali possiamo calcolare i costi ridotti, analogamente a come avevamo fatto sui flussi non capacitati:
$$
c^\pi_{ij} = c_{ij} - \pi_i + \pi_j
$$
e dimostrare una variante del teorema di Bellman:

\begin{theorem}{di Bellman capacitato}
	Supponiamo di avere una tripartizione $T, L, U$ che generi un flusso di base ammissibile.
	Se la soluzione è anche ottima, vale riguardo ai costi ridotti:
	\[
		\begin{cases}
			c_ {ij}^\pi, \geq 0 \quad \forall (i,j) \in L \\ 
			c_ {ij}^\pi \leq 0, \quad \forall (i,j) \in U
		\end{cases}
	\]	
\end{theorem}

\end{document}
