
\documentclass[a4paper,11pt]{article}
\usepackage[a4paper, margin=8em]{geometry}

% usa i pacchetti per la scrittura in italiano
\usepackage[french,italian]{babel}
\usepackage[T1]{fontenc}
\usepackage[utf8]{inputenc}
\frenchspacing 

% usa i pacchetti per la formattazione matematica
\usepackage{amsmath, amssymb, amsthm, amsfonts}

% usa altri pacchetti
\usepackage{gensymb}
\usepackage{hyperref}
\usepackage{standalone}

% imposta il titolo
\title{Appunti Ricerca Operativa}
\author{Luca Seggiani}
\date{2024}

% disegni
\usepackage{pgfplots}
\pgfplotsset{width=10cm,compat=1.9}

% imposta lo stile
% usa helvetica
\usepackage[scaled]{helvet}
% usa palatino
\usepackage{palatino}
% usa un font monospazio guardabile
\usepackage{lmodern}

\renewcommand{\rmdefault}{ppl}
\renewcommand{\sfdefault}{phv}
\renewcommand{\ttdefault}{lmtt}

% disponi il titolo
\makeatletter
\renewcommand{\maketitle} {
	\begin{center} 
		\begin{minipage}[t]{.8\textwidth}
			\textsf{\huge\bfseries \@title} 
		\end{minipage}%
		\begin{minipage}[t]{.2\textwidth}
			\raggedleft \vspace{-1.65em}
			\textsf{\small \@author} \vfill
			\textsf{\small \@date}
		\end{minipage}
		\par
	\end{center}

	\thispagestyle{empty}
	\pagestyle{fancy}
}
\makeatother

% disponi teoremi
\usepackage{tcolorbox}
\newtcolorbox[auto counter, number within=section]{theorem}[2][]{%
	colback=blue!10, 
	colframe=blue!40!black, 
	sharp corners=northwest,
	fonttitle=\sffamily\bfseries, 
	title=Teorema~\thetcbcounter: #2, 
	#1
}

% disponi definizioni
\newtcolorbox[auto counter, number within=section]{definition}[2][]{%
	colback=red!10,
	colframe=red!40!black,
	sharp corners=northwest,
	fonttitle=\sffamily\bfseries,
	title=Definizione~\thetcbcounter: #2,
	#1
}

% disponi problemi
\newtcolorbox[auto counter, number within=section]{problem}[2][]{%
	colback=green!10,
	colframe=green!40!black,
	sharp corners=northwest,
	fonttitle=\sffamily\bfseries,
	title=Problema~\thetcbcounter: #2,
	#1
}

% disponi codice
\usepackage{listings}
\usepackage[table]{xcolor}

\lstdefinestyle{codestyle}{
		backgroundcolor=\color{black!5}, 
		commentstyle=\color{codegreen},
		keywordstyle=\bfseries\color{magenta},
		numberstyle=\sffamily\tiny\color{black!60},
		stringstyle=\color{green!50!black},
		basicstyle=\ttfamily\footnotesize,
		breakatwhitespace=false,         
		breaklines=true,                 
		captionpos=b,                    
		keepspaces=true,                 
		numbers=left,                    
		numbersep=5pt,                  
		showspaces=false,                
		showstringspaces=false,
		showtabs=false,                  
		tabsize=2
}

\lstdefinestyle{shellstyle}{
		backgroundcolor=\color{black!5}, 
		basicstyle=\ttfamily\footnotesize\color{black}, 
		commentstyle=\color{black}, 
		keywordstyle=\color{black},
		numberstyle=\color{black!5},
		stringstyle=\color{black}, 
		showspaces=false,
		showstringspaces=false, 
		showtabs=false, 
		tabsize=2, 
		numbers=none, 
		breaklines=true
}

\lstdefinelanguage{javascript}{
	keywords={typeof, new, true, false, catch, function, return, null, catch, switch, var, if, in, while, do, else, case, break},
	keywordstyle=\color{blue}\bfseries,
	ndkeywords={class, export, boolean, throw, implements, import, this},
	ndkeywordstyle=\color{darkgray}\bfseries,
	identifierstyle=\color{black},
	sensitive=false,
	comment=[l]{//},
	morecomment=[s]{/*}{*/},
	commentstyle=\color{purple}\ttfamily,
	stringstyle=\color{red}\ttfamily,
	morestring=[b]',
	morestring=[b]"
}

% disponi sezioni
\usepackage{titlesec}

\titleformat{\section}
	{\sffamily\Large\bfseries} 
	{\thesection}{1em}{} 
\titleformat{\subsection}
	{\sffamily\large\bfseries}   
	{\thesubsection}{1em}{} 
\titleformat{\subsubsection}
	{\sffamily\normalsize\bfseries} 
	{\thesubsubsection}{1em}{}

% disponi alberi
\usepackage{forest}

\forestset{
	rectstyle/.style={
		for tree={rectangle,draw,font=\large\sffamily}
	},
	roundstyle/.style={
		for tree={circle,draw,font=\large}
	}
}

% disponi algoritmi
\usepackage{algorithm}
\usepackage{algorithmic}
\makeatletter
\renewcommand{\ALG@name}{Algoritmo}
\makeatother

% disponi numeri di pagina
\usepackage{fancyhdr}
\fancyhf{} 
\fancyfoot[L]{\sffamily{\thepage}}

\makeatletter
\fancyhead[L]{\raisebox{1ex}[0pt][0pt]{\sffamily{\@title \ \@date}}} 
\fancyhead[R]{\raisebox{1ex}[0pt][0pt]{\sffamily{\@author}}}
\makeatother

\begin{document}

% sezione (data)
\section{Lezione del 17-10-24}

% stili pagina
\thispagestyle{empty}
\pagestyle{fancy}

% testo

\subsection{Piani di taglio}

Abbiamo visto finora il problema:

\[
	\begin{cases}
		\max c^T x \\
		Ax \leq b
		x \in \mathbb{Z}^n_+
	\end{cases}
\]

Dove si può definire il poliedro $P$ del rilassato continuo:
$$
P = \{ x \in \mathbb{R}^n : Ax \leq b \} \rightarrow s_{PL}, \ x_{RC}
$$

L'insieme dei punti $\Omega$:
$$
\Omega = \{ x \in \mathbb{Z}^n : Ax \leq b \} \rightarrow v_{PLI}, \ \bar{x}
$$

E un problema di PL associato, "ristretto" sul convesso dell'insieme $\Omega$, di cui però non sapevamo caratterizzare le disequazioni (salvo alcuni casi specifici in problemi di PL sui grafi): 
$$
\mathrm{conv}(\Omega) \rightarrow v_{\mathrm{conv}(\Omega)}, \ \bar{x} 
$$

Sappiamo valere, fra queste soluzioni, la catena di diseguaglianze:
$$ v_{PLI} = v_{\mathrm{conv}(\Omega)} \leq v_{PL} $$

Decidiamo di ridurre le dimensioni dell'insieme $P$, visto che $\mathrm{conv}(\Omega) \subseteq P$, per approssimare $\mathrm{conv}(\Omega)$ stesso.
Facciamo ciò sfruttando le cosiddette \textbf{disuguaglianze valide} (DV):
\begin{definition}{Diseguaglianza valida}
	Una diseguaglianza è una DV di un'insieme $\Omega \subset \mathbb{Z}^n$ se rispetta la forma:
	$$ \gamma^T x \leq \gamma_0, \quad \forall x \in \Omega $$
\end{definition}

Abbiamo quindi che una diseguaglianza è valida se contiene $\Omega$ completamente nel semispazio che definisce.
Si può poi dimostrare il fatto piuttosto ovvio:

\begin{theorem}{Diseguaglianza valida dell'involucro convesso}	
una diseguaglianza valida per un'insieme $\Omega$ lo è anche per $\mathrm{conv}(\Omega)$.
\end{theorem}

Una tecnica di base per il calcolo delle diseguaglianze valide è quella dell'approssimazione per difetto delle diseguaglianze che definiscono $P$.
Questo però è poco utile nel caso di problemi già definiti con diseguaglianze in componenti intere.
Introduciamo quindi l'idea di un \textbf{piano di taglio}:

\begin{definition}{Piano di taglio}	
Su un problema di $PLI$ con soluzione $x_{RC}$ al rilassato continuo, una DV in forma:
$$ \gamma x \leq \gamma_0, \quad \forall x \in S \ \ : \ \ \gamma x_{RC} > \gamma_0 $$
si dice \textbf{piano di taglio}.
\end{definition}

L'idea dei piani di taglio è quella di rimuovere successivamente soluzioni di $P$ fuori da $\mathrm{conv}(\Omega)$.
Supponiamo di avere il problema:
\[
	\begin{cases}			
		\max c^T x \\ 
		Ax \leq b \\ 
		\gamma x \leq \gamma_0
	\end{cases}
	\rightarrow V_{P_{n}}
\]

Vogliamo ricavare un poliedro $P_n$, che contiene l'ottimo $v_{PLI}$, ma che è più piccolo del rilassato continuo:
$$
v_{PLI} \leq v_{P_n} \leq v_{PL}
$$

Per fare ciò, prendiamo una DV che non contiene $x_{RC}$ soluzione del rilassato continuo. 
A questo punto si avrà che, se il problema di $v_{P_n}$ ha soluzione a componenti intere, allora coinciderà con $v_{\mathrm{conv}(\Omega)}$, altrimenti si dovranno ripetere i passi precedenti.

Abbiamo quindi che qualsiasi problema trovato iterativamente coi piani di taglio è una riduzione della regione ammissibile.
Il sottoinsieme dei piani di taglio che studieremo è quello dei \textbf{piani di taglio di Gomory}.

\subsubsection{Costruzione dei piani di taglio di Gomory}
Prendiamo un problema di ILP in forma:
\[
	\begin{cases}
		\max c^T x \\ 
		Ax = b \\ 
		x \geq 0 \\ 
		x \in \mathbb{Z}^n
	\end{cases}
\]
cioè in una forma simile al duale standard.
Possiamo convertire qualsiasi problema in questa forma introducendo eventuali variabili di surplus per portare le disequazioni in equazioni.
Ad esempio, sul problema presentato un paio di lezioni fa:
\[
	\begin{cases}
		\max (0.3x_1 + 0.4x_2) \\ 
		3 x_1 + 5 x_2 \leq 15 \\ 
		4 x_1 + 4 x_2 \leq 16 \\ 
		x_i \geq 0 \\ 
		x \in \mathbb{Z}^2
	\end{cases}
\]
si ricava:
\[
	\begin{cases}
		\max (0.3x_1 + 0.4x_2) \\ 
		3 x_1 + 5 x_2 + x_3 = 15 \\ 
		4 x_1 + 4 x_2 + x_4 = 16 \\ 
		x_i \geq 0 \\ 
		x \in \mathbb{Z}^2
	\end{cases}
\]

Dalla soluzione del rilassato continuo $x_{RC}$, abbiamo una base ottima $B$, tale per cui si può porre:
$$
A = (A_B \  A_N), \quad x_{RC} = \binom{x_B}{x_N}
$$
cioè, $A_B$ sono le colonne di matrice corrispondenti alla base e $A_N$ le altre, e $x_B$ sono le variabili di base e $x_N$ le altre. 

Definiamo allora la matrice $\tilde{A}$:
$$ \tilde{A} = A_B^{-1} A_N $$

Prendiamo quindi un indice $r \in B$, tale che $x_{RC}$ a $r$ ha una componente non intera.
Possiamo a questo punto enunciare il teorema:

\begin{theorem}{Teorema di Gomory}
	Si ha che:
	$$
		\sum_{j \in N} \{ \tilde{a}_{rj} \} x_j \geq \{ ( x_{B} )_r \}
	$$
	è un piano di taglio.
\end{theorem}

Occorre fare qualche chiarimento sulla notazione: le graffe rappresentano l'operatore componente frazionaria, mentre l'$r$ a pedice di $x_{B}$ indica di prendere la componente all'indice $r$, cioè quella non intera.

Applichiamo quindi questo metodo al problema di prima.
Avremo che:
$$
A= \begin{pmatrix}
	3 & 5 & 1 & 0 \\ 4 & 4 & 0 & 1
\end{pmatrix}, \quad 
b= \begin{pmatrix}
15 \\ 16
\end{pmatrix}
$$
e la soluzione ottima $x_{RC} = \left( \frac{5}{2}, \frac{3}{2} \right)$ sulla base ottima $B = \{1,2\}$.
Abbiamo quindi che:
$$
A_B = \begin{pmatrix}
	3 & 5 \\ 4 & 4 
\end{pmatrix}, \quad 
A_N = \begin{pmatrix}
	1 & 0 \\ 0 & 1
\end{pmatrix}, \quad 
x_B = \left( \frac{5}{2}, \frac{3}{2} \right), \quad 
x_N = ( 0, 0 )
$$

e la matrice $\tilde{A}$ è:
$$
\tilde{A} = A_B^{-1} A_N = A_B^{-1} = \begin{pmatrix}
	-\frac{1}{2} & \frac{5}{8} \\ \frac{1}{2} & -\frac{3}{8}
\end{pmatrix} 
$$

Notiamo che in $x_B$ entrambi i componenti sono non interi, ergo possiamo prendere i due casi:
\begin{itemize}
	\item $r=1$, si ha $\left\{ -\frac{1}{2}  \right\} x_3 + \left\{ \frac{5}{8} \right\} x_4 \geq \left\{ \frac{5}{2} \right\}$. Applicando l'operatore parte frazionaria, e sostituendo a $x_3$ e $x_4$ le loro espressioni in funzione di $x_1$ e $x_2$ (ricavate dalle equazioni del poliedro) si ha:
		$$ 4x_1 + 5x_2 \leq 17 $$
	\item $r=2$, si ha con calcoli simili, lo stesso taglio:
		$$ 4x_1 + 5x_2 \leq 17 $$
\end{itemize}

Quindi entrambe le componenti danno lo stesso taglio (si noti che questa non è la norma).
Possiamo visualizzare sul grafico cosa significa il taglio introdotto:

\begin{center}

\begin{tikzpicture}
\begin{axis}[
    axis lines = middle,
		grid=both,
    xlabel = {$x_1$},
    ylabel = {$x_2$},
    xmin=0, xmax=6.9,
    ymin=0, ymax=3.9,
    domain=0:10,
    samples=100,
    width=15cm, height=9cm,
    legend pos=north east
  ]

% regione ammissibile

\addplot[fill=gray, opacity=0.4, forget plot] 
	coordinates {
		(0,0)
		(0,3)
		(2.5, 1.5)
		(4, 0)
	};

% rette

\addplot[domain=0:2.5, thick, red] {3 - 0.6*x};
\addlegendentry{$ 3 x_1 + 5 x_2 \leq 15 $}

\addplot[domain=2.5:4, thick, blue] {4 - x};
\addlegendentry{$ 4 x_1 + 4 x_2 \leq 16 $}

\addplot[domain=0:10, thick, green] {3.4 - 0.8*x};
\addlegendentry{$ 4 x_1 + 5 x_2 \leq 17 $}

\addplot[
    only marks,  % Only marks the points without connecting lines
    mark=*,
    color=violet,
    mark size=2pt
] coordinates {
	(0,0)
	(0,1)
	(0,2)
	(0,3)
	(1,0)
	(1,1)
	(1,2)
	(2,0)
	(2,1)
	(3,0)
	(3,1)
	(4,0)
};
\addlegendentry{$\Omega$}

\addplot[
	only marks,
	mark=*,
	mark size=2pt
] coordinates {
	(2.5, 1.5)
};

\node at (axis cs:2.5, 1.5) [above right] {$x_{V_S}$};

\addplot[
	only marks,
	mark=*,
	mark size=2pt
] coordinates {
	(3, 1)
};

\node at (axis cs:3, 1) [above right] {$x_{\mathrm{conv}(\Omega)}$};

\end{axis}
\end{tikzpicture}

\end{center}

Notiamo che il taglio, se non ha ridotto completamente il poliedro a $\mathrm{conv}(\Omega)$, ha portato la soluzione ottima a quella di $\mathrm{conv}(\Omega)$.

\end{document}
