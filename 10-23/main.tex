
\documentclass[a4paper,11pt]{article}
\usepackage[a4paper, margin=8em]{geometry}

% usa i pacchetti per la scrittura in italiano
\usepackage[french,italian]{babel}
\usepackage[T1]{fontenc}
\usepackage[utf8]{inputenc}
\frenchspacing 

% usa i pacchetti per la formattazione matematica
\usepackage{amsmath, amssymb, amsthm, amsfonts}

% usa altri pacchetti
\usepackage{gensymb}
\usepackage{hyperref}
\usepackage{standalone}

% imposta il titolo
\title{Appunti Ricerca Operativa}
\author{Luca Seggiani}
\date{2024}

% disegni
\usepackage{pgfplots}
\pgfplotsset{width=10cm,compat=1.9}

% imposta lo stile
% usa helvetica
\usepackage[scaled]{helvet}
% usa palatino
\usepackage{palatino}
% usa un font monospazio guardabile
\usepackage{lmodern}

\renewcommand{\rmdefault}{ppl}
\renewcommand{\sfdefault}{phv}
\renewcommand{\ttdefault}{lmtt}

% disponi il titolo
\makeatletter
\renewcommand{\maketitle} {
	\begin{center} 
		\begin{minipage}[t]{.8\textwidth}
			\textsf{\huge\bfseries \@title} 
		\end{minipage}%
		\begin{minipage}[t]{.2\textwidth}
			\raggedleft \vspace{-1.65em}
			\textsf{\small \@author} \vfill
			\textsf{\small \@date}
		\end{minipage}
		\par
	\end{center}

	\thispagestyle{empty}
	\pagestyle{fancy}
}
\makeatother

% disponi teoremi
\usepackage{tcolorbox}
\newtcolorbox[auto counter, number within=section]{theorem}[2][]{%
	colback=blue!10, 
	colframe=blue!40!black, 
	sharp corners=northwest,
	fonttitle=\sffamily\bfseries, 
	title=Teorema~\thetcbcounter: #2, 
	#1
}

% disponi definizioni
\newtcolorbox[auto counter, number within=section]{definition}[2][]{%
	colback=red!10,
	colframe=red!40!black,
	sharp corners=northwest,
	fonttitle=\sffamily\bfseries,
	title=Definizione~\thetcbcounter: #2,
	#1
}

% disponi problemi
\newtcolorbox[auto counter, number within=section]{problem}[2][]{%
	colback=green!10,
	colframe=green!40!black,
	sharp corners=northwest,
	fonttitle=\sffamily\bfseries,
	title=Problema~\thetcbcounter: #2,
	#1
}

% disponi codice
\usepackage{listings}
\usepackage[table]{xcolor}

\lstdefinestyle{codestyle}{
		backgroundcolor=\color{black!5}, 
		commentstyle=\color{codegreen},
		keywordstyle=\bfseries\color{magenta},
		numberstyle=\sffamily\tiny\color{black!60},
		stringstyle=\color{green!50!black},
		basicstyle=\ttfamily\footnotesize,
		breakatwhitespace=false,         
		breaklines=true,                 
		captionpos=b,                    
		keepspaces=true,                 
		numbers=left,                    
		numbersep=5pt,                  
		showspaces=false,                
		showstringspaces=false,
		showtabs=false,                  
		tabsize=2
}

\lstdefinestyle{shellstyle}{
		backgroundcolor=\color{black!5}, 
		basicstyle=\ttfamily\footnotesize\color{black}, 
		commentstyle=\color{black}, 
		keywordstyle=\color{black},
		numberstyle=\color{black!5},
		stringstyle=\color{black}, 
		showspaces=false,
		showstringspaces=false, 
		showtabs=false, 
		tabsize=2, 
		numbers=none, 
		breaklines=true
}

\lstdefinelanguage{javascript}{
	keywords={typeof, new, true, false, catch, function, return, null, catch, switch, var, if, in, while, do, else, case, break},
	keywordstyle=\color{blue}\bfseries,
	ndkeywords={class, export, boolean, throw, implements, import, this},
	ndkeywordstyle=\color{darkgray}\bfseries,
	identifierstyle=\color{black},
	sensitive=false,
	comment=[l]{//},
	morecomment=[s]{/*}{*/},
	commentstyle=\color{purple}\ttfamily,
	stringstyle=\color{red}\ttfamily,
	morestring=[b]',
	morestring=[b]"
}

% disponi sezioni
\usepackage{titlesec}

\titleformat{\section}
	{\sffamily\Large\bfseries} 
	{\thesection}{1em}{} 
\titleformat{\subsection}
	{\sffamily\large\bfseries}   
	{\thesubsection}{1em}{} 
\titleformat{\subsubsection}
	{\sffamily\normalsize\bfseries} 
	{\thesubsubsection}{1em}{}

% disponi alberi
\usepackage{forest}

\forestset{
	rectstyle/.style={
		for tree={rectangle,draw,font=\large\sffamily}
	},
	roundstyle/.style={
		for tree={circle,draw,font=\large}
	}
}

% disponi algoritmi
\usepackage{algorithm}
\usepackage{algorithmic}
\makeatletter
\renewcommand{\ALG@name}{Algoritmo}
\makeatother

% disponi numeri di pagina
\usepackage{fancyhdr}
\fancyhf{} 
\fancyfoot[L]{\sffamily{\thepage}}

\makeatletter
\fancyhead[L]{\raisebox{1ex}[0pt][0pt]{\sffamily{\@title \ \@date}}} 
\fancyhead[R]{\raisebox{1ex}[0pt][0pt]{\sffamily{\@author}}}
\makeatother

\begin{document}

% sezione (data)
\section{Lezione del 23-10-24}

% stili pagina
\thispagestyle{empty}
\pagestyle{fancy}

% testo
Prendiamo in considerazione il poliedro generico di ILP:

\[
	\begin{cases}
		Ax \leq b \\ 
		x \in \mathbb{Z}^n
	\end{cases}
\]

Facciamo un riassunto sulle tecniche greedy per il ricavo di una soluzione ammissibile (che danno un limite inferiore $v_I$) e sul ricavo del rilassato (che dà un limite superiore $v_S$), con riferimento alla ILP.
Bisogna notare che un algoritmo greedy da vincolo inferiore su problemi di massimo, e superiore altrimenti.
Allo stesso modo, il rilassato dà vincolo superiore su problemi di massimo, e inferiore altrimenti.

\subsubsection{Algoritmi per valutazioni inferiori}
\begin{itemize}
	\item \textbf{Zaino:} abbiamo visto gli \textbf{algoritmi dei rendimenti} per il caricamento di problemi di zaino in ILP, cioè per uno zaino di vettore valore $v$ e peso $p$, con peso massimo $P$, si calcolano i rendimenti:
		$$
		r_i = \frac{v_i}{p_i}
		$$
		e si selezionano uno o più elementi di rendimento massimo (dipende anche dal tipo booleano o meno di problema), saturando fino a frazione se si è nel rilassato, e fino a intero in caso contrario;
	\item \textbf{Bin-packing:} per la ricerca di soluzioni ammissibili dei problemi di bin-packing abbiamo visto tre algoritmi di ricerca, \textbf{next-fit decreasing}, \textbf{first-fit decreasing} e \textbf{best-fit decreasing};
	\item \textbf{Algoritmo delle toppe:} l'algoritmo delle toppe per i problemi TSP asimmetrici rappresenta ancora un metodo euristico per il calcolo di cicli hamiltoniani a costo minimo. 
\end{itemize}

\subsubsection{Algoritmi per valutazioni superiori}
Abbiamo introdotto il concetto di rilassamento di vincoli, in modo da trovare un sovrainsieme della regione ammissibile di ILP da dove è più facile ricavare una soluzione ammissibile.
In particolare, la tecnica più immediata è quella di rimuovere il vincolo di interezza ($x \in \mathbb{Z}^n$) o booleano ($x \in \{0, 1\}^n$).


\subsection{TSP simmetrico}
Veniamo quindi ai problemi di TSP simmetrico, cioè dove la tabella dei costi è simmetrica, ovvero in forma:
$$
c =
\begin{pmatrix}
	27 & 23 & 24 & 26 \\ 
	- & 21 & 32 & 33 \\ 
	- & - & 41 & 42 \\ 
	- & - & - & 47
\end{pmatrix}
$$
dove non si riporta la parte in basso a sinistra in quanto è la specchiata della matrice triangolare in alto a destra.

Siccome questo è un caso particolare del TSP, possiamo effettivamente usare tutti i metodi studiati per il TSP asimmetrico.
Vediamo però che ci sono delle semplificazioni particolari che possiamo fare sul problema.

Innanzitutto, visto che possiamo prendere gli archi come non orientati, le variabili possibili sono dimezzano.
Se nello scorso problema avevamo $5 \cdot 5 = 25$ variabili meno $5$ della diagonale, ergo $n^2 - n$, adesso ne abbiamo soltanto $\frac{n^2 - n}{2}$. 

Abbiamo poi che i vincoli si presentano in forma:
\[
	\begin{cases}
		\min c^T x \\ 
		\sum\limits_{h,i \in A} x_{hi} + \sum\limits_{i,k \in A} = 2, \quad \forall i \in N \\
		\sum\limits_{i \in S, \, j \notin S} x_{ij} + \sum\limits_{i \notin S, \, j \in S} x_{ij} \geq 1, \quad \forall S \subset N, \quad 2 \leq |S| \leq \left\lceil \frac{|N|}{2} \right\rceil
	\end{cases}
\]
La prima serie di vincoli rappresenta il fatto che ogni nodo è parte di al massimo due archi, ergo uno uscente e entrante (anche se questa definizione non ha molto significato per grafi non orientati). 
La seconda serie rappresenta invece i vincoli di connessione, che in questo caso prendiamo in entrambe le direzioni, e che limitiamo in dimensioni di sottoinsieme alla cardinalità $|N|$ fratto 2, al limite approssimata all'eccesso, in quanto il vincolo per $S$ vale anche per $N \setminus S$. 
Anche in questo caso si evitano gli insiemi singoletti con $|S| = 1$, in quanto rappresentano i vincoli già espressi nella seconda riga.

\subsubsection{Valutazioni inferiori e superiori}

Un'approccio greedy alla soluzione, che fornisce un vincolo superiore $v_S$ (siamo in minimo), è quello di scegliere un nodo ad arbitrio e proseguire da lì in poi scegliendo il nodo con $\min(c_{xj})$ di adiacenza, ovvero:

\begin{algorithm}[H]
\caption{del nodo vicino}
\begin{algorithmic}
	\STATE \textbf{Input:} un insieme $N$ di nodi % input
	\STATE \textbf{Output:} una valutazione superiore $v_S$ del TSP simmetrico % output
	% body
	\STATE Parti da un nodo ad arbitrio;
	\STATE \textsf{ciclo:}
	\STATE Scegli il nodo adiacente più vicino e unifica
	\IF{ci sono altri nodi}
	\STATE Torna a \textsf{ciclo}
	\ENDIF
\end{algorithmic}
\end{algorithm}

Per il calcolo di una valutazione inferiore avremo bisogno di un rilassamento.
Scegliamo di rilassare i vincoli sul grado di tutti i nodi tranne uno, detto $k$: 
\[
	\begin{cases}
		\min c^T x \\ 
		\sum\limits_{h,r \in A} x_{hi} + \sum\limits_{r,k \in A} = 2, \quad \forall i \in N \\
		\sum\limits_{i \in S, \, j \notin S} x_{ij} + \sum\limits_{i \notin S, \, j \in S} x_{ij} \geq 1, \quad \forall S \subset N \setminus \{r\}, \quad 2 \leq |S| \leq \left\lceil \frac{|N|}{2} \right\rceil
	\end{cases}
\]

Diamo quindi la seguente definizione:
\begin{definition}{K-albero}
	Scelto un nodo $k$, chiamiamo $k$-albero un insieme di $n$ archi dove $n-2$ archi formano un albero di copertura sul sottografo formato dai nodi $N \setminus \{k\}$, e 2 archi incidono sul nodo $k$.	
\end{definition}
In sostanza, un $k$-albero è un \textbf{albero di copertura} con un ciclo.
Bisogna notare che questa definizione di $k$-albero differisce da quella comunemente usata sia in informatica (albero con ramificazione $k$) che in teoria dei grafi (un altro tipo di grafo non diretto).
Inoltre, si trova anche sotto altri nomi, tra cui $r$-albero, che però è ancora conflitto con gli $r$-grafi informatici (che sono un tipo di albero per l'organizzazione di informazione spaziale).

Tolta quest'ultima precisazione, abbiamo che ciò che vogliamo è un $k$-albero minimo.
Ricordiamo quindi il seguente algoritmo:
\subsubsection{Algoritmo di Kruskal}
Possiamo usare l'\textbf{algoritmo di Kruskal} per trovare l'albero di copertura minimale per un insieme $N$ di nodi.
Assumendo grafi connessi, si può formulare come:

\begin{algorithm}[H]
\caption{di Kruskal}
\begin{algorithmic}
	\STATE \textbf{Input:} un insieme $N$ di nodi % input
	\STATE \textbf{Output:} un albero di copertura minimale di $N$% output
	% body
	\STATE \textsf{ciclo:}
	\STATE Ordina gli archi in maniera crescente rispetto al peso
	\STATE Scegli il primo arco. 
	\IF{aggiungere l'arco non crea cicli} 
		\STATE Aggiungi l'arco e unifica le componenti
	\ELSE		
		\STATE Scegli il prossimo arco e riprova
	\ENDIF
	\IF{non hai finito gli archi} 
		\STATE Vai a \textsf{ciclo}
	\ENDIF
\end{algorithmic}
\end{algorithm}
Questo algoritmo trova sempre una soluzione, cioè un albero di copertura minimale, e visto che sceglie sempre archi a costo minimo, trova sempre la soluzione ottima.
\par\medskip
Ritornando al problema dei $k$-alberi minimi, possiamo finalmente presentare il seguente algoritmo per il calcolo di un $k$-albero di costo minimo su qualsiasi nodo $k$:
\begin{algorithm}
\caption{del $k$-albero}
\begin{algorithmic}
	\STATE \textbf{Input:} un insieme $N$ nodi, di cui $k \in N$  % input
	\STATE \textbf{Output:} un $k$-albero di costo minimo % output
	% body
	\STATE Trova l'albero di copertura minimale $C_H$ con Kruskal del sottografo $N \setminus \{k\}$
	\STATE Connetti $k$ a $C_H$ attraverso i due nodi a costo minimo
\end{algorithmic}
\end{algorithm}

Possiamo quindi dire perchè abbiamo introdotto il concetto di $k$-albero: il problema rilassato posto prima, dove si erano rilassate le cardinalità degli archi su tutti i nodi tranne $k$, ha come soluzione il $k$-albero a costo minimo.
Trovando questo $k$-albero, abbiamo una valutazione superiore del problema di TSP simmetrico di partenza.

\end{document}
