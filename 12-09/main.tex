
\documentclass[a4paper,11pt]{article}
\usepackage[a4paper, margin=8em]{geometry}

% usa i pacchetti per la scrittura in italiano
\usepackage[french,italian]{babel}
\usepackage[T1]{fontenc}
\usepackage[utf8]{inputenc}
\frenchspacing 

% usa i pacchetti per la formattazione matematica
\usepackage{amsmath, amssymb, amsthm, amsfonts}

% usa altri pacchetti
\usepackage{gensymb}
\usepackage{hyperref}
\usepackage{standalone}

% imposta il titolo
\title{Appunti Ricerca Operativa}
\author{Luca Seggiani}
\date{2024}

% disegni
\usepackage{pgfplots}
\pgfplotsset{width=10cm,compat=1.9}

% imposta lo stile
% usa helvetica
\usepackage[scaled]{helvet}
% usa palatino
\usepackage{palatino}
% usa un font monospazio guardabile
\usepackage{lmodern}

\renewcommand{\rmdefault}{ppl}
\renewcommand{\sfdefault}{phv}
\renewcommand{\ttdefault}{lmtt}

% disponi il titolo
\makeatletter
\renewcommand{\maketitle} {
	\begin{center} 
		\begin{minipage}[t]{.8\textwidth}
			\textsf{\huge\bfseries \@title} 
		\end{minipage}%
		\begin{minipage}[t]{.2\textwidth}
			\raggedleft \vspace{-1.65em}
			\textsf{\small \@author} \vfill
			\textsf{\small \@date}
		\end{minipage}
		\par
	\end{center}

	\thispagestyle{empty}
	\pagestyle{fancy}
}
\makeatother

% disponi teoremi
\usepackage{tcolorbox}
\newtcolorbox[auto counter, number within=section]{theorem}[2][]{%
	colback=blue!10, 
	colframe=blue!40!black, 
	sharp corners=northwest,
	fonttitle=\sffamily\bfseries, 
	title=Teorema~\thetcbcounter: #2, 
	#1
}

% disponi definizioni
\newtcolorbox[auto counter, number within=section]{definition}[2][]{%
	colback=red!10,
	colframe=red!40!black,
	sharp corners=northwest,
	fonttitle=\sffamily\bfseries,
	title=Definizione~\thetcbcounter: #2,
	#1
}

% disponi problemi
\newtcolorbox[auto counter, number within=section]{problem}[2][]{%
	colback=green!10,
	colframe=green!40!black,
	sharp corners=northwest,
	fonttitle=\sffamily\bfseries,
	title=Problema~\thetcbcounter: #2,
	#1
}

% disponi codice
\usepackage{listings}
\usepackage[table]{xcolor}

\lstdefinestyle{codestyle}{
		backgroundcolor=\color{black!5}, 
		commentstyle=\color{codegreen},
		keywordstyle=\bfseries\color{magenta},
		numberstyle=\sffamily\tiny\color{black!60},
		stringstyle=\color{green!50!black},
		basicstyle=\ttfamily\footnotesize,
		breakatwhitespace=false,         
		breaklines=true,                 
		captionpos=b,                    
		keepspaces=true,                 
		numbers=left,                    
		numbersep=5pt,                  
		showspaces=false,                
		showstringspaces=false,
		showtabs=false,                  
		tabsize=2
}

\lstdefinestyle{shellstyle}{
		backgroundcolor=\color{black!5}, 
		basicstyle=\ttfamily\footnotesize\color{black}, 
		commentstyle=\color{black}, 
		keywordstyle=\color{black},
		numberstyle=\color{black!5},
		stringstyle=\color{black}, 
		showspaces=false,
		showstringspaces=false, 
		showtabs=false, 
		tabsize=2, 
		numbers=none, 
		breaklines=true
}

\lstdefinelanguage{javascript}{
	keywords={typeof, new, true, false, catch, function, return, null, catch, switch, var, if, in, while, do, else, case, break},
	keywordstyle=\color{blue}\bfseries,
	ndkeywords={class, export, boolean, throw, implements, import, this},
	ndkeywordstyle=\color{darkgray}\bfseries,
	identifierstyle=\color{black},
	sensitive=false,
	comment=[l]{//},
	morecomment=[s]{/*}{*/},
	commentstyle=\color{purple}\ttfamily,
	stringstyle=\color{red}\ttfamily,
	morestring=[b]',
	morestring=[b]"
}

% disponi sezioni
\usepackage{titlesec}

\titleformat{\section}
	{\sffamily\Large\bfseries} 
	{\thesection}{1em}{} 
\titleformat{\subsection}
	{\sffamily\large\bfseries}   
	{\thesubsection}{1em}{} 
\titleformat{\subsubsection}
	{\sffamily\normalsize\bfseries} 
	{\thesubsubsection}{1em}{}

% disponi alberi
\usepackage{forest}

\forestset{
	rectstyle/.style={
		for tree={rectangle,draw,font=\large\sffamily}
	},
	roundstyle/.style={
		for tree={circle,draw,font=\large}
	}
}

% disponi algoritmi
\usepackage{algorithm}
\usepackage{algorithmic}
\makeatletter
\renewcommand{\ALG@name}{Algoritmo}
\makeatother

% disponi numeri di pagina
\usepackage{fancyhdr}
\fancyhf{} 
\fancyfoot[L]{\sffamily{\thepage}}

\makeatletter
\fancyhead[L]{\raisebox{1ex}[0pt][0pt]{\sffamily{\@title \ \@date}}} 
\fancyhead[R]{\raisebox{1ex}[0pt][0pt]{\sffamily{\@author}}}
\makeatother

\begin{document}

% sezione (data)
\section{Lezione del 09-12-24}

% stili pagina
\thispagestyle{empty}
\pagestyle{fancy}

% testo
\subsection{Problema di ottimizzazione quadratico}
Vediamo un problema di esempio sull'ottimizzazione quadratica.
Sia la funzione:
$$
f(x_1, x_2) = 2x_1^2 + 4x_2^2 + 4x_1 - x_2
$$
sul poliedro $P$ di vertici:
$$
V = \left\{ (0, 1), (4, 1), (4, 3), (-1, 5) \right\}
$$
Ci interessiamo a trovare massimi e minimi globali.

\begin{itemize}
	\item \textbf{\textsf{Massimi}} \\
		La funzione è concava e definita su un poliedro, quindi i possibili massimi non possono che stare sui vertici.
		Si calcola allora per enumerazione diretta:
		\[
			\begin{cases}		
				f(0,1) = 3 \\ 
				f(4,1) = 51 \\ 
				f(4,3) = 81 \\
				f(-1, 5) = 93
			\end{cases}
		\]
		da cui risulta chiaro il massimo globale in $(-1, 5)$, che è anche unico.
	\item \textbf{\textsf{Minimi}} \\
		Per il calcolo dei minimi non possiamo sfruttare la concavità della funzione e la convessità del poliedro, in quanto potremmo avere minimi in \textbf{punti interni} o \textbf{punti sul bordo}.

		Il vertice del paraboloide si trova come $\left(\frac{c}{2a}, \frac{d}{2b}\right) = \left( -1, -\frac{1}{8} \right)$. che non ricade in $P$, ergo siamo nel secondo caso.

		Si presentano quindi due metodi di risoluzione: uno per via geometrica, e uno che sfrutta il sistema LKT.
		\begin{enumerate}
			\item \textit{(via grafica)} La $f$ definisce un paraboloide concavo.
				In generale, il paraboloide avrà un punto di minimo globale libero $\bar{x}$, e il minimo vincolato sarà il punto $x \in P$ che minimizza la distanza $d(x, \bar{x})$, dove $d$ è scalata per le dimensioni degli elissoidi dati dall'insieme di livello di $\mathrm{lev}_{=k}(f)$.
				Avevamo adottato un approccio simile negli esempi precedenti, notando il fatto che parlavamo di circonferenze, e quindi $d$ era la comune norma euclidea.
			
				Cerchiamo innanzitutto di trovare un equazione che espliciti i semiassi e il punto centrale dell'insieme di livello stesso.
				Per un paraboloide qualsiasi con semiassi paralleli agli assi cartesiani, cioè in forma:
				$$
				f(x_1, x_2) = ax_1^2 + bx_2^2 + cx_1 + dx_2 + e
				$$
				siamo interessati all'insieme:
				$$
				\mathrm{lev}_{=k} (f) = \left\{ (x_1, x_2) : ax_1^2 + bx_2^2 + cx_1 + dx_2 + e = k \right\}
				$$
				
				Converrà riportare la $f$ in una forma del tipo:
				$$
				f(x_1, x_2) = \frac{(x_1 - p_1)^2}{\alpha^2} + \frac{(x_2 - p_2)^2}{\beta^2} - r^2
				$$
				in modo da poter esprimere gli insiemi di livello:
				$$
				\mathrm{lev}_{=k} (f) = \left\{ (x_1, x_2) :  \frac{(x_1 - p_1)^2}{\alpha^2} + \frac{(x_2 - p_2)^2}{\beta^2} = r^2 \right\}
				$$
				che non è altro che il fascio di ellissi centrate in $(p_1, p_2)$, con semiassi di dimensione $\alpha r$ e $\beta r$.

				Troviamo quindi i parametri di tali ellissi, dato:
				$$
				f(x_1, x_2)	= \frac{(x_1 - p_1)^2}{\alpha^2} + \frac{(x_2 - p_2)^2}{\beta^2} - r^2
				$$
				$$
				= \frac{1}{\alpha^2} x_1^2 + \frac{1}{\beta^2} x_2^2 + \left( -\frac{2 p_1}{\alpha^2} \right) x_1 + \left( -\frac{2 p_2}{\beta^2} \right) x_2 + \frac{p_1^2}{\alpha^2} + \frac{p_2^2}{\beta^2} - r^2
				$$
				Questo si traduce nel sistema:
				\[
					\begin{cases}
						a = \frac{1}{\alpha^2} \\ 	
						b = \frac{1}{\beta^2} \\ 	
						c = - \frac{2 p_1}{\alpha^2} \\ 	
						d = - \frac{2 p_2}{\beta^2} \\ 
						e - k = \frac{p_1^2}{\alpha^2} + \frac{p_2^2}{\beta^2} - r^2
					\end{cases}
				\]
				da cui si ha:
				\[
					\begin{cases}
						\alpha = \frac{1}{\sqrt{a}} \\ 	
						\beta = \frac{1}{\sqrt{b}} \\
						p_1 = - \frac{c}{2a} \\
						p_2 = - \frac{d}{2b} \\
						r = \sqrt{\frac{c^{2}}{4a}+\frac{d^{2}}{4b}-g+k}
					\end{cases}
				\]
				che sostituendo i valori dati dall'esempio diventa:
				\[
					\begin{cases}
						\alpha = \frac{\sqrt{2}}{2} \\ 
						\beta = \frac{1}{2} \\ 
						p_1 = -1 \\ 
						p_2 = \frac{1}{8} \\
						r = \sqrt{\frac{33}{16} + k}
					\end{cases}
				\]
				
				Da qui potremmo già applicare un approccio grafico variando $k$ e trovando la prima intersezione con il poliedro $P$, sia questa sul bordo o su un vertice.
				
				Un metodo più informato prevede di definire una funzione distanza $d(a, b)$, scalata sulle dimensioni degli elissoidi di $\mathrm{lev}_{=k}(f)$.
				A questo punto basterà minimizzare $d(p_m, p_\Omega) \in \mathbb{R}$, con $p_m = (p_1, p_2)$ e $p_\Omega$ un punto qualsiasi del bordo di $P$.
				Definiamo quindi $d$ come:
				$$
				d(a, b) = \sqrt{\frac{(a_x - b_x)^2}{\alpha^2} + \frac{(a_y - b_y)^2}{\beta^2}}
				$$
				Dove gli $\alpha$ e $\beta$ sono gli stessi semiassi $/ r$ di prima.
				
				Questa definizione deriva dalla \textit{distanza di Mahalanobis} su matrici di covarianza, presa nel caso di matrici diagonali (quindi semiassi paralleli agli assi cartesiani).

				Prendiamo quindi il punto $p_\Omega$ come uno qualsiasi fra i:
				$$
				\phi_{ij}(t) = v_i (1 - t) + v_j t
				$$
				segmenti del bordo di $P$.
				Consideremo solo i segmenti $(1, 2)$ e $(1, 4)$, cioè $(0,1) \rightarrow (4,1)$ e $(0, 1) \rightarrow (-1, 5)$, per esemplificare i calcoli.
				Una discussione completa richiederebbe la verifica di tutti i segmenti.
				Si hanno quindi le parametrizzazioni:
				$$
				\phi_{12}(t) = v_1 (1 - t) + v_2 t
				$$
				$$
				\phi_{14}(t) = v_1 (1 - t) + v_4 t
				$$
				e si vogliono calcolare:
				$$
				\min d(p_m, \phi_{12}(t))
				$$
				$$
				\min d(p_m, \phi_{14}(t))
				$$

				Prendiamo la $d(p_m, \phi_{12}(t))$. Svolgendo le composizioni, si ha:
				$$
				d(p_m, \phi_{12}(t)) = \sqrt{\frac{(p_{1}-4t)^{2}}{\alpha^{2}}+\frac{\left(p_{2}-\left(\left(1-t\right)+t\right)\right)^{2}}{\beta^{2}}}
				$$
				$$
				 = \sqrt{\frac{p_{1}^{2}-8p_{1}t+16t^{2}}{\alpha^{2}}+\frac{\left(p_{2}-1\right)^{2}}{\beta^{2}}} = \sqrt{32t^{2}+16t+\frac{81}{16}} 
				$$
				Con passaggi simili, la $d(p_m, \phi_{14}(t))$ dà:
				$$
				d(p_m, \phi_{14}(t)) = \sqrt{66t^{2}+24t+\frac{81}{16}}
				$$

				Queste sono funzioni reali di variabile reale $t$, e hanno entrambe minimo in $0$ di valore $\frac{9}{4}$.
				Fossero state in disaccordo sul punto di minimo, si sarebbe preso il minimo minore (che significa distanza minore).
				Il fatto che qui i minimi sono identici e sullo stesso punto significa che il minimo è su un vertice.
				In particolare, secondo le parametrizzazioni riportate sopra, il vertice è $v_1$, quindi $(0, 1)$.

				Notiamo come ad $r = \frac{9}{4}$ si ha $k = 3$, e l'insieme $\mathrm{lev}_{=3}(f)$ è proprio quello che intercetta per primo il vertice $(0, 1)$ del poliedro.
				Questa è una conseguenza di aver scelto una norma $d$ che è correttamente scalata sui semiassi delle ellissi.

				Si riporta un grafico:
\begin{center}
\begin{tikzpicture}
	\begin{axis}[
			axis lines = center,
			xlabel = $x$, ylabel = $y$,
			width=14cm,
			height=8cm,
			xmin = -2.9,
			xmax = 4.9,
			ymin = -1.9, 
			ymax = 5.9,
			grid = minor
	]
	
	\addplot[
		domain=0:360,
		samples=100, 
		thick,
		color=red,
	] ({1.58114*cos(x) - 1}, {1.125*sin(x) + 0.125});
	\addlegendentry{$\mathrm{lev}_{=3}(f)$};

	\addplot[
        thick,
        blue,
    ] coordinates {(0,1) (4,1)};
	
	\addplot[
        thick,
        blue,
    ] coordinates {(0,1) (-1,5)};
	
	\addplot[
		only marks,
		mark=*,
		mark size = 2pt,
		] coordinates{(0,1)};

	\addplot[
		only marks,
		mark=*,
		mark size = 2pt,
		] coordinates{(4,1)};

	\addplot[
		only marks,
		mark=*,
		mark size = 2pt,
		] coordinates{(-1,5)};

	\node[anchor=south, yshift=5pt] at (axis cs:-1,5) {$v_4$};
	\node[anchor=south, yshift=5pt, xshift=10pt] at (axis cs:0,1) {$v_1$};
	\node[anchor=south, yshift=5pt] at (axis cs:4,1) {$v_2$};

	\addplot[
		only marks,
		mark=*,
		mark size = 2pt,
		] coordinates{(-1, 0.125)};
	
	\node[anchor=south, yshift=5pt] at (axis cs:-1,0.125) {$p_m$};

\end{axis}
\end{tikzpicture}

\end{center}

	\item \textit{(Sistema LKT)} Si esplicitano i vincoli del poliedro come segue:
		\[
			\begin{cases}
				g_1(x_1, x_2) = 1 - x_2 \\   	
				g_2(x_1, x_2) = x_1 - 4 \\   	
				g_3(x_1, x_2) = \frac{2}{5}x_1 + x_2 - \frac{23}{5} \\   	
				g_4(x_1, x_2) = 1 - 4 x_1 - x_2 \\   	
			\end{cases}
		\]
		rispetto ai segmenti $(v_1, v_2)$, $(v_2, v_3)$, $(v_3, v_4)$, $(v_1, v_4)$.

		Dai gradienti di $f$ e le $g_i$ si ricava il sistema LKT:
		\[
			\begin{cases}			
				4 x_1 + 4 + \lambda_2 + \frac{2}{5} \lambda_3 - 4 \lambda_4 = 0 \\ 
				8 x_2 - 1 - \lambda_1 + \lambda_3 - \lambda_4 = 0 \\ 
				\lambda_1 (1 - x_2) = 0 \\ 
				\lambda_2 (x_1 - 4) = 0 \\ 
				\lambda_3 (\frac{2}{5}x_1 + x_2 - \frac{23}{5}) = 0 \\ 
				\lambda_4 (1 - 4 x_1 - x_2) = 0
			\end{cases}
		\]
		\end{enumerate}
		I vincoli rappresentano semipiani, ergo la soluzione del sistema è agile se si considerano prima le coppie di rette incidenti (che per $n=4$ rette sono $\frac{n(n-1)}{2} \big|_{n=4} = 6$), cioè:
	\begin{table}[H]
		\center 
		\begin{tabular} { c c | c c c c }
			$x_1$ & $x_2$ & $\lambda_1$ & $\lambda_2$ & $\lambda_3$ & $\lambda_4$ \\ 
			\hline 
			$4$ & $1$ & $7$ & $-20$ & $0$ & $0$ \\ 
			$9$ & $1$ & $107$ & $0$ & $-100$ & $0$ \\ 
			$0$ & $1$ & $1$ & $0$ & $0$ & $6$ \\ 
			$4$ & $3$ & $0$ & $-\frac{54}{5}$ & $-23$ & $0$ \\ 
			$4$ & $-15$ & $0$ & $482$ & $0$ & $121$ \\ 
			$-1$ & $5$ & $0$ & $0$ & $-\frac{130}{3}$ & $-\frac{13}{3}$
		\end{tabular}
	\end{table}
	da cui rimuoviamo subito il 2° e il 5° punto in quanto esterni a $P$.
	Notiamo inoltre di aver già trovato il punto di massimo in $(-1, 5)$.
	Si hanno poi i 4 punti giacienti sulle rette, cioè:
	\begin{table}[H]
		\center 
		\begin{tabular} { c c | c c c c }
			$x_1$ & $x_2$ & $\lambda_1$ & $\lambda_2$ & $\lambda_3$ & $\lambda_4$ \\ 
			\hline 
			$-1$ & $1$ & $7$ & $0$ & $0$ & $0$ \\ 
			$4$ & $\frac{1}{8}$ & $0$ & $-8$ & $0$ & $0$ \\ 
			$\frac{43}{22}$ & $\frac{42}{11}$ & $0$ & $0$ & $-\frac{325}{11}$ & $0$ \\ 
			$\frac{2}{11}$ & $\frac{3}{11}$ & $0$ & $0$ & $0$ & $\frac{13}{11}$
		\end{tabular}
	\end{table}
	di cui solo il 3° è interno a $P$.
	Calcoliamo quindi per enumerazione diretta:
	\[
		\begin{cases}
			f(0,1) = 3 \\ 
			f(4,1) = 51 \\ 
			f(4,3) = 81 \\
			f(-1, 5) = 93 \\ 
			f(-1, 1) = 1 \\ 
			f(\frac{43}{22}, \frac{32}{11}) \approx 46.4 \\ 
		\end{cases}
	\]	
	da cui si riconferma il minimo globale in $(0,1)$.


		Si riporta un grafico che evidenzia come il punto interno al bordo trovato è dato dall'intersezione fra il bordo di $P$ e l'insieme di livello in $\sim 70$:
\begin{center}
\begin{tikzpicture}
	\begin{axis}[
			axis lines = center,
			xlabel = $x$, ylabel = $y$,
			width=14cm,
			height=8cm,
			xmin = -7.9,
			xmax = 5.9,
			ymin = -5.9, 
			ymax = 5.9,
			grid = minor
	]
	
	\addplot[
		domain=0:360,
		samples=100, 
		thick,
		color=red,
	] ({6*cos(x) - 1}, {4.244*sin(x) + 0.125});
	\addlegendentry{$\mathrm{lev}_{\approx70}(f)$};

	\addplot[
        thick,
        blue,
    ] coordinates {(0,1) (4,1)};
	
	\addplot[
        thick,
        blue,
    ] coordinates {(0,1) (-1,5)};

	\addplot[
        thick,
        blue,
    ] coordinates {(-1,5) (4,3)};
	
	\addplot[
        thick,
        blue,
    ] coordinates {(4,1) (4,3)};
	
	\addplot[
		only marks,
		mark=*,
		mark size = 2pt,
		] coordinates{(0,1)};

	\addplot[
		only marks,
		mark=*,
		mark size = 2pt,
		] coordinates{(4,1)};

	\addplot[
		only marks,
		mark=*,
		mark size = 2pt,
		] coordinates{(-1,5)};

	\addplot[
		only marks,
		mark=*,
		mark size = 2pt,
		] coordinates{(4,3)};

	\addplot[
		only marks,
		mark=*,
		mark size = 2pt,
		] coordinates{(1.9545,3.8182)};

	\node[anchor=south, yshift=5pt] at (axis cs:-1,5) {$v_4$};
	\node[anchor=south, yshift=5pt, xshift=10pt] at (axis cs:0,1) {$v_1$};
	\node[anchor=south, yshift=5pt] at (axis cs:4,1) {$v_2$};
	\node[anchor=south, yshift=5pt] at (axis cs:4,3) {$v_3$};
	
	\node[anchor=south, yshift=5pt] at (axis cs:1.9545,3.8182) {$(\frac{43}{22}, \frac{42}{11})$};

	\addplot[
		only marks,
		mark=*,
		mark size = 2pt,
		] coordinates{(-1, 0.125)};
	
	\node[anchor=south, yshift=5pt] at (axis cs:-1,0.125) {$p_m$};

\end{axis}
\end{tikzpicture}

\end{center}
\end{itemize}

\subsection{Algoritmo del gradiente proiettato}
Sia dato un problema di ottimizzazione non lineare simile al precedente, di massimizzazione, cioè:
\[
	\begin{cases}
		\max f(x) \\ 
		x \in P
	\end{cases}
\]
dove $P$ è un poliedro limitato.
Di questo poniamo di conoscere il gradiente $\nabla f(x_k)$ ad un certo punto $x_k$ sul bordo di $P$.

L'idea è quella di \textbf{proiettare} il gradiente $\nabla f(x_k)$ sul poliedro $P$, o meglio sulla retta data dal vincolo attivo di $P$ nel punto $x_k$.
Se la proiezione è non nulla, l'angolo $\theta_k$ che il gradiente proiettato ha con il gradiente proprio è $< \frac{\pi}{2}$, e quindi:
$$
\nabla f(x_k) \cdot d_k > 0
$$
posta $d_k$ come la proiezione ottenuta, che è esattamente l'ipotesi di una direzione di crescita per un algoritmo di PNL.

Resta quindi il problema di calcolare tale proiezione.
Avevamo quindi che il poliedro $P$ è definito come:
$$
Ax \leq b
$$
e gli indici attivi erano definiti come:
$$
A(x_k) = \left\{ i \in I: A_ix_k = b \right\}
$$

Definiamo quindi $M$ come la sottomatrice di $A$ ottenuta selezionando le righe di indice $i \in A(x_k)$. 
Si introduce allora la \textbf{matrice di proiezione}:
\begin{definition}{Matrice di proiezione}
	Definiamo la matrice di proiezione in un punto $x_k$ di un poliedro $P$ di un problema di PNL come: 
	$$
		H = 
			\begin{cases}
				I , \quad \quad \quad \quad \quad \quad \quad \quad \quad  A(x_k) = \emptyset \\ 
				I - M^T(M M^T)^{-1} M, \quad \text{altrimenti}
			\end{cases}
	$$
	data $M$ matrice dei vincoli attivi.
\end{definition}

Si può dimostrare il seguente teorema:
\begin{theorem}{Teorema della proiezione}
	Data una matrice di proiezione $H$, $H y$ è la proiezione di $y$ sulla varietà geometrica $M$.
\end{theorem}

Nel nostro caso, potremo sfruttare il teorema per calcolare la proiezione del gradiente su $M$, cioè:
$$
H \cdot \nabla f(x_k) = d_k
$$


Posto $x_{k +1} = x_k + t d_k$, vorremo seguire tale direzione $d_k$ fino al punto di uscita dal poliedro (diciamo $\hat{t}_k$).
Questo punto sarà dato dal problema LP:
\[
	\hat{t}_k = 
	\begin{cases}
			\max t \\ 
			A\cdot(x_k + t d_k) \leq b
	\end{cases}
\]
su $m$ disequazioni per $m$ vincoli.
Massimizzeremo poi la funzione sulla direzione $d_k$ fino a un massimo di $\hat{t}_k$ cioè:
$$
t_k = \mathrm{argmax}_{[0, \hat{t}_k]} f(x_k + t d_k) = \mathrm{argmax}_{[0, \hat{t}_k]} f(x_k + t \cdot H \cdot \nabla f(x_k)) 
$$



\end{document}
