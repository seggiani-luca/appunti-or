
\documentclass[a4paper,11pt]{article}
\usepackage[a4paper, margin=8em]{geometry}

% usa i pacchetti per la scrittura in italiano
\usepackage[french,italian]{babel}
\usepackage[T1]{fontenc}
\usepackage[utf8]{inputenc}
\frenchspacing 

% usa i pacchetti per la formattazione matematica
\usepackage{amsmath, amssymb, amsthm, amsfonts}

% usa altri pacchetti
\usepackage{gensymb}
\usepackage{hyperref}
\usepackage{standalone}

% imposta il titolo
\title{Appunti Ricerca Operativa}
\author{Luca Seggiani}
\date{2024}

% disegni
\usepackage{pgfplots}
\pgfplotsset{width=10cm,compat=1.9}

% imposta lo stile
% usa helvetica
\usepackage[scaled]{helvet}
% usa palatino
\usepackage{palatino}
% usa un font monospazio guardabile
\usepackage{lmodern}

\renewcommand{\rmdefault}{ppl}
\renewcommand{\sfdefault}{phv}
\renewcommand{\ttdefault}{lmtt}

% disponi il titolo
\makeatletter
\renewcommand{\maketitle} {
	\begin{center} 
		\begin{minipage}[t]{.8\textwidth}
			\textsf{\huge\bfseries \@title} 
		\end{minipage}%
		\begin{minipage}[t]{.2\textwidth}
			\raggedleft \vspace{-1.65em}
			\textsf{\small \@author} \vfill
			\textsf{\small \@date}
		\end{minipage}
		\par
	\end{center}

	\thispagestyle{empty}
	\pagestyle{fancy}
}
\makeatother

% disponi teoremi
\usepackage{tcolorbox}
\newtcolorbox[auto counter, number within=section]{theorem}[2][]{%
	colback=blue!10, 
	colframe=blue!40!black, 
	sharp corners=northwest,
	fonttitle=\sffamily\bfseries, 
	title=Teorema~\thetcbcounter: #2, 
	#1
}

% disponi definizioni
\newtcolorbox[auto counter, number within=section]{definition}[2][]{%
	colback=red!10,
	colframe=red!40!black,
	sharp corners=northwest,
	fonttitle=\sffamily\bfseries,
	title=Definizione~\thetcbcounter: #2,
	#1
}

% disponi problemi
\newtcolorbox[auto counter, number within=section]{problem}[2][]{%
	colback=green!10,
	colframe=green!40!black,
	sharp corners=northwest,
	fonttitle=\sffamily\bfseries,
	title=Problema~\thetcbcounter: #2,
	#1
}

% disponi codice
\usepackage{listings}
\usepackage[table]{xcolor}

\lstdefinestyle{codestyle}{
		backgroundcolor=\color{black!5}, 
		commentstyle=\color{codegreen},
		keywordstyle=\bfseries\color{magenta},
		numberstyle=\sffamily\tiny\color{black!60},
		stringstyle=\color{green!50!black},
		basicstyle=\ttfamily\footnotesize,
		breakatwhitespace=false,         
		breaklines=true,                 
		captionpos=b,                    
		keepspaces=true,                 
		numbers=left,                    
		numbersep=5pt,                  
		showspaces=false,                
		showstringspaces=false,
		showtabs=false,                  
		tabsize=2
}

\lstdefinestyle{shellstyle}{
		backgroundcolor=\color{black!5}, 
		basicstyle=\ttfamily\footnotesize\color{black}, 
		commentstyle=\color{black}, 
		keywordstyle=\color{black},
		numberstyle=\color{black!5},
		stringstyle=\color{black}, 
		showspaces=false,
		showstringspaces=false, 
		showtabs=false, 
		tabsize=2, 
		numbers=none, 
		breaklines=true
}

\lstdefinelanguage{javascript}{
	keywords={typeof, new, true, false, catch, function, return, null, catch, switch, var, if, in, while, do, else, case, break},
	keywordstyle=\color{blue}\bfseries,
	ndkeywords={class, export, boolean, throw, implements, import, this},
	ndkeywordstyle=\color{darkgray}\bfseries,
	identifierstyle=\color{black},
	sensitive=false,
	comment=[l]{//},
	morecomment=[s]{/*}{*/},
	commentstyle=\color{purple}\ttfamily,
	stringstyle=\color{red}\ttfamily,
	morestring=[b]',
	morestring=[b]"
}

% disponi sezioni
\usepackage{titlesec}

\titleformat{\section}
	{\sffamily\Large\bfseries} 
	{\thesection}{1em}{} 
\titleformat{\subsection}
	{\sffamily\large\bfseries}   
	{\thesubsection}{1em}{} 
\titleformat{\subsubsection}
	{\sffamily\normalsize\bfseries} 
	{\thesubsubsection}{1em}{}

% disponi alberi
\usepackage{forest}

\forestset{
	rectstyle/.style={
		for tree={rectangle,draw,font=\large\sffamily}
	},
	roundstyle/.style={
		for tree={circle,draw,font=\large}
	}
}

% disponi algoritmi
\usepackage{algorithm}
\usepackage{algorithmic}
\makeatletter
\renewcommand{\ALG@name}{Algoritmo}
\makeatother

% disponi numeri di pagina
\usepackage{fancyhdr}
\fancyhf{} 
\fancyfoot[L]{\sffamily{\thepage}}

\makeatletter
\fancyhead[L]{\raisebox{1ex}[0pt][0pt]{\sffamily{\@title \ \@date}}} 
\fancyhead[R]{\raisebox{1ex}[0pt][0pt]{\sffamily{\@author}}}
\makeatother

\begin{document}

% sezione (data)
\section{Lezione del 07-11-24}

% stili pagina
\thispagestyle{empty}
\pagestyle{fancy}

% testo
\subsection{Simplesso per flussi}
Vediamo come applicare l'algoritmo del simplesso ai problemi di ottimizzazione sui grafi, in particolare per risolvere problemi di flusso minimo. 

Partiamo dall'assunto fondamentale che una soluzione ammissibile è rappresentata da un albero di copertura che genera flusso ammissibile, che chiameremo \textbf{albero di copertura ammissibile}.
Potremo quindi partizionare l'insieme $A$ degli archi in due sottoinsiemi:
\begin{itemize}
	\item $T$, formato dagli archi che formano l'albero (e che denota quindi l'albero stesso);
	\item $L$, formato dagli archi rimanenti.
\end{itemize}

L'insieme di possibili partizioni $(T, L)$ rappresenta le basi dei vertici di un poliedro, cioè qualsiasi albero di copertura ammissibile $T$ rappresenta una base del poliedro.

Si possono quindi calcolare, dato $T$, i costi ridotti $c_{ij}^\pi$ su tutti gli archi che comprende:
$$
c_{ij}^\pi = c_{ij} + \pi_i - \pi_j
$$
Si ha che, dal teorema di Bellman, se $\forall (i, j) \in L : c_{ij}^\pi \geq 0$, allora la base duale è ammissibile e siamo all'ottimo.
Altrimenti, dovrà essere che $\exists (i,j) \in L : c_{ij}^\pi < 0$. Scegliamo questo $(i, j)$ come \textbf{arco entrante}.

Si ha che l'arco entrante $(i, j)$ forma un ciclo con gli archi dell'albero $T$: la selezione di un arco uscente coinciderà nel simplesso su grafi ad un processo di \textit{eliminazione di cicli}.
Si sceglie allora una direzione di percorrenza del ciclo concorde a $(i, j)$, e si partizionano gli archi del ciclo in $\mathcal{C}^+$ per gli archi concordi a questa direzione, e $\mathcal{C}^-$ per gli archi discordi.

L'obiettivo è quello di "spingere" più unità di flusso $\vartheta$ possibili nella direzione del ciclo, sfruttando l'arco appena introdotto.
Riguardo ai flussi, questo significherà aggiungere $\vartheta$ unità di flusso agli archi in $\mathcal{C}^+$, e rimuovere $\vartheta$ unità di flusso dagli archi in $\mathcal{C}^-$.
Se $\mathcal{C}^-$ è vuoto, non c'è limite al flusso che possiamo spingere, e l'ottimo è $-\infty$. 
Altrimenti, il valore massimo di $\vartheta$ sarà quello che non viola i vincoli, portando un arco in $\mathcal{C}^-$ a valori $<0$. 
Questo valore non sarà altro che il flusso minimo già impegnato sugli archi $x_{ij}$ con $(i, j) \in C^-$, cioè il flusso dell'arco che si "svuoterà" per primo spingendo flusso nella direzione opposta.
Chiamiamo questo arco \textbf{arco uscente}.

Si aggiorna quindi la base rimuovendo l'arco uscente e introducendo l'arco entrante, e si ripete fino all'ottimo.

Possiamo quindi formulare l'algoritmo per intero:

\begin{algorithm}[H]
\caption{del simplesso per flussi}
\begin{algorithmic}
	\STATE \textbf{Input:} un problema di flusso di costo minimo 
	\STATE \textbf{Output:} la soluzione ottima 
	\STATE Trova una partizione degli archi $(T,L)$ con $T$ albero di copertura che genera un flusso ammissibile
	\STATE \textsf{ciclo:}
	\STATE Calcola il flusso di base $\bar{x}_T = E_T^{-1} b$, posti gli $\bar{x}_L = 0$, e il potenziale di base $\bar{\pi}_T = c_T^\intercal E_T^{-1}$
	\STATE Calcola i costi ridotti $c_{ij}^{\bar{\pi}} = c_{ij} + \bar{\pi}_{ij} - \bar{\pi}_{ij}$ per ogni arco
	\IF{$c_{ij}^{\overline{\pi}} \geq 0 \ \ \forall (i, j) \in L$}
	\STATE Fermati, $\bar{x}$ è un flusso ottimo e $\bar{\pi}$ è un potenziale ottimo
\ELSE
		\STATE Calcola l'arco entrante: 
		$$
		(p, q) = \min \left\{ (i, j) \in L : c_{ij}^{\bar{\pi}} < 0 \right\}
		$$
		stabilito l'ordinamento lessicografico degli archi
		\STATE Chiama $\mathcal{C}$ il ciclo che l'arco $(p, q)$ forma con gli archi in $T$
		\STATE Fissa un orientamento concorde a $(p,q)$ su $\mathcal{C}$ e partiziona $\mathcal{C}$ in $\mathcal{C^+}$ archi concordi e $\mathcal{C^-}$ archi discordi a tale orientamento
	\ENDIF
	\IF{$\mathcal{C}^- = \emptyset$}
		\STATE Fermati, il flusso di costo minimo ha valore $-\infty$ 
	\ELSE
		\STATE Calcola:
		$$
		\vartheta = \min\{ \bar{x}_{ij} : (i, j) \in \mathcal{C}^- \}
		$$
		e trova l'arco uscente: 
		$$ 
		(r, s) = \min\{ (i, j) \in \mathcal{C}^- : \bar{x}_{ij} = \vartheta \} 
		$$
		stabilito un ordinamento lessicografico degli archi
	\ENDIF
	\STATE Aggiorna le partizioni come:
	$$
	T = T \setminus \left\{ (r,s) \right\} \cup \left\{ (p, q) \right\} \quad L = L \setminus \left\{ (p, q) \right\} \cup \left\{ (r, s) \right\}
	$$
	\STATE Torna a \textsf{ciclo}
\end{algorithmic}
\end{algorithm}

\subsubsection{Ottimizzazioni del simplesso per i flussi}
Possiamo fare alcune ottimizzazioni considerevoli sull'algoritmo del simplesso per i flussi:
\begin{itemize}
	\item Non è necessario calcolare i costi ridotti $c^\pi_{ij}$ per ogni arco $\in A$, ma soltanto in $L$, in quanto sappiamo gli altri (quelli $\in T$) varranno 0;
	\item Non è necessario calcolare l'insieme $\mathcal{C}^*$: le rimozioni potranno essere effettuate solo da $\mathcal{C}^-$;
	\item Il calcolo dei potenziali di base dopo l'aggiornamento delle partizioni può essere effettuato senza inversioni di matrici.
		Si ha che, presi l'arco entrante $(p, q)$ e l'arco uscente $(r, s)$, la rimozione di $(r,s)$ (prima dell'introduzione di $(p,q)$) partiziona effettivamente i nodi del grafo in due sottoinsiemi $N_p$ e $N_q$, con $p \in N_p$ e $q \in N_q$.
		Avremo allora che i nuovi potenziali di base $\pi'_i$ saranno:
		\[
			\pi'_i =
			\begin{cases}
				\pi_i, \quad i \in N_p \\ 
				\pi_i'+ c_{pq}^\pi, \quad i \in N_q
			\end{cases}
		\]
		cioè la partizione $p$ resta invariata, e la partizione $q$ sale di potenziale pari al costo ridotto $c^\pi_{pq}$.
	\item Una volta variati i potenziali, si possono ricalcolare anche i costi ridotti $c_{ij}^{\pi'}$ con una regola simile:
		\[
			c_{ij}^{\pi'} =
			\begin{cases}
				c_{ij}^{\pi'}, \quad (i,j) \in N_p \vee (i,j) \in N_q \\ 
				c_{ij}^{\pi'} - c_{pq}^\pi, \quad (i,j) \in N_p \wedge (i,j) \in N_q \\ 
				c_{ij}^{\pi'} + c_{pq}^\pi, \quad (i,j) \in N_q \wedge (i,j) \in N_p
			\end{cases}
		\]
		cioè, se l'arco è interamente contenuto in $N_p$ o $N_q$, il costo ridotto resta invariato.
		Altrimenti, se l'arco "salta" dalla partizione $p$ alla partizione $q$, si aggiunge o si sottrae, a seconda della direzione, il costo ridotto di $c_pq^\pi$.
	\end{itemize}

\subsubsection{Ricavare le variazioni dei cambi di partizione}
Potremmo voler ricavare le variazioni in termini di valore ottimo che si ottengono dopo un passo del simplesso sui flussi, quindi dopo l'aggiornamento delle partizioni.
Abbiamo che la variazione del valore ottimo è quindi il flusso introdotto sul ciclo, cioè $\vartheta$, moltiplicato per il costo ridotto dell'arco introdotto, cioè il valore si evolve di $v$ in $v'$ come:
$$
v' = V - \vartheta \cdot c_{ij}^\pi
$$

\end{document}
