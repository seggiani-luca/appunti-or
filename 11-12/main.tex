
\documentclass[a4paper,11pt]{article}
\usepackage[a4paper, margin=8em]{geometry}

% usa i pacchetti per la scrittura in italiano
\usepackage[french,italian]{babel}
\usepackage[T1]{fontenc}
\usepackage[utf8]{inputenc}
\frenchspacing 

% usa i pacchetti per la formattazione matematica
\usepackage{amsmath, amssymb, amsthm, amsfonts}

% usa altri pacchetti
\usepackage{gensymb}
\usepackage{hyperref}
\usepackage{standalone}

% imposta il titolo
\title{Appunti Ricerca Operativa}
\author{Luca Seggiani}
\date{2024}

% disegni
\usepackage{pgfplots}
\pgfplotsset{width=10cm,compat=1.9}

% imposta lo stile
% usa helvetica
\usepackage[scaled]{helvet}
% usa palatino
\usepackage{palatino}
% usa un font monospazio guardabile
\usepackage{lmodern}

\renewcommand{\rmdefault}{ppl}
\renewcommand{\sfdefault}{phv}
\renewcommand{\ttdefault}{lmtt}

% disponi il titolo
\makeatletter
\renewcommand{\maketitle} {
	\begin{center} 
		\begin{minipage}[t]{.8\textwidth}
			\textsf{\huge\bfseries \@title} 
		\end{minipage}%
		\begin{minipage}[t]{.2\textwidth}
			\raggedleft \vspace{-1.65em}
			\textsf{\small \@author} \vfill
			\textsf{\small \@date}
		\end{minipage}
		\par
	\end{center}

	\thispagestyle{empty}
	\pagestyle{fancy}
}
\makeatother

% disponi teoremi
\usepackage{tcolorbox}
\newtcolorbox[auto counter, number within=section]{theorem}[2][]{%
	colback=blue!10, 
	colframe=blue!40!black, 
	sharp corners=northwest,
	fonttitle=\sffamily\bfseries, 
	title=Teorema~\thetcbcounter: #2, 
	#1
}

% disponi definizioni
\newtcolorbox[auto counter, number within=section]{definition}[2][]{%
	colback=red!10,
	colframe=red!40!black,
	sharp corners=northwest,
	fonttitle=\sffamily\bfseries,
	title=Definizione~\thetcbcounter: #2,
	#1
}

% disponi problemi
\newtcolorbox[auto counter, number within=section]{problem}[2][]{%
	colback=green!10,
	colframe=green!40!black,
	sharp corners=northwest,
	fonttitle=\sffamily\bfseries,
	title=Problema~\thetcbcounter: #2,
	#1
}

% disponi codice
\usepackage{listings}
\usepackage[table]{xcolor}

\lstdefinestyle{codestyle}{
		backgroundcolor=\color{black!5}, 
		commentstyle=\color{codegreen},
		keywordstyle=\bfseries\color{magenta},
		numberstyle=\sffamily\tiny\color{black!60},
		stringstyle=\color{green!50!black},
		basicstyle=\ttfamily\footnotesize,
		breakatwhitespace=false,         
		breaklines=true,                 
		captionpos=b,                    
		keepspaces=true,                 
		numbers=left,                    
		numbersep=5pt,                  
		showspaces=false,                
		showstringspaces=false,
		showtabs=false,                  
		tabsize=2
}

\lstdefinestyle{shellstyle}{
		backgroundcolor=\color{black!5}, 
		basicstyle=\ttfamily\footnotesize\color{black}, 
		commentstyle=\color{black}, 
		keywordstyle=\color{black},
		numberstyle=\color{black!5},
		stringstyle=\color{black}, 
		showspaces=false,
		showstringspaces=false, 
		showtabs=false, 
		tabsize=2, 
		numbers=none, 
		breaklines=true
}

\lstdefinelanguage{javascript}{
	keywords={typeof, new, true, false, catch, function, return, null, catch, switch, var, if, in, while, do, else, case, break},
	keywordstyle=\color{blue}\bfseries,
	ndkeywords={class, export, boolean, throw, implements, import, this},
	ndkeywordstyle=\color{darkgray}\bfseries,
	identifierstyle=\color{black},
	sensitive=false,
	comment=[l]{//},
	morecomment=[s]{/*}{*/},
	commentstyle=\color{purple}\ttfamily,
	stringstyle=\color{red}\ttfamily,
	morestring=[b]',
	morestring=[b]"
}

% disponi sezioni
\usepackage{titlesec}

\titleformat{\section}
	{\sffamily\Large\bfseries} 
	{\thesection}{1em}{} 
\titleformat{\subsection}
	{\sffamily\large\bfseries}   
	{\thesubsection}{1em}{} 
\titleformat{\subsubsection}
	{\sffamily\normalsize\bfseries} 
	{\thesubsubsection}{1em}{}

% disponi alberi
\usepackage{forest}

\forestset{
	rectstyle/.style={
		for tree={rectangle,draw,font=\large\sffamily}
	},
	roundstyle/.style={
		for tree={circle,draw,font=\large}
	}
}

% disponi algoritmi
\usepackage{algorithm}
\usepackage{algorithmic}
\makeatletter
\renewcommand{\ALG@name}{Algoritmo}
\makeatother

% disponi numeri di pagina
\usepackage{fancyhdr}
\fancyhf{} 
\fancyfoot[L]{\sffamily{\thepage}}

\makeatletter
\fancyhead[L]{\raisebox{1ex}[0pt][0pt]{\sffamily{\@title \ \@date}}} 
\fancyhead[R]{\raisebox{1ex}[0pt][0pt]{\sffamily{\@author}}}
\makeatother

\begin{document}

% sezione (data)
\section{Lezione del 12-11-24}

% stili pagina
\thispagestyle{empty}
\pagestyle{fancy}

% testo
\subsection{Flusso di costo minimo capacitato}
In realtà, il modello di flusso di costo minimo prevede, su ogni arco, un'altra quantità: la \textbf{portata} (o \textbf{capacità}) massima dell'arco.
Avremo quindi che, oltre oltre agli $n$ bilanci $b_i$ e agli $m$ costi $c$ per $n$ nodi e $m$ archi, dovremo introdurre un nuovo vettore $u$ di dimensione $m$ per le \textbf{capacità superiori} di ogni arco.

Porremo quindi il nostro sistema come:
\[
	\begin{cases}
		\max c^T x \\ 
		Ex = b \\ 
		0 \leq x \leq u
	\end{cases}
\]
dove l'ultimo vincolo è l'espressione in forma vettoriale di $0 \leq x_{ij} \leq u_{ij} \ \forall (i, j) \in A$.

Avevamo notato che un problema di flusso di costo minimo \textit{non capacitato} si esprime agilmente in forma duale standard. 
Convverrà quindi esprimere il vincolo delle capacità superiori come uguaglianza, per ricondursi allo stesso formato:
\[
	\begin{cases}
		\max c^T x \\ 
		Ex = b \\
		x + w = u \\ 
		x \geq 0 \\ 
		w \geq 0
	\end{cases}
\]

Questa forma, con $w \in \mathbb{R^m}$, intende $x_ {ij} + w_{ij} = u_{ij}$, cioè gli $w_{ij}$ di ogni arco sono gli \textbf{scarti} dal valore di capacità massima. A $w$ negativi si avranno $x > u$, ergo saremo oltre la portata massima, mentre a $w = 0$ avremo "fissato" la capacità al massimo per la $x$ corrispondente.

Se volessimo restare nella forma matriciale, potremmo riscrivere il vincolo di capacità come:
$$
Ix + Iw = u
$$
sulla matrice identità $I$, e quindi esprimere a blocchi:
$$
\begin{pmatrix}
	E & 0 \\ 
	I & I
\end{pmatrix}
\begin{pmatrix}
	x \\ w
\end{pmatrix}
=
\begin{pmatrix}
	b \\ u
\end{pmatrix}
$$
dove la prima matrice ha dimensioni $(m + n) \times 2n$.

Se avevamo dimostrato che il rango di $Ex = b$ è $n - 1$, cioè quello delle matrici di base (che danno alberi di copertura), sarà abbastanza intuitivo che il rango di questa nuova matrice sarà $m + n - 1$.
In particolare, si avrà la seguente caratterizzazione delle basi:
\begin{theorem}{Caratterizzazione di base di matrici di incidenza capacitate}
	Presa la matrice di vincoli data da un problema di flusso di costo minimo capacitato, espressa in forma:
$$
\begin{pmatrix}
	E & 0 \\ 
	I & I
\end{pmatrix}
$$
si suppone di avere una \textbf{tripartizione} $T, L, U$ degli archi del grafo (cioè delle colonne della matrice), dove $T$ è un \textit{albero di copertura}. 
Si chiamano poi $T', L', U'$ le colonne corrispondenti alle variabili di scarto $w$.

A questo punto, $B = T \cup U \cup T' \cup L'$ sarà un base. 

\end{theorem}

Abbiamo che la tripartizione $T, L, U, T', L', U'$ è effettivamente una \textit{esapartizione} negli insiemi di colonne $T, L, U$ e le colonne con le $w$ associate.
Notiamo che, da $T$ di dimensione $n-1$ e $T'$ + $L'$ di dimensione pari a $m$, si ha che la dimensione della base è $m + n - 1$, come volevamo.

Vediamo quindi come calcolare il flusso di base corrispondente a una base:
$$
(x, w) = \left( E_T^{-1} (b - E_U u_U), 0, u_U, u_T - x_T, u_L, 0 \right)
$$

# qua sopra dubbi

\end{document}
