
\documentclass[a4paper,11pt]{article}
\usepackage[a4paper, margin=8em]{geometry}

% usa i pacchetti per la scrittura in italiano
\usepackage[french,italian]{babel}
\usepackage[T1]{fontenc}
\usepackage[utf8]{inputenc}
\frenchspacing 

% usa i pacchetti per la formattazione matematica
\usepackage{amsmath, amssymb, amsthm, amsfonts}

% usa altri pacchetti
\usepackage{gensymb}
\usepackage{hyperref}
\usepackage{standalone}

% imposta il titolo
\title{Appunti /home/luca/Desktop/Uni/appunti/Ricerca Operativa}
\author{Luca Seggiani}
\date{2024}

% disegni
\usepackage{pgfplots}
\pgfplotsset{width=10cm,compat=1.9}

% imposta lo stile
% usa helvetica
\usepackage[scaled]{helvet}
% usa palatino
\usepackage{palatino}
% usa un font monospazio guardabile
\usepackage{lmodern}

\renewcommand{\rmdefault}{ppl}
\renewcommand{\sfdefault}{phv}
\renewcommand{\ttdefault}{lmtt}

% disponi il titolo
\makeatletter
\renewcommand{\maketitle} {
	\begin{center} 
		\begin{minipage}[t]{.8\textwidth}
			\textsf{\huge\bfseries \@title} 
		\end{minipage}%
		\begin{minipage}[t]{.2\textwidth}
			\raggedleft \vspace{-1.65em}
			\textsf{\small \@author} \vfill
			\textsf{\small \@date}
		\end{minipage}
		\par
	\end{center}

	\thispagestyle{empty}
	\pagestyle{fancy}
}
\makeatother

% disponi teoremi
\usepackage{tcolorbox}
\newtcolorbox[auto counter, number within=section]{theorem}[2][]{%
	colback=blue!10, 
	colframe=blue!40!black, 
	sharp corners=northwest,
	fonttitle=\sffamily\bfseries, 
	title=Teorema~\thetcbcounter: #2, 
	#1
}

% disponi definizioni
\newtcolorbox[auto counter, number within=section]{definition}[2][]{%
	colback=red!10,
	colframe=red!40!black,
	sharp corners=northwest,
	fonttitle=\sffamily\bfseries,
	title=Definizione~\thetcbcounter: #2,
	#1
}

% disponi problemi
\newtcolorbox[auto counter, number within=section]{problem}[2][]{%
	colback=green!10,
	colframe=green!40!black,
	sharp corners=northwest,
	fonttitle=\sffamily\bfseries,
	title=Problema~\thetcbcounter: #2,
	#1
}

% disponi codice
\usepackage{listings}
\usepackage[table]{xcolor}

\lstdefinestyle{codestyle}{
		backgroundcolor=\color{black!5}, 
		commentstyle=\color{codegreen},
		keywordstyle=\bfseries\color{magenta},
		numberstyle=\sffamily\tiny\color{black!60},
		stringstyle=\color{green!50!black},
		basicstyle=\ttfamily\footnotesize,
		breakatwhitespace=false,         
		breaklines=true,                 
		captionpos=b,                    
		keepspaces=true,                 
		numbers=left,                    
		numbersep=5pt,                  
		showspaces=false,                
		showstringspaces=false,
		showtabs=false,                  
		tabsize=2
}

\lstdefinestyle{shellstyle}{
		backgroundcolor=\color{black!5}, 
		basicstyle=\ttfamily\footnotesize\color{black}, 
		commentstyle=\color{black}, 
		keywordstyle=\color{black},
		numberstyle=\color{black!5},
		stringstyle=\color{black}, 
		showspaces=false,
		showstringspaces=false, 
		showtabs=false, 
		tabsize=2, 
		numbers=none, 
		breaklines=true
}

\lstdefinelanguage{javascript}{
	keywords={typeof, new, true, false, catch, function, return, null, catch, switch, var, if, in, while, do, else, case, break},
	keywordstyle=\color{blue}\bfseries,
	ndkeywords={class, export, boolean, throw, implements, import, this},
	ndkeywordstyle=\color{darkgray}\bfseries,
	identifierstyle=\color{black},
	sensitive=false,
	comment=[l]{//},
	morecomment=[s]{/*}{*/},
	commentstyle=\color{purple}\ttfamily,
	stringstyle=\color{red}\ttfamily,
	morestring=[b]',
	morestring=[b]"
}

% disponi sezioni
\usepackage{titlesec}

\titleformat{\section}
	{\sffamily\Large\bfseries} 
	{\thesection}{1em}{} 
\titleformat{\subsection}
	{\sffamily\large\bfseries}   
	{\thesubsection}{1em}{} 
\titleformat{\subsubsection}
	{\sffamily\normalsize\bfseries} 
	{\thesubsubsection}{1em}{}

% disponi alberi
\usepackage{forest}

\forestset{
	rectstyle/.style={
		for tree={rectangle,draw,font=\large\sffamily}
	},
	roundstyle/.style={
		for tree={circle,draw,font=\large}
	}
}

% disponi algoritmi
\usepackage{algorithm}
\usepackage{algorithmic}
\makeatletter
\renewcommand{\ALG@name}{Algoritmo}
\makeatother

% disponi numeri di pagina
\usepackage{fancyhdr}
\fancyhf{} 
\fancyfoot[L]{\sffamily{\thepage}}

\makeatletter
\fancyhead[L]{\raisebox{1ex}[0pt][0pt]{\sffamily{\@title \ \@date}}} 
\fancyhead[R]{\raisebox{1ex}[0pt][0pt]{\sffamily{\@author}}}
\makeatother

\begin{document}

% sezione (data)
\section{Lezione del 25-09-24}

% stili pagina
\thispagestyle{empty}
\pagestyle{fancy}

% testo
\subsection{Assegnamento di costo minimo}
Vediamo un problema:
\begin{problem}{Assegnamento}
	Quattro agenzie possono occuparsi di 4 progetti.
	Ogni agenzia presenta il costo stimato per la realizzazione di ogni progetto, in migliaia di euro.
	In forma tabulare, si riportano i valori:

	\center \rowcolors{2}{green!10}{green!40!black!20}
	\begin{tabular} { | c | c | c | c | c | }
		\hline
		& \bfseries Agenzia 1 & \bfseries Agenzia 2 & \bfseries Agenzia 3 & \bfseries Agenzia 4 \\
		\hline 
		\bfseries Progetto 1 & 20 & 17 & 16 & 14 \\
		\bfseries Progetto 2 & 22 & 16 & 19 & 15 \\
		\bfseries Progetto 3 & 21 & 17 & 15 & 23 \\ 
		\bfseries Progetto 4 & 19 & 18 & 14 & 24 \\
		\hline
	\end{tabular}

	\par\bigskip
	\raggedright
	
	Vogliamo capire quale agenzia deve occuparsi di quale progetto per minimizzare i costi.

\end{problem}

Con $n$ agenzie e progetti ci sono $n!$ possibili combinazioni, ergo dobbiamo trovare un algoritmo efficiente. 
Applicando il modello studiato finora, abbiamo la matrice dei costi $c$:

$$
\begin{pmatrix}
 20 & 17 & 16 & 14 \\
 22 & 16 & 19 & 15 \\
 21 & 17 & 15 & 23 \\
 19 & 18 & 14 & 24 \\
\end{pmatrix}
$$

che possiamo portare a:
$$ c: ( -18, +18 + ... + 24 ) $$
come linearizzazione lessicografica della tabella sopra riportata (notare che sarebbe un vettore colonna).

Adesso dobbiamo solo trovare un metodo per esplicitare i vincoli del problema:
\begin{itemize}
	\item Ogni agenzia può occuparsi solo di un progetto;
	\item Ogni progetto richiede solo un'agenzia.
\end{itemize}

Possiamo rappresentare la corrispondenza fra elementi come un grafo, e quindi riportarne una matrice d'adiacenza.
Assumendo di appaiare elementi con lo stesso numero, avremo:
$$
\begin{pmatrix}
	1 & 0 & 0 & 0 \\ 
	0 & 1 & 0 & 0 \\ 
	0 & 0 & 1 & 0 \\ 
	0 & 0 & 0 & 1 \\ 
\end{pmatrix}
$$

La caratteristica di questa matrice, chiamiamola $x$, e che ogni elemento $x_{ij}$ è:
$$
x_{ij} = 
	\begin{cases}
		0 \\ 1	
	\end{cases}
$$

Decidiamo di trattare la $x$ come un vettore linearizzato lessicograficamente dalla matrice, proprio come avevamo fatto per il vettore costo.
Per una matrice di adiacenza $n \times n$, di $n$ elementi in ogni categoria, questo vettore ha dimensione $n^2$. 
Questo semplifica la notazione del problema, e sopratutto della matrice $A$, che sarebbe lasciando $x$ matrice effettivamente un tensore.

Si dimostra quindi facilmente che i vincoli riportati prima possono quindi esprimersi come:
\[
	\begin{cases}
		x_{11} + x_{12} + x_{13} + x_{14} = 1	\\
		x_{21} + x_{22} + x_{23} + x_{24} = 1 \\ 
		x_{31} + x_{32} + x_{33} + x_{34} = 1 \\ 
		x_{41} + x_{42} + x_{43} + x_{44} = 1 \\ 
	\end{cases}
\]

per il primo punto, e:
\[
	\begin{cases}
		x_{11} + x_{21} + x_{31} + x_{41} = 1	\\
		x_{12} + x_{22} + x_{32} + x_{42} = 1 \\ 
		x_{13} + x_{23} + x_{33} + x_{43} = 1 \\ 
		x_{14} + x_{24} + x_{34} + x_{44} = 1 \\ 
	\end{cases}
\]

per il secondo.
Imponendo la positività, si hanno quindi le matrici $A$ e $b$:

$$
A: 
\begin{pmatrix}
	1 & 1 & 1 & 1 & 0... & &  & & & ...0 \\
	0... & & ...0 & 1 & 1 & 1 & 1 & 0... & & ...0 \\
	... \\
	0... & &  & & & ...0 & 1 & 1 & 1 & 1 \\
	...\\
	-1 & 0... & & & & & & & & ...0 \\
	...\\
	0... & & & & & & & & & -1 \\
\end{pmatrix}, \quad 
b: ( 1, ..., 1, 0, ..., 0)
$$

Si nota che il numero di vincoli necessari per $n$ elementi è $2n + n^2$.

\subsubsection{Assegnamento cooperativo e non cooperativo}
A questo punto conviene fare una distinzione.
Abbiamo definito finora il modello:

\[
	\begin{cases}
		\min{c^T \cdot x} \\
		x_{11} + x_{12} + x_{13} + x_{14} = 1	\\
		...\\
		x_{41} + x_{42} + x_{43} + x_{44} = 1 \\
		x_{11} + x_{21} + x_{31} + x_{41} = 1	\\
		...\\
		x_{14} + x_{24} + x_{34} + x_{44} = 1 \\ 
 
	\end{cases}
\]

che così scritto non nega la possibilità di $x$ con componenti reali.
Nell'esempio ciò significa sono ammesse soluzioni dove più agenzie danno contributi reali ai progetti, che possiamo semanticamente interpretare come condividere il carico di lavoro, pur rispettando i vincoli imposti.
Decidiamo che questo è corretto se si parla di un problema di \textbf{assegnamento cooperativo}.
Visto che il problema posto non era di questo tipo, ma era di \textbf{assegnamento non cooperativo}, si introduce un'ulteriore vincolo:

$$
	x \in \mathbb{Z}^n
$$

Adesso ogni azienda darà un contributo intero al suo progetto, ergo coi vincoli imposti prima, ogni azienda sarà unica nel dirigere un solo progetto.

Più formalmente, possiamo dire che il passaggio ad assegnamento cooperativo comporta un \textbf{rilassamento} dei vincoli del problema.
Ovvero, in generale, se un problema non cooperativo ha minimo ottimale $nc$, il suo associato cooperativo avrà minimo ottimale $c$ con:

$$  c \leq nc $$

\subsubsection{Forma primale standard}
Portiamo quindi questo problema in una forma primale simile a quella vista per altri problemi LP, concesso il vincolo $x \in \mathbb{Z}^n$.

Finora avevamo usato le trasformazioni equivalenti per problemi LP:

\begin{enumerate}
	\item Trasformazione delle disuguaglianze: $ \geq \ \leftrightarrow \ \leq $
	\item Trasformazione delle uguaglianze:
		$$
			f(x) = c \ \rightarrow \
		\begin{cases}
			f(x) \leq c \\ 
			-f(x) \leq -c
		\end{cases}
		$$
	\item Trasformazione minimo / massimo: 
		$$
		\max{f} = -\min{f} \ \text{e sopratutto} \ 
		\bar{x} \in \mathrm{argmax}(f) \Leftrightarrow \bar{x} \in \mathrm{argmin}(-f)
		$$
\end{enumerate}

Possiamo applicare queste trasformazioni al modello, in particolare la (2), che porta il numero di vincoli a $4n + n^2$.

\subsection{Introduzione di surplus}
Vediamo un ulteriore tecnica per trasformare problemi LP: si può portare una disequazione del tipo:
$$ a_1x_1 + a_2x_2 + ... + a_nx_n \leq b $$
in un uguaglianza introducendo una variabile ausiliaria $s$:
$$ a_1x_1 + a_2x_2 + ... + a_nx_n + s = b $$
$s$ prende il nome di \textbf{slack}, in italiano scarto, o \textit{surplus}.
\end{document}
