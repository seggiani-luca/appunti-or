
\documentclass[a4paper,11pt]{article}
\usepackage[a4paper, margin=8em]{geometry}

% usa i pacchetti per la scrittura in italiano
\usepackage[french,italian]{babel}
\usepackage[T1]{fontenc}
\usepackage[utf8]{inputenc}
\frenchspacing 

% usa i pacchetti per la formattazione matematica
\usepackage{amsmath, amssymb, amsthm, amsfonts}

% usa altri pacchetti
\usepackage{gensymb}
\usepackage{hyperref}
\usepackage{standalone}

% imposta il titolo
\title{Appunti Ricerca Operativa}
\author{Luca Seggiani}
\date{2024}

% disegni
\usepackage{pgfplots}
\pgfplotsset{width=10cm,compat=1.9}

% imposta lo stile
% usa helvetica
\usepackage[scaled]{helvet}
% usa palatino
\usepackage{palatino}
% usa un font monospazio guardabile
\usepackage{lmodern}

\renewcommand{\rmdefault}{ppl}
\renewcommand{\sfdefault}{phv}
\renewcommand{\ttdefault}{lmtt}

% disponi il titolo
\makeatletter
\renewcommand{\maketitle} {
	\begin{center} 
		\begin{minipage}[t]{.8\textwidth}
			\textsf{\huge\bfseries \@title} 
		\end{minipage}%
		\begin{minipage}[t]{.2\textwidth}
			\raggedleft \vspace{-1.65em}
			\textsf{\small \@author} \vfill
			\textsf{\small \@date}
		\end{minipage}
		\par
	\end{center}

	\thispagestyle{empty}
	\pagestyle{fancy}
}
\makeatother

% disponi teoremi
\usepackage{tcolorbox}
\newtcolorbox[auto counter, number within=section]{theorem}[2][]{%
	colback=blue!10, 
	colframe=blue!40!black, 
	sharp corners=northwest,
	fonttitle=\sffamily\bfseries, 
	title=Teorema~\thetcbcounter: #2, 
	#1
}

% disponi definizioni
\newtcolorbox[auto counter, number within=section]{definition}[2][]{%
	colback=red!10,
	colframe=red!40!black,
	sharp corners=northwest,
	fonttitle=\sffamily\bfseries,
	title=Definizione~\thetcbcounter: #2,
	#1
}

% disponi problemi
\newtcolorbox[auto counter, number within=section]{problem}[2][]{%
	colback=green!10,
	colframe=green!40!black,
	sharp corners=northwest,
	fonttitle=\sffamily\bfseries,
	title=Problema~\thetcbcounter: #2,
	#1
}

% disponi codice
\usepackage{listings}
\usepackage[table]{xcolor}

\lstdefinestyle{codestyle}{
		backgroundcolor=\color{black!5}, 
		commentstyle=\color{codegreen},
		keywordstyle=\bfseries\color{magenta},
		numberstyle=\sffamily\tiny\color{black!60},
		stringstyle=\color{green!50!black},
		basicstyle=\ttfamily\footnotesize,
		breakatwhitespace=false,         
		breaklines=true,                 
		captionpos=b,                    
		keepspaces=true,                 
		numbers=left,                    
		numbersep=5pt,                  
		showspaces=false,                
		showstringspaces=false,
		showtabs=false,                  
		tabsize=2
}

\lstdefinestyle{shellstyle}{
		backgroundcolor=\color{black!5}, 
		basicstyle=\ttfamily\footnotesize\color{black}, 
		commentstyle=\color{black}, 
		keywordstyle=\color{black},
		numberstyle=\color{black!5},
		stringstyle=\color{black}, 
		showspaces=false,
		showstringspaces=false, 
		showtabs=false, 
		tabsize=2, 
		numbers=none, 
		breaklines=true
}

\lstdefinelanguage{javascript}{
	keywords={typeof, new, true, false, catch, function, return, null, catch, switch, var, if, in, while, do, else, case, break},
	keywordstyle=\color{blue}\bfseries,
	ndkeywords={class, export, boolean, throw, implements, import, this},
	ndkeywordstyle=\color{darkgray}\bfseries,
	identifierstyle=\color{black},
	sensitive=false,
	comment=[l]{//},
	morecomment=[s]{/*}{*/},
	commentstyle=\color{purple}\ttfamily,
	stringstyle=\color{red}\ttfamily,
	morestring=[b]',
	morestring=[b]"
}

% disponi sezioni
\usepackage{titlesec}

\titleformat{\section}
	{\sffamily\Large\bfseries} 
	{\thesection}{1em}{} 
\titleformat{\subsection}
	{\sffamily\large\bfseries}   
	{\thesubsection}{1em}{} 
\titleformat{\subsubsection}
	{\sffamily\normalsize\bfseries} 
	{\thesubsubsection}{1em}{}

% disponi alberi
\usepackage{forest}

\forestset{
	rectstyle/.style={
		for tree={rectangle,draw,font=\large\sffamily}
	},
	roundstyle/.style={
		for tree={circle,draw,font=\large}
	}
}

% disponi algoritmi
\usepackage{algorithm}
\usepackage{algorithmic}
\makeatletter
\renewcommand{\ALG@name}{Algoritmo}
\makeatother

% disponi numeri di pagina
\usepackage{fancyhdr}
\fancyhf{} 
\fancyfoot[L]{\sffamily{\thepage}}

\makeatletter
\fancyhead[L]{\raisebox{1ex}[0pt][0pt]{\sffamily{\@title \ \@date}}} 
\fancyhead[R]{\raisebox{1ex}[0pt][0pt]{\sffamily{\@author}}}
\makeatother

\begin{document}

% sezione (data)
\section{Lezione del 03-12-24}

% stili pagina
\thispagestyle{empty}
\pagestyle{fancy}

% testo
\subsection{Analisi locale}
Vediamo come applicare i metodi visti finora per ricavare massimi, minimi e punti di sella di due problemi di ottimizzazione vincolata.
\begin{enumerate}
	\item Sia la funzione:
$$
f(x_1, x_2) = (x_1 - 1)^2 + x_2^2
$$
sottoposta al vincolo:
$$
g_1(x_1, x_2) = x_1 - x_2^2
$$
Dai gradienti:
$$
\nabla f(x_1, x_2) = (2x_1 - 2, 2x_2)
$$
$$
\nabla g_1(x_1, x_2) = (1, -2x_2)
$$
si ricava il sistema LKT come:
\[
	\begin{cases}
		2x_1 - 2 + \lambda_1 = 0 \\ 
		2x_2 - 2 \lambda_1 x_2 = 0 \\ 
		\lambda_1 (x_1 - x_2^2) = 0
	\end{cases}
\]

Le ultime due equazioni del sistema determinano completamente i valori di $\lambda_1$.
Abbiamo infatti tre casi:
\begin{itemize}
	\item $\lambda_1 = 1 \implies x_1 - x_2^2 = 0$;
	\item $\lambda_1 = 0 \implies x_2 = 0$;
	\item $\lambda_1 = 2 \implies x_1, x_2 = 0$.
\end{itemize}
Non sono possibili altre combinazioni in quanto la seconda equazione vincola $\lambda_1$ a $1$ quando $x_2 \neq 0$, l'ultima vincola $x_1 = x_2^2$ quando $\lambda_1 \neq 0$, e infine $\lambda_1 = 2$ è forzato quando $x_1 = x_2= 0$ dalla prima equazione.
Le coppie punti - moltiplicatori sono quindi:
\begin{table}[h!]
	\center 
	\begin{tabular} { c c | c }
		$x_1$ & $x_2$ & $\lambda_1$ \\
		\hline 
		$1$ & $0$ & $0$ \\ 
		$\frac{1}{2}$ & $\frac{\sqrt{2}}{2}$ & $1$ \\ 
		$\frac{1}{2}$ & $-\frac{\sqrt{2}}{2}$ & $1$ \\ 
		$0$ & $0$ & $2$
	\end{tabular}
\end{table}

Notiamo subito che il primo punto è inammissibile in quanto $1 - 0 > 0$, e quindi non si rispetta $g_1$.
Restano quindi 3 punti, che per enumerazione diretta danno:
$$
f\left(\frac{1}{2}, \frac{\sqrt{2}}{2}\right) = \frac{3}{4}
$$
$$
f\left(\frac{1}{2}, -\frac{\sqrt{2}}{2}\right) = \frac{3}{4}
$$
$$
f\left(0, 0\right) = 1
$$
da cui risulta subito che $\left(\frac{1}{2}, \frac{\sqrt{2}}{2}\right)$ e $\left(\frac{1}{2}, -\frac{\sqrt{2}}{2}\right)$ sono punti di minimo globale.

Per valutare il punto in $(0,0)$ possiamo prendere due percorsi, ad esempio $\phi_1(t) = (t^2, t)$ e $\phi_2(t) = (t, 0)$.
Si ha:
\begin{itemize}
	\item Per il primo percorso, $\phi_1(t) = (t^2, t)$, cioè il bordo della parabola del vincolo $g_1$: 
		$$
		f(\phi_1(t)) = f(t^2, t) = (t^2 -1)^2 + t^2 = t^4 - 2t^2 + 1 + t^2 = t^4 -t^2 + 1
		$$
		da cui derivando:
		$$
		D \left[ f(t^2, t) \right] = 4t^3 - 2t = t(4t^2 -2)
		$$
		si ottengono i punti stazionari $t = 0$, $t = \pm \frac{\sqrt{2}}{2}$, di cui quello in $t = 0$ è \textbf{massimo} (fra l'altro si ricavano anche i minimi locali, che stanno comunque sul bordo).
	\item Per il secondo percorso, $\phi_2(t) = (t, 0)$ cioè l'asse delle $x$, dobbiamo imporre il vincolo $g_1$:
		$$
		(x_1, x_2) \leftarrow (t, 0), \ x_1 - x_2^2 \leq 0 \implies t < 0
		$$
		Studiamo quindi:
		$$
		f(\phi_2(t)) = (t - 1)^2 = t^2 - 2t + 1
		$$
		cioè la parabola di vertice a coordinata $t_0 = 1$.
		Essendo $1 \geq 0$ e il segno del termine di secondo grado positivo, si ha che il punto in $(0,0)$ è \textbf{minimo}.
\end{itemize}

Visto che considerando due percorsi diversi per approcciare il punto $(0,0)$ abbiamo ottenuto un massimo e un minimo, il punto è una \textbf{sella}.

Un'interpretazione secondaria è che la funzione obiettivo è un paraboloide di vertice in $(1, 0)$, definito sulla regione \textit{"esterna"} della parabola $x^2 = x_1$.
Sarà allora che gli insiemi di livello della $f$ sono circonferenze concentriche di centro $(1,0)$, cioè nella forma generale:
$$
(x - 1)^2 + y^2 = r^2
$$

A raggi $r$ minori, valori minori della funzione.
Siamo quindi interessati a trovare i valori più vicini al punto $(1, 0)$ fra quelli definiti del dominio.
Calcolando le norme dei punti dati dall'LKT si ha:
$$
\left|(0,1) - \left( \frac{1}{2}, \frac{\sqrt{2}}{2}\right) \right| = \sqrt{\left(\frac{1}{2}\right)^2 + \left(\frac{\sqrt{2}}{2}\right)^2} = \sqrt{\frac{3}{4}}
$$
$$
\left|(0,1) - \left( \frac{1}{2}, -\frac{\sqrt{2}}{2}\right) \right| = \sqrt{\left(\frac{1}{2}\right)^2 + \left(-\frac{\sqrt{2}}{2}\right)^2} = \sqrt{\frac{3}{4}}
$$
$$
|(0,1) - (0,0)| = 1
$$
cioè i primi due punti sono i più vicini a $(1,0)$, e quindi necessariamente i minimi globali.

Si riporta un grafico con la circonferenza a raggio $\sqrt{\frac{3}{4}}$:
\begin{center}
\begin{tikzpicture}
	\begin{axis}[
			axis lines = center,
			xlabel = $x$, ylabel = $y$,
			width=14cm,
			height=8cm,
			xmin = 0,
			xmax = 5.9,
			ymin = -1.4,
			ymax = 1.4,
			grid = minor
	]
	\addplot[
		samples=100
	]({x^2}, x);

	\addplot[
		domain=0:360,
		samples=100, 
		thick,
		color=red,
	] ({sqrt(3/4)*cos(x) + 1}, {sqrt(3/4)*sin(x)});
	
	\addplot[
		only marks,
		mark=*,
		mark size = 2pt,
		] coordinates{({1/2}, {sqrt(1/2)})};
	
	\addplot[
		only marks,
		mark=*,
		mark size = 2pt,
	] coordinates{({1/2}, {-sqrt(1/2)})};

	\node[anchor=south, yshift=5pt] at (axis cs:{1/2},{sqrt(1/2)}) {$(\frac{1}{2}, \frac{\sqrt{2}}{2})$};
	\node[anchor=north, yshift=-10pt] at (axis cs:{1/2},{-sqrt(1/2)}) {$(\frac{1}{2}, -\frac{\sqrt{2}}{2})$};

	\end{axis}
\end{tikzpicture}
\end{center}
	\item  Sia la funzione:
$$
f(x_1, x_2) = x_1 + x_2
$$
sottoposta ai vincoli di diseguaglianza e uguaglianza:
\[
	\begin{cases}
		g_1(x_1, x_2) = x_1^2 + x_2^2 - 2 \\
		h_1(x_1, x_2) = x_2^2 - x_1
	\end{cases}
\]
Dai gradienti:
$$
\nabla f(x_1, x_2) = (1, 1)
$$
$$
\nabla g_1(x_1, x_2) = (2x_1, 2x_2)
$$
$$
\nabla h_1(x_1, x_2) = (-1, 2x_2)
$$
si ricava il sistema LKT come:
\[
	\begin{cases}
		1 + 2 \lambda_1 x_1 - \mu_1 = 0 \\ 
		1 + 2 \lambda_1 x_2 - 2 \mu_1 x_2 = 0\\ 
		\lambda_1 (x_2^2 - x_1) = 0
	\end{cases}
\]

Il sistema si risolve, in maniera non banale, imponendo separatamente o entrambi i vincoli ($\lambda_1$ e $\mu_1 \neq 0$) o il solo vincolo di uguaglianza ($\lambda = 0$), ottendendo:
\begin{table}[h!]
	\center 
	\begin{tabular} { c c | c c }
		$x_1$ & $x_2$ & $\lambda_1$ & $\mu_1$ \\
		\hline 
		$1$ & $1$ & $\frac{1}{2}$ & $0$ \\ 
		$1$ & $-1$ & $-\frac{1}{6}$ & $\frac{2}{3}$ \\ 
		$\frac{1}{4}$ & $-\frac{1}{2}$ & $0$ & $1$
 \end{tabular}
\end{table}

Un'interpretazione geometrica più immediata si può avere valutando i vincoli: $g_1$ definisce i punt interni alla circonferenza centrata sull'origine di raggio $\sqrt{2}$, mentre $h_1$ definisce la parabola con asse sull'asse x dell'esempio precedente.
Vogliamo prendere le intersezioni dei bordi dei domini (i primi due punti dei moltiplicatori, e il punto dove la parabola è perpendicolare al gradiente di $f$ (l'ultimo punto).
Si riporta un grafico:
\begin{center}
\begin{tikzpicture}
	\begin{axis}[
			axis lines = center,
			xlabel = $x$, ylabel = $y$,
			width=14cm,
			height=8cm,
			xmin = -2,
			xmax = 6.9,
			ymin = -2.4,
			ymax = 2.4,
			grid = minor
	]
	\addplot[
		samples=100
	]({x^2}, x);

	\addplot[
		domain=0:360,
		samples=100, 
		thick,
		color=red,
	] ({sqrt(2)*cos(x)}, {sqrt(2)*sin(x)});
	
	\addplot[
		only marks,
		mark=*,
		mark size = 2pt,
		] coordinates{(1,1)};
	
	\addplot[
		only marks,
		mark=*,
		mark size = 2pt,
		] coordinates{(1,-1)};

	\node[anchor=south, yshift=5pt] at (axis cs:1,1) {$(1,1)$};
	\node[anchor=north, yshift=-10pt] at (axis cs:1,-1) {$(1,-1)$};

	\end{axis}
\end{tikzpicture}

\end{center}
Da qui in poi, per valutazione diretta, si ottiene:
$$
f(1,1) = 2
$$
$$
f(1, -1) = 0
$$
$$
f\left(\frac{1}{4}, \frac{1}{2} \right) = -\frac{1}{4}
$$
da cui massimo in $(1,1)$ e minimo in $(\frac{1}{4}, \frac{-1}{2})$ (come poteva essere chiaro anche tracciando gli insiemi di livello di $f$).
Il punto in $(1, -1)$ può essere valutato solo sulla parabola stessa, dove risulta sicuramente crescente (il minimo è toccato in un unico punto).
Si ha quindi che è un massimo locale.

Un approccio alternativo al problema può essere poi quello di parametrizzare, ad esempio attraverso una funzione $\phi(t)$ che traccia la parabola sul piano $x,y$, su cui imponiamo il vincolo della $g_1$:
$$
\phi(t) = (t^2, t) \implies f(\phi(t)) = t^2 + t
$$
dove si nota che i limiti del dominio sono sempre i punti $(1,1)$ e $(1, -1)$.
Con $\phi(1) = (1,1)$ e $\phi(-1) = (1, -2)$, avremo che il dominio di definizione di $\phi$ (e quindi di $f \circ \phi$) è $[-1, 1]$.
Possiamo quindi studiare la funzione su tale dominio per ottenere gli stessi punti di massimo, globale e locale, e minimo globale.


\end{enumerate}

\subsection{Algoritmi di PNL}
Sia $f: \mathbb{R}^n \rightarrow \mathbb{R}$: vogliamo definire algoritmi che ricavino, ad esempio, $\max\limits_{x \in \mathbb{R}^n} f(x)$.
L'idea fondamentale è quella di creare una \textbf{relazione di ricorrenza}, cioè una successione di iterazioni che converga verso il punto di massimo.

\subsubsection{Direzioni di crescita}
Partiamo quindi da una certa $x_k$ e convergiamo verso una $x_{k+n}$ corrispondente al punto desiderato.
La domanda è come calcolare $x_{k+1}$.
Scegliamo quindi una direzione di spostamento $d_k$ e modifichiamo $x_k$ come:
$$
x_{k + 1} = x_k + t \, d_k 
$$
dove $t$ è detto \textit{step size} o \textbf{passo}, ed è un parametro $\geq 0$ che determina la \textit{velocità} di convergenza dell'algoritmo (vedremo però che $t$ sempre più grandi non sono sempre ottimi, in quanto potrebbero essere meno precisi se non addirittura oltrepassare completamente il punto di massimo).

Sostituendo la successione in $f(x)$ si ottiene:
$$
f(x_k + t \, d_k) = \phi(t)
$$
che corrisponde, per $x \in \mathbb{R}^n$, ad un \textit{taglio} della funzione su $n$ variabili lungo una curva determinata dai $d_k$ e campionata ad intervalli $t$.

Ci chiediamo quindi come scegliere i $d_k$ cioè le \textbf{direzioni di crescita}: scorrendo lungo queste direzioni ci aspettiamo che, almeno localmente, la funzione salga.
Notiamo che, ad esempio, nella PL questa era stata la colonna corrispondente a vincolo uscente della matrice $W = -A_B^{-1}$. 
Su funzioni nonlineari arbitrarie, questa sarà invece il \textbf{gradiente} $\nabla f$ della funzione calcolato in $x_k$.
Finchè questo sarà $\neq 0$, la $\phi(t)$ sarà monotona crescente per $t$ successivi, cioè per ogni $x_k$ varrà $x_k \leq x_{k+1}$.

Notiamo che il gradiente $\nabla f \geq 0$ vale come direzione di crescita per la $\phi(t)$.
Applicando la regola della derivazione a catena sulla variabile $t$ abbiamo infatti:
$$
\nabla \phi(t) = \nabla f(x_k + t \, d_k) = \nabla f(x_k + t \, d_k) \cdot \mathrm{D} \left[x_k + t \, d_k\right] = \nabla f(x_k + t \, d_k) \cdot  d_k
$$

Ricaviamo quindi che, se $d_k$ è tale che $\nabla f(x_k) \cdot \, d_k > 0$, allora $d_k$ è una direzione di crescita.
Una direzione che sicuramente ha prodotto scalare positivo col gradiente sarà il gradiente stesso, da cui quanto sopra, con:
$$
\nabla \phi(t) = \nabla f(x_k + t \, d_k) \cdot \, d_k = |\nabla f(x_k + t \, d_k)|^2
$$

\par\smallskip 

Notiamo che una direzione di sallita non resta tale per ogni valore di $t$, bensì rappresenta effettivamente un accrescimento della funzione solo per un certo intervallo, cioè:
$$
\exists \bar{t}, \ \forall t \in (0, \bar{t}] : f(x_k + t \, d_k) > f(x_k) 
$$

Possiamo riprendere il conto precedente per determinare se una direzione è di crescita.
In particolare, se $ \nabla f(x_k) \cdot d_k > 0$, allora $d_k$ è di crescita.
Notiamo ancora che $d_k = \nabla f(x_k)$ soddisfa questa disequazione, quindi è una direzione di crescita.

\end{document}
