
\documentclass[a4paper,11pt]{article}
\usepackage[a4paper, margin=8em]{geometry}

% usa i pacchetti per la scrittura in italiano
\usepackage[french,italian]{babel}
\usepackage[T1]{fontenc}
\usepackage[utf8]{inputenc}
\frenchspacing 

% usa i pacchetti per la formattazione matematica
\usepackage{amsmath, amssymb, amsthm, amsfonts}

% usa altri pacchetti
\usepackage{gensymb}
\usepackage{hyperref}
\usepackage{standalone}

% imposta il titolo
\title{Appunti Ricerca Operativa}
\author{Luca Seggiani}
\date{2024}

% disegni
\usepackage{pgfplots}
\pgfplotsset{width=10cm,compat=1.9}

% imposta lo stile
% usa helvetica
\usepackage[scaled]{helvet}
% usa palatino
\usepackage{palatino}
% usa un font monospazio guardabile
\usepackage{lmodern}

\renewcommand{\rmdefault}{ppl}
\renewcommand{\sfdefault}{phv}
\renewcommand{\ttdefault}{lmtt}

% disponi il titolo
\makeatletter
\renewcommand{\maketitle} {
	\begin{center} 
		\begin{minipage}[t]{.8\textwidth}
			\textsf{\huge\bfseries \@title} 
		\end{minipage}%
		\begin{minipage}[t]{.2\textwidth}
			\raggedleft \vspace{-1.65em}
			\textsf{\small \@author} \vfill
			\textsf{\small \@date}
		\end{minipage}
		\par
	\end{center}

	\thispagestyle{empty}
	\pagestyle{fancy}
}
\makeatother

% disponi teoremi
\usepackage{tcolorbox}
\newtcolorbox[auto counter, number within=section]{theorem}[2][]{%
	colback=blue!10, 
	colframe=blue!40!black, 
	sharp corners=northwest,
	fonttitle=\sffamily\bfseries, 
	title=Teorema~\thetcbcounter: #2, 
	#1
}

% disponi definizioni
\newtcolorbox[auto counter, number within=section]{definition}[2][]{%
	colback=red!10,
	colframe=red!40!black,
	sharp corners=northwest,
	fonttitle=\sffamily\bfseries,
	title=Definizione~\thetcbcounter: #2,
	#1
}

% disponi problemi
\newtcolorbox[auto counter, number within=section]{problem}[2][]{%
	colback=green!10,
	colframe=green!40!black,
	sharp corners=northwest,
	fonttitle=\sffamily\bfseries,
	title=Problema~\thetcbcounter: #2,
	#1
}

% disponi codice
\usepackage{listings}
\usepackage[table]{xcolor}

\lstdefinestyle{codestyle}{
		backgroundcolor=\color{black!5}, 
		commentstyle=\color{codegreen},
		keywordstyle=\bfseries\color{magenta},
		numberstyle=\sffamily\tiny\color{black!60},
		stringstyle=\color{green!50!black},
		basicstyle=\ttfamily\footnotesize,
		breakatwhitespace=false,         
		breaklines=true,                 
		captionpos=b,                    
		keepspaces=true,                 
		numbers=left,                    
		numbersep=5pt,                  
		showspaces=false,                
		showstringspaces=false,
		showtabs=false,                  
		tabsize=2
}

\lstdefinestyle{shellstyle}{
		backgroundcolor=\color{black!5}, 
		basicstyle=\ttfamily\footnotesize\color{black}, 
		commentstyle=\color{black}, 
		keywordstyle=\color{black},
		numberstyle=\color{black!5},
		stringstyle=\color{black}, 
		showspaces=false,
		showstringspaces=false, 
		showtabs=false, 
		tabsize=2, 
		numbers=none, 
		breaklines=true
}

\lstdefinelanguage{javascript}{
	keywords={typeof, new, true, false, catch, function, return, null, catch, switch, var, if, in, while, do, else, case, break},
	keywordstyle=\color{blue}\bfseries,
	ndkeywords={class, export, boolean, throw, implements, import, this},
	ndkeywordstyle=\color{darkgray}\bfseries,
	identifierstyle=\color{black},
	sensitive=false,
	comment=[l]{//},
	morecomment=[s]{/*}{*/},
	commentstyle=\color{purple}\ttfamily,
	stringstyle=\color{red}\ttfamily,
	morestring=[b]',
	morestring=[b]"
}

% disponi sezioni
\usepackage{titlesec}

\titleformat{\section}
	{\sffamily\Large\bfseries} 
	{\thesection}{1em}{} 
\titleformat{\subsection}
	{\sffamily\large\bfseries}   
	{\thesubsection}{1em}{} 
\titleformat{\subsubsection}
	{\sffamily\normalsize\bfseries} 
	{\thesubsubsection}{1em}{}

% disponi alberi
\usepackage{forest}

\forestset{
	rectstyle/.style={
		for tree={rectangle,draw,font=\large\sffamily}
	},
	roundstyle/.style={
		for tree={circle,draw,font=\large}
	}
}

% disponi algoritmi
\usepackage{algorithm}
\usepackage{algorithmic}
\makeatletter
\renewcommand{\ALG@name}{Algoritmo}
\makeatother

% disponi numeri di pagina
\usepackage{fancyhdr}
\fancyhf{} 
\fancyfoot[L]{\sffamily{\thepage}}

\makeatletter
\fancyhead[L]{\raisebox{1ex}[0pt][0pt]{\sffamily{\@title \ \@date}}} 
\fancyhead[R]{\raisebox{1ex}[0pt][0pt]{\sffamily{\@author}}}
\makeatother

\begin{document}

% sezione (data)
\section{Lezione del 30-10-24}

% stili pagina
\thispagestyle{empty}
\pagestyle{fancy}

% testo
\subsection{Problema di copertura}
\begin{problem}{di copertura}
	Supponiamo che l'ASL debba dislocare ambulanze su 5 sedi distribuite sul territorio, in 9 diverse località.
	La matrice di adiacenza fra sedi e località è la seguente:

	\center \rowcolors{2}{green!10}{green!40!black!20}
	\begin{tabular} { | c | c | c | c | c | c | }
		\hline
		& \bfseries Sede 1 & \bfseries Sede 2 & \bfseries Sede 3 & \bfseries Sede 4 & \bfseries Sede 5\\
		\hline 
		\bfseries Località 1 & 1 & 0 & 1 & 0 & 1 \\
		\bfseries Località 2 & 0 & 1 & 0 & 1 & 0 \\
		\bfseries Località 3 & 1 & 1 & 0 & 0 & 0 \\
		\bfseries Località 4 & 0 & 1 & 1 & 0 & 1 \\
		\bfseries Località 5 & 0 & 0 & 0 & 0 & 1 \\
		\bfseries Località 6 & 1 & 1 & 1 & 1 & 1 \\
		\bfseries Località 7 & 1 & 1 & 0 & 0 & 1 \\
		\bfseries Località 8 & 0 & 1 & 1 & 1 & 0 \\
		\bfseries Località 9 & 0 & 0 & 1 & 0 & 1 \\
		\hline
	\end{tabular}	
	\par\bigskip
	\raggedright
	Vogliamo capire in quali sedi dovremo dislocare le ambulanze in modo da coprire tutte le località.
\end{problem}
Il problema è risolvibile attraverso la ILP.
Possiamo assumere che la matrice di adiacenza sia stata ricavata da qualche matrice dei tempi (rappresentante il tempo necessario a raggiungere una località), e quindi tagliata su un tempo massimo di, ad esempio, 20 minuti, con il risultato che le località a meno di 20 minuti di distanza risultano connesse e quelle a più di 20 minuti disconnesse.

Possiamo quindi definire alcune \textbf{regole di riduzione} per semplificare la  matrice di adiacenza:
\begin{enumerate}
	\item Le righe di soli zeri ($0, ..., 0$) possono essere eliminate in quanto rappresentano una soluzione impossibile del problema;
	\item Le righe di soli uni ($1, ..., 1$) possono essere eliminate in quanto possono essere risolte da qualsiasi servizio; 
	\item Se una riga contiene un solo 1, apro un "servizio" per quella riga (cioè gli dedico una delle colonne), la elimino, e elimino tutte le righe che hanno 1 sulla stessa colonna.
	\item Se per due colonne $r$ e $k$, $r_{ij} \geq k_{kj}$ per ogni riga, allora si diche che $r$ \textbf{domina} $k$, e quindi che $k$ può essere scartata.
		Questo vale solo se l'apertura è a costo costante su tutte le colonne: in caso contrario potremmo scartare opzioni viabili di ottimizzazione. 
\end{enumerate}

	Vogliamo capire quali località dovranno essere servite da quale sede in modo da coprirle tutte. 
Si ricavano quindi i vincoli su ogni riga:
\[
	\begin{cases}
		x_1 + x_3 + x_5 \geq 1 \\
		x_2 + x_4 \geq 1 \\ 
		... \\
		x_3+x_5 \geq 1 \\ 
		x_i \in \{ 0, 1 \}
	\end{cases}
\]

ergo si ricava un problema di ILP in forma:
\[
	\begin{cases}
		\min c^T \cdot x \\ 
		Ax \geq 1 \\ 
		x \in \{ 0, 1 \}
	\end{cases}
\]

dove $c$ rappresenta un vettore costo, nel problema riportato assunto come costante.

Quello che facciamo effettivamente è scegliere fra $n$ sottoinsiemi, scegliendo l'insieme minimo di questi per coprire l'interezza degli elementi (detti anche nodi).

\subsubsection{Valutazioni inferiori e superiori}
Vediamo quali valutazioni possiamo usare per trovare soluzioni approssimate. 
\begin{itemize}
	\item \textbf{Valutazione inferiore:} una valutazione inferiore $v_I$ può essere agevolmente calcolata dal rilassato continuo del problema ILP:
		\[
			\begin{cases}
				\min c^T \cdot x \\ 
				Ax \geq 1 \\ 
				0 \leq x \leq 1
			\end{cases}
		\]
	\item \textbf{Valutazione superiore:} si può applicare un algoritmo greedy per il calcolo di una valutazione superiore, assunti costi costanti:
\begin{algorithm}[H]
\caption{di copertura a costi costanti}
\begin{algorithmic}
	\STATE \textbf{Input:} un problema di copertura % input
	\STATE \textbf{Output:} una valutazione superiore $v_S$ % output
	% body
	\STATE \textsf{ciclo:}
	\STATE $r_i \leftarrow$ la somma dei valori su ogni colonna
	\STATE Rimuovi la colonna con $r_i$ maggiore, scegli quella colonna nella soluzione, e rimuovi le righe servite da quella colonna
	\STATE Torna a \textsf{ciclo}
\end{algorithmic}
\end{algorithm}
Notiamo che $r_i$ rappresenta effettivamente un rendimento, come avevamo visto nei problemi di zaino.
Possiamo infatti applicare un'altro algoritmo, più sofisticato, nel caso dei costi variabili:
	\begin{algorithm}[H]
\caption{di Chvatal}
\begin{algorithmic}
	\STATE \textbf{Input:} un problema di copertura % input
	\STATE \textbf{Output:} una valutazione superiore $v_S$% output
	% body
	\STATE \textsf{ciclo:}
	\STATE $r_i \leftarrow$ il rapporto fra il costo di ogni colonna e la somma dei valori su quella colonna
	\STATE Rimuovi la colonna con $r_i$ maggiore, scegli quella colonna nella soluzione, e rimuovi le righe servite da quella colonna
	\STATE Torna a \textsf{ciclo}
\end{algorithmic}
\end{algorithm}
\end{itemize}

\subsubsection{Problemi di massima copertura}
Supponiamo di avere una stima del valore (che possiamo assimilare alla domanda, o al guadagno associato all'apertura di un servizio su un certo nodo) su ogni riga (nell'esempio precedente di ogni località), e di voler prediligere soluzioni che coprono righe con valore maggiore, supponendo di avere un numero limitato di $k$ risorse (in questo caso ambulanze) a disposizione.
Prendiamo ad esempio la tabella:

\begin{table}[h!]
	\center \rowcolors{2}{white}{black!10}
	\begin{tabular} {  c || c || c | c | c | c | c  }
		& \bfseries Abitanti & \bfseries Sede 1 & \bfseries Sede 2 & \bfseries Sede 3 & \bfseries Sede 4 & \bfseries Sede 5\\
		\hline 
		\bfseries Località 1 & 2000 & 1 & 0 & 1 & 0 & 1 \\
		\bfseries Località 2 & 1000 & 0 & 1 & 0 & 1 & 0 \\
		\bfseries Località 3 & 800 & 1 & 1 & 0 & 0 & 0 \\
		\bfseries Località 4 & 3000 & 0 & 1 & 1 & 0 & 1 \\
		\bfseries Località 5 & 2500 & 0 & 0 & 0 & 0 & 1 \\
		\bfseries Località 6 & 1200 & 1 & 1 & 1 & 1 & 1 \\
		\bfseries Località 7 & 1700 & 1 & 1 & 0 & 0 & 1 \\
		\bfseries Località 8 & 400 & 0 & 1 & 1 & 1 & 0 \\
		\bfseries Località 9 & 150 & 0 & 0 & 1 & 0 & 1 \\
	\end{tabular}	
\end{table}

Il problema potrebbe essere di assegnare ambulanze alle sedi che offrono copertura della maggior popolazione possibile. 
Formuleremo allora il problema di ILP:
\[
	\begin{cases}
		\max h_1 z_1 + ... + h_9 z_9 \\
		x_1 + x_3 + x_5 \geq z_1 \\
		x_2 + x_4 \geq z_2 \\ 
		... \\ 
		x_3 + x_5 \geq z_9 \\
		x_1 + x_2 + x_3 + x_4 = k \\ 
		x \in \{ 0, 1 \} \\ 
		z \in \{ 0, 1 \}
	\end{cases}
\] 

Questo rappresenta un \textbf{problema di massima copertura}.
I coefficienti $h_1, ..., h_n$ sono i valori associati ad ogni nodo (qui ad ogni località): la funzione obiettivo risponde meglio all'apertura dei servizi ai nodi con valore alto.
I successivi $n$ vincoli rappresentano il fatto che, presa un nodo $z$, cioè una riga, dobbiamo rendere uguale a 1 almeno una fra le colonne che la servono, cioè dobbiamo aprirgli almeno un servizio.
Infine, il vincolo $x_1 + ... + x_n =k$ rappresenta il fatto che vogliamo prendere $k$ servizi, cioè colonne (nel nostro caso distribuire $k$ ambulanze).

Quello che facciamo, effettivamente, è scegliere fra i sottoinsiemi $x$ per massimizzare la copertura sugli elementi $z$, massimizzando il guadagno ottenuto in base alla scelta degli $z$ (scelti gli $z$ i successivi vincoli determinano quali $x$ dobbiamo aprire per servirli).

\end{document}
