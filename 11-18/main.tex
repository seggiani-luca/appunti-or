
\documentclass[a4paper,11pt]{article}
\usepackage[a4paper, margin=8em]{geometry}

% usa i pacchetti per la scrittura in italiano
\usepackage[french,italian]{babel}
\usepackage[T1]{fontenc}
\usepackage[utf8]{inputenc}
\frenchspacing 

% usa i pacchetti per la formattazione matematica
\usepackage{amsmath, amssymb, amsthm, amsfonts}

% usa altri pacchetti
\usepackage{gensymb}
\usepackage{hyperref}
\usepackage{standalone}

% imposta il titolo
\title{Appunti Ricerca Operativa}
\author{Luca Seggiani}
\date{2024}

% disegni
\usepackage{pgfplots}
\pgfplotsset{width=10cm,compat=1.9}

% imposta lo stile
% usa helvetica
\usepackage[scaled]{helvet}
% usa palatino
\usepackage{palatino}
% usa un font monospazio guardabile
\usepackage{lmodern}

\renewcommand{\rmdefault}{ppl}
\renewcommand{\sfdefault}{phv}
\renewcommand{\ttdefault}{lmtt}

% disponi il titolo
\makeatletter
\renewcommand{\maketitle} {
	\begin{center} 
		\begin{minipage}[t]{.8\textwidth}
			\textsf{\huge\bfseries \@title} 
		\end{minipage}%
		\begin{minipage}[t]{.2\textwidth}
			\raggedleft \vspace{-1.65em}
			\textsf{\small \@author} \vfill
			\textsf{\small \@date}
		\end{minipage}
		\par
	\end{center}

	\thispagestyle{empty}
	\pagestyle{fancy}
}
\makeatother

% disponi teoremi
\usepackage{tcolorbox}
\newtcolorbox[auto counter, number within=section]{theorem}[2][]{%
	colback=blue!10, 
	colframe=blue!40!black, 
	sharp corners=northwest,
	fonttitle=\sffamily\bfseries, 
	title=Teorema~\thetcbcounter: #2, 
	#1
}

% disponi definizioni
\newtcolorbox[auto counter, number within=section]{definition}[2][]{%
	colback=red!10,
	colframe=red!40!black,
	sharp corners=northwest,
	fonttitle=\sffamily\bfseries,
	title=Definizione~\thetcbcounter: #2,
	#1
}

% disponi problemi
\newtcolorbox[auto counter, number within=section]{problem}[2][]{%
	colback=green!10,
	colframe=green!40!black,
	sharp corners=northwest,
	fonttitle=\sffamily\bfseries,
	title=Problema~\thetcbcounter: #2,
	#1
}

% disponi codice
\usepackage{listings}
\usepackage[table]{xcolor}

\lstdefinestyle{codestyle}{
		backgroundcolor=\color{black!5}, 
		commentstyle=\color{codegreen},
		keywordstyle=\bfseries\color{magenta},
		numberstyle=\sffamily\tiny\color{black!60},
		stringstyle=\color{green!50!black},
		basicstyle=\ttfamily\footnotesize,
		breakatwhitespace=false,         
		breaklines=true,                 
		captionpos=b,                    
		keepspaces=true,                 
		numbers=left,                    
		numbersep=5pt,                  
		showspaces=false,                
		showstringspaces=false,
		showtabs=false,                  
		tabsize=2
}

\lstdefinestyle{shellstyle}{
		backgroundcolor=\color{black!5}, 
		basicstyle=\ttfamily\footnotesize\color{black}, 
		commentstyle=\color{black}, 
		keywordstyle=\color{black},
		numberstyle=\color{black!5},
		stringstyle=\color{black}, 
		showspaces=false,
		showstringspaces=false, 
		showtabs=false, 
		tabsize=2, 
		numbers=none, 
		breaklines=true
}

\lstdefinelanguage{javascript}{
	keywords={typeof, new, true, false, catch, function, return, null, catch, switch, var, if, in, while, do, else, case, break},
	keywordstyle=\color{blue}\bfseries,
	ndkeywords={class, export, boolean, throw, implements, import, this},
	ndkeywordstyle=\color{darkgray}\bfseries,
	identifierstyle=\color{black},
	sensitive=false,
	comment=[l]{//},
	morecomment=[s]{/*}{*/},
	commentstyle=\color{purple}\ttfamily,
	stringstyle=\color{red}\ttfamily,
	morestring=[b]',
	morestring=[b]"
}

% disponi sezioni
\usepackage{titlesec}

\titleformat{\section}
	{\sffamily\Large\bfseries} 
	{\thesection}{1em}{} 
\titleformat{\subsection}
	{\sffamily\large\bfseries}   
	{\thesubsection}{1em}{} 
\titleformat{\subsubsection}
	{\sffamily\normalsize\bfseries} 
	{\thesubsubsection}{1em}{}

% disponi alberi
\usepackage{forest}

\forestset{
	rectstyle/.style={
		for tree={rectangle,draw,font=\large\sffamily}
	},
	roundstyle/.style={
		for tree={circle,draw,font=\large}
	}
}

% disponi algoritmi
\usepackage{algorithm}
\usepackage{algorithmic}
\makeatletter
\renewcommand{\ALG@name}{Algoritmo}
\makeatother

% disponi numeri di pagina
\usepackage{fancyhdr}
\fancyhf{} 
\fancyfoot[L]{\sffamily{\thepage}}

\makeatletter
\fancyhead[L]{\raisebox{1ex}[0pt][0pt]{\sffamily{\@title \ \@date}}} 
\fancyhead[R]{\raisebox{1ex}[0pt][0pt]{\sffamily{\@author}}}
\makeatother

\begin{document}

% sezione (data)
\section{Lezione del 18-11-24}

% stili pagina
\thispagestyle{empty}
\pagestyle{fancy}

% testo
\subsection{Problema di flusso massimo}
Poniamo di avere un grafo su cui riportiamo solamente le capacità superiori $u_{ij}$ sui singoli archi: 

\begin{center}
	\begin{tikzpicture}
		\node[circle, draw=black] (1) at (0,0) {1};
		\node[circle, draw=black] (2) at (2,1) {2};
		\node[circle, draw=black] (3) at (2,-1) {3};
		\node[circle, draw=black] (4) at (4,1) {4};
		\node[circle, draw=black] (5) at (4,-1) {5};
		\node[circle, draw=black] (6) at (6, 0) {6};
		\draw[->, to path={-| (\tikztotarget)}] (1) -- (2);
		\draw[->, to path={-| (\tikztotarget)}] (1) -- (3);
		\draw[->, to path={-| (\tikztotarget)}] (2) -- (3);
		\draw[->, to path={-| (\tikztotarget)}] (2) -- (4);
		\draw[->, to path={-| (\tikztotarget)}] (3) -- (5);
		\draw[->, to path={-| (\tikztotarget)}] (5) -- (4);
		\draw[->, to path={-| (\tikztotarget)}] (3) -- (4);
		\draw[->, to path={-| (\tikztotarget)}] (4) -- (6);
		\draw[->, to path={-| (\tikztotarget)}] (5) -- (6);

		\node at (1, 1)  {$8$};
		\node at (1, -1)  {$9$};

		\node at (3, 1.5)  {$2$};
		\node at (3, -1.5)  {$7$};
		\node at (3, 0.5)  {$8$};

		\node at (1.5, 0)  {$3$};
		\node at (4.5, 0)  {$6$};

		\node at (5, 1)  {$10$};
		\node at (5, -1)  {$4$};
	\end{tikzpicture}
\end{center}

Prendiamo due nodi $s$ e $t$, e cerchiamo di \textbf{massimizzare} il flusso da $s$ a $t$.
L'idea fondamentale che il flusso che parte da $s$ dovrà essere uguale al flusso che arriva in $t$.
Potremo partire da un flusso ammissibile qualsiasi, cioè che si limita a rispettare le capacità:
$$
x = \left( 2, 4, 0, 2, 0, 4, 2, 0, 4 \right)
$$
Notando che le capacità sugli \textit{ultimi} archi influenzano quelle sugli archi precedenti (ad esempio, $(3,5)$ è limitato a 4 da $(5,6)$ con $u_{56} = 4$).
Poniamo allora, più intelligentemente, un problema di PL:
\[
	\begin{cases}
		\max v \\ 
		Ex = b \\ 
		0 \leq x \leq u
	\end{cases}
\]
dove $E$ è la matrice di incidenza della rete, e i bilanci $b_i$ stessi dipendono dalla variabile $v$: 
\[
	b_i = 
	\begin{cases}
		-v, \quad i = s \\ 
		0, \quad i \neq s \wedge i \neq t \\ 
		v, \quad i = t
	\end{cases}
\]

Potremmo avere dubbi sul fatto che la  variabile $v$ compare nei bilanci $b$.
Scriviamo per esteso le uguaglianze dei vincoli:
\[
	\begin{cases}
		-x_{12} - x_{13} = -v \\ 
		x_{12} - x_{23} - x_{24} = 0 \\ 
		x_{13} + x_{23} - x_{34} - x_{35} = 0 \\ 
		x_{24} + x_{34} + x_{54} - x_{46} = 0 \\ 
		x_{35} - x_{54} - x_{56} = 0 \\ 
		x_{46} + x_{56} = v
	\end{cases}
\]

Notiamo che questo problema è effettivamente un problema di flusso minimo (massimo) su $n+1$ variabili per $n$ nodi, dove la $n$-esima variabile è proprio il flusso $v$.
Inoltre, con capacità massime intere, anche $v$ sarà necessariamente intero (dato da somma di interi) e quindi la matrice \textbf{unimodulare}, con la conseguenza già vista che $\text{PL} = \text{ILP}$.

Inoltre, possiam portare i termini $v$ di $b$ a sinistra delle rispettive equazioni, ottenendo effettivamente la matrice:
$$
\begin{pmatrix}
 & 1 \\ 
E & 0 \\ 
 & -1
\end{pmatrix}
\begin{pmatrix}
x \\ v
\end{pmatrix}
= 0
$$
cioè dove si è introdotto la un \textbf{arco fittizio}, quello che parte da $t$ e arriva in $s$.
Capovolgendo la funzione obiettivo (e riportando in vista i termini di costo nullo su ogni arco $x_{ij}$), otteniamo quindi:
\[
	\begin{cases}
		\min 0 \cdot x - v \\ 	
\begin{pmatrix}
 & 1 \\ 
E & 0 \\ 
 & -1
\end{pmatrix}
\begin{pmatrix}
x \\ v
\end{pmatrix}
= 0 \\ 
0 \leq x \leq u
	\end{cases}
\]
dove per l'ultimo arco fittizio la capacità massima $u$ è un $M$ molto grande (o $\infty$).

Questo è un problema di flusso di costo minimo capacitato che include l'arco fittizio come unico arco a costo diverso da 0 (per giunta negativo), cioè che è "costretto" a imporre flusso massimo da $t$ ad $s$, e visto che tutti i nodi sono a bilancio 0, a riportarlo in direzione opposta da $s$ a $t$ lungo i nodi della rete vera e propria.

A questo punto possiamo proporre la soluzione ammissibile:
$$
(x,v) = \left( 2, 9, 0, 2, 5, 4, 7, 0, 4, 11 \right)
$$
da cui si ricavano le partizioni, controllando quali archi si svuotano e quali saturano:
$$
T = \left\{ (1,2), (3,4), (3,5), (4,6), (6,1) \right\}, \quad L = \left\{ (2,3), (5,4) \right\}, \quad U = \left\{ (1,3), (2,4), (5,6) \right\}
$$
Possiamo quindi ricavare il potenziale dell'albero di copertura, notando che dividerà necessariamente i nodi in partizione di potenziale 0 e potenziale 1:
$$
\pi = \left( 0, 0, 1, 1 ,1 \right)
$$
e calcolare i costi ridotti, che troviamo negativi su $(2,3) \in L$, quindi arco entrante:
$$
c^\pi_{23} = 0 + 0 - 1 = -1
$$
che è quanto ci aspettavamo, in quanto chiaramente $v = 14$ (dagli archi entranti nel nodo 6).

Procediamo quindi con l'eliminazione del ciclo, distinguendo innanzitutto le pratizioni $\mathcal{C}^+$ e $\mathcal{C}^-$:
$$
\mathcal{C}^+ = \left\{ (1,2), (2,3), (3,4), (4,6), (6, 1) \right\}, \quad \mathcal{C}^- = \emptyset
$$
Su $\mathcal{C}^-$ mettiamo $\vartheta^- = \infty$ (essenzialmente vogliamo considerare solo $\vartheta^+$).
Calcoliamo allora $\vartheta^+$ dagli $u_{ij} - x_{ij}$, tenendo conto che la capacità $u$ di $(6,1)$ (arco fittizio) è $\infty$:
$$
\vartheta^+ = \max \left\{ 6, 3, 3, 3, \infty \right\}
$$
da cui prendiamo $\vartheta^+ = 3$ e l'arco uscente $(2,3)$ per l'ordinamento lessicografico.
Notiamo di essere nel caso particolare dove l'arco \textit{entra} ed \textit{esce} (in questo caso si sposta da L a U).

Aggiungendo quindi il $\vartheta$ agli archi in $\mathcal{C}^+$ otteniamo il flusso:
$$
(x, v) = \left( 5, 9, 3, 2, 8, 4, 10, 0, 4, 14 \right)
$$
da cui $v=14$, come ci aspettavamo.

\end{document}
