
\documentclass[a4paper,11pt]{article}
\usepackage[a4paper, margin=8em]{geometry}

% usa i pacchetti per la scrittura in italiano
\usepackage[french,italian]{babel}
\usepackage[T1]{fontenc}
\usepackage[utf8]{inputenc}
\frenchspacing 

% usa i pacchetti per la formattazione matematica
\usepackage{amsmath, amssymb, amsthm, amsfonts}

% usa altri pacchetti
\usepackage{gensymb}
\usepackage{hyperref}
\usepackage{standalone}

% cose fluttuanti
\usepackage{float}

% imposta il titolo
\title{Appunti Ricerca Operativa}
\author{Luca Seggiani}
\date{2024}

% disegni
\usepackage{pgfplots}
\pgfplotsset{width=10cm,compat=1.9}

% imposta lo stile
% usa helvetica
\usepackage[scaled]{helvet}
% usa palatino
\usepackage{palatino}
% usa un font monospazio guardabile
\usepackage{lmodern}

\renewcommand{\rmdefault}{ppl}
\renewcommand{\sfdefault}{phv}
\renewcommand{\ttdefault}{lmtt}

% disponi il titolo
\makeatletter
\renewcommand{\maketitle} {
	\begin{center} 
		\begin{minipage}[t]{.8\textwidth}
			\textsf{\huge\bfseries \@title} 
		\end{minipage}%
		\begin{minipage}[t]{.2\textwidth}
			\raggedleft \vspace{-1.65em}
			\textsf{\small \@author} \vfill
			\textsf{\small \@date}
		\end{minipage}
		\par
	\end{center}

	\thispagestyle{empty}
	\pagestyle{fancy}
}
\makeatother

% disponi teoremi
\usepackage{tcolorbox}
\newtcolorbox[auto counter, number within=section]{theorem}[2][]{%
	colback=blue!10, 
	colframe=blue!40!black, 
	sharp corners=northwest,
	fonttitle=\sffamily\bfseries, 
	title=Teorema~\thetcbcounter: #2, 
	#1
}

% disponi definizioni
\newtcolorbox[auto counter, number within=section]{definition}[2][]{%
	colback=red!10,
	colframe=red!40!black,
	sharp corners=northwest,
	fonttitle=\sffamily\bfseries,
	title=Definizione~\thetcbcounter: #2,
	#1
}

% disponi problemi
\newtcolorbox[auto counter, number within=section]{problem}[2][]{%
	colback=green!10,
	colframe=green!40!black,
	sharp corners=northwest,
	fonttitle=\sffamily\bfseries,
	title=Problema~\thetcbcounter: #2,
	#1
}

% disponi codice
\usepackage{listings}
\usepackage[table]{xcolor}

\lstdefinestyle{codestyle}{
		backgroundcolor=\color{black!5}, 
		commentstyle=\color{codegreen},
		keywordstyle=\bfseries\color{magenta},
		numberstyle=\sffamily\tiny\color{black!60},
		stringstyle=\color{green!50!black},
		basicstyle=\ttfamily\footnotesize,
		breakatwhitespace=false,         
		breaklines=true,                 
		captionpos=b,                    
		keepspaces=true,                 
		numbers=left,                    
		numbersep=5pt,                  
		showspaces=false,                
		showstringspaces=false,
		showtabs=false,                  
		tabsize=2
}

\lstdefinestyle{shellstyle}{
		backgroundcolor=\color{black!5}, 
		basicstyle=\ttfamily\footnotesize\color{black}, 
		commentstyle=\color{black}, 
		keywordstyle=\color{black},
		numberstyle=\color{black!5},
		stringstyle=\color{black}, 
		showspaces=false,
		showstringspaces=false, 
		showtabs=false, 
		tabsize=2, 
		numbers=none, 
		breaklines=true
}

\lstdefinelanguage{javascript}{
	keywords={typeof, new, true, false, catch, function, return, null, catch, switch, var, if, in, while, do, else, case, break},
	keywordstyle=\color{blue}\bfseries,
	ndkeywords={class, export, boolean, throw, implements, import, this},
	ndkeywordstyle=\color{darkgray}\bfseries,
	identifierstyle=\color{black},
	sensitive=false,
	comment=[l]{//},
	morecomment=[s]{/*}{*/},
	commentstyle=\color{purple}\ttfamily,
	stringstyle=\color{red}\ttfamily,
	morestring=[b]',
	morestring=[b]"
}

% disponi sezioni
\usepackage{titlesec}

\titleformat{\section}
	{\sffamily\Large\bfseries} 
	{\thesection}{1em}{} 
\titleformat{\subsection}
	{\sffamily\large\bfseries}   
	{\thesubsection}{1em}{} 
\titleformat{\subsubsection}
	{\sffamily\normalsize\bfseries} 
	{\thesubsubsection}{1em}{}

% disponi alberi
\usepackage{forest}

\forestset{
	rectstyle/.style={
		for tree={rectangle,draw,font=\large\sffamily}
	},
	roundstyle/.style={
		for tree={circle,draw,font=\large}
	}
}

% disponi algoritmi
\usepackage{algorithm}
\usepackage{algorithmic}
\makeatletter
\renewcommand{\ALG@name}{Algoritmo}
\makeatother

% disponi numeri di pagina
\usepackage{fancyhdr}
\fancyhf{} 
\fancyfoot[L]{\sffamily{\thepage}}

\makeatletter
\fancyhead[L]{\raisebox{1ex}[0pt][0pt]{\sffamily{\@title \ \@date}}} 
\fancyhead[R]{\raisebox{1ex}[0pt][0pt]{\sffamily{\@author}}}
\makeatother

\begin{document}

\pagestyle{fancy}
\thispagestyle{empty}
\renewcommand{\thispagestyle}[1]{}

\maketitle
\documentclass[a4paper,11pt]{article}
\usepackage[a4paper, margin=8em]{geometry}

% usa i pacchetti per la scrittura in italiano
\usepackage[french,italian]{babel}
\usepackage[T1]{fontenc}
\usepackage[utf8]{inputenc}
\frenchspacing 

% usa i pacchetti per la formattazione matematica
\usepackage{amsmath, amssymb, amsthm, amsfonts}

% usa altri pacchetti
\usepackage{gensymb}
\usepackage{hyperref}
\usepackage{standalone}

% imposta il titolo
\title{Appunti Ricerca Operativa}
\author{Luca Seggiani}
\date{23-09-24}

% imposta lo stile
% usa helvetica
\usepackage[scaled]{helvet}
% usa palatino
\usepackage{palatino}
% usa un font monospazio guardabile
\usepackage{lmodern}

\renewcommand{\rmdefault}{ppl}
\renewcommand{\sfdefault}{phv}
\renewcommand{\ttdefault}{lmtt}

% disponi teoremi
\usepackage{tcolorbox}
\newtcolorbox[auto counter, number within=section]{theorem}[2][]{%
	colback=blue!10, 
	colframe=blue!40!black, 
	sharp corners=northwest,
	fonttitle=\sffamily\bfseries, 
	title=Teorema~\thetcbcounter: #2, 
	#1
}

% disponi definizioni
\newtcolorbox[auto counter, number within=section]{definition}[2][]{%
	colback=red!10,
	colframe=red!40!black,
	sharp corners=northwest,
	fonttitle=\sffamily\bfseries,
	title=Definizione~\thetcbcounter: #2,
	#1
}

% disponi problemi
\newtcolorbox[auto counter, number within=section]{problem}[2][]{%
	colback=green!10,
	colframe=green!40!black,
	sharp corners=northwest,
	fonttitle=\sffamily\bfseries,
	title=Problema~\thetcbcounter: #2,
	#1
}

% disponi codice
\usepackage{listings}
\usepackage[table]{xcolor}

\lstdefinestyle{codestyle}{
		backgroundcolor=\color{black!5}, 
		commentstyle=\color{codegreen},
		keywordstyle=\bfseries\color{magenta},
		numberstyle=\sffamily\tiny\color{black!60},
		stringstyle=\color{green!50!black},
		basicstyle=\ttfamily\footnotesize,
		breakatwhitespace=false,         
		breaklines=true,                 
		captionpos=b,                    
		keepspaces=true,                 
		numbers=left,                    
		numbersep=5pt,                  
		showspaces=false,                
		showstringspaces=false,
		showtabs=false,                  
		tabsize=2
}

\lstdefinestyle{shellstyle}{
		backgroundcolor=\color{black!5}, 
		basicstyle=\ttfamily\footnotesize\color{black}, 
		commentstyle=\color{black}, 
		keywordstyle=\color{black},
		numberstyle=\color{black!5},
		stringstyle=\color{black}, 
		showspaces=false,
		showstringspaces=false, 
		showtabs=false, 
		tabsize=2, 
		numbers=none, 
		breaklines=true
}

\lstdefinelanguage{javascript}{
	keywords={typeof, new, true, false, catch, function, return, null, catch, switch, var, if, in, while, do, else, case, break},
	keywordstyle=\color{blue}\bfseries,
	ndkeywords={class, export, boolean, throw, implements, import, this},
	ndkeywordstyle=\color{darkgray}\bfseries,
	identifierstyle=\color{black},
	sensitive=false,
	comment=[l]{//},
	morecomment=[s]{/*}{*/},
	commentstyle=\color{purple}\ttfamily,
	stringstyle=\color{red}\ttfamily,
	morestring=[b]',
	morestring=[b]"
}

% disponi sezioni
\usepackage{titlesec}

\titleformat{\section}
	{\sffamily\Large\bfseries} 
	{\thesection}{1em}{} 
\titleformat{\subsection}
	{\sffamily\large\bfseries}   
	{\thesubsection}{1em}{} 
\titleformat{\subsubsection}
	{\sffamily\normalsize\bfseries} 
	{\thesubsubsection}{1em}{}

% disponi alberi
\usepackage{forest}

\forestset{
	rectstyle/.style={
		for tree={rectangle,draw,font=\large\sffamily}
	},
	roundstyle/.style={
		for tree={circle,draw,font=\large}
	}
}

% disponi algoritmi
\usepackage{algorithm}
\usepackage{algorithmic}
\makeatletter
\renewcommand{\ALG@name}{Algoritmo}
\makeatother

% disponi numeri di pagina
\usepackage{fancyhdr}
\fancyhf{} 
\fancyfoot[L]{\sffamily{\thepage}}

\makeatletter
\fancyhead[L]{\raisebox{1ex}[0pt][0pt]{\sffamily{\@title \ \@date}}} 
\fancyhead[R]{\raisebox{1ex}[0pt][0pt]{\sffamily{\@author}}}
\makeatother

% disegni
\usepackage{pgfplots}
\pgfplotsset{width=10cm,compat=1.9}

\begin{document}
% sezione (data)
\section{Lezione del 23-09-24}

% stili pagina
\thispagestyle{empty}
\pagestyle{fancy}

% testo
\subsection{Introduzione}

\subsubsection{Programma del corso}
Il corso di ricerca operativa si divide in 4 parti:

\begin{enumerate}
	\item Modello di Programmazione Lineare;
	\item Programmazione Lineare su reti, ergo programmazione lineare su grafi;
	\item Programmazione Lineare intera, ergo programmazione lineare col vincolo $x \in \mathbb{Z}^n$;
	\item Programmazione Non Lineare.
\end{enumerate}

Le prime 3 parti hanno come prerequisiti l'algebra lineare: in particolare operazioni matriciali, prodotti scalari, sistemi lineari, teorema di Rouché-Capelli.
La quarta parte richiede invece conoscenze di Analisi II.

\subsubsection{Un problema di programmazione lineare}

La ricerca operativa si occupa di risolvere problemi di ottimizzazione con variabili decisionali e risorse limitate.
Poniamo un problema di esempio:

\begin{problem}{Produzione}
Una ditta produce due prodotti: \textbf{laminato A} e \textbf{laminato B}.
Ogni prodotto deve passare attraverso diversi reparti: il reparto \textbf{materie prime}, il reparto \textbf{taglio}, il reparto \textbf{finiture A} e il reparto \textbf{finiture B}.
Il guadagno è rispettivamente di 8.4 e 11.2 (unità di misura irrilevante) per ogni tipo di laminato.

Ora, nel reparto materie prime, il laminato A occupa 30, ore, e lo B 20 ore.
Nel reparto taglio il laminato A occupa 10 ore e lo B 20 ore.
Il laminato A occupa poi 20 ore nel reparto finiture A, mentre il laminato B occupa 30 ore nel reparto finiture B.
I reparti hanno a disposizione, rispettivamente, 120, 80, 62 e 105 ore.
Possiamo porre queste informazioni in forma tabulare:

	\center \rowcolors{2}{green!10}{green!40!black!20}
	\begin{tabular} { | c || c | c | c | }
		\hline
		\bfseries Reparto & \bfseries Capienza & \bfseries Laminato A & \bfseries Laminato B \\
		\hline 
		Materie prime & 120 & 30 & 20 \\
		Taglio & 80 & 10 & 20 \\
		Finiture A & 62 & 20 & / \\
		Finiture B & 105 & / & 30 \\
		\hline
		\textbf{Guadagno} & & 8.4 & 11.2 \\
		\hline
	\end{tabular}

	\par\bigskip

Quello che ci interessa è chiaramente massimizzare il guadagno.
\end{problem}

Decidiamo di modellizzare questa situazione con un modello matematico.

Il guadagno che abbiamo dai laminati rappresenta una \textbf{funzione obiettivo}, ovvero la funzione che vogliamo ottimizzare.
Ottimizzare significa trovare il modo migliore di massimizzare o minimizzare i valori della funzione agendo sulle variabili decisionali.
La funzione obiettivo va ottimizzata rispettando determinati \textbf{vincoli}, che modellizzano il fatto che le risorse sono limitate.
Una \textbf{soluzione ammissibile} è una qualsiasi soluzione che rispetta i vincoli del problema.
Chiamiamo quindi \textbf{regione ammissibile} l'insieme di tutte le soluzioni ammissibili.
All'interno della regione ammissibile c'è la soluzione che cerchiamo, ovvero la \textbf{soluzione ottima}.

Decidiamo quindi le \textbf{variabili decisionali}, ed esplicitiamo la funzione obiettivo e i vincoli.

In questo caso le variabili decisionali saranno le quantità di laminato A e B da produrre, che individuano un punto in $ \mathbb{R}^2 $ denominato $ ( x_A, x_B ) $. 
Decidere di usare la soluzione $ (1,1) $ significa decidere di produrre 1 unità di laminato A e 1 unità di laminato B, per un guadagno complessivo di $ 8.4 + 11.2 = 19.6 $.

La funzione obiettivo sarà quindi:

$$ f(x_A, x_B) = 8.4 x_A + 11.2 x_B, \quad f: \mathbb{R}^2 \rightarrow \mathbb{R} $$

lineare, e noi saremo interessati a:

$$ \max(f(x_A, x_B)) $$

rispettando i vincoli, ergo nella regione ammissibile.
Per esprimere questi vincoli, cioè il tempo limitato all'interno di ogni reparto, introduciamo il sistema di disequazioni:

\[
	\begin{cases}
		30 x_A + 20 x_B \leq 120 \\
		10 x_A + 20 x_B \leq 80	\\
		20 x_A + 0 x_B \leq 62 \\	
		0 x_A + 30 x_B \leq 105 \\
		- x_A \leq 0 \\
		- x_B \leq 0 \\
	\end{cases}
\]

dove notiamo le ultime due disequazioni indicano la positività di $x_A$ e $x_B$, in forma $ f(x_A, x_B) \leq b $.
Questo sistema non indica altro che la regione ammissibile.

Possiamo riscrivere questo modello usando la notazione dell'algebra lineare.
La funzione obiettiva e i vincoli diventano semplicemente:

\[
	\begin{cases}
		\max(c^\intercal \cdot x) \\
		A \cdot x \leq b	
	\end{cases}
\]

dove $c$ rappresenta il vettore dei costi, $A$ rappresenta la matrice dei costi a $b$ il vettore dei vincoli.
$c$ è trasposto per indicare prodotto fra vettori.

Possiamo scrivere $A$, $b$ e $c$ per esteso:

$$
A:
\begin{pmatrix}
	30 & 20 \\
	10 & 20 \\
	20 & 0 \\
	0 & 30 \\
	-1 & 0 \\
	0 & -1
\end{pmatrix}, \quad
b:
\begin{pmatrix}
	120 \\
	80 \\
	62 \\ 
	105 \\ 
	0 \\ 
	0 
\end{pmatrix}, \quad 
c:
\begin{pmatrix}
	8.4 \\
	11.2 \\
\end{pmatrix}
$$

Notiamo come $A$ e $b$ hanno dimensione verticale $ 4 + 2 = 6 $, dai 4 vincoli superiori e i 2 vincoli inferiori.

A questo punto, possiamo disegnare la regione ammissibile come l'intersezione dei semipiani individuati da ogni singola disuguaglianza.

Si riporta un grafico:

\begin{center}

\begin{tikzpicture}
\begin{axis}[
    axis lines = middle,
    xlabel = {$x_A$},
    ylabel = {$x_B$},
    xmin=0, xmax=7.9,
    ymin=0, ymax=3.9,
    domain=0:10,
    samples=100,
    width=15cm, height=7cm,
    legend pos=north east
  ]

% regione ammissibile

	\addplot[fill=gray, opacity=0.4, forget plot] 
    coordinates {
			(0, 0)
			(3.1, 0)
			(3.1, 1.35)
			(2,3)
			(1, 3.5)
			(0, 3.5)
		};

% rette

\addplot[domain=2:3.1, thick, blue] {6 - 1.5*x}; 
\addlegendentry{$ 30 x_A + 20 x_B \leq 120 $}

\addplot[domain=1:2, thick, green] {4 - 0.5*x}; 
\addlegendentry{$ 10 x_A + 20 x_B \leq 80 $}

\addplot[thick, purple] coordinates {(3.1, 0) (3.1, 1.35)};
\addlegendentry{$ 20 x_A + 0 x_B \leq 62 $}

\addplot[domain=0:1, thick, red] {3.5}; 
\addlegendentry{$ 0 x_A + 30 x_B \leq 105 $}
	
\end{axis}
\end{tikzpicture}

\end{center}

In diversi colori sono riportati i margini delle disequazioni, mentre in grigio è evidenziata la regione ammissibile.
Qualsiasi punto all'interno della regione ammissibile vale come soluzione, e almeno uno di essi è soluzione ottimale.

\par\smallskip

Il modello finora descritto prende il nome di modello di programmazione lineare, e permette di formulare problemi di programmazione lineare (LP).

\begin{definition}{Problema di programmazione lineare (1)}
Un problema di programmazione lineare (LP) riguarda l'ottimizzazione di una funzione lineare in più variabili
soggetta a vincoli di tipo $ =, \ \leq $ e $ \geq $, ovvero in forma:
\[
	\begin{cases}
			\min / \max(c^\intercal \cdot x) \\
			A_i x \leq b \\
			B_j x \geq d \\
			C_k x = e \\
	\end{cases}
\]
\end{definition}

"Programmazione" qui non ha alcun legame col concetto di programmazione informatica, ma si riferisce al fatto che il modello è effettivamente \textit{programmabile}.

"Lineare" si riferisce alla linearità dei problemi che ci permette di risolvere (e quindi del modello).




\end{document}

\documentclass[a4paper,11pt]{article}
\usepackage[a4paper, margin=8em]{geometry}

% usa i pacchetti per la scrittura in italiano
\usepackage[french,italian]{babel}
\usepackage[T1]{fontenc}
\usepackage[utf8]{inputenc}
\frenchspacing 

% usa i pacchetti per la formattazione matematica
\usepackage{amsmath, amssymb, amsthm, amsfonts}

% usa altri pacchetti
\usepackage{gensymb}
\usepackage{hyperref}
\usepackage{standalone}

% imposta il titolo
\title{Appunti Ricerca Operativa}
\author{Luca Seggiani}
\date{24-09-24}

% imposta lo stile
% usa helvetica
\usepackage[scaled]{helvet}
% usa palatino
\usepackage{palatino}
% usa un font monospazio guardabile
\usepackage{lmodern}

\renewcommand{\rmdefault}{ppl}
\renewcommand{\sfdefault}{phv}
\renewcommand{\ttdefault}{lmtt}

% disponi teoremi
\usepackage{tcolorbox}
\newtcolorbox[auto counter, number within=section]{theorem}[2][]{%
	colback=blue!10, 
	colframe=blue!40!black, 
	sharp corners=northwest,
	fonttitle=\sffamily\bfseries, 
	title=Teorema~\thetcbcounter: #2, 
	#1
}

% disponi definizioni
\newtcolorbox[auto counter, number within=section]{definition}[2][]{%
	colback=red!10,
	colframe=red!40!black,
	sharp corners=northwest,
	fonttitle=\sffamily\bfseries,
	title=Definizione~\thetcbcounter: #2,
	#1
}

% disponi codice
\usepackage{listings}
\usepackage[table]{xcolor}

\lstdefinestyle{codestyle}{
		backgroundcolor=\color{black!5}, 
		commentstyle=\color{codegreen},
		keywordstyle=\bfseries\color{magenta},
		numberstyle=\sffamily\tiny\color{black!60},
		stringstyle=\color{green!50!black},
		basicstyle=\ttfamily\footnotesize,
		breakatwhitespace=false,         
		breaklines=true,                 
		captionpos=b,                    
		keepspaces=true,                 
		numbers=left,                    
		numbersep=5pt,                  
		showspaces=false,                
		showstringspaces=false,
		showtabs=false,                  
		tabsize=2
}

\lstdefinestyle{shellstyle}{
		backgroundcolor=\color{black!5}, 
		basicstyle=\ttfamily\footnotesize\color{black}, 
		commentstyle=\color{black}, 
		keywordstyle=\color{black},
		numberstyle=\color{black!5},
		stringstyle=\color{black}, 
		showspaces=false,
		showstringspaces=false, 
		showtabs=false, 
		tabsize=2, 
		numbers=none, 
		breaklines=true
}

\lstdefinelanguage{javascript}{
	keywords={typeof, new, true, false, catch, function, return, null, catch, switch, var, if, in, while, do, else, case, break},
	keywordstyle=\color{blue}\bfseries,
	ndkeywords={class, export, boolean, throw, implements, import, this},
	ndkeywordstyle=\color{darkgray}\bfseries,
	identifierstyle=\color{black},
	sensitive=false,
	comment=[l]{//},
	morecomment=[s]{/*}{*/},
	commentstyle=\color{purple}\ttfamily,
	stringstyle=\color{red}\ttfamily,
	morestring=[b]',
	morestring=[b]"
}

% disponi sezioni
\usepackage{titlesec}

\titleformat{\section}
	{\sffamily\Large\bfseries} 
	{\thesection}{1em}{} 
\titleformat{\subsection}
	{\sffamily\large\bfseries}   
	{\thesubsection}{1em}{} 
\titleformat{\subsubsection}
	{\sffamily\normalsize\bfseries} 
	{\thesubsubsection}{1em}{}

% disponi alberi
\usepackage{forest}

\forestset{
	rectstyle/.style={
		for tree={rectangle,draw,font=\large\sffamily}
	},
	roundstyle/.style={
		for tree={circle,draw,font=\large}
	}
}

% disponi algoritmi
\usepackage{algorithm}
\usepackage{algorithmic}
\makeatletter
\renewcommand{\ALG@name}{Algoritmo}
\makeatother

% disponi numeri di pagina
\usepackage{fancyhdr}
\fancyhf{} 
\fancyfoot[L]{\sffamily{\thepage}}

\makeatletter
\fancyhead[L]{\raisebox{1ex}[0pt][0pt]{\sffamily{\@title \ \@date}}} 
\fancyhead[R]{\raisebox{1ex}[0pt][0pt]{\sffamily{\@author}}}
\makeatother

\begin{document}
% sezione (data)
\section{Lezione del 24-09-24}

% stili pagina
\thispagestyle{empty}
\pagestyle{fancy}

% testo

\subsection{Forma primale standard}
Ciò che abbiamo formulato finora è un problema di programmazione lineare.
Possiamo dire che la forma:
\[
	\begin{cases}
			\max(c^\intercal \cdot x) \\
			A x \leq b
	\end{cases}
\]
rappresenta un problema LP in forma \textbf{primale standard}, ricordando che $c$ è il vettore dei coefficienti della funzione obiettivo, $A$ la matrice dei coefficienti per ogni vincolo, e $b$ il vettore dei termini noti per ogni vincolo.

\begin{definition}{Forma primale standard}
	Un problema di programmazione lineare si dice in forma primale standard quando è espresso in forma:
	
	\[
		\begin{cases}
			\max(c^\intercal \cdot x) \\
			Ax \leq b \\
		\end{cases}
	\]

\end{definition}
\par\smallskip
Si adotta una forma primale standard in quanto si può trasformare ogni problema LP in una forma di questo tipo.

\subsubsection{Normalizzazione di un problema LP}
Un modo per portare un problema LP qualsiasi in forma primale standard è:

\begin{enumerate}
	\item Si trasformano le disuguaglianze: $ \geq \ \leftrightarrow \ \leq $
	\item Si riscrivono le uguaglianze come coppie di diseguaglianze:
		$$
			f(x) = c \ \rightarrow \
		\begin{cases}
			f(x) \leq c \\ 
			f(x) \geq c
		\end{cases}
		$$
		da cui la (1):
		$$
			f(x) = c \ \rightarrow \
S		\begin{cases}
			f(x) \leq c \\ 
			-f(x) \leq -c
		\end{cases}
		$$
	\item Se il problema richiede il minimo, si nota che $ \max(f) = -\min(-f) $, e sopratutto:
		$$
		\bar{x} \in \mathrm{argmax}(f) \Leftrightarrow \bar{x} \in \mathrm{argmin}(-f)
		$$
		con $ \mathrm{argmax}(f) $ e $ \mathrm{argmin}(-f) $ rispettivamente gli insiemi dei punti di massimo e minimo.
		Questo significa che posso semplicemente cambiare di segno la funzione obiettivo per trovare da massimi minimi, e viceversa.
\end{enumerate}

Notiamo inoltre che, nella forma primale standard, si ha:
$$
	x \in R^n, \quad
	A \in R^{n \times m}, \quad
	b \in R^m, \quad
	c \in R^n
$$

\subsection{Proprietà generali di un problema LP}
La regione ammissibile di un problema PL si chiama \textbf{poliedro}.
Si può dare agilmente una definizione algebrica di poliedro:
\begin{definition}{Definizione algebrica di poliedro}
	Algebricamente, un poliedro è l'insieme delle soluzioni di un sistema di disequazioni lineari in $\mathbb{R}^n$ variabili:
	$$
		P = \{ x \in \mathbb{R}^n : Ax \leq b \}
	$$
\end{definition}

Questa regione in un problema LP prende il nome di regione ammissibile.

\begin{definition}{Definizione geometrica di poliedro}
	Geometricamente, un poliedro è l'intersezione di un numero finito di semispazi chiusi.
\end{definition}

I semispazi chiaramente sono lineari, e in $\mathbb{R}^2$ rappresenterebbero semipiani.
Chiusi significa che nelle disequazioni che descrivono i vincoli compargono solo $\leq$ e non $<$, ergo la regione ammissibile contiene la sua frontiera.

Possiamo dimostrare 4 proprietà dei poliedri:

\begin{enumerate}
	\item 
		Un'osservazione fondamentale è la seguente:
		\begin{theorem}{Soluzione ottimale di un problema LP}
			La soluzione ottimale di un problema LP è contenuta nella frontiera della regione ammissibile.
		\end{theorem}
		Questo si può ricavare dai teoremi di Fermat e di Weierstrass, e dalla convessità della regione ammissibile.
		Inanzitutto, si è stabilito che la soluzione ottimale non è altro che il massimo o minimo assoluto all'interno della regione ammissibile del problema.
		Il gradiente della funzione obiettiva è $\neq 0$ in ogni suo punto (funzione lineare a gradiente costante). 
		Da Fermat, i massimi e minimi hanno sempre gradiente $0$, ergo massimi o minimi locali (che esistono per Weierstrass) possono trovarsi solo sulla frontiera.
		A questo punto, possiamo imporre la convessità per asserire che quei punti di massimo o minimo sono anche globali. 

	\item 
		Prendiamo in esempio il poliedro dato da:
		\[
			\begin{cases}
				x_A > 0 \\ 
				x_B > 0
			\end{cases}
		\]
		o se vogliamo, in forma primale standard, dato dalle matrici $A$ e $b$:
		$$
		A:
		\begin{pmatrix}
			-1 & 0 \\	
			0 & -1 \\	
		\end{pmatrix}
		, \quad b:
		\begin{pmatrix}
			0 \\ 
			0
		\end{pmatrix}
		$$
		questo poliedro non è limitato nella direzione positiva, ergo può arrivare a valori di $x_A$ e $x_B$ che tendono a $+\infty$.
		Da ciò si ha che può accadere che un problema LP ammetta soluzioni $x$ tali che $x \rightarrow \pm \infty $,
		ovvero che il poliedro sia illimitato.
		In particolare, un poliedro limitato si dice \textbf{politopo}.
	\item Notiamo poi che la soluzione di un problema LP può non essere unica.
		Questo accade ad esempio quando la soluzione sta su un segmento di frontiera: a quel punto tutti i punti del segmento sono soluzione.
		Da questo segue che:
		\begin{theorem}{Unicità della soluzione ottimale di un problema LP}
			Se un problema LP ha almeno 2 soluzioni, allora ne ha infinite.
		\end{theorem}
		Ciò si può dimostrare come segue.
		Si riporta innanzitutto la notazione parametrica del segmento $\bar{zw}$, dati i due vettori di estremo $z$ e $w$:
		$$
			\lambda z + (1 - \lambda)w, \quad \lambda \in [ 0, 1 ]
		$$
		A questo punto si pone che $z$ e $w$ sono entrambi soluzioni ottime, ergo: 
		$$ 
			\max(c^\intercal \cdot x) = c^\intercal z = c^\intercal w = v 
		$$
		da cui si può dire che:
		$$ 
			c^\intercal\left(\lambda z + (1 - \lambda)w\right) = \lambda c^\intercal z + (1 - \lambda) c^\intercal w = \lambda v + (1 - \lambda) v = v 
		$$
		Ovvero ogni punto sul segmento porta la funzione obiettiva a massimo assoluto, quindi è soluzione ottimale.
	\item Infine, notiamo che il poliedro della regione ammissible di un problema LP può essere vuoto, ergo $P = \emptyset$.
		In questo caso, si ha che $ \max(c^\intercal \cdot x) = -\infty $ e $ \min(c^\intercal \cdot x) = \infty $. 
		Un poliedro vuoto significa che i vincoli stessi vanno modificati. 
		Questo solitamente si fa risovendo una versione semplificata del problema LP.
\end{enumerate}

Si può fare un'altro esempio per sottolineare l'importanza del punto di massimo (o minimo), e non quel massimo (o minimo).
Finché nella funzione obiettivo i coefficienti compargono nello stesso rapporto (ergo finché si scelgono vettori $c$ linearmente dipendenti), il punto di massimo (o minimo) non cambia, per via della linearità (e si presume omogeneità) della funzione obiettiva stessa.
Sarà solo il massimo (o minimo) a variare di un rapporto pari a quello di cui variano i coefficienti.

\subsection{Gradiente e linee di isocosto}
Si può dimostrare il seguente teorema:
\begin{theorem}{Gradiente della funzione obiettivo}
	Il gradiente di una funzione obiettivo definita come $ f(x) = c^\intercal \cdot x $ sulla base di un qualche vettore $c$ è in ogni punto il vettore $c$ stesso.
\end{theorem}
Da questo gradiente si possono ricavare le cosiddette linee di isocosto (in dimensioni $>2$ sarebbero superfici), cioè linee a valore costante della funzione obiettivo.
\begin{definition}{Linea di isocosto}
	Si definisce linea di isocosto di una funzione obiettivo con vettore $c$ una retta (o superficie):	
	$$ f(x) = c^\intercal \cdot x = k $$
	per un qualsiasi $k$ costante.
\end{definition}

\subsection{Cono di competenza}
Dovrebbe essere chiaro adesso che i punti di soluzione ottima stanno tutti su un segmento o su un punto della frontiera.
Nel caso si abbia un vettore gradiente perpendicolare ad un segmento della frontiera, quel segmento sarà soluzione ottima. In caso contrario, spostandoci a destra avremo l'estremo destro del segmento, e spostandoci a sinistra viceversa, finché non si diventerà perpendicolari a qualche altro segmento di frontiera.

Il cono (in $R^2$, angolo) di valori possibili del gradiente che rendono un punto ottimale prende il nome di \textbf{cono di competenza}.
\begin{definition}{Cono di competenza}
	Il cono di competenza di un punto $x^*$ è il cono, ovvero l'insieme di vettori gradiente, tale per cui il punto $x^*$ conserva l'ottimalità sulla funzione obiettivo coi vincoli imposti.
\end{definition}

Vedremo in seguito l'importanza di una nozione di cono per i problemi LP.

\end{document}


\documentclass[a4paper,11pt]{article}
\usepackage[a4paper, margin=8em]{geometry}

% usa i pacchetti per la scrittura in italiano
\usepackage[french,italian]{babel}
\usepackage[T1]{fontenc}
\usepackage[utf8]{inputenc}
\frenchspacing 

% usa i pacchetti per la formattazione matematica
\usepackage{amsmath, amssymb, amsthm, amsfonts}

% usa altri pacchetti
\usepackage{gensymb}
\usepackage{hyperref}
\usepackage{standalone}

% imposta il titolo
\title{Appunti /home/luca/Desktop/Uni/appunti/Ricerca Operativa}
\author{Luca Seggiani}
\date{2024}

% disegni
\usepackage{pgfplots}
\pgfplotsset{width=10cm,compat=1.9}

% imposta lo stile
% usa helvetica
\usepackage[scaled]{helvet}
% usa palatino
\usepackage{palatino}
% usa un font monospazio guardabile
\usepackage{lmodern}

\renewcommand{\rmdefault}{ppl}
\renewcommand{\sfdefault}{phv}
\renewcommand{\ttdefault}{lmtt}

% disponi il titolo
\makeatletter
\renewcommand{\maketitle} {
	\begin{center} 
		\begin{minipage}[t]{.8\textwidth}
			\textsf{\huge\bfseries \@title} 
		\end{minipage}%
		\begin{minipage}[t]{.2\textwidth}
			\raggedleft \vspace{-1.65em}
			\textsf{\small \@author} \vfill
			\textsf{\small \@date}
		\end{minipage}
		\par
	\end{center}

	\thispagestyle{empty}
	\pagestyle{fancy}
}
\makeatother

% disponi teoremi
\usepackage{tcolorbox}
\newtcolorbox[auto counter, number within=section]{theorem}[2][]{%
	colback=blue!10, 
	colframe=blue!40!black, 
	sharp corners=northwest,
	fonttitle=\sffamily\bfseries, 
	title=Teorema~\thetcbcounter: #2, 
	#1
}

% disponi definizioni
\newtcolorbox[auto counter, number within=section]{definition}[2][]{%
	colback=red!10,
	colframe=red!40!black,
	sharp corners=northwest,
	fonttitle=\sffamily\bfseries,
	title=Definizione~\thetcbcounter: #2,
	#1
}

% disponi problemi
\newtcolorbox[auto counter, number within=section]{problem}[2][]{%
	colback=green!10,
	colframe=green!40!black,
	sharp corners=northwest,
	fonttitle=\sffamily\bfseries,
	title=Problema~\thetcbcounter: #2,
	#1
}

% disponi codice
\usepackage{listings}
\usepackage[table]{xcolor}

\lstdefinestyle{codestyle}{
		backgroundcolor=\color{black!5}, 
		commentstyle=\color{codegreen},
		keywordstyle=\bfseries\color{magenta},
		numberstyle=\sffamily\tiny\color{black!60},
		stringstyle=\color{green!50!black},
		basicstyle=\ttfamily\footnotesize,
		breakatwhitespace=false,         
		breaklines=true,                 
		captionpos=b,                    
		keepspaces=true,                 
		numbers=left,                    
		numbersep=5pt,                  
		showspaces=false,                
		showstringspaces=false,
		showtabs=false,                  
		tabsize=2
}

\lstdefinestyle{shellstyle}{
		backgroundcolor=\color{black!5}, 
		basicstyle=\ttfamily\footnotesize\color{black}, 
		commentstyle=\color{black}, 
		keywordstyle=\color{black},
		numberstyle=\color{black!5},
		stringstyle=\color{black}, 
		showspaces=false,
		showstringspaces=false, 
		showtabs=false, 
		tabsize=2, 
		numbers=none, 
		breaklines=true
}

\lstdefinelanguage{javascript}{
	keywords={typeof, new, true, false, catch, function, return, null, catch, switch, var, if, in, while, do, else, case, break},
	keywordstyle=\color{blue}\bfseries,
	ndkeywords={class, export, boolean, throw, implements, import, this},
	ndkeywordstyle=\color{darkgray}\bfseries,
	identifierstyle=\color{black},
	sensitive=false,
	comment=[l]{//},
	morecomment=[s]{/*}{*/},
	commentstyle=\color{purple}\ttfamily,
	stringstyle=\color{red}\ttfamily,
	morestring=[b]',
	morestring=[b]"
}

% disponi sezioni
\usepackage{titlesec}

\titleformat{\section}
	{\sffamily\Large\bfseries} 
	{\thesection}{1em}{} 
\titleformat{\subsection}
	{\sffamily\large\bfseries}   
	{\thesubsection}{1em}{} 
\titleformat{\subsubsection}
	{\sffamily\normalsize\bfseries} 
	{\thesubsubsection}{1em}{}

% disponi alberi
\usepackage{forest}

\forestset{
	rectstyle/.style={
		for tree={rectangle,draw,font=\large\sffamily}
	},
	roundstyle/.style={
		for tree={circle,draw,font=\large}
	}
}

% disponi algoritmi
\usepackage{algorithm}
\usepackage{algorithmic}
\makeatletter
\renewcommand{\ALG@name}{Algoritmo}
\makeatother

% disponi numeri di pagina
\usepackage{fancyhdr}
\fancyhf{} 
\fancyfoot[L]{\sffamily{\thepage}}

\makeatletter
\fancyhead[L]{\raisebox{1ex}[0pt][0pt]{\sffamily{\@title \ \@date}}} 
\fancyhead[R]{\raisebox{1ex}[0pt][0pt]{\sffamily{\@author}}}
\makeatother

\begin{document}

% sezione (data)
\section{Lezione del 25-09-24}

% stili pagina
\thispagestyle{empty}
\pagestyle{fancy}

% testo
\subsection{Assegnamento di costo minimo}
Vediamo un problema:
\begin{problem}{Assegnamento}
	Quattro agenzie possono occuparsi di 4 progetti.
	Ogni agenzia presenta il costo stimato per la realizzazione di ogni progetto, in migliaia di euro.
	In forma tabulare, si riportano i valori:

	\center \rowcolors{2}{green!10}{green!40!black!20}
	\begin{tabular} { | c | c | c | c | c | }
		\hline
		& \bfseries Agenzia 1 & \bfseries Agenzia 2 & \bfseries Agenzia 3 & \bfseries Agenzia 4 \\
		\hline 
		\bfseries Progetto 1 & 20 & 17 & 16 & 14 \\
		\bfseries Progetto 2 & 22 & 16 & 19 & 15 \\
		\bfseries Progetto 3 & 21 & 17 & 15 & 23 \\ 
		\bfseries Progetto 4 & 19 & 18 & 14 & 24 \\
		\hline
	\end{tabular}

	\par\bigskip
	\raggedright
	
	Vogliamo capire quale agenzia deve occuparsi di quale progetto per minimizzare i costi.

\end{problem}

Con $n$ agenzie e progetti ci sono $n!$ possibili combinazioni, ergo dobbiamo trovare un algoritmo efficiente. 
Applicando il modello studiato finora, abbiamo la matrice dei costi $c$:

$$
\begin{pmatrix}
 20 & 17 & 16 & 14 \\
 22 & 16 & 19 & 15 \\
 21 & 17 & 15 & 23 \\
 19 & 18 & 14 & 24 \\
\end{pmatrix}
$$

che possiamo portare a:
$$ c: ( -18, +18 + ... + 24 ) $$
come linearizzazione lessicografica della tabella sopra riportata (notare che sarebbe un vettore colonna).

Adesso dobbiamo solo trovare un metodo per esplicitare i vincoli del problema:
\begin{itemize}
	\item Ogni agenzia può occuparsi solo di un progetto;
	\item Ogni progetto richiede solo un'agenzia.
\end{itemize}

Possiamo rappresentare la corrispondenza fra elementi come un grafo, e quindi riportarne una matrice d'adiacenza.
Assumendo di appaiare elementi con lo stesso numero, avremo:
$$
\begin{pmatrix}
	1 & 0 & 0 & 0 \\ 
	0 & 1 & 0 & 0 \\ 
	0 & 0 & 1 & 0 \\ 
	0 & 0 & 0 & 1 \\ 
\end{pmatrix}
$$

La caratteristica di questa matrice, chiamiamola $x$, e che ogni elemento $x_{ij}$ è:
$$
x_{ij} = 
	\begin{cases}
		0 \\ 1	
	\end{cases}
$$

Decidiamo di trattare la $x$ come un vettore linearizzato lessicograficamente dalla matrice, proprio come avevamo fatto per il vettore costo.
Per una matrice di adiacenza $n \times n$, di $n$ elementi in ogni categoria, questo vettore ha dimensione $n^2$. 
Questo semplifica la notazione del problema, e sopratutto della matrice $A$, che sarebbe lasciando $x$ matrice effettivamente un tensore.

Si dimostra quindi facilmente che i vincoli riportati prima possono quindi esprimersi come:
\[
	\begin{cases}
		x_{11} + x_{12} + x_{13} + x_{14} = 1	\\
		x_{21} + x_{22} + x_{23} + x_{24} = 1 \\ 
		x_{31} + x_{32} + x_{33} + x_{34} = 1 \\ 
		x_{41} + x_{42} + x_{43} + x_{44} = 1 \\ 
	\end{cases}
\]

per il primo punto, e:
\[
	\begin{cases}
		x_{11} + x_{21} + x_{31} + x_{41} = 1	\\
		x_{12} + x_{22} + x_{32} + x_{42} = 1 \\ 
		x_{13} + x_{23} + x_{33} + x_{43} = 1 \\ 
		x_{14} + x_{24} + x_{34} + x_{44} = 1 \\ 
	\end{cases}
\]

per il secondo.
Imponendo la positività, si hanno quindi le matrici $A$ e $b$:

$$
A: 
\begin{pmatrix}
	1 & 1 & 1 & 1 & 0... & &  & & & ...0 \\
	0... & & ...0 & 1 & 1 & 1 & 1 & 0... & & ...0 \\
	... \\
	0... & &  & & & ...0 & 1 & 1 & 1 & 1 \\
	...\\
	-1 & 0... & & & & & & & & ...0 \\
	...\\
	0... & & & & & & & & & -1 \\
\end{pmatrix}, \quad 
b: ( 1, ..., 1, 0, ..., 0)
$$

Si nota che il numero di vincoli necessari per $n$ elementi è $2n + n^2$.

\subsubsection{Assegnamento cooperativo e non cooperativo}
A questo punto conviene fare una distinzione.
Abbiamo definito finora il modello:

\[
	\begin{cases}
		\min{c^\intercal \cdot x} \\
		x_{11} + x_{12} + x_{13} + x_{14} = 1	\\
		...\\
		x_{41} + x_{42} + x_{43} + x_{44} = 1 \\
		x_{11} + x_{21} + x_{31} + x_{41} = 1	\\
		...\\
		x_{14} + x_{24} + x_{34} + x_{44} = 1 \\ 
 
	\end{cases}
\]

che così scritto non nega la possibilità di $x$ con componenti reali.
Nell'esempio ciò significa sono ammesse soluzioni dove più agenzie danno contributi reali ai progetti, che possiamo semanticamente interpretare come condividere il carico di lavoro, pur rispettando i vincoli imposti.
Decidiamo che questo è corretto se si parla di un problema di \textbf{assegnamento cooperativo}.
Visto che il problema posto non era di questo tipo, ma era di \textbf{assegnamento non cooperativo}, si introduce un'ulteriore vincolo:

$$
	x \in \mathbb{Z}^n
$$

Adesso ogni azienda darà un contributo intero al suo progetto, ergo coi vincoli imposti prima, ogni azienda sarà unica nel dirigere un solo progetto.

Più formalmente, possiamo dire che il passaggio ad assegnamento cooperativo comporta un \textbf{rilassamento} dei vincoli del problema.
Ovvero, in generale, se un problema non cooperativo ha minimo ottimale $nc$, il suo associato cooperativo avrà minimo ottimale $c$ con:

$$  c \leq nc $$

\subsubsection{Forma primale standard}
Portiamo quindi questo problema in una forma primale simile a quella vista per altri problemi LP, concesso il vincolo $x \in \mathbb{Z}^n$.

Finora avevamo usato le trasformazioni equivalenti per problemi LP:

\begin{enumerate}
	\item Trasformazione delle disuguaglianze: $ \geq \ \leftrightarrow \ \leq $
	\item Trasformazione delle uguaglianze:
		$$
			f(x) = c \ \rightarrow \
		\begin{cases}
			f(x) \leq c \\ 
			-f(x) \leq -c
		\end{cases}
		$$
	\item Trasformazione minimo / massimo: 
		$$
		\max{f} = -\min{f} \ \text{e sopratutto} \ 
		\bar{x} \in \mathrm{argmax}(f) \Leftrightarrow \bar{x} \in \mathrm{argmin}(-f)
		$$
\end{enumerate}

Possiamo applicare queste trasformazioni al modello, in particolare la (2), che porta il numero di vincoli a $4n + n^2$.

\subsection{Introduzione di surplus}
Vediamo un ulteriore tecnica per trasformare problemi LP: si può portare una disequazione del tipo:
$$ a_1x_1 + a_2x_2 + ... + a_nx_n \leq b $$
in un uguaglianza introducendo una variabile ausiliaria $s$:
$$ a_1x_1 + a_2x_2 + ... + a_nx_n + s = b $$
$s$ prende il nome di \textbf{slack}, in italiano scarto, o \textit{surplus}.
\end{document}


\documentclass[a4paper,11pt]{article}
\usepackage[a4paper, margin=8em]{geometry}

% usa i pacchetti per la scrittura in italiano
\usepackage[french,italian]{babel}
\usepackage[T1]{fontenc}
\usepackage[utf8]{inputenc}
\frenchspacing 

% usa i pacchetti per la formattazione matematica
\usepackage{amsmath, amssymb, amsthm, amsfonts}

% usa altri pacchetti
\usepackage{gensymb}
\usepackage{hyperref}
\usepackage{standalone}

% imposta il titolo
\title{Appunti Ricerca Operativa}
\author{Luca Seggiani}
\date{2024}

% disegni
\usepackage{pgfplots}
\pgfplotsset{width=10cm,compat=1.9}

% imposta lo stile
% usa helvetica
\usepackage[scaled]{helvet}
% usa palatino
\usepackage{palatino}
% usa un font monospazio guardabile
\usepackage{lmodern}

\renewcommand{\rmdefault}{ppl}
\renewcommand{\sfdefault}{phv}
\renewcommand{\ttdefault}{lmtt}

% disponi il titolo
\makeatletter
\renewcommand{\maketitle} {
	\begin{center} 
		\begin{minipage}[t]{.8\textwidth}
			\textsf{\huge\bfseries \@title} 
		\end{minipage}%
		\begin{minipage}[t]{.2\textwidth}
			\raggedleft \vspace{-1.65em}
			\textsf{\small \@author} \vfill
			\textsf{\small \@date}
		\end{minipage}
		\par
	\end{center}

	\thispagestyle{empty}
	\pagestyle{fancy}
}
\makeatother

% disponi teoremi
\usepackage{tcolorbox}
\newtcolorbox[auto counter, number within=section]{theorem}[2][]{%
	colback=blue!10, 
	colframe=blue!40!black, 
	sharp corners=northwest,
	fonttitle=\sffamily\bfseries, 
	title=Teorema~\thetcbcounter: #2, 
	#1
}

% disponi definizioni
\newtcolorbox[auto counter, number within=section]{definition}[2][]{%
	colback=red!10,
	colframe=red!40!black,
	sharp corners=northwest,
	fonttitle=\sffamily\bfseries,
	title=Definizione~\thetcbcounter: #2,
	#1
}

% disponi problemi
\newtcolorbox[auto counter, number within=section]{problem}[2][]{%
	colback=green!10,
	colframe=green!40!black,
	sharp corners=northwest,
	fonttitle=\sffamily\bfseries,
	title=Problema~\thetcbcounter: #2,
	#1
}

% disponi codice
\usepackage{listings}
\usepackage[table]{xcolor}

\lstdefinestyle{codestyle}{
		backgroundcolor=\color{black!5}, 
		commentstyle=\color{codegreen},
		keywordstyle=\bfseries\color{magenta},
		numberstyle=\sffamily\tiny\color{black!60},
		stringstyle=\color{green!50!black},
		basicstyle=\ttfamily\footnotesize,
		breakatwhitespace=false,         
		breaklines=true,                 
		captionpos=b,                    
		keepspaces=true,                 
		numbers=left,                    
		numbersep=5pt,                  
		showspaces=false,                
		showstringspaces=false,
		showtabs=false,                  
		tabsize=2
}

\lstdefinestyle{shellstyle}{
		backgroundcolor=\color{black!5}, 
		basicstyle=\ttfamily\footnotesize\color{black}, 
		commentstyle=\color{black}, 
		keywordstyle=\color{black},
		numberstyle=\color{black!5},
		stringstyle=\color{black}, 
		showspaces=false,
		showstringspaces=false, 
		showtabs=false, 
		tabsize=2, 
		numbers=none, 
		breaklines=true
}

\lstdefinelanguage{javascript}{
	keywords={typeof, new, true, false, catch, function, return, null, catch, switch, var, if, in, while, do, else, case, break},
	keywordstyle=\color{blue}\bfseries,
	ndkeywords={class, export, boolean, throw, implements, import, this},
	ndkeywordstyle=\color{darkgray}\bfseries,
	identifierstyle=\color{black},
	sensitive=false,
	comment=[l]{//},
	morecomment=[s]{/*}{*/},
	commentstyle=\color{purple}\ttfamily,
	stringstyle=\color{red}\ttfamily,
	morestring=[b]',
	morestring=[b]"
}

% disponi sezioni
\usepackage{titlesec}

\titleformat{\section}
	{\sffamily\Large\bfseries} 
	{\thesection}{1em}{} 
\titleformat{\subsection}
	{\sffamily\large\bfseries}   
	{\thesubsection}{1em}{} 
\titleformat{\subsubsection}
	{\sffamily\normalsize\bfseries} 
	{\thesubsubsection}{1em}{}

% disponi alberi
\usepackage{forest}

\forestset{
	rectstyle/.style={
		for tree={rectangle,draw,font=\large\sffamily}
	},
	roundstyle/.style={
		for tree={circle,draw,font=\large}
	}
}

% disponi algoritmi
\usepackage{algorithm}
\usepackage{algorithmic}
\makeatletter
\renewcommand{\ALG@name}{Algoritmo}
\makeatother

% disponi numeri di pagina
\usepackage{fancyhdr}
\fancyhf{} 
\fancyfoot[L]{\sffamily{\thepage}}

\makeatletter
\fancyhead[L]{\raisebox{1ex}[0pt][0pt]{\sffamily{\@title \ \@date}}} 
\fancyhead[R]{\raisebox{1ex}[0pt][0pt]{\sffamily{\@author}}}
\makeatother

\begin{document}

% sezione (data)
\section{Lezione del 26-09-24}

% stili pagina
\thispagestyle{empty}
\pagestyle{fancy}

% testo
\subsection{Geometria dei poliedri}
Introduciamo progressivamente i tipi di \textbf{combinazione} che ci sono utili nello studio dei problemi di programmazione lineare.

\subsubsection{Combinazioni lineari}
\begin{definition}{Combinazione lineare}
	Dati $ x_1, x_2, ..., x_k \in \mathbb{R}^n $ punti, $y$ si dice \textbf{combinazione lineare} di $ x_1, x_2, ..., x_k $ se:
	$$
	\exists \lambda_i \quad (i = 1, ..., k) \quad \text{t.c.} \quad y = \sum_{i=1}^k \lambda_i x_i 
	$$
\end{definition}

Le combinazioni lineari sono utili per esprimere la funzione obiettiva sulla base dei vettori costo, ma non bastano a trovarne una soluzione ottimale.

\subsubsection{Combinazioni convesse}
Si introduce quindi il concetto di:
\begin{definition}{Combinazione convessa}
	Dati $ x_1, x_2, ..., x_k \in \mathbb{R}^n $ punti, $y$ si dice \textbf{combinazione convessa} di $ x_1, x_2, ..., x_k $ se:
	$$
	\exists \lambda_i \in [0, 1] \quad (i = 1, ..., k), \quad \sum_{i=1}^k \lambda_i = 1 \quad \text{t.c.} \quad y = \sum_{i=1}^k \lambda_i x_i 
	$$
\end{definition}

Possiamo dare un esempio di cos'è la combinazione convessa di due punti in $\mathbb{R}^2$.
Posti $x_1$ e $x_2$, si ha:
$$
\lambda_1 + \lambda_2 = 1 \Rightarrow \lambda_2 = (1 - \lambda_1), \quad y = \lambda x_1 + (1 - \lambda ) x_2, \quad \lambda \in [0, 1]
$$
che riconosciamo essere l'equazione di un segmento $\bar{x_1x_2}$ (primo grafico).

Possiamo provare con tre punti: si avrà:
$$
y = \lambda_1 x_1 + \lambda_2 x_2 + \lambda_3 x_3, \quad \lambda_1 + \lambda_2 + \lambda_3 = 1, \quad \lambda_i \in [0, 1]
$$
che si riconduce all'equazione del triangolo di vertici $x_1$, $x_2$, $x_3$ (secondo grafico).
\par\medskip

\begin{minipage}{0.45\textwidth}
    % Segment on a graph
	\begin{tikzpicture}[scale=0.75]
    \begin{axis}[
        axis lines = center,
        xlabel = $x$, ylabel = $y$,
        xmin= -0.8, xmax=4.8, ymin=-0.8, ymax=4.8,
        grid = minor,
				title = {Combinazione di $x_1$ e $x_2$}
    ]
        \addplot[thick] coordinates {(1,2) (4,2)};
        \addplot[only marks, mark=*] coordinates {(1,2)} node[anchor=north] {$x_1$};
        \addplot[only marks, mark=*] coordinates {(4,2)} node[anchor=north] {$x_2$};
    \end{axis}
    \end{tikzpicture}
\end{minipage}
\hfill
\begin{minipage}{0.45\textwidth}
    % Triangle on a graph
    \begin{tikzpicture}[scale=0.75]
    \begin{axis}[
        axis lines = center,
        xlabel = $x$, ylabel = $y$,
        xmin= -0.8, xmax=4.8, ymin=-0.8, ymax=4.8,
        grid = minor,
				title = {Combinazione di $x_1$, $x_2$ e $x_3$}
    ]
        \addplot[thick] coordinates {(1,1) (4,1) (2,3) (1,1)};
        \addplot[only marks, mark=*] coordinates {(1,1)} node[anchor=north] {$x_1$};
        \addplot[only marks, mark=*] coordinates {(4,1)} node[anchor=north] {$x_2$};
        \addplot[only marks, mark=*] coordinates {(2,3)} node[anchor=south] {$x_3$};
    \end{axis}
    \end{tikzpicture}
\end{minipage}
\par\medskip

Dai grafici si nota come una combinazione convessa descrive una parte di spazio, che si può definire:
\begin{definition}{Involucro convesso}
	L'involucro convesso $\mathrm{conv}(K)$ di un'insieme di punti $K = \{x_1, x_2, ..., x_n\}, \quad x_i \in \mathbb{R}^n$ è definito come il luogo di tutte le loro combinazioni convesse.
\end{definition}

Si nota che l'involucro convesso negli esempi precedenti è effettivamente un poliedro convesso che contiene tutti i punti che lo formano.
Si può infatti dire:
\begin{theorem}{Minimalità dell'involucro convesso}
	L'insieme $\mathrm{conv}(K)$ di tutte le combinazioni convesse di $n$ punti è l'insieme convesso minimale che li contiene tutti.
\end{theorem}

In $R^n$, possiamo esprimere un insieme convesso lineare come un poliedro convesso, e quindi dire che l'involucro convesso di $n$ punti è il più piccolo (in termini di inclusione) poliedro convesso che li contiene tutti.
Inoltre, un'insieme è convesso se e solo se corrisponde al suo involucro convesso.

Si noti che non è detto che a $n$ punti corrisponda un poligono di $n$ vertici.
Potrebbe infatti accadere che uno dei punti (chiamiamolo $x_j$) sia già parte dell'involucro convesso, ergo $x_j \in \mathrm{conv}(K)$.

Le combinazioni convesse ci permettono di descrivere parte delle regioni ammissibili (poliedri) dei problemi di programmazione lineare, ma restano ancora in sospeso problemi che ammettono regioni illimitate.
Per descrivere tali regioni, si introduce un altro tipo di combinazione.

\subsubsection{Combinazioni coniche}
\begin{definition}{Combinazione conica}
	Dati $ x_1, x_2, ..., x_k \in \mathbb{R}^n $ punti, $y$ si dice \textbf{combinazione conica} di $ x_1, x_2, ..., x_k $ se:
	$$
	\exists \lambda_i \geq 0 \quad (i = 1, ..., k) \quad \text{t.c.} \quad y = \sum_{i=1}^k \lambda_i x_i 
	$$
\end{definition}

La combinazione conica di più punti non è più il poliedro convesso che li contiene, ma il cono con vertice nell'origine, convesso o meno, che li contiene, definito come:
\begin{definition}{Involucro conico}
	L'involucro conico $\mathrm{cono}(K)$ di un'insieme di punti $K = \{x_1, x_2, ..., x_n\} \in \mathbb{R}^n$ è definito come il luogo di tutte le loro combinazioni coniche.
\end{definition}

Questo cono si estende fino all'infinito ($\lambda_i \geq 0$) nelle direzioni dei vettori che lo formano.
Il concetto è simile a quello di spazio somma, ma con la differenza che non si va ovunque nello span dei due vettori, ma si seguono le semirette che essi conducono.

Si può dire, analogamente alle combinazioni convesse, che il cono di $n$ punti è il pi+ piccolo (in termini di inclusione) cono convesso che li contiene.
Inoltre, un'insieme è un cono convesso se e solo se corrisponde al suo involucro conico.

\subsection{Poliedri}
Abbiamo definito un poliedro come la regione definita da un sistema di disequazioni lineari, o geometricamente come l'intersezione di un numero finito di semipiani chiusi in $\mathbb{R}^n$.
Si dimostra che un poliedro, in quanto intersezione di insiemi convessi, è lui stesso convesso: potremo applicare la definizione di convessità per dire che, presi due punti $x_1, x_2 \in C$ nell'intersezione $C = C_1 \cap C_2$, si avrà che entrambi appartengono sia a $C_1$ che a $C_2$, ergo il segmento che li congiunge completamente in ciascuno di quei due insiemi è interamente contenuto in $C$.

Un poliedro che è anche cono si chiama cono poliedrico. Si dimostra che:
\begin{theorem}{Cono poliedrico}
	Se $P$ è un cono poliedrico allora:
	$$
	\exists Q \quad \text{t.c.} \quad P = \{ x \in \mathbb{R}^n : Qx \leq 0 \}
	$$
	con $Q$ matrice.
\end{theorem}

Senza dimostrazioni, questo è chiaro dal fatto che le disequazioni che compongono il poliedrico sono omogenee (hanno frontiere che passano dall'origine).

Si definiscono poi i \textbf{vertici} del poliedro:
\begin{definition}{Vertice}
	Un vertice di un poliedro è un punto che non si può esprimere come combinazione convessa propria di altri punti del poliedro.
	Si indica l'insieme dei vettori di un poliedro $P$ come $\mathrm{vert}(P)$.
\end{definition}
Notiamo che i vertici di un poliedro limitato corrispondono ai punti che formano la combinazione convessa equivalente al poliedro.
Per poliedri illimitati, introduciamo invece:
\begin{definition}{Direzione di recessione}
	Un vettore $d$ è la direzione di recessione di un poliedro se:
	$$
	x + \lambda d \in P \quad \forall x \in P, \quad \forall \lambda \geq 0
	$$
	Si indica come $\mathrm{rec}(P)$ l'insieme delle direzioni di recessione di un poliedro.
\end{definition}

Chiaramente, per ogni poliedro $P$, $0 \in \mathrm{rec}(P)$ e per i poliedri limitati, $\mathrm{rec}(P) = \{0\}$.
Notiamo che le direzioni di recessione determinano i vettori del cono che coincide (almeno a distanze abbastanza grandi dall'origine) con i poliedri illimitati.

Inoltre, l'insieme $\mathrm{rec}(p)$ di un poliedro è in sé un cono poliedrico. In particolare: 

\begin{theorem}{Cono di recessione}	
Il cono di recessione di un poliedro $P$ è dato da:
$$
P = \{ x \in \mathbb{R}^n : Ax \leq b \} \Rightarrow \mathrm{rec}(P) = \{ x \in \mathbb{R}^n : Ax \leq 0 \}
$$
\end{theorem}

\par\medskip
\noindent
\textbf{\textsf{Dimostrazione}} \\
Questo teorema vale perchè, per definizione, una direzione di recessione è $x + \lambda d$, che sostituita al sistema dà:
$$
A(x + \lambda d) \leq b \Rightarrow Ax + \lambda Ad \leq b
$$

Possiamo quindi sottrarre $Ax$ da entrambi i lati:
$$
\lambda Ad \leq b - Ax
$$

Da ipotesi, $Ax \leq b$ e quindi $b - Ax \geq 0$.
Chiamiamo questa quantità postiva $k^+$ e scriviamo:
$$
\lambda Ad \leq k^+
$$

Visto che $\lambda \rightarrow +\infty$, dovrà essere verò che $Ad \leq 0$, ergo con $d = x$, il teorema è dimostrato.

\par\medskip

Un'altro teorema importante lega il cono ai suoi "raggi estremi", cioè quell'insieme finito di vettori la cui combinazione conica corrisponde all'intero cono:
\begin{theorem}{Raggi estremi}
	Un cono poliedrico $P = \{ x \in \mathbb{R^n} : Ax \leq 0 \}$ è l'involucro conico di un'insieme limitato dei suoi punti. Questi punti prendono il nome di \textit{raggi estremi}.
\end{theorem}

\par\medskip
\noindent
\textbf{\textsf{Dimostrazione}} \\
Diamo una dimostrazione più o meno intuitiva di questo risultato.
Indichiamo con $A_1, ..., A_m$ le righe della matrice $A$. 
Queste formeranno dei vettori che indicano la direzione su cui "agisce" la diseguaglianza.
Si può semplicemente osservare che questi vettori sono perpendicolari agli estremi del cono.
Prendiamo quindi il cono dei vettori, che formerà un'altro cono detto \textit{duale} del cono di partenza:
$$
\mathrm{cono}(A_1, ..., A_m) = \{ x \in \mathbb{R}^n : Qx \leq 0 \}
$$

I vettori $Q_1, ..., Q_t$, presi come righe della matrice, sono perpendicolari ai vettori $A_1, ..., A_n$ di partenza, ergo paralleli ai raggi estremi del cono di partenza.
Possiamo quindi dire che:
$$
P = \mathrm{cono}(Q_1, ..., Q_t)
$$

\subsubsection{Spazio di linealità}

Definiamo infine lo spazio di linealità:
\begin{definition}{Spazio di linealità}
	Lo spazio di linealità di un poliedro illimitato $P$ è il più piccolo sottospazio contenuto interamente in $P$.
\end{definition}
Ogni vettore di base di uno spazio di linealità è un vettore $d$ tale che:
$$
d \in \mathrm{rec}(P), \quad - d \in \mathrm{rec}(P)
$$
ovvero un vettore che è contenuto sia positivo che negativo nelle direzioni di recessione del poliedro.

Questa distinzione è importante in quanto non si può dimostrare completamente il prossimo teorema su poliedri con spazio di linealità $\neq {0}$.

\subsubsection{Teorema di rappresentazione dei poliedri}
Gli strumenti che abbiamo stabilito finora ci permettono di enunciare un'importante risultato, noto come \textbf{teorema di rappresentazione dei poliedri}, o teorema di Minkowski-Weyl.

\begin{theorem}{Rappresentazione dei poliedri}
	Dato un poliedro $P$ definito come $P = \{ x \in \mathbb{R}^n : Ax \leq b \}$, si ha:
	$$
	\exists V = \{ v_1, ..., v_k \} \in \mathrm{vert}(P), \quad \exists E = {e_1, ..., e_p} \in \mathrm{rec}(P) \quad \text{t.c.} \quad P = \mathrm{conv}(V) + \mathrm{cono}(E)
	$$
\end{theorem}

Questo significa che è possibile rappresentare qualsiasi poliedro attraverso i suoi vertici, e le direzioni in cui si estende all'infinito (ergo le sue direzioni di recessione).

\subsubsection{H-rappresentazioni e V-rappresentazioni}
Possiamo intendere il risultato precedente come segue: per ogni poliedro di un problema di programmazione LP, abbiamo due possibili rappresentazioni:
\begin{itemize}
	\item \textbf{H-rappresentazione:} come intersezione di semispazi, ergo come insieme delle soluzioni di un sistema di disequazioni lineari.
	\item \textbf{V-rappresentazione:} come un'insieme di vertici e direzioni di regressione, quindi un involucro convesso e un cono di regressione.
\end{itemize}

\par\smallskip

Non diamo una dimostrazione del teorema ma possiamo, noti i concetti di H-rappresentazione e V-rappresentazione, dire che:
\begin{itemize}
	\item Da un \textbf{H-rappresentazione} si può ricavare una \textbf{V-rappresentazione}:
		abbiamo un poliedro dato dall'H-rappresentazione $Ax \leq b$.
		Dividiamo questo poliedro in una parte limitata e una parte illimitata: la parte limitata sarà data dai vertici, mentre la parte illimitata sarà data dal cono di regressione:
		$$ P = \mathrm{conv}(v_1,...,v_k) + rec(P)$$
		A questo punto possiamo esprimere, come dallo scorso lemma, come combinazione conica dei suoi raggi estremi:
		$$ P = \mathrm{conv}(v_1, ..., v_k) + \mathrm{cono}(e_1, ..., e_p) $$

	\item Da un \textbf{V-rappresentazione} si può ricavare una \textbf{H-rappresentazione}:
		abbiamo un poliedro dato dalla V-rappresentazione $P = \mathrm{conv}(v_1, ..., v_k) + \mathrm{cone}(e_1, ..., e_p)$.
		Prendiamo separatamente gli involucri convessi e conici: avremo che per l'involucro convesso possiamo immediatamente riportare in una forma ad intersezione di semipiani (disequazioni):
		$$ \mathrm{conv}(v_1, ..., v_k) = \{ x \in \mathbb{R}^n : A'x \leq b' \} $$
		mentre per l'involucro conico possiamo portare in una forma ad intersezione di disequazioni omogenee:
		$$ \mathrm{cone}(e_1, ..., e_p) = \{ x \in \mathbb{R}^n : A''x \leq 0 \} $$
		Per trovare l'H-rappresentazione, basterà combinare queste due rappresentazioni in una forma del tipo:
		$$ P = \{ x \in \mathbb{R}^n : A'x \leq b', A''x \leq 0 \}$$
\end{itemize}

La somma in $P = \mathrm{conv}(V) + \mathrm{cono}(E)$ si riferisce alla somma vettoriale fra tutti i possibili punti di $\mathrm{conv}(V)$ e $\mathrm{conv}(E)$, come quella studiata sui sottospazi vettoriali (anche se nessuno dei due insiemi è un sottospazio vettoriale).
Per un dato insieme $\mathrm{conv}(V)$, quindi, l'aggiunta di $\mathrm{cono}(E)$ rappresenta la "proiezione" di tale insieme nelle direzioni di recessione indicate dal cono.

Più propriamente, posto $\mathrm{lineal}(P) = {0}$, si ha:

\begin{theorem}{Rappresentazione dei poliedri non lineali}
	Dato un poliedro $P$ definito come $P = \{ x \in \mathbb{R}^n : Ax \leq b \}$, tale che $\mathrm{lineal}(P)$, si ha:
	$$
	P = \mathrm{conv}(\mathrm{vert}(P)) + \mathrm{rec}(P)
	$$
\end{theorem}
la limitazione di linealità è necessaria in quanto un poliedro lineale potrebbe non essere rappresentato, nelle sue dimensioni infinite, dal semplice insieme dei suoi vettori.
In verità, è possibile dimostrare che:

\begin{theorem}{Linealità di poliedri}
	Per ogni poliedro $P$ non vuoto si ha:
	$$
		\mathrm{vert}(P) \neq \emptyset \Leftrightarrow \mathrm{lineal}(P) = {0}
	$$
\end{theorem}
ergo applicando lo scorso teorema potremmo provare a rappresentare un poliedro attraverso un'insieme di vettori vuoto.

Per i poliedri che otteniamo dai problemi di programmazione lineare, però, abbiamo i corollari:
\begin{itemize}
	\item Un poliedro limitato è l'involucro convesso dei suoi vertici;
	\item Se il poliedro ha vincoli di positività sulle sue variabili, allora non è lineale, ergo si applica il teorema di rappresentazione.
		Questo è il tipo di poliedri a cui siamo abituati.
\end{itemize}

\subsection{Teorema fondamentale della PL}
Quanto riportare finora sulla geometria dei poliedri può essere usato per dimostrare il seguente teorema:
\begin{theorem}{Teorema fondamentale della PL}
	Sia dato un poliedro $P$ rappresentato come:
	$$ 
	P = \mathrm{conv}(V) + \mathrm{cono}(E), \quad V = \{ v_1, ..., v_k \}, \quad E = \{ e_1, ..., e_p \}
	$$
	Se il problema $\mathcal{P}$ con regione ammissibile $P$ ha valore ottimo finito, allora esiste $s \in \{ 1, ..., k \}$ tale che $v_k$ è soluzione ottima di $\mathcal{P}$.
\end{theorem}

In sostanza, se un problema LP ha soluzione, essa si trova su uno dei vertici del poliedro della regione ammissibile.

\par\medskip
\noindent
\textbf{\textsf{Dimostrazione}} \\
Sia dato un problema LP $\mathcal{P}$ in forma primale standard, ergo posto come:
\[
	\begin{cases}
		\max{c^\intercal \cdot x} \\ 
		Ax \leq b	
	\end{cases}
\]
ergo con regione ammissibile rappresentata da un poliedro $P$.

Dal teorema della rappresentazione, possiamo esprimere il poliedro come:
$$
	P = \mathrm{conv}(V) + \mathrm{cono}(E), \quad V = \{ v_1, ..., v_k \}, \quad E = \{ e_1, ..., e_p \}
$$

Combiniamo le due equazioni, ergo esprimiamo prima il punto $\bar{x}$ generico del poliedro applicando le definizioni di involucro convesso e conico:
$$
\bar{x} \in P : P = \mathrm{conv}(V) + \mathrm{cono}(E), \quad \bar{x} = \sum_{i=1}^k \lambda_i v_i + \sum_{j=i}^p \mu_j e_j
$$
ed esprimiamo quindi la funzione obiettivo come il prodotto scalare fra il vettore costo e il punto $\bar{x}$ del poliedro:
$$
c^\intercal \cdot \bar{x} = c^\intercal \cdot \left( \sum_{i=1}^k \lambda_i v_i + \sum_{j=i}^p \mu_j e_j \right) = \sum_{i=1}^k \lambda_i c^\intercal v_i + \sum_{j=i}^p \mu_j c^\intercal e_j
$$

A questo punto conviene chiarire su cosa significa che il problema ha valore ottimo finito.
Il secondo termine è la combinazione conica della sommatoria dei vettori di recessione scalati dal vettore costo.
Se almeno uno dei $c^\intercal e_j > 0$, si avrà che portando $\mu_j \rightarrow +\infty$ la funzione avrà massimo $= \infty$.
Geometricamente, questo significa che esiste una direzione illimitata del poliedro dove i vettori costo permettono alla funzione di crescere all'infinito.

Dunque sarà vero che $c^\intercal e_j \leq 0 \quad \forall j \in \{ 1, ..., p \}$ se vogliamo che la funzione abbia valore ottimo finito.

Possiamo quindi usare questa ipotesi per dire:
$$
c^\intercal \cdot \bar{x} = \sum_{i=1}^k \lambda_i c^\intercal v_i + \sum_{j=i}^p \mu_j c^\intercal e_j \leq \sum_{i=1}^k \lambda_i c^\intercal v_i \leq \sum_{i=1}^k \max_{1 \leq i \leq p} \left(\lambda_i c^\intercal v_i\right) 
$$

$$
= \left( \max_{1 \leq i \leq p} c^\intercal v_i\right) \sum_{i=1}^k \lambda_i = \max_{1\leq i \leq p} c^\intercal v_i = c^\intercal v_k
$$

E quindi $\max_{x \in P} c^\intercal \cdot x \leq c^\intercal v_k$.
A questo punto, visto che $\bar{x}$ è effettivamente un punto della regione ammissibile, sarà vero che:
$$
c^\intercal v_k \leq \max_{x \in P} c^\intercal \cdot x
$$

E dunque $\max_{x \in P} c^\intercal \cdot x = c^\intercal v_k$, come volevasi dimostrare. 

\end{document}


\documentclass[a4paper,11pt]{article}
\usepackage[a4paper, margin=8em]{geometry}

% usa i pacchetti per la scrittura in italiano
\usepackage[french,italian]{babel}
\usepackage[T1]{fontenc}
\usepackage[utf8]{inputenc}
\frenchspacing 

% usa i pacchetti per la formattazione matematica
\usepackage{amsmath, amssymb, amsthm, amsfonts}

% usa altri pacchetti
\usepackage{gensymb}
\usepackage{hyperref}
\usepackage{standalone}

% imposta il titolo
\title{Appunti Ricerca Operativa}
\author{Luca Seggiani}
\date{2024}

% disegni
\usepackage{pgfplots}
\pgfplotsset{width=10cm,compat=1.9}

% imposta lo stile
% usa helvetica
\usepackage[scaled]{helvet}
% usa palatino
\usepackage{palatino}
% usa un font monospazio guardabile
\usepackage{lmodern}

\renewcommand{\rmdefault}{ppl}
\renewcommand{\sfdefault}{phv}
\renewcommand{\ttdefault}{lmtt}

% disponi il titolo
\makeatletter
\renewcommand{\maketitle} {
	\begin{center} 
		\begin{minipage}[t]{.8\textwidth}
			\textsf{\huge\bfseries \@title} 
		\end{minipage}%
		\begin{minipage}[t]{.2\textwidth}
			\raggedleft \vspace{-1.65em}
			\textsf{\small \@author} \vfill
			\textsf{\small \@date}
		\end{minipage}
		\par
	\end{center}

	\thispagestyle{empty}
	\pagestyle{fancy}
}
\makeatother

% disponi teoremi
\usepackage{tcolorbox}
\newtcolorbox[auto counter, number within=section]{theorem}[2][]{%
	colback=blue!10, 
	colframe=blue!40!black, 
	sharp corners=northwest,
	fonttitle=\sffamily\bfseries, 
	title=Teorema~\thetcbcounter: #2, 
	#1
}

% disponi definizioni
\newtcolorbox[auto counter, number within=section]{definition}[2][]{%
	colback=red!10,
	colframe=red!40!black,
	sharp corners=northwest,
	fonttitle=\sffamily\bfseries,
	title=Definizione~\thetcbcounter: #2,
	#1
}

% disponi problemi
\newtcolorbox[auto counter, number within=section]{problem}[2][]{%
	colback=green!10,
	colframe=green!40!black,
	sharp corners=northwest,
	fonttitle=\sffamily\bfseries,
	title=Problema~\thetcbcounter: #2,
	#1
}

% disponi codice
\usepackage{listings}
\usepackage[table]{xcolor}

\lstdefinestyle{codestyle}{
		backgroundcolor=\color{black!5}, 
		commentstyle=\color{codegreen},
		keywordstyle=\bfseries\color{magenta},
		numberstyle=\sffamily\tiny\color{black!60},
		stringstyle=\color{green!50!black},
		basicstyle=\ttfamily\footnotesize,
		breakatwhitespace=false,         
		breaklines=true,                 
		captionpos=b,                    
		keepspaces=true,                 
		numbers=left,                    
		numbersep=5pt,                  
		showspaces=false,                
		showstringspaces=false,
		showtabs=false,                  
		tabsize=2
}

\lstdefinestyle{shellstyle}{
		backgroundcolor=\color{black!5}, 
		basicstyle=\ttfamily\footnotesize\color{black}, 
		commentstyle=\color{black}, 
		keywordstyle=\color{black},
		numberstyle=\color{black!5},
		stringstyle=\color{black}, 
		showspaces=false,
		showstringspaces=false, 
		showtabs=false, 
		tabsize=2, 
		numbers=none, 
		breaklines=true
}

\lstdefinelanguage{javascript}{
	keywords={typeof, new, true, false, catch, function, return, null, catch, switch, var, if, in, while, do, else, case, break},
	keywordstyle=\color{blue}\bfseries,
	ndkeywords={class, export, boolean, throw, implements, import, this},
	ndkeywordstyle=\color{darkgray}\bfseries,
	identifierstyle=\color{black},
	sensitive=false,
	comment=[l]{//},
	morecomment=[s]{/*}{*/},
	commentstyle=\color{purple}\ttfamily,
	stringstyle=\color{red}\ttfamily,
	morestring=[b]',
	morestring=[b]"
}

% disponi sezioni
\usepackage{titlesec}

\titleformat{\section}
	{\sffamily\Large\bfseries} 
	{\thesection}{1em}{} 
\titleformat{\subsection}
	{\sffamily\large\bfseries}   
	{\thesubsection}{1em}{} 
\titleformat{\subsubsection}
	{\sffamily\normalsize\bfseries} 
	{\thesubsubsection}{1em}{}

% disponi alberi
\usepackage{forest}

\forestset{
	rectstyle/.style={
		for tree={rectangle,draw,font=\large\sffamily}
	},
	roundstyle/.style={
		for tree={circle,draw,font=\large}
	}
}

% disponi algoritmi
\usepackage{algorithm}
\usepackage{algorithmic}
\makeatletter
\renewcommand{\ALG@name}{Algoritmo}
\makeatother

% disponi numeri di pagina
\usepackage{fancyhdr}
\fancyhf{} 
\fancyfoot[L]{\sffamily{\thepage}}

\makeatletter
\fancyhead[L]{\raisebox{1ex}[0pt][0pt]{\sffamily{\@title \ \@date}}} 
\fancyhead[R]{\raisebox{1ex}[0pt][0pt]{\sffamily{\@author}}}
\makeatother

\begin{document}

% sezione (data)
\section{Lezione del 18-11-24}

% stili pagina
\thispagestyle{empty}
\pagestyle{fancy}

% testo
\subsection{Problema di flusso massimo}
Poniamo di avere un grafo su cui riportiamo solamente le capacità superiori $u_{ij}$ sui singoli archi: 

\begin{center}
	\begin{tikzpicture}
		\node[circle, draw=black] (1) at (0,0) {1};
		\node[circle, draw=black] (2) at (2,1) {2};
		\node[circle, draw=black] (3) at (2,-1) {3};
		\node[circle, draw=black] (4) at (4,1) {4};
		\node[circle, draw=black] (5) at (4,-1) {5};
		\node[circle, draw=black] (6) at (6, 0) {6};
		\draw[->, to path={-| (\tikztotarget)}] (1) -- (2);
		\draw[->, to path={-| (\tikztotarget)}] (1) -- (3);
		\draw[->, to path={-| (\tikztotarget)}] (2) -- (3);
		\draw[->, to path={-| (\tikztotarget)}] (2) -- (4);
		\draw[->, to path={-| (\tikztotarget)}] (3) -- (5);
		\draw[->, to path={-| (\tikztotarget)}] (5) -- (4);
		\draw[->, to path={-| (\tikztotarget)}] (3) -- (4);
		\draw[->, to path={-| (\tikztotarget)}] (4) -- (6);
		\draw[->, to path={-| (\tikztotarget)}] (5) -- (6);

		\node at (1, 1)  {$8$};
		\node at (1, -1)  {$9$};

		\node at (3, 1.5)  {$2$};
		\node at (3, -1.5)  {$7$};
		\node at (3, 0.5)  {$8$};

		\node at (1.5, 0)  {$3$};
		\node at (4.5, 0)  {$6$};

		\node at (5, 1)  {$10$};
		\node at (5, -1)  {$4$};
	\end{tikzpicture}
\end{center}

Prendiamo due nodi $s$ e $t$, e cerchiamo di \textbf{massimizzare} il flusso da $s$ a $t$.
L'idea fondamentale che il flusso che parte da $s$ dovrà essere uguale al flusso che arriva in $t$.
Potremo partire da un flusso ammissibile qualsiasi, cioè che si limita a rispettare le capacità:
$$
x = \left( 2, 4, 0, 2, 0, 4, 2, 0, 4 \right)
$$
Notando che le capacità sugli \textit{ultimi} archi influenzano quelle sugli archi precedenti (ad esempio, $(3,5)$ è limitato a 4 da $(5,6)$ con $u_{56} = 4$).
Poniamo allora, più intelligentemente, un problema di PL:
\[
	\begin{cases}
		\max v \\ 
		Ex = b \\ 
		0 \leq x \leq u
	\end{cases}
\]
dove $E$ è la matrice di incidenza della rete, e i bilanci $b_i$ stessi dipendono dalla variabile $v$: 
\[
	b_i = 
	\begin{cases}
		-v, \quad i = s \\ 
		0, \quad i \neq s \wedge i \neq t \\ 
		v, \quad i = t
	\end{cases}
\]

Potremmo avere dubbi sul fatto che la  variabile $v$ compare nei bilanci $b$.
Scriviamo per esteso le uguaglianze dei vincoli:
\[
	\begin{cases}
		-x_{12} - x_{13} = -v \\ 
		x_{12} - x_{23} - x_{24} = 0 \\ 
		x_{13} + x_{23} - x_{34} - x_{35} = 0 \\ 
		x_{24} + x_{34} + x_{54} - x_{46} = 0 \\ 
		x_{35} - x_{54} - x_{56} = 0 \\ 
		x_{46} + x_{56} = v
	\end{cases}
\]

Notiamo che questo problema è effettivamente un problema di flusso minimo (massimo) su $n+1$ variabili per $n$ nodi, dove la $n$-esima variabile è proprio il flusso $v$.
Inoltre, con capacità massime intere, anche $v$ sarà necessariamente intero (dato da somma di interi) e quindi la matrice \textbf{unimodulare}, con la conseguenza già vista che $\text{PL} = \text{ILP}$.

Inoltre, possiamo portare i termini $v$ di $b$ a sinistra delle rispettive equazioni, ottenendo effettivamente la matrice:
$$
\begin{pmatrix}
 & 1 \\ 
E & 0 \\ 
 & -1
\end{pmatrix}
\begin{pmatrix}
x \\ v
\end{pmatrix}
= 0
$$
cioè dove si è introdotto la un \textbf{arco fittizio}, quello che parte da $t$ e arriva in $s$.
Capovolgendo la funzione obiettivo (e riportando in vista i termini di costo nullo su ogni arco $x_{ij}$), otteniamo quindi:
\[
	\begin{cases}
		\min 0 \cdot x - v \\ 	
\begin{pmatrix}
 & 1 \\ 
E & 0 \\ 
 & -1
\end{pmatrix}
\begin{pmatrix}
x \\ v
\end{pmatrix}
= 0 \\ 
0 \leq x \leq u
	\end{cases}
\]
dove per l'ultimo arco fittizio la capacità massima $u$ è un $M$ molto grande (o $\infty$).

Questo è un problema di flusso di costo minimo capacitato che include l'arco fittizio come unico arco a costo diverso da 0 (per giunta negativo), cioè che è "costretto" a imporre flusso massimo da $t$ ad $s$, e visto che tutti i nodi sono a bilancio 0, a riportarlo in direzione opposta da $s$ a $t$ lungo i nodi della rete vera e propria.

A questo punto possiamo proporre la soluzione ammissibile:
$$
(x,v) = \left( 2, 9, 0, 2, 5, 4, 7, 0, 4, 11 \right)
$$
da cui si ricavano le partizioni, controllando quali archi si svuotano e quali saturano:
$$
T = \left\{ (1,2), (3,4), (3,5), (4,6), (6,1) \right\}, \quad L = \left\{ (2,3), (5,4) \right\}, \quad U = \left\{ (1,3), (2,4), (5,6) \right\}
$$
Possiamo quindi ricavare il potenziale dell'albero di copertura, notando che dividerà necessariamente i nodi in partizione di potenziale 0 e potenziale 1:
$$
\pi = \left( 0, 0, 1, 1 ,1 \right)
$$
e calcolare i costi ridotti, che troviamo negativi su $(2,3) \in L$, quindi arco entrante:
$$
c^\pi_{23} = 0 + 0 - 1 = -1
$$
che è quanto ci aspettavamo, in quanto chiaramente $v = 14$ (dagli archi entranti nel nodo 6).

Procediamo quindi con l'eliminazione del ciclo, distinguendo innanzitutto le pratizioni $\mathcal{C}^+$ e $\mathcal{C}^-$:
$$
\mathcal{C}^+ = \left\{ (1,2), (2,3), (3,4), (4,6), (6, 1) \right\}, \quad \mathcal{C}^- = \emptyset
$$
Su $\mathcal{C}^-$ mettiamo $\vartheta^- = \infty$ (essenzialmente vogliamo considerare solo $\vartheta^+$).
Calcoliamo allora $\vartheta^+$ dagli $u_{ij} - x_{ij}$, tenendo conto che la capacità $u$ di $(6,1)$ (arco fittizio) è $\infty$:
$$
\vartheta^+ = \max \left\{ 6, 3, 3, 3, \infty \right\}
$$
da cui prendiamo $\vartheta^+ = 3$ e l'arco uscente $(2,3)$ per l'ordinamento lessicografico.
Notiamo di essere nel caso particolare dove l'arco \textit{entra} ed \textit{esce} (in questo caso si sposta da L a U).

Aggiungendo quindi il $\vartheta$ agli archi in $\mathcal{C}^+$ otteniamo il flusso:
$$
(x, v) = \left( 5, 9, 3, 2, 8, 4, 10, 0, 4, 14 \right)
$$
da cui $v=14$, come ci aspettavamo.
Vedremo in seguito un algoritmo per il calcolo del flusso massimo su reti più efficiente dell'applicazione diretta del simplesso.

\end{document}


\documentclass[a4paper,11pt]{article}
\usepackage[a4paper, margin=8em]{geometry}

% usa i pacchetti per la scrittura in italiano
\usepackage[french,italian]{babel}
\usepackage[T1]{fontenc}
\usepackage[utf8]{inputenc}
\frenchspacing 

% usa i pacchetti per la formattazione matematica
\usepackage{amsmath, amssymb, amsthm, amsfonts}

% usa altri pacchetti
\usepackage{gensymb}
\usepackage{hyperref}
\usepackage{standalone}

% imposta il titolo
\title{Appunti Ricerca Operativa}
\author{Luca Seggiani}
\date{2024}

% disegni
\usepackage{pgfplots}
\pgfplotsset{width=10cm,compat=1.9}

% imposta lo stile
% usa helvetica
\usepackage[scaled]{helvet}
% usa palatino
\usepackage{palatino}
% usa un font monospazio guardabile
\usepackage{lmodern}

\renewcommand{\rmdefault}{ppl}
\renewcommand{\sfdefault}{phv}
\renewcommand{\ttdefault}{lmtt}

% disponi il titolo
\makeatletter
\renewcommand{\maketitle} {
	\begin{center} 
		\begin{minipage}[t]{.8\textwidth}
			\textsf{\huge\bfseries \@title} 
		\end{minipage}%
		\begin{minipage}[t]{.2\textwidth}
			\raggedleft \vspace{-1.65em}
			\textsf{\small \@author} \vfill
			\textsf{\small \@date}
		\end{minipage}
		\par
	\end{center}

	\thispagestyle{empty}
	\pagestyle{fancy}
}
\makeatother

% disponi teoremi
\usepackage{tcolorbox}
\newtcolorbox[auto counter, number within=section]{theorem}[2][]{%
	colback=blue!10, 
	colframe=blue!40!black, 
	sharp corners=northwest,
	fonttitle=\sffamily\bfseries, 
	title=Teorema~\thetcbcounter: #2, 
	#1
}

% disponi definizioni
\newtcolorbox[auto counter, number within=section]{definition}[2][]{%
	colback=red!10,
	colframe=red!40!black,
	sharp corners=northwest,
	fonttitle=\sffamily\bfseries,
	title=Definizione~\thetcbcounter: #2,
	#1
}

% disponi problemi
\newtcolorbox[auto counter, number within=section]{problem}[2][]{%
	colback=green!10,
	colframe=green!40!black,
	sharp corners=northwest,
	fonttitle=\sffamily\bfseries,
	title=Problema~\thetcbcounter: #2,
	#1
}

% disponi codice
\usepackage{listings}
\usepackage[table]{xcolor}

\lstdefinestyle{codestyle}{
		backgroundcolor=\color{black!5}, 
		commentstyle=\color{codegreen},
		keywordstyle=\bfseries\color{magenta},
		numberstyle=\sffamily\tiny\color{black!60},
		stringstyle=\color{green!50!black},
		basicstyle=\ttfamily\footnotesize,
		breakatwhitespace=false,         
		breaklines=true,                 
		captionpos=b,                    
		keepspaces=true,                 
		numbers=left,                    
		numbersep=5pt,                  
		showspaces=false,                
		showstringspaces=false,
		showtabs=false,                  
		tabsize=2
}

\lstdefinestyle{shellstyle}{
		backgroundcolor=\color{black!5}, 
		basicstyle=\ttfamily\footnotesize\color{black}, 
		commentstyle=\color{black}, 
		keywordstyle=\color{black},
		numberstyle=\color{black!5},
		stringstyle=\color{black}, 
		showspaces=false,
		showstringspaces=false, 
		showtabs=false, 
		tabsize=2, 
		numbers=none, 
		breaklines=true
}

\lstdefinelanguage{javascript}{
	keywords={typeof, new, true, false, catch, function, return, null, catch, switch, var, if, in, while, do, else, case, break},
	keywordstyle=\color{blue}\bfseries,
	ndkeywords={class, export, boolean, throw, implements, import, this},
	ndkeywordstyle=\color{darkgray}\bfseries,
	identifierstyle=\color{black},
	sensitive=false,
	comment=[l]{//},
	morecomment=[s]{/*}{*/},
	commentstyle=\color{purple}\ttfamily,
	stringstyle=\color{red}\ttfamily,
	morestring=[b]',
	morestring=[b]"
}

% disponi sezioni
\usepackage{titlesec}

\titleformat{\section}
	{\sffamily\Large\bfseries} 
	{\thesection}{1em}{} 
\titleformat{\subsection}
	{\sffamily\large\bfseries}   
	{\thesubsection}{1em}{} 
\titleformat{\subsubsection}
	{\sffamily\normalsize\bfseries} 
	{\thesubsubsection}{1em}{}

% disponi alberi
\usepackage{forest}

\forestset{
	rectstyle/.style={
		for tree={rectangle,draw,font=\large\sffamily}
	},
	roundstyle/.style={
		for tree={circle,draw,font=\large}
	}
}

% disponi algoritmi
\usepackage{algorithm}
\usepackage{algorithmic}
\makeatletter
\renewcommand{\ALG@name}{Algoritmo}
\makeatother

% disponi numeri di pagina
\usepackage{fancyhdr}
\fancyhf{} 
\fancyfoot[L]{\sffamily{\thepage}}

\makeatletter
\fancyhead[L]{\raisebox{1ex}[0pt][0pt]{\sffamily{\@title \ \@date}}} 
\fancyhead[R]{\raisebox{1ex}[0pt][0pt]{\sffamily{\@author}}}
\makeatother

\begin{document}

% sezione (data)
\section{Lezione del 01-10-24}

% stili pagina
\thispagestyle{empty}
\pagestyle{fancy}

% testo
\subsection{Soluzioni di base primali degeneri}
Abbiamo dato un teorema di caratterizzazione dei vertici primali.
Questo teorema si basava sulla nozione di \textbf{soluzione di base primale}.
Possiamo fare una distinzione fra soluzione di base degeneri e non degeneri:
\begin{definition}{Soluzione di base degenere}
Quando una soluzione di base è soluzione di più combinazioni delle disequazioni del problema, essa si dice degenere.
\end{definition}

Questa definizione è esatta ma non particolarmente utile.
Sostanzialmente, ci dice soltanto che una soluzione degenere è \textbf{ridondante} su più combinazioni di disequazioni (cioè risolve $A_Bx = b_B$ su più permutazioni degli $1,...,m$ elementi in classi $n$ in $B$).
Si noti che ridondante non significa \textbf{eliminabile}: questa affermazione purtroppo è vera soltanto in $R^2$, dove effettivamente si può rimuovere una delle disequazioni ridondanti ed avere sempre lo stesso risultato.

Diamo quindi una caratterizzazione delle soluzioni di base primali degeneri appoggiandoci al teorema di caratterizzazione dei vertici, ergo al concetto di soluzione di base:
\begin{theorem}{Caratterizzazione delle soluzioni di base primali degeneri}
	Se una soluzione è di base, ergo scelto $B = \{ 1, ..., m \}$ con $\mathrm{card}(B) = n$ è data da $A_Bx = b_B$, possiamo dire che è pure degenere quando $\exists i \in N$ t.c. $A_i x = b_i$, con $I$ = $\{1, ..., m\} - B$. 
\end{theorem}
Quindi, una soluzione di base è degenere quando almeno una variabile di base si annulla per almeno una delle disequazioni non di base indicate dagli indici $I$, che sono tutti gli indici fra $\{1,...,m\}$ non contenuti in $B$. 

Sulla stessa linea di pensiero, possiamo dimostrare un'altro teorema, stavolta sul concetto piuttosto intuitivo di ammissibilità.
Potremmo infatti dire che una soluzione di base ammissibile, cioè che rientra all'interno della regione ammissibile, è tale se:
\begin{theorem}{Caratterizzazione delle soluzioni di base primali ammissibili}
	Se una soluzione è di base, ergo scelto $B = \{ 1, ..., m \}$ con $\mathrm{card}(B) = n$ è data da $A_Bx = b_B$, possiamo dire che è ammissible quando $\forall i \in N$ si ha $A_i x \leq b_i$, con $I$ = $\{1, ..., m\} - B$. 
\end{theorem}
cioè banalmente rispetta tutte le disequazioni.

\subsubsection{Considerazioni numeriche sui numeri di soluzioni base}
Solitamente un problema con $n$ variabili decisionali a $m \geq n$ vincoli.
Posti questi vincoli, visto che per calcolare $\mathrm{vert}(P)$ prendiamo effettivamente tutte le combinazioni degli $m$ vincoli classe $n$ variabili decisionali, possiamo usare il coefficiente binomiale per calcolare il numero massimo di potenziali vertici: 
$$ \mathrm{card}(\mathrm{vert}(P)) \sim \binom{m}{n} = \frac{m!}{n!(m-n)!} $$

In verità, i vertici sono solitamente meno, in quanto possiamo rimuovere le soluzioni non ammissibili.
Inoltre, le soluzioni degeneri non contribuiscono al risultato, ergo anche quelle non sono rilevanti.

\subsection{Riassunto delle trasformazioni equivalenti}
Riassumiamo adesso le trasformazioni equivalenti che abbiamo individuato finora per le disequazioni di problemi LP:
\begin{itemize}
	\item $\mathrm{min}(C^\intercal \cdot x) \leftrightarrow \mathrm{max}(C^\intercal \cdot x)$: trasformiamo problemi di massimo in problemi di minimo invertendo i segni;
	\item $Ax \geq b \leftrightarrow -Ax \leq -b$: invertiamo il verso della diseguaglianza moltiplicando per $-1$;
	\item $Ax = b \rightarrow Ax \leq b \wedge Ax \geq b$: convertiamo un'uguaglianza in una coppia di diseguaglianze;
	\item $Ax \leq b \rightarrow Ax + s = b$: convertiamo una diseguaglianza in un'uguaglianza introducendo una variabile di surplus. Si nota che la variabile di surplus può essere riconosciuta per essere rimossa, da:
		\begin{itemize}
			\item $s > 0$;
			\item Compare in un solo vincolo, che è di uguaglianza;
			\item Ha coefficiente $0$ nella funzione obiettivo, e $1$ nell'equazione dove compare;
		\end{itemize}
	\item $ x \mathrel{\text{\tikz[baseline]{
    \draw (0,1.1ex)--(1.1em,0.1ex);
    \node[scale=1] at (0.6em,0.3em) {$\geq$};
  }}} 0 \rightarrow x = x^+ - x^-, \quad x^+ \geq 0, \quad x^- \geq 0$: aggiriamo il vincolo di positività introducendo parti positive e negative delle variabili decisionali, con rispettivi vincoli di positività.
\end{itemize}

Usiamo queste trasformazioni per portare i problemi LP in forme standard.
Esistono molteplici forme standard, ma in questo corso ci riguardano: il formato \textit{linprog}, usato dal pacchetto software \textit{MATLAB}, le forme standard primale (già vista) e duale (che vedremo fra poco). 

\end{document}


\documentclass[a4paper,11pt]{article}
\usepackage[a4paper, margin=8em]{geometry}

% usa i pacchetti per la scrittura in italiano
\usepackage[french,italian]{babel}
\usepackage[T1]{fontenc}
\usepackage[utf8]{inputenc}
\frenchspacing 

% usa i pacchetti per la formattazione matematica
\usepackage{amsmath, amssymb, amsthm, amsfonts}

% usa altri pacchetti
\usepackage{gensymb}
\usepackage{hyperref}
\usepackage{standalone}

% imposta il titolo
\title{Appunti Ricerca Operativa}
\author{Luca Seggiani}
\date{2024}

% disegni
\usepackage{pgfplots}
\pgfplotsset{width=10cm,compat=1.9}

% imposta lo stile
% usa helvetica
\usepackage[scaled]{helvet}
% usa palatino
\usepackage{palatino}
% usa un font monospazio guardabile
\usepackage{lmodern}

\renewcommand{\rmdefault}{ppl}
\renewcommand{\sfdefault}{phv}
\renewcommand{\ttdefault}{lmtt}

% disponi il titolo
\makeatletter
\renewcommand{\maketitle} {
	\begin{center} 
		\begin{minipage}[t]{.8\textwidth}
			\textsf{\huge\bfseries \@title} 
		\end{minipage}%
		\begin{minipage}[t]{.2\textwidth}
			\raggedleft \vspace{-1.65em}
			\textsf{\small \@author} \vfill
			\textsf{\small \@date}
		\end{minipage}
		\par
	\end{center}

	\thispagestyle{empty}
	\pagestyle{fancy}
}
\makeatother

% disponi teoremi
\usepackage{tcolorbox}
\newtcolorbox[auto counter, number within=section]{theorem}[2][]{%
	colback=blue!10, 
	colframe=blue!40!black, 
	sharp corners=northwest,
	fonttitle=\sffamily\bfseries, 
	title=Teorema~\thetcbcounter: #2, 
	#1
}

% disponi definizioni
\newtcolorbox[auto counter, number within=section]{definition}[2][]{%
	colback=red!10,
	colframe=red!40!black,
	sharp corners=northwest,
	fonttitle=\sffamily\bfseries,
	title=Definizione~\thetcbcounter: #2,
	#1
}

% disponi problemi
\newtcolorbox[auto counter, number within=section]{problem}[2][]{%
	colback=green!10,
	colframe=green!40!black,
	sharp corners=northwest,
	fonttitle=\sffamily\bfseries,
	title=Problema~\thetcbcounter: #2,
	#1
}

% disponi codice
\usepackage{listings}
\usepackage[table]{xcolor}

\lstdefinestyle{codestyle}{
		backgroundcolor=\color{black!5}, 
		commentstyle=\color{codegreen},
		keywordstyle=\bfseries\color{magenta},
		numberstyle=\sffamily\tiny\color{black!60},
		stringstyle=\color{green!50!black},
		basicstyle=\ttfamily\footnotesize,
		breakatwhitespace=false,         
		breaklines=true,                 
		captionpos=b,                    
		keepspaces=true,                 
		numbers=left,                    
		numbersep=5pt,                  
		showspaces=false,                
		showstringspaces=false,
		showtabs=false,                  
		tabsize=2
}

\lstdefinestyle{shellstyle}{
		backgroundcolor=\color{black!5}, 
		basicstyle=\ttfamily\footnotesize\color{black}, 
		commentstyle=\color{black}, 
		keywordstyle=\color{black},
		numberstyle=\color{black!5},
		stringstyle=\color{black}, 
		showspaces=false,
		showstringspaces=false, 
		showtabs=false, 
		tabsize=2, 
		numbers=none, 
		breaklines=true
}

\lstdefinelanguage{javascript}{
	keywords={typeof, new, true, false, catch, function, return, null, catch, switch, var, if, in, while, do, else, case, break},
	keywordstyle=\color{blue}\bfseries,
	ndkeywords={class, export, boolean, throw, implements, import, this},
	ndkeywordstyle=\color{darkgray}\bfseries,
	identifierstyle=\color{black},
	sensitive=false,
	comment=[l]{//},
	morecomment=[s]{/*}{*/},
	commentstyle=\color{purple}\ttfamily,
	stringstyle=\color{red}\ttfamily,
	morestring=[b]',
	morestring=[b]"
}

% disponi sezioni
\usepackage{titlesec}

\titleformat{\section}
	{\sffamily\Large\bfseries} 
	{\thesection}{1em}{} 
\titleformat{\subsection}
	{\sffamily\large\bfseries}   
	{\thesubsection}{1em}{} 
\titleformat{\subsubsection}
	{\sffamily\normalsize\bfseries} 
	{\thesubsubsection}{1em}{}

% disponi alberi
\usepackage{forest}

\forestset{
	rectstyle/.style={
		for tree={rectangle,draw,font=\large\sffamily}
	},
	roundstyle/.style={
		for tree={circle,draw,font=\large}
	}
}

% disponi algoritmi
\usepackage{algorithm}
\usepackage{algorithmic}
\makeatletter
\renewcommand{\ALG@name}{Algoritmo}
\makeatother

% disponi numeri di pagina
\usepackage{fancyhdr}
\fancyhf{} 
\fancyfoot[L]{\sffamily{\thepage}}

\makeatletter
\fancyhead[L]{\raisebox{1ex}[0pt][0pt]{\sffamily{\@title \ \@date}}} 
\fancyhead[R]{\raisebox{1ex}[0pt][0pt]{\sffamily{\@author}}}
\makeatother

\begin{document}

% sezione (data)
\section{Lezione del 02-10-24}

% stili pagina
\thispagestyle{empty}
\pagestyle{fancy}

% testo
\subsection{Trasporto}
Poniamo il seguente problema:

\begin{problem}{Trasporto}
	Due centrali del latte di Firenze producono	rispettivamente 50 e 60 mila litri di latte al giorno.
	Le centrali servono tre quartieri, che consumano rispettivamente 30, 30 e 20 mila litri di latte al giorno.
	Si conosce il costo necessario per portare un migliaio di litri di latte da ogni centrale a ogni quartiere, riportato nella seguente tabella:
	
	\center{} \rowcolors{2}{green!10}{green!40!black!20}
	\begin{tabular} { | c || c | c | c | }
		\hline
		& \bfseries Novoli & \bfseries Statuto & \bfseries Rifredi \\ 
		\hline
		\bfseries Centrale A & 6 & 8 & 4 \\
		\bfseries Centrale B & 7 & 3 & 9 \\
		\hline
	\end{tabular}

	\par\bigskip

	Vogliamo capire quanto latte deve spedire ogni centrale ad ogni quartiere.

	\raggedright
	\par\smallskip

	\tiny{Nota simpatica: secondo l'indagine INRAN-SCAI 2005-06, l'italiano medio consuma $0.115 \mathrm{g}$ di latte al giorno, che per un peso specifico di circa $1.040 \mathrm{kg}/\mathrm{L}$ fanno $0.11 \mathrm{L}$ di latte al giorno. Al 2024, il comune di Firenze ha $364\ 073$ abitanti, ergo dovrebbe avere bisogno di approssimativamente $40\ 258 \mathrm{L}$ di latte al giorno. I fiorentini nell'esempio devono avere le ossa veramente forti!}

\end{problem}

Possiamo esprimere il problema dell'esempio come un problema LP.
Abbiamo innanzitutto che i costi di trasporto formano una matrice:
$$
C_{matr} =
\begin{pmatrix}
6 & 8 & 4 \\
7 & 3 & 9 \\
\end{pmatrix}
$$
che possiamo linearizzare, come avevamo fatto nei problemi di assegnamento di costo minimo, in un vettore costo:
$$
C = (6, 8, 4, 7, 3, 9)
$$

Questo vettore moltiplica il vettore delle variabili decisionali, che è la linearizzazione della matrice:
$$
x_{matr} =
\begin{pmatrix}
	x_{13} & x_{14} & x_{15} \\ 
	x_{23} & x_{24} & x_{25}
\end{pmatrix}
$$

Questa matrice non rappresenta altro che quanto latte mandare ad ogni quartiere.

A questo punto, possiamo stabilire i vincoli.
Innanzitutto, non si può avere più latte di quanto viene prodotto, ergo:
\[
	\begin{cases}
		x_{13} + x_{14} + x_{15} \leq 50 \\ 
		x_{23} + x_{14} + x_{15} \leq 60
	\end{cases}
\]
inoltre, si vuole fornire ad ogni quartiere il fabbisogno richiesto, ergo:
\[
	\begin{cases}
		x_{13} + x_{23} \geq 30 \\	
		x_{14} + x_{24} \geq 30	\\
		x_{15} + x_{25} \geq 20	
	\end{cases}
\]

Questo è un problema di programmazione lineare.

In generale, quindi, un problema di trasporto minimizza la funzione obiettivo data da una matrice di costo in $n \times m$ variabili, con $m$ vincoli di riga sul vettore $o_j$ dei limiti di produzione, e $n$ vincoli di colonna sul vettore $d_j$ della domanda, in forma:
\[
	\begin{cases}
		\min{\sum^m_{i=1} \sum^n_{j=1} c_{ij}x_{ij}} = C^\intercal \cdot x \\
		\sum^m_{i=1} x_{ij} \geq d_{j} \quad \forall j = 1,...,n \\ 
		\sum^n_{j=1} x_{ij} \leq o_{i} \quad \forall i = 1,...,m \\ 
		x \geq 0
	\end{cases}
\]

Non ci sono soluzioni se la domanda supera l'offerta, cioè sé:
$$
\sum_{j=1}^m d_j \geq \sum_{i=1}^m o_i
$$

Mentre in caso di eccessi di produzione, potremmo trasformare le diseguaglianze in uguaglianze, e aggiugnere un carico "fittizio" con costo zero dove deviare il surplus di produzione.

Inoltre, come avevamo detto per i problemi di assegnamento di costo minimo, anche qui potremmo scegliere di distinguere fra trasporti divisibili (nello spazio $x = \mathbb{R}^n$) e indivisibili (col vincolo aggiunto $x = \mathbb{Z}^n$).

\subsection{Forma duale standard}
Avevamo definito la forma primale standard:
\[
	\begin{cases}
		\max{C^\intercal \cdot x} \\
		Ax \leq b
	\end{cases}
\]

Introduciamo adesso la forma duale standard:
\begin{definition}{Forma duale standard}
	Un problema di programmazione lineare si dice in forma duale standard quando è espresso in forma:
	
	\[
		\begin{cases}
			\min(c^\intercal \cdot x) \\
			Ax = b \\
			x \geq 0
		\end{cases}
	\]

\end{definition}

\subsubsection{Vertici del duale}
Sulle forme duali è semplice il calcolo dei vertici. 
Possiamo infatti avere, come avevamo fatto sulla primale:
\begin{definition}{Soluzione di base duale}
	Sia dato un problema LP $\mathcal{P}$ in forma duale standard.
	Sia $B \subseteq \{ 1, ..., n \}$ un sottoinsieme di indici di variabili decisionali tale che $\mathrm{card}(B) = m$.
	Chiamiamo $x_B$ l'insieme delle variabili decisionali individuate da $B$, e $x_N$ l'insieme delle $n - m$ variabili decisionali rimanenti:
	$$ x = \{x_B, x_N\}$$
	Impostiamo quindi tutte le $x_N$ a 0: avremo un sistema di $m$ variabili in $m$ equazioni, quindi determinato.
	La soluzione di quel sistema è detta soluzione di base duale di $\mathcal{P}$.
\end{definition}

Indichiamo spesso questo vertice come $(bA_B^{-1}, 0)$.
Questa definizione porta ad una caratterizzazione dei vertici del tutto analoga a quella dichiarata sui problemi in forma primale standard:

\begin{theorem}{Caratterizzazione dei vertici duali}
	Su un problema in forma duale standard, un punto $x$ del poliedro $P$ è un vertice di $P$ se e solo se è una soluzione di base duale ammissibile, ovvero:
	$$ 
	x \in \mathrm{vert}(P) \Leftrightarrow \text{$x$ è soluzione di base duale}
	$$
\end{theorem}

\subsubsection{Soluzioni di base duali degeneri}
Possiamo ricavare il concetto di soluzione degenere (e anche di soluzione ammissibile) sui vertici del poliedro del duale. Si ha:

\begin{theorem}{Caratterizzazione delle soluzioni di base duali degeneri}
	Se una soluzione è di base, ergo scelto $B = \{ 1, ..., n \}$ con $\mathrm{card}(B) = m$ è data da $(bA_B^{-1}, 0)$, possiamo dire che è pure degenere quando $\exists i \in B$ tale che almeno una componente si annulla. 
\end{theorem}
e riguardo l'ammissibilità:
\begin{theorem}{Caratterizzazione delle soluzioni di base duali ammissibili}
	Se una soluzione è di base, ergo scelto $B = \{ 1, ..., n \}$ con $\mathrm{card}(B) = m$ è data da $(bA_B^{-1}, 0)$, possiamo dire che è ammissibile quando il vettore soluzione è $\geq 0$. 
\end{theorem}

\end{document}


\documentclass[a4paper,11pt]{article}
\usepackage[a4paper, margin=8em]{geometry}

% usa i pacchetti per la scrittura in italiano
\usepackage[french,italian]{babel}
\usepackage[T1]{fontenc}
\usepackage[utf8]{inputenc}
\frenchspacing 

% usa i pacchetti per la formattazione matematica
\usepackage{amsmath, amssymb, amsthm, amsfonts}

% usa altri pacchetti
\usepackage{gensymb}
\usepackage{hyperref}
\usepackage{standalone}

% imposta il titolo
\title{Appunti Ricerca Operativa}
\author{Luca Seggiani}
\date{2024}

% disegni
\usepackage{pgfplots}
\pgfplotsset{width=10cm,compat=1.9}

% imposta lo stile
% usa helvetica
\usepackage[scaled]{helvet}
% usa palatino
\usepackage{palatino}
% usa un font monospazio guardabile
\usepackage{lmodern}

\renewcommand{\rmdefault}{ppl}
\renewcommand{\sfdefault}{phv}
\renewcommand{\ttdefault}{lmtt}

% disponi il titolo
\makeatletter
\renewcommand{\maketitle} {
	\begin{center} 
		\begin{minipage}[t]{.8\textwidth}
			\textsf{\huge\bfseries \@title} 
		\end{minipage}%
		\begin{minipage}[t]{.2\textwidth}
			\raggedleft \vspace{-1.65em}
			\textsf{\small \@author} \vfill
			\textsf{\small \@date}
		\end{minipage}
		\par
	\end{center}

	\thispagestyle{empty}
	\pagestyle{fancy}
}
\makeatother

% disponi teoremi
\usepackage{tcolorbox}
\newtcolorbox[auto counter, number within=section]{theorem}[2][]{%
	colback=blue!10, 
	colframe=blue!40!black, 
	sharp corners=northwest,
	fonttitle=\sffamily\bfseries, 
	title=Teorema~\thetcbcounter: #2, 
	#1
}

% disponi definizioni
\newtcolorbox[auto counter, number within=section]{definition}[2][]{%
	colback=red!10,
	colframe=red!40!black,
	sharp corners=northwest,
	fonttitle=\sffamily\bfseries,
	title=Definizione~\thetcbcounter: #2,
	#1
}

% disponi problemi
\newtcolorbox[auto counter, number within=section]{problem}[2][]{%
	colback=green!10,
	colframe=green!40!black,
	sharp corners=northwest,
	fonttitle=\sffamily\bfseries,
	title=Problema~\thetcbcounter: #2,
	#1
}

% disponi codice
\usepackage{listings}
\usepackage[table]{xcolor}

\lstdefinestyle{codestyle}{
		backgroundcolor=\color{black!5}, 
		commentstyle=\color{codegreen},
		keywordstyle=\bfseries\color{magenta},
		numberstyle=\sffamily\tiny\color{black!60},
		stringstyle=\color{green!50!black},
		basicstyle=\ttfamily\footnotesize,
		breakatwhitespace=false,         
		breaklines=true,                 
		captionpos=b,                    
		keepspaces=true,                 
		numbers=left,                    
		numbersep=5pt,                  
		showspaces=false,                
		showstringspaces=false,
		showtabs=false,                  
		tabsize=2
}

\lstdefinestyle{shellstyle}{
		backgroundcolor=\color{black!5}, 
		basicstyle=\ttfamily\footnotesize\color{black}, 
		commentstyle=\color{black}, 
		keywordstyle=\color{black},
		numberstyle=\color{black!5},
		stringstyle=\color{black}, 
		showspaces=false,
		showstringspaces=false, 
		showtabs=false, 
		tabsize=2, 
		numbers=none, 
		breaklines=true
}

\lstdefinelanguage{javascript}{
	keywords={typeof, new, true, false, catch, function, return, null, catch, switch, var, if, in, while, do, else, case, break},
	keywordstyle=\color{blue}\bfseries,
	ndkeywords={class, export, boolean, throw, implements, import, this},
	ndkeywordstyle=\color{darkgray}\bfseries,
	identifierstyle=\color{black},
	sensitive=false,
	comment=[l]{//},
	morecomment=[s]{/*}{*/},
	commentstyle=\color{purple}\ttfamily,
	stringstyle=\color{red}\ttfamily,
	morestring=[b]',
	morestring=[b]"
}

% disponi sezioni
\usepackage{titlesec}

\titleformat{\section}
	{\sffamily\Large\bfseries} 
	{\thesection}{1em}{} 
\titleformat{\subsection}
	{\sffamily\large\bfseries}   
	{\thesubsection}{1em}{} 
\titleformat{\subsubsection}
	{\sffamily\normalsize\bfseries} 
	{\thesubsubsection}{1em}{}

% disponi alberi
\usepackage{forest}

\forestset{
	rectstyle/.style={
		for tree={rectangle,draw,font=\large\sffamily}
	},
	roundstyle/.style={
		for tree={circle,draw,font=\large}
	}
}

% disponi algoritmi
\usepackage{algorithm}
\usepackage{algorithmic}
\makeatletter
\renewcommand{\ALG@name}{Algoritmo}
\makeatother

% disponi numeri di pagina
\usepackage{fancyhdr}
\fancyhf{} 
\fancyfoot[L]{\sffamily{\thepage}}

\makeatletter
\fancyhead[L]{\raisebox{1ex}[0pt][0pt]{\sffamily{\@title \ \@date}}} 
\fancyhead[R]{\raisebox{1ex}[0pt][0pt]{\sffamily{\@author}}}
\makeatother

\begin{document}

% sezione (data)
\section{Lezione del 03-10-24}

% stili pagina
\thispagestyle{empty}
\pagestyle{fancy}

% testo
\subsection{Teoria della dualità}
Introduciamo adesso uno dei concetti più importanti della programmazione lineare.
Avevamo posto problemi LP in forma primale standard come:

\[
	\begin{cases}
		\min(c^\intercal \cdot x) \\
		Ax \leq b
	\end{cases}
\]

Ottimizzare questo problema significa partire dal basso e avvicinarsi verso un punto di massimo.
Potremmo scegliere di seguire il percorso opposto: cercare di estrapolare un limite superiore per la soluzione dai vincoli, e minimizzarlo.

Per fare ciò introduciamo $m$ variabili, una per ogni disequazione, che denoteremo come $y_1, ..., y_m$.
Moltiplichiamo ogni disequazione per la $y_i$ corrispondente a destra e a sinistra.
Su un semplice problema $n, m = 2$, questo darà una forma del tipo:
\[
	\begin{cases}
		a_{11} x_1 + a_{12} x_2 \leq b_1 \\
		a_{21} x_1 + a_{22} x_2 \leq b_2
	\end{cases}
	\rightarrow
	\begin{cases}
	y_1 \cdot \left(	a_{11} x_1 + a_{12} x_2 \right) \leq b_1 y_1 \\
	y_2 \cdot \left( a_{21} x_1 + a_{22} x_2 \right) \leq b_2 y_2
	\end{cases}
\]

Se vincoliamo gli $y_i$ in modo che ogni variabile decisionale $x_i$ del sistema abbia un coefficiente del costo $\geq c_i$ corrispondente, otterremo una disequazione che ha a sinistra una situazione di valore uguale o addirittura migliore di quella data dalla funzione costo, e a destra un massimo (che era ciò che stavamo cercando, un limite superiore).
Abbiamo quindi una serie di variabili vincolate:
\[
	\begin{cases}			
		y_1 a_11 + y_2 a_21 \geq c_1 \\ 
		y_1 a_12 + y_2 a_22 \geq c_2 
	\end{cases}
\]
e una funzione da minimizzare:
$$
\min(b_1 y_1 + b_2 y)
$$

Cioè, ci siamo ricondotti ad un altro problema LP.
Possiamo formalizzare questo risultato:
\begin{definition}{Duale di un problema LP}
	Per un qualsiasi problema LP $\mathcal{P}$, detto primale, con $m \geq n$, possiamo definire il duale $\mathcal{D}$:
	\[
		P:
		\begin{cases}
			\max(c^\intercal \cdot x) \\ 
			Ax \leq b
		\end{cases}
	\rightarrow
		D:
		\begin{cases}
			\min(b^\intercal \cdot y) \\ 
			A^\intercal y = c \\
			y \geq 0
		\end{cases}
	\]
	dove si nota $x \in \mathbb{R}^n$ e $y \in \mathbb{R}^m$.
\end{definition}

Il duale viene posto in forma duale standard in quanto ciò che ci interessa è \textit{stringere} il limite superiore fino al suo minimo, in un modo che fa combaciare perfettamente le variabili con il loro vettore costo, da cui le uguaglianze.

Si può dimostrare che l'operazione del calcolo del duale è involutoria: il duale del duale è nuovamente il primale, e così via.

\subsubsection{Dualità debole}
Visto che abbiamo costruito il duale per avere un limite superiore dei valori ottenuti dalla funzione obiettivo del primale, potremo dimostrare facilmente:
\begin{theorem}{Dualità debole}
	Se i poliedri $P$ e il suo duale $D$ non sono vuoti, allora:
	$$ c^\intercal x \leq y^\intercal b \quad \forall x \in P, \forall y \in D$$
\end{theorem}
Cioè il duale è sempre maggiore del primale.

\subsubsection{Dualità forte}
Idealmente, ciò che vorremmo è che primale e duale convergessero verso un punto comune, ergo l'ottimo di entrambi.
Effettivamente, questo risultato è verificato:
\begin{theorem}{Dualità forte}
	Se i poliedri $P$ e il suo duale $D$ non sono vuoti, allora:
	$$ -\infty \leq \min_{y \in D} b^\intercal y = \max_{x \in P} c^\intercal x \leq +\infty $$
\end{theorem}

Il teorema della dualità forte afferma che, se entrambi i poliedri (primale e duale) sono non vuoti, allora condividono l'ottimo, e anzi, che due soluzioni nel primale e nel duale sono ottime solo se hanno lo stesso valore.
Se invece solo il primale (solo il duale) è vuoto, si ha che entrambi condividono ottimo $-\infty$ ($\infty$).
Quando entrambi sono vuoti non si ha soluzione condivisa.

\par\medskip

Si noti che la dualità è univocamente determinata solo nel caso di soluzioni non degeneri.
Nel caso di soluzioni degeneri si può avere che a una soluzione del primale ne corrispondono multiple del duale, e viceversa.
In questo caso, è possibile che si raggiunga un massimo (o un minimo) nel primale (nel duale), a partire da più vertici complementari, o addirittura senza che la soluzione di base complementare sia ammissibile.
Più formalmente, se l'esistenza di una soluzione complementare ammissibile è \textbf{condizione necessaria e sufficiente} per problemi con soluzioni non degeneri, è solo \textbf{condizione sufficiente} per problemi con soluzioni degeneri.

Ad esempio, si noti il problema duale:
\[
	\begin{cases}
		\min 3y_1 - 7y_2 + 5y_3 + 22y_4 + 14y_5 + 15y_6 \\ 
-y_1 - y_2 +3 y_4 + 2y_5 + 2y_6 = 2 \\
y_1 - 4y_2 + y_3 + 2y_4 + y_5 - 2 y_6 = 1 \\e
y_i \geq 0	
	\end{cases}
\]
con il primale associato:
\[
	\begin{cases}
			
\max 2x_1 + x_2 \\ 
-x_1 + x_1 \leq 3 \\ 
-x_1 -4x_2 \leq -7 \\ 
x_2 \leq 5 \\ 
3 x_1 + 2 x_2 \leq 22 \\ 
2 x_1 + x_2 \leq 14 \\ 
2 x_1 - 2 x_2 \leq 5
	\end{cases}
\]

Con la base $B = \{ 1, 5 \}$, si hanno le soluzioni $\bar{x} = (4, 5)$ e $\bar{y} = (0, 0, 0,  0, 1, 0)$.
Si dimostra che $\bar{y}$ è ottima a $D$, mentre $\bar{x}$ è non ammissibile su $P$.
Ciò deriva dal fatto che $\bar{y}$ in $D$ è degenere: descrive un vincolo perpendicolare al vettore costo (di indice $i = 5$), e quindi soddisfatto da infinite soluzioni del primale.
Il punto $\bar{x}$ nel primale ha come complementare l'ottimo del duale perchè sta proprio su questo vincolo. 

\subsubsection{Scarti complementari}
Si può dimostrare il seguente teorema:
\begin{theorem}{Scarti complementari}
	Se le soluzioni $x$ e $y$ dei problemi primale e duale $\mathcal{P}$ e $\mathcal{D}$ sono entrambe ottime, allora vale:
	$$ y^\intercal (b - Ax) = 0 $$
\end{theorem}
Questo si ricava dal fatto che, per la dualità forte, si ha che:
$$
c^\intercal x = y^\intercal Ax = y^\intercal b \Rightarrow y^\intercal(b - Ax) = 0
$$

Il significato del teorema è che, se una disequazione nel primale è \textit{stretta}, allora la corrispondente variabile nel duale è $\neq 0$, e viceversa.

\subsubsection{Soluzioni di base}
Avevamo dato una definizione di soluzione di base per problemi LP in forma sia primale che duale.
Possiamo dimostrare che non solo questa nozione esiste su entrambe le formule, ma è analoga su coppie primale / duale.

Avevamo posto che la formazione di una certa base $B \in \{ 1, ..., m \}$ per ricavare soluzioni di base.
Per il primale, questo significa partizionare la matrice e i termini noti:
$$
A = \left( A_B \over A_N \right), \quad b = \left( b_A \over b_N \right)
$$
mentre per il duale, significherà partizionare le variabili introdotte:
$$
y = \left( y_B \over y_N \right)
$$
noto il numero di $y_1, ..., y_m$ uguale a $m$.

Questo significa che possiamo trovare due soluzioni di base corrispondenti per un'unica base su primale e duale.
Queste sono:
\begin{itemize}
	\item Soluzione di base primale: $ x = A_B^{-1} b_B $;
	\item Soluzione di base duale: $ y_B^\intercal = c^\intercal A_B^-1, \quad y_N = 0 $;
\end{itemize}
Si dice che le soluzioni di base sono \textbf{complementari}.

\par\smallskip
\noindent 
\textbf{\textsf{Dimostrazione}}
Vogliamo che $y^\intercal(b - Ax)$ sia $=0$ soddisfatte le condizioni di base.
Applichiamo quindi la base:
$$
y^\intercal(b - Ax) = \left( y_B^\intercal, y_N^\intercal \right) \binom{b_B - A_B x}{b_N - A_N x} = \left( c^\intercal A_B^{-1}, 0 \right) \binom{b_B - A_B A_B^{-1} b_B}{b_N - A_N A_B^{-1} b_B}
$$
$$
= \left( c^\intercal A_B^{-1}, 0 \right) \binom{0}{b_N - A_N A_B^{-1} b_B} = 0
$$

Questo nome non è a caso, in quanto si può dimostrare le due soluzioni sono in scarti complementari.
Da questo risultato, si ha che se entrambe le soluzioni sono ammissibili, cioé:
\begin{itemize}
	\item La primale è ammissibile: 
		$$ \forall i \in N \ \text{si ha} \ A_i x \leq b_i $$
		ergo i vincoli sono soddisfatti;
	\item La duale è ammissibile:
		$$ y \geq 0$$
\end{itemize}
questo è condizione sufficiente perche la soluzione sia ottima, e dagli scorsi corollari, sia l'ottima sia del primale che del duale.

Formalizziamo quanto detto in un teorema:
\begin{theorem}{Condizioni di ottimalità di soluzione di base}
	Dato un vertice del primale, ottenuto da una certa base, si può costruire il complemento duale sulla stessa base.
	Se entrambi i vertici ottenuti sono ammissibili, allora sono uguali e ottimi dei rispettivi problemi.
\end{theorem}

\end{document}


\documentclass[a4paper,11pt]{article}
\usepackage[a4paper, margin=8em]{geometry}

% usa i pacchetti per la scrittura in italiano
\usepackage[french,italian]{babel}
\usepackage[T1]{fontenc}
\usepackage[utf8]{inputenc}
\frenchspacing 

% usa i pacchetti per la formattazione matematica
\usepackage{amsmath, amssymb, amsthm, amsfonts}

% usa altri pacchetti
\usepackage{gensymb}
\usepackage{hyperref}
\usepackage{standalone}

% cose fluttuanti
\usepackage{float}

% imposta il titolo
\title{Appunti Ricerca Operativa}
\author{Luca Seggiani}
\date{2024}

% disegni
\usepackage{pgfplots}
\pgfplotsset{width=10cm,compat=1.9}

% imposta lo stile
% usa helvetica
\usepackage[scaled]{helvet}
% usa palatino
\usepackage{palatino}
% usa un font monospazio guardabile
\usepackage{lmodern}

\renewcommand{\rmdefault}{ppl}
\renewcommand{\sfdefault}{phv}
\renewcommand{\ttdefault}{lmtt}

% disponi il titolo
\makeatletter
\renewcommand{\maketitle} {
	\begin{center} 
		\begin{minipage}[t]{.8\textwidth}
			\textsf{\huge\bfseries \@title} 
		\end{minipage}%
		\begin{minipage}[t]{.2\textwidth}
			\raggedleft \vspace{-1.65em}
			\textsf{\small \@author} \vfill
			\textsf{\small \@date}
		\end{minipage}
		\par
	\end{center}

	\thispagestyle{empty}
	\pagestyle{fancy}
}
\makeatother

% disponi teoremi
\usepackage{tcolorbox}
\newtcolorbox[auto counter, number within=section]{theorem}[2][]{%
	colback=blue!10, 
	colframe=blue!40!black, 
	sharp corners=northwest,
	fonttitle=\sffamily\bfseries, 
	title=Teorema~\thetcbcounter: #2, 
	#1
}

% disponi definizioni
\newtcolorbox[auto counter, number within=section]{definition}[2][]{%
	colback=red!10,
	colframe=red!40!black,
	sharp corners=northwest,
	fonttitle=\sffamily\bfseries,
	title=Definizione~\thetcbcounter: #2,
	#1
}

% disponi problemi
\newtcolorbox[auto counter, number within=section]{problem}[2][]{%
	colback=green!10,
	colframe=green!40!black,
	sharp corners=northwest,
	fonttitle=\sffamily\bfseries,
	title=Problema~\thetcbcounter: #2,
	#1
}

% disponi codice
\usepackage{listings}
\usepackage[table]{xcolor}

\lstdefinestyle{codestyle}{
		backgroundcolor=\color{black!5}, 
		commentstyle=\color{codegreen},
		keywordstyle=\bfseries\color{magenta},
		numberstyle=\sffamily\tiny\color{black!60},
		stringstyle=\color{green!50!black},
		basicstyle=\ttfamily\footnotesize,
		breakatwhitespace=false,         
		breaklines=true,                 
		captionpos=b,                    
		keepspaces=true,                 
		numbers=left,                    
		numbersep=5pt,                  
		showspaces=false,                
		showstringspaces=false,
		showtabs=false,                  
		tabsize=2
}

\lstdefinestyle{shellstyle}{
		backgroundcolor=\color{black!5}, 
		basicstyle=\ttfamily\footnotesize\color{black}, 
		commentstyle=\color{black}, 
		keywordstyle=\color{black},
		numberstyle=\color{black!5},
		stringstyle=\color{black}, 
		showspaces=false,
		showstringspaces=false, 
		showtabs=false, 
		tabsize=2, 
		numbers=none, 
		breaklines=true
}

\lstdefinelanguage{javascript}{
	keywords={typeof, new, true, false, catch, function, return, null, catch, switch, var, if, in, while, do, else, case, break},
	keywordstyle=\color{blue}\bfseries,
	ndkeywords={class, export, boolean, throw, implements, import, this},
	ndkeywordstyle=\color{darkgray}\bfseries,
	identifierstyle=\color{black},
	sensitive=false,
	comment=[l]{//},
	morecomment=[s]{/*}{*/},
	commentstyle=\color{purple}\ttfamily,
	stringstyle=\color{red}\ttfamily,
	morestring=[b]',
	morestring=[b]"
}

% disponi sezioni
\usepackage{titlesec}

\titleformat{\section}
	{\sffamily\Large\bfseries} 
	{\thesection}{1em}{} 
\titleformat{\subsection}
	{\sffamily\large\bfseries}   
	{\thesubsection}{1em}{} 
\titleformat{\subsubsection}
	{\sffamily\normalsize\bfseries} 
	{\thesubsubsection}{1em}{}

% disponi alberi
\usepackage{forest}

\forestset{
	rectstyle/.style={
		for tree={rectangle,draw,font=\large\sffamily}
	},
	roundstyle/.style={
		for tree={circle,draw,font=\large}
	}
}

% disponi algoritmi
\usepackage{algorithm}
\usepackage{algorithmic}
\makeatletter
\renewcommand{\ALG@name}{Algoritmo}
\makeatother

% disponi numeri di pagina
\usepackage{fancyhdr}
\fancyhf{} 
\fancyfoot[L]{\sffamily{\thepage}}

\makeatletter
\fancyhead[L]{\raisebox{1ex}[0pt][0pt]{\sffamily{\@title \ \@date}}} 
\fancyhead[R]{\raisebox{1ex}[0pt][0pt]{\sffamily{\@author}}}
\makeatother

\begin{document}

% sezione (data)
\section{Lezione del 07-10-24}

% stili pagina
\thispagestyle{empty}
\pagestyle{fancy}

% testo
\subsection{Algoritmo del simplesso primale}
Supponiamo di avere un problema LP in formato primale standard con $n = 8$ vincoli, espresso come:

\[
	\begin{cases}
		\max(c^\intercal \cdot x) \\
		Ax \leq b
	\end{cases}
\]

e con poliedro:

\begin{center}
	\begin{tikzpicture}
	\begin{axis}[
			axis lines = middle,
			xlabel = {$x_A$},
			ylabel = {$x_B$},
			xmin=0, xmax=7.9,
			ymin=0, ymax=3.9,
			samples=100,
			width=13cm, height=7cm,
			legend pos=north east
		]

	\addplot[blue, thick] coordinates {(0,1) (1,0)};  
	\addplot[blue, thick] coordinates {(1,0) (2,0)};  
	\addplot[blue, thick] coordinates {(0,1) (0,2)};  
	\addplot[blue, thick] coordinates {(0,2) (1,3)};  
	\addplot[blue, thick] coordinates {(1,3) (2,3)};  
	\addplot[blue, thick] coordinates {(1,3) (2,3)};  
	\addplot[blue, thick] coordinates {(2,3) (3,2)};  
	\addplot[blue, thick] coordinates {(3,2) (3,1)};  
	\addplot[blue, thick] coordinates {(3,1) (2,0)}; 

	\node at (axis cs:0.7,0.7) [anchor=center] {1};
	\node at (axis cs:1.5,0.3) [anchor=center] {2};
	\node at (axis cs:2.3,0.7) [anchor=center] {3};
	\node at (axis cs:2.7,1.5) [anchor=center] {4};
	\node at (axis cs:2.3,2.3) [anchor=center] {5};
	\node at (axis cs:1.5,2.7) [anchor=center] {6};
	\node at (axis cs:0.7,2.3) [anchor=center] {7};
	\node at (axis cs:0.3,1.5) [anchor=center] {8};
		
	\end{axis}
	\end{tikzpicture}
\end{center}

Scegliamo un vertice di partenza, per adesso ad arbitrio: diciamo $\bar{x} = (0, 1)$ (vedremo in seguito un'algoritmo particolare per ricavare un vertice, che ci permetterà anche di determinare se il poliedro è vuoto o meno).
Ci chiediamo se questo vertice $\bar{x}$ è ottimo.
Visto che è vertice, abbiamo che per una matrice $A_B$ e un vettore $b_B$ di base:
$$
\bar{x} = A_B^{-1} b_B
$$
e che possiamo costruire il complementare duale $\bar{y}$, impostando a zero le variabili fuori base e risolvendo il sistema:
$$
\bar{y} = (cA_B^{-1}, 0)
$$
e applicare il test di ottimalità, cioè vedere se:
$$
cA_B^{-1} \geq 0
$$
ergo $\bar{y} \in D$, quindi il complementare duale esiste e il vertice è ottimo.
Se questa condizione risulta verificata, possiamo fermarci, in quanto abbiamo trovato la soluzione ottimale.

In caso contrario, avremo $\exists k \in B$ tale che $\bar{y}_k < 0$.
Dovremo quindi spostarci verso un'altro vertice, magari \textit{adiacente}, che dal punto di vista delle basi, significa cambiare un solo indice di base, conservando gli altri.
Possiamo formalizzare questa affermazione definendo un \textbf{indice uscente} $h$ ed un \textbf{indice entrante} $k$.
Sostituire un indice di base significa effettuare il cambio di base:
$$
B := B \setminus \{h\} \cup \{k\}
$$

Resta la domanda di \textit{quale} spigolo scegliere: in uno spazio vettoriale $\mathbb{R}^n$, ho a disposizione $n$ spigoli che si staccano dallo stesso vertice.
Ovviamente, vorrei scegliere uno spigolo che accresce la funzione obiettivo, e si può dimostrare che ne esiste almeno uno: altrimenti sarei già all'ottimo.
Inoltre, avendo un metodo per scegliere sempre lo spigolo di crescita maggiore potrei dire 2 cose: l'algoritmo tende all'ottimo (il vertice da cui non si staccano spigoli che accrescono la funzione obiettivo), e termina in un numero finito di passi (prima o poi raggiungerà inevitabilmente un vertice che massimizza la funzione).

Prima però dobbiamo chiarire una questione: scegliere un nuovo spigolo significa trovare 2 indici base, uno da eliminare e uno da inserire.
Si può dire che il primo indice, quello uscente, indica anche la direzione di spostamento: allentando un vincolo ci spostiamo sulla semiretta del prossimo.
Allo stesso tempo, scegliere un indice da rimuovere non basta: dobbiamo scegliere quale introdurre, che geometricamente significa capire \textit{quanto} ci possiamo spostare lungo la semiretta prima di uscire dalla regione di ammissibilità.
Vediamo quindi questi due passaggi in ordine.

\begin{itemize}
	\item \textbf{\textsf{Indice uscente}} \\
Diciamo:
$$
W = \left( -A_B^{-1} \right)
$$
e prendiamo le colonne $W^i$ corrispondenti agli indici di base scelti.

Possiamo allora dire che l'equazione degli spigoli dati dalle disequazioni all'indice $i$ sono:
$$
\bar{x} + \lambda W^i
$$

Mettiamo questa equazione nella funzione costo:
$$
c\left( \bar{x} + \lambda W^i \right) = c \bar{x} + \lambda c W^{i}
$$

Qui abbiamo $c\bar{x}$, che è il valore nel vertice, e un'altro termine scalato da $\lambda$.
Ricordiamo poi che $cA_B^{-1} = \bar{y}_B$, e che $W = \left( -A_B^{-1} \right)$, ergo $c W^{i} = -\bar{y}_B$:
$$
c \bar{x} + \lambda c W^{i} = c \bar{x} - \lambda \bar{y}_B
$$
Vogliamo quindi "allentare" l'indice (e il corrispettivo vertice) che ci dà $\bar{y}_i < 0$, in quanto è quello che restituisce un $c W^{i} > 0$, e quindi un accrescimento della funzione. 
Definiamo allora questo indice:
\begin{definition}{Indice uscente primale}
Chiamiamo indice uscente $h$, da una certa soluzione della base $B$:
$$h := \min\{ i \in B \ \text{t.c.} \ \bar{y_i} < 0  \}$$
\end{definition}
Il $\min$ significa che in caso di più $i$ negativi, si adotta la regola anticiclo (di Bland) di scegliere il primo.
In caso di nessun $i$ negativo, la complementare duale esiste e siamo sull'ottimo.

	\item \textbf{\textsf{Indice entrante}} \\
Adesso cerchiamo per quali $\lambda$ lo spigolo $\bar{x} + \lambda W^h$ resta ammissibile, ergo soddisfa:
$$
A_i \left( \bar{x} + \lambda W^h \right) \leq b_i, \quad i \in N 
$$
Questo significa effettivamente vedere qual'è il primo vincolo che "stringiamo", o che incontriamo, spostandoci lungo la semiretta ottenuta allentando il vincolo dato dall'indice uscente.

Possiamo dire:
$$
A_i \left( \bar{x} + \lambda W^h \right) = A_i \bar{x} + \lambda A_i W^h \leq b_i
$$
da cui si ricava (e si risolve) la disequazione di primo grado:
$$
\lambda A_i W^h \leq b_i - A_i \bar{x} \Rightarrow \lambda \leq \frac{b_i - A_i \bar{x}}{A_i W^h}
$$
Notiamo che se fosse $A_i W^h \leq 0, \ \forall i \in N$, avremmo che l'indice rappresenta una direzione di regressione, in quanto $\lambda \rightarrow +\infty$.
Si ha quindi che il duale non ha soluzione, e il primale $\rightarrow +\infty$.
In caso contrario, noi vogliamo trovare il primo vincolo che si va a stringere, quindi dovremo calcolare tutti gli $r_i$:
$$
r_i = \frac{b_i - A_i \bar{x}}{A_i W^h}, \quad i \in N, \quad A_i W^h > 0
$$
e scegliere l'indice che dà $\vartheta = \min(r_i)$.
Definiamo allora anche questo indice:
\begin{definition}{Indice entrante primale}
	Chiamiamo indice entrante $k$, da una certa soluzione della base $B$ e un certo indice uscente $h$:
	$$
	k := \min\{ i \in N \ \text{t.c.} \ A_i W^h > 0, \quad \frac{b_i - A_i \bar{x}}{A_i W^h} = \vartheta \}	
	$$
\end{definition}
Anche qui, il $\min$ serve a selezionare il primo indice valido, ed è una regola anticiclo (di Bland).
Notiamo due possibili situazioni:
\begin{itemize}
	\item Si potrebbero avere più $r_i$ uguali: questi rappresentano soluzioni di base degenere \textit{in arrivo}, in quanto sono più modi di arrivare allo stesso vertice stringendo vincoli diversi;
	\item Si potrebbe avere un $r_i$ nullo: questo significa che il vertice è sullo stesso vertice da dove siamo partiti, ergo rappresenta una soluzione di base degenere \textit{in partenza}.
\end{itemize}
Come prima, le regole anticiclo di Bland assicurano anche che l'algoritmo non si blocchi a ciclare su queste soluzioni degeneri.
\end{itemize}

Abbiamo quindi tutti gli strumenti necessari alla formulazione dell'algoritmo del simplesso:
\begin{algorithm}[H]
\caption{del simplesso primale}
\begin{algorithmic}
	\STATE \textbf{Input:} un problema LP in forma primale standard
	\STATE \textbf{Output:} la soluzione ottima 
	\STATE Trova una base B che genera una soluzione di base primale ammissibile
	\STATE \textsf{ciclo:}
	\STATE Calcola la soluzione di base primale $\bar{x} = A_b^{-1} b_B$ e la soluzione di base duale $\bar{y} = (cA_b^{-1}, 0)$
	\IF{$\bar{y_B} \geq 0$}
		\STATE Fermati, $\bar{x}$ è ottima per $P$ e $\bar{y}$ è ottima per $D$
\ELSE
		\STATE Calcola l'indice uscente: 
		$$
		h := \min\{ i \in B \ \text{t.c.} \ \bar{y_i} < 0 \}
		$$
		poni $W := -A_B^{-1}$ e indica con $W^h$ la $h$-esima colonna di $W$
	\ENDIF
	\IF{$A_i W^h \leq 0 \quad \forall i \in N$}
		\STATE Fermati, $P \rightarrow +\infty$ e $D$ non ha soluzione ottima
	\ELSE
		\STATE Calcola:
		$$
		\vartheta = \min\{ \frac{b_i - A_i \bar{x}}{A_i W^h} \text{t.c.} \quad i \in N, \quad A_i W^h > 0 \}
		$$
		e trova l'indice entrante: 
		$$ 
		k := \min\{ i \in N \ \text{t.c.} \ A_i W^h > 0, \quad \frac{b_i - A_i \bar{x}}{A_i W^h} = \vartheta \} 
		$$
	\ENDIF
	\STATE Aggiorna la base come:
	$$
	B := B \setminus \{h\} \cup \{k\}
	$$
	\STATE Torna a \textsf{ciclo}
\end{algorithmic}
\end{algorithm}

\end{document}


\documentclass[a4paper,11pt]{article}
\usepackage[a4paper, margin=8em]{geometry}

% usa i pacchetti per la scrittura in italiano
\usepackage[french,italian]{babel}
\usepackage[T1]{fontenc}
\usepackage[utf8]{inputenc}
\frenchspacing 

% usa i pacchetti per la formattazione matematica
\usepackage{amsmath, amssymb, amsthm, amsfonts}

% usa altri pacchetti
\usepackage{gensymb}
\usepackage{hyperref}
\usepackage{standalone}

% imposta il titolo
\title{Appunti Ricerca Operativa}
\author{Luca Seggiani}
\date{2024}

% disegni
\usepackage{pgfplots}
\pgfplotsset{width=10cm,compat=1.9}

% imposta lo stile
% usa helvetica
\usepackage[scaled]{helvet}
% usa palatino
\usepackage{palatino}
% usa un font monospazio guardabile
\usepackage{lmodern}

\renewcommand{\rmdefault}{ppl}
\renewcommand{\sfdefault}{phv}
\renewcommand{\ttdefault}{lmtt}

% disponi il titolo
\makeatletter
\renewcommand{\maketitle} {
	\begin{center} 
		\begin{minipage}[t]{.8\textwidth}
			\textsf{\huge\bfseries \@title} 
		\end{minipage}%
		\begin{minipage}[t]{.2\textwidth}
			\raggedleft \vspace{-1.65em}
			\textsf{\small \@author} \vfill
			\textsf{\small \@date}
		\end{minipage}
		\par
	\end{center}

	\thispagestyle{empty}
	\pagestyle{fancy}
}
\makeatother

% disponi teoremi
\usepackage{tcolorbox}
\newtcolorbox[auto counter, number within=section]{theorem}[2][]{%
	colback=blue!10, 
	colframe=blue!40!black, 
	sharp corners=northwest,
	fonttitle=\sffamily\bfseries, 
	title=Teorema~\thetcbcounter: #2, 
	#1
}

% disponi definizioni
\newtcolorbox[auto counter, number within=section]{definition}[2][]{%
	colback=red!10,
	colframe=red!40!black,
	sharp corners=northwest,
	fonttitle=\sffamily\bfseries,
	title=Definizione~\thetcbcounter: #2,
	#1
}

% disponi problemi
\newtcolorbox[auto counter, number within=section]{problem}[2][]{%
	colback=green!10,
	colframe=green!40!black,
	sharp corners=northwest,
	fonttitle=\sffamily\bfseries,
	title=Problema~\thetcbcounter: #2,
	#1
}

% disponi codice
\usepackage{listings}
\usepackage[table]{xcolor}

\lstdefinestyle{codestyle}{
		backgroundcolor=\color{black!5}, 
		commentstyle=\color{codegreen},
		keywordstyle=\bfseries\color{magenta},
		numberstyle=\sffamily\tiny\color{black!60},
		stringstyle=\color{green!50!black},
		basicstyle=\ttfamily\footnotesize,
		breakatwhitespace=false,         
		breaklines=true,                 
		captionpos=b,                    
		keepspaces=true,                 
		numbers=left,                    
		numbersep=5pt,                  
		showspaces=false,                
		showstringspaces=false,
		showtabs=false,                  
		tabsize=2
}

\lstdefinestyle{shellstyle}{
		backgroundcolor=\color{black!5}, 
		basicstyle=\ttfamily\footnotesize\color{black}, 
		commentstyle=\color{black}, 
		keywordstyle=\color{black},
		numberstyle=\color{black!5},
		stringstyle=\color{black}, 
		showspaces=false,
		showstringspaces=false, 
		showtabs=false, 
		tabsize=2, 
		numbers=none, 
		breaklines=true
}

\lstdefinelanguage{javascript}{
	keywords={typeof, new, true, false, catch, function, return, null, catch, switch, var, if, in, while, do, else, case, break},
	keywordstyle=\color{blue}\bfseries,
	ndkeywords={class, export, boolean, throw, implements, import, this},
	ndkeywordstyle=\color{darkgray}\bfseries,
	identifierstyle=\color{black},
	sensitive=false,
	comment=[l]{//},
	morecomment=[s]{/*}{*/},
	commentstyle=\color{purple}\ttfamily,
	stringstyle=\color{red}\ttfamily,
	morestring=[b]',
	morestring=[b]"
}

% disponi sezioni
\usepackage{titlesec}

\titleformat{\section}
	{\sffamily\Large\bfseries} 
	{\thesection}{1em}{} 
\titleformat{\subsection}
	{\sffamily\large\bfseries}   
	{\thesubsection}{1em}{} 
\titleformat{\subsubsection}
	{\sffamily\normalsize\bfseries} 
	{\thesubsubsection}{1em}{}

% disponi alberi
\usepackage{forest}

\forestset{
	rectstyle/.style={
		for tree={rectangle,draw,font=\large\sffamily}
	},
	roundstyle/.style={
		for tree={circle,draw,font=\large}
	}
}

% disponi algoritmi
\usepackage{algorithm}
\usepackage{algorithmic}
\makeatletter
\renewcommand{\ALG@name}{Algoritmo}
\makeatother

% disponi numeri di pagina
\usepackage{fancyhdr}
\fancyhf{} 
\fancyfoot[L]{\sffamily{\thepage}}

\makeatletter
\fancyhead[L]{\raisebox{1ex}[0pt][0pt]{\sffamily{\@title \ \@date}}} 
\fancyhead[R]{\raisebox{1ex}[0pt][0pt]{\sffamily{\@author}}}
\makeatother

\begin{document}

% sezione (data)
\section{Lezione del 08-10-24}

% stili pagina
\thispagestyle{empty}
\pagestyle{fancy}

% testo
Abbiamo trovato un'algoritmo per ricavare una soluzione ottimale partendo da una soluzione di base primale ammissibile.
Nel fare ciò, si è dato per scontato che l'algoritmo di partenza ci avrebbe fornito una soluzione di base primale ammissibile, e che qualsiasi successivo passo del simplesso ci avrebbe restituito altre soluzioni di base primali ammissibili.
In verita, le soluzioni di base trovate possono avere le seguenti combinazioni di ammissibilità su $P$ primale e $D$ duale:
\begin{itemize}
	\item \textbf{Ammissibile su $P$ e $D$:} in questo caso siamo all'ottimo dalla dualità forte;
	\item \textbf{Ammissibile su $P$ ma non su $D$:} in questo caso siamo su un comune punto ammissibile non ottimo;
	\item \textbf{Non ammissibile su $P$ o $D$:} in questo caso si scarta la soluzione;
	\item \textbf{Ammissibile su $D$ ma non su $P$:} effettivamente, ancora non abbiamo un modo per gestire questa situazione.
\end{itemize}

Per questo motivo estendiamo l'algoritmo del simplesso al duale.

\subsection{Algoritmo del simplesso duale}
Intendiamo mantenere, ad ogni passo, la soluzione di base duale come ammissibile e controllare l'ammissibilità di quella primale.
Se la soluzione di base primale è ammissibile (l'inverso di come avevamo visto per il simplesso primale), allora siamo sull'ottimo.
Altrimenti, si cambia base, cercando di minimizzare la funzione obiettivo su una nuova soluzione di base duale.

Abbiamo quindi un vertice del poliedro duale, cioè una soluzione di base ammissibile del duale. 
Ci chiediamo se questo vertice $\bar{y}$ è ottimo.
Visto che è vertice del duale, abbiamo che per una matrice $A_B$ e un vettore costo $c$ del primale:
$$
\bar{y} = (cA_B^{-1}, 0), \quad \text{con} \ cA_B^{-1} \geq 0 
$$
dove ricordiamo questa notazione significa impostare le variabili non di base $\in N$ a $0$ e risolvere il sistema in $m$ variabili rimasto.

Adesso possiamo costruire il complementare primale $\bar{x}$, ponendo:
$$
\bar{x} = A_B^{-1}b_b 
$$
e applicare il test di ammissibilità, cioè vedere se:
$$
A_N(A_B^{-1} b_B) \leq b_N
$$
ergo $\bar{x} \in P$, quindi il complementare duale esiste e il vertice è ottimo.
Come prima, se questa condizione risulta verificata possiamo fermarci, in quanto abbiamo trovato la soluzione ottimale.

In caso contrario, avremo $\exists i \in N$ tale che $b_i - A_i (A_B^{-1} - b_B) < 0$.
Questo equivale a ciò che avevamo trovato per il primale per quanto riguardava gli indici uscenti, con una sola differenza: per risolvere il duale, si trova \textbf{prima} l'\textbf{indice entrante}, e \textbf{poi} l'\textbf{indice uscente}.
Vediamo quindi i passaggi in ordine:

\begin{itemize}
	\item \textbf{\textsf{Indice entrante}} \\
Abbiamo che, se il vertice non è ottimo, vale:
$$
\exists i \in N \ \text{t.c.} \ b_i - A_i (A_B^{-1} - b_B) < 0
$$
Vogliamo che l'indice entrante sia quell che raggiunge questo valore negativo, quindi:
\begin{definition}{Indice entrante duale}
Chiamiamo indice entrante $k$, da una certa soluzione della base $B$:
$$k := \min\{ i \in N \ \text{t.c.} \ b_i - A_i \bar{x} < 0 \}$$
\end{definition}
Come sempre, $\min$ significa che in caso di più $i$ negativi, si adotta la regola anticiclo (di Bland) di scegliere il primo.
In caso di nessun $i$ negativo, la complementare duale esiste e siamo sull'ottimo.

	\item \textbf{\textsf{Indice uscente}} \\
Cerchiamo quindi l'indice uscente.
Dovremo prima definire la matrice $W$ come $ W = -A_B^{-1}$, e  calcolare il prodotto (perlopiù analogo al primale, ma si noti il segno della diseguaglianza capovolto):
$$ 
A_k W^i < 0
$$

Questo ci fornisce una regola, come nel simplesso primale, per l'esistenza di una soluzione: nel caso non sia verificato, si ha che il duale $\rightarrow -\infty$ e che il primale è vuoto.
In caso contrario, si possono calcolare i rapporti, come:
$$
r_i = \frac{-\bar{y}_i}{A_kW^i}, \quad i \in B, A_k W^i < 0
$$
e scegliere l'indice che da $\vartheta = \min{r_i}$, cioè:
\begin{definition}{Indice uscente duale}
	Chiamiamo indice uscente $h$, da una certa soluzione della base $B$ e un certo indice entrante $k$:
	$$
	h := \min\{ i \in B \ \text{t.c.} \ A_k W^i < 0, \quad \frac{-\bar{y}_i}{A_kW^i} = \vartheta \}	
	$$
\end{definition}
Anche qui, il $\min$ serve a selezionare il primo indice valido, ed è una regola anticiclo.
\end{itemize}

Abbiamo quindi revisionato tutti gli strumenti necessari alla formulazione dell'algoritmo del simplesso, stavolta duale:
\begin{algorithm}[H]
\caption{del simplesso duale}
\begin{algorithmic}
	\STATE \textbf{Input:} un problema LP in forma primale standard
	\STATE \textbf{Output:} la soluzione ottima 
	\STATE Trova una base B che genera una soluzione di base duale ammissibile.
	\STATE \textsf{ciclo:}
	\STATE Calcola la soluzione di base duale $\bar{y} = (cA_B^{-1}, 0)$ e la soluzione di base primale $\bar{x} = A_B^{-1}b_B$
	\IF{$A_N(A_B^{-1} b_B) \leq b_N$}
		\STATE Fermati, $\bar{y}$ è ottima per $D$ e $\bar{x}$ è ottima per $P$
	\ELSE
		\STATE Calcola l'indice entrante: 
		$$
		k := \min\{ i \in N \ \text{t.c.} \ b_i - A_i \bar{x} < 0 \}
		$$
		poni $W := -A_B^{-1}$ e indica con $W^i$ la $i$-esima colonna di $W$
	\ENDIF
	\IF{$A_k W^i \geq 0 \quad \forall i \in B$}
		\STATE Fermati, $D \rightarrow -\infty$ e $P$ non ha soluzione ottima
	\ELSE
		\STATE Calcola:
		$$
		\vartheta = \min\{ \frac{-\bar{y}_i}{A_kW^i} \text{t.c.} \quad i \in B, \quad A_k W^i < 0 \}
		$$
		e trova l'indice uscente: 
		$$ 
		h := \min\{ i \in B \ \text{t.c.} \ A_k W^i < 0, \quad \frac{-\bar{y}_i}{A_kW^i} = \vartheta \}	
		$$
	\ENDIF
	\STATE Aggiorna la base come:
	$$
	B := B \setminus \{h\} \cup \{k\}
	$$
	\STATE Torna a \textsf{ciclo}
\end{algorithmic}
\end{algorithm}

\end{document}



\documentclass[a4paper,11pt]{article}
\usepackage[a4paper, margin=8em]{geometry}

% usa i pacchetti per la scrittura in italiano
\usepackage[french,italian]{babel}
\usepackage[T1]{fontenc}
\usepackage[utf8]{inputenc}
\frenchspacing 

% usa i pacchetti per la formattazione matematica
\usepackage{amsmath, amssymb, amsthm, amsfonts}

% usa altri pacchetti
\usepackage{gensymb}
\usepackage{hyperref}
\usepackage{standalone}

% imposta il titolo
\title{Appunti Ricerca Operativa}
\author{Luca Seggiani}
\date{2024}

% disegni
\usepackage{pgfplots}
\pgfplotsset{width=10cm,compat=1.9}

% imposta lo stile
% usa helvetica
\usepackage[scaled]{helvet}
% usa palatino
\usepackage{palatino}
% usa un font monospazio guardabile
\usepackage{lmodern}

\renewcommand{\rmdefault}{ppl}
\renewcommand{\sfdefault}{phv}
\renewcommand{\ttdefault}{lmtt}

% disponi il titolo
\makeatletter
\renewcommand{\maketitle} {
	\begin{center} 
		\begin{minipage}[t]{.8\textwidth}
			\textsf{\huge\bfseries \@title} 
		\end{minipage}%
		\begin{minipage}[t]{.2\textwidth}
			\raggedleft \vspace{-1.65em}
			\textsf{\small \@author} \vfill
			\textsf{\small \@date}
		\end{minipage}
		\par
	\end{center}

	\thispagestyle{empty}
	\pagestyle{fancy}
}
\makeatother

% disponi teoremi
\usepackage{tcolorbox}
\newtcolorbox[auto counter, number within=section]{theorem}[2][]{%
	colback=blue!10, 
	colframe=blue!40!black, 
	sharp corners=northwest,
	fonttitle=\sffamily\bfseries, 
	title=Teorema~\thetcbcounter: #2, 
	#1
}

% disponi definizioni
\newtcolorbox[auto counter, number within=section]{definition}[2][]{%
	colback=red!10,
	colframe=red!40!black,
	sharp corners=northwest,
	fonttitle=\sffamily\bfseries,
	title=Definizione~\thetcbcounter: #2,
	#1
}

% disponi problemi
\newtcolorbox[auto counter, number within=section]{problem}[2][]{%
	colback=green!10,
	colframe=green!40!black,
	sharp corners=northwest,
	fonttitle=\sffamily\bfseries,
	title=Problema~\thetcbcounter: #2,
	#1
}

% disponi codice
\usepackage{listings}
\usepackage[table]{xcolor}

\lstdefinestyle{codestyle}{
		backgroundcolor=\color{black!5}, 
		commentstyle=\color{codegreen},
		keywordstyle=\bfseries\color{magenta},
		numberstyle=\sffamily\tiny\color{black!60},
		stringstyle=\color{green!50!black},
		basicstyle=\ttfamily\footnotesize,
		breakatwhitespace=false,         
		breaklines=true,                 
		captionpos=b,                    
		keepspaces=true,                 
		numbers=left,                    
		numbersep=5pt,                  
		showspaces=false,                
		showstringspaces=false,
		showtabs=false,                  
		tabsize=2
}

\lstdefinestyle{shellstyle}{
		backgroundcolor=\color{black!5}, 
		basicstyle=\ttfamily\footnotesize\color{black}, 
		commentstyle=\color{black}, 
		keywordstyle=\color{black},
		numberstyle=\color{black!5},
		stringstyle=\color{black}, 
		showspaces=false,
		showstringspaces=false, 
		showtabs=false, 
		tabsize=2, 
		numbers=none, 
		breaklines=true
}

\lstdefinelanguage{javascript}{
	keywords={typeof, new, true, false, catch, function, return, null, catch, switch, var, if, in, while, do, else, case, break},
	keywordstyle=\color{blue}\bfseries,
	ndkeywords={class, export, boolean, throw, implements, import, this},
	ndkeywordstyle=\color{darkgray}\bfseries,
	identifierstyle=\color{black},
	sensitive=false,
	comment=[l]{//},
	morecomment=[s]{/*}{*/},
	commentstyle=\color{purple}\ttfamily,
	stringstyle=\color{red}\ttfamily,
	morestring=[b]',
	morestring=[b]"
}

% disponi sezioni
\usepackage{titlesec}

\titleformat{\section}
	{\sffamily\Large\bfseries} 
	{\thesection}{1em}{} 
\titleformat{\subsection}
	{\sffamily\large\bfseries}   
	{\thesubsection}{1em}{} 
\titleformat{\subsubsection}
	{\sffamily\normalsize\bfseries} 
	{\thesubsubsection}{1em}{}

% disponi alberi
\usepackage{forest}

\forestset{
	rectstyle/.style={
		for tree={rectangle,draw,font=\large\sffamily}
	},
	roundstyle/.style={
		for tree={circle,draw,font=\large}
	}
}

% disponi algoritmi
\usepackage{algorithm}
\usepackage{algorithmic}
\makeatletter
\renewcommand{\ALG@name}{Algoritmo}
\makeatother

% disponi numeri di pagina
\usepackage{fancyhdr}
\fancyhf{} 
\fancyfoot[L]{\sffamily{\thepage}}

\makeatletter
\fancyhead[L]{\raisebox{1ex}[0pt][0pt]{\sffamily{\@title \ \@date}}} 
\fancyhead[R]{\raisebox{1ex}[0pt][0pt]{\sffamily{\@author}}}
\makeatother

\begin{document}

% sezione (data)
\section{Lezione del 18-11-24}

% stili pagina
\thispagestyle{empty}
\pagestyle{fancy}

% testo
\subsection{Problema di flusso massimo}
Poniamo di avere un grafo su cui riportiamo solamente le capacità superiori $u_{ij}$ sui singoli archi: 

\begin{center}
	\begin{tikzpicture}
		\node[circle, draw=black] (1) at (0,0) {1};
		\node[circle, draw=black] (2) at (2,1) {2};
		\node[circle, draw=black] (3) at (2,-1) {3};
		\node[circle, draw=black] (4) at (4,1) {4};
		\node[circle, draw=black] (5) at (4,-1) {5};
		\node[circle, draw=black] (6) at (6, 0) {6};
		\draw[->, to path={-| (\tikztotarget)}] (1) -- (2);
		\draw[->, to path={-| (\tikztotarget)}] (1) -- (3);
		\draw[->, to path={-| (\tikztotarget)}] (2) -- (3);
		\draw[->, to path={-| (\tikztotarget)}] (2) -- (4);
		\draw[->, to path={-| (\tikztotarget)}] (3) -- (5);
		\draw[->, to path={-| (\tikztotarget)}] (5) -- (4);
		\draw[->, to path={-| (\tikztotarget)}] (3) -- (4);
		\draw[->, to path={-| (\tikztotarget)}] (4) -- (6);
		\draw[->, to path={-| (\tikztotarget)}] (5) -- (6);

		\node at (1, 1)  {$8$};
		\node at (1, -1)  {$9$};

		\node at (3, 1.5)  {$2$};
		\node at (3, -1.5)  {$7$};
		\node at (3, 0.5)  {$8$};

		\node at (1.5, 0)  {$3$};
		\node at (4.5, 0)  {$6$};

		\node at (5, 1)  {$10$};
		\node at (5, -1)  {$4$};
	\end{tikzpicture}
\end{center}

Prendiamo due nodi $s$ e $t$, e cerchiamo di \textbf{massimizzare} il flusso da $s$ a $t$.
L'idea fondamentale che il flusso che parte da $s$ dovrà essere uguale al flusso che arriva in $t$.
Potremo partire da un flusso ammissibile qualsiasi, cioè che si limita a rispettare le capacità:
$$
x = \left( 2, 4, 0, 2, 0, 4, 2, 0, 4 \right)
$$
Notando che le capacità sugli \textit{ultimi} archi influenzano quelle sugli archi precedenti (ad esempio, $(3,5)$ è limitato a 4 da $(5,6)$ con $u_{56} = 4$).
Poniamo allora, più intelligentemente, un problema di PL:
\[
	\begin{cases}
		\max v \\ 
		Ex = b \\ 
		0 \leq x \leq u
	\end{cases}
\]
dove $E$ è la matrice di incidenza della rete, e i bilanci $b_i$ stessi dipendono dalla variabile $v$: 
\[
	b_i = 
	\begin{cases}
		-v, \quad i = s \\ 
		0, \quad i \neq s \wedge i \neq t \\ 
		v, \quad i = t
	\end{cases}
\]

Potremmo avere dubbi sul fatto che la  variabile $v$ compare nei bilanci $b$.
Scriviamo per esteso le uguaglianze dei vincoli:
\[
	\begin{cases}
		-x_{12} - x_{13} = -v \\ 
		x_{12} - x_{23} - x_{24} = 0 \\ 
		x_{13} + x_{23} - x_{34} - x_{35} = 0 \\ 
		x_{24} + x_{34} + x_{54} - x_{46} = 0 \\ 
		x_{35} - x_{54} - x_{56} = 0 \\ 
		x_{46} + x_{56} = v
	\end{cases}
\]

Notiamo che questo problema è effettivamente un problema di flusso minimo (massimo) su $n+1$ variabili per $n$ nodi, dove la $n$-esima variabile è proprio il flusso $v$.
Inoltre, con capacità massime intere, anche $v$ sarà necessariamente intero (dato da somma di interi) e quindi la matrice \textbf{unimodulare}, con la conseguenza già vista che $\text{PL} = \text{ILP}$.

Inoltre, possiamo portare i termini $v$ di $b$ a sinistra delle rispettive equazioni, ottenendo effettivamente la matrice:
$$
\begin{pmatrix}
 & 1 \\ 
E & 0 \\ 
 & -1
\end{pmatrix}
\begin{pmatrix}
x \\ v
\end{pmatrix}
= 0
$$
cioè dove si è introdotto la un \textbf{arco fittizio}, quello che parte da $t$ e arriva in $s$.
Capovolgendo la funzione obiettivo (e riportando in vista i termini di costo nullo su ogni arco $x_{ij}$), otteniamo quindi:
\[
	\begin{cases}
		\min 0 \cdot x - v \\ 	
\begin{pmatrix}
 & 1 \\ 
E & 0 \\ 
 & -1
\end{pmatrix}
\begin{pmatrix}
x \\ v
\end{pmatrix}
= 0 \\ 
0 \leq x \leq u
	\end{cases}
\]
dove per l'ultimo arco fittizio la capacità massima $u$ è un $M$ molto grande (o $\infty$).

Questo è un problema di flusso di costo minimo capacitato che include l'arco fittizio come unico arco a costo diverso da 0 (per giunta negativo), cioè che è "costretto" a imporre flusso massimo da $t$ ad $s$, e visto che tutti i nodi sono a bilancio 0, a riportarlo in direzione opposta da $s$ a $t$ lungo i nodi della rete vera e propria.

A questo punto possiamo proporre la soluzione ammissibile:
$$
(x,v) = \left( 2, 9, 0, 2, 5, 4, 7, 0, 4, 11 \right)
$$
da cui si ricavano le partizioni, controllando quali archi si svuotano e quali saturano:
$$
T = \left\{ (1,2), (3,4), (3,5), (4,6), (6,1) \right\}, \quad L = \left\{ (2,3), (5,4) \right\}, \quad U = \left\{ (1,3), (2,4), (5,6) \right\}
$$
Possiamo quindi ricavare il potenziale dell'albero di copertura, notando che dividerà necessariamente i nodi in partizione di potenziale 0 e potenziale 1:
$$
\pi = \left( 0, 0, 1, 1 ,1 \right)
$$
e calcolare i costi ridotti, che troviamo negativi su $(2,3) \in L$, quindi arco entrante:
$$
c^\pi_{23} = 0 + 0 - 1 = -1
$$
che è quanto ci aspettavamo, in quanto chiaramente $v = 14$ (dagli archi entranti nel nodo 6).

Procediamo quindi con l'eliminazione del ciclo, distinguendo innanzitutto le pratizioni $\mathcal{C}^+$ e $\mathcal{C}^-$:
$$
\mathcal{C}^+ = \left\{ (1,2), (2,3), (3,4), (4,6), (6, 1) \right\}, \quad \mathcal{C}^- = \emptyset
$$
Su $\mathcal{C}^-$ mettiamo $\vartheta^- = \infty$ (essenzialmente vogliamo considerare solo $\vartheta^+$).
Calcoliamo allora $\vartheta^+$ dagli $u_{ij} - x_{ij}$, tenendo conto che la capacità $u$ di $(6,1)$ (arco fittizio) è $\infty$:
$$
\vartheta^+ = \max \left\{ 6, 3, 3, 3, \infty \right\}
$$
da cui prendiamo $\vartheta^+ = 3$ e l'arco uscente $(2,3)$ per l'ordinamento lessicografico.
Notiamo di essere nel caso particolare dove l'arco \textit{entra} ed \textit{esce} (in questo caso si sposta da L a U).

Aggiungendo quindi il $\vartheta$ agli archi in $\mathcal{C}^+$ otteniamo il flusso:
$$
(x, v) = \left( 5, 9, 3, 2, 8, 4, 10, 0, 4, 14 \right)
$$
da cui $v=14$, come ci aspettavamo.
Vedremo in seguito un algoritmo per il calcolo del flusso massimo su reti più efficiente dell'applicazione diretta del simplesso.

\end{document}


\documentclass[a4paper,11pt]{article}
\usepackage[a4paper, margin=8em]{geometry}

% usa i pacchetti per la scrittura in italiano
\usepackage[french,italian]{babel}
\usepackage[T1]{fontenc}
\usepackage[utf8]{inputenc}
\frenchspacing 

% usa i pacchetti per la formattazione matematica
\usepackage{amsmath, amssymb, amsthm, amsfonts}

% usa altri pacchetti
\usepackage{gensymb}
\usepackage{hyperref}
\usepackage{standalone}

% imposta il titolo
\title{Appunti Ricerca Operativa}
\author{Luca Seggiani}
\date{2024}

% disegni
\usepackage{pgfplots}
\pgfplotsset{width=10cm,compat=1.9}

% imposta lo stile
% usa helvetica
\usepackage[scaled]{helvet}
% usa palatino
\usepackage{palatino}
% usa un font monospazio guardabile
\usepackage{lmodern}

\renewcommand{\rmdefault}{ppl}
\renewcommand{\sfdefault}{phv}
\renewcommand{\ttdefault}{lmtt}

% disponi il titolo
\makeatletter
\renewcommand{\maketitle} {
	\begin{center} 
		\begin{minipage}[t]{.8\textwidth}
			\textsf{\huge\bfseries \@title} 
		\end{minipage}%
		\begin{minipage}[t]{.2\textwidth}
			\raggedleft \vspace{-1.65em}
			\textsf{\small \@author} \vfill
			\textsf{\small \@date}
		\end{minipage}
		\par
	\end{center}

	\thispagestyle{empty}
	\pagestyle{fancy}
}
\makeatother

% disponi teoremi
\usepackage{tcolorbox}
\newtcolorbox[auto counter, number within=section]{theorem}[2][]{%
	colback=blue!10, 
	colframe=blue!40!black, 
	sharp corners=northwest,
	fonttitle=\sffamily\bfseries, 
	title=Teorema~\thetcbcounter: #2, 
	#1
}

% disponi definizioni
\newtcolorbox[auto counter, number within=section]{definition}[2][]{%
	colback=red!10,
	colframe=red!40!black,
	sharp corners=northwest,
	fonttitle=\sffamily\bfseries,
	title=Definizione~\thetcbcounter: #2,
	#1
}

% disponi problemi
\newtcolorbox[auto counter, number within=section]{problem}[2][]{%
	colback=green!10,
	colframe=green!40!black,
	sharp corners=northwest,
	fonttitle=\sffamily\bfseries,
	title=Problema~\thetcbcounter: #2,
	#1
}

% disponi codice
\usepackage{listings}
\usepackage[table]{xcolor}

\lstdefinestyle{codestyle}{
		backgroundcolor=\color{black!5}, 
		commentstyle=\color{codegreen},
		keywordstyle=\bfseries\color{magenta},
		numberstyle=\sffamily\tiny\color{black!60},
		stringstyle=\color{green!50!black},
		basicstyle=\ttfamily\footnotesize,
		breakatwhitespace=false,         
		breaklines=true,                 
		captionpos=b,                    
		keepspaces=true,                 
		numbers=left,                    
		numbersep=5pt,                  
		showspaces=false,                
		showstringspaces=false,
		showtabs=false,                  
		tabsize=2
}

\lstdefinestyle{shellstyle}{
		backgroundcolor=\color{black!5}, 
		basicstyle=\ttfamily\footnotesize\color{black}, 
		commentstyle=\color{black}, 
		keywordstyle=\color{black},
		numberstyle=\color{black!5},
		stringstyle=\color{black}, 
		showspaces=false,
		showstringspaces=false, 
		showtabs=false, 
		tabsize=2, 
		numbers=none, 
		breaklines=true
}

\lstdefinelanguage{javascript}{
	keywords={typeof, new, true, false, catch, function, return, null, catch, switch, var, if, in, while, do, else, case, break},
	keywordstyle=\color{blue}\bfseries,
	ndkeywords={class, export, boolean, throw, implements, import, this},
	ndkeywordstyle=\color{darkgray}\bfseries,
	identifierstyle=\color{black},
	sensitive=false,
	comment=[l]{//},
	morecomment=[s]{/*}{*/},
	commentstyle=\color{purple}\ttfamily,
	stringstyle=\color{red}\ttfamily,
	morestring=[b]',
	morestring=[b]"
}

% disponi sezioni
\usepackage{titlesec}

\titleformat{\section}
	{\sffamily\Large\bfseries} 
	{\thesection}{1em}{} 
\titleformat{\subsection}
	{\sffamily\large\bfseries}   
	{\thesubsection}{1em}{} 
\titleformat{\subsubsection}
	{\sffamily\normalsize\bfseries} 
	{\thesubsubsection}{1em}{}

% disponi alberi
\usepackage{forest}

\forestset{
	rectstyle/.style={
		for tree={rectangle,draw,font=\large\sffamily}
	},
	roundstyle/.style={
		for tree={circle,draw,font=\large}
	}
}

% disponi algoritmi
\usepackage{algorithm}
\usepackage{algorithmic}
\makeatletter
\renewcommand{\ALG@name}{Algoritmo}
\makeatother

% disponi numeri di pagina
\usepackage{fancyhdr}
\fancyhf{} 
\fancyfoot[L]{\sffamily{\thepage}}

\makeatletter
\fancyhead[L]{\raisebox{1ex}[0pt][0pt]{\sffamily{\@title \ \@date}}} 
\fancyhead[R]{\raisebox{1ex}[0pt][0pt]{\sffamily{\@author}}}
\makeatother

\begin{document}

% sezione (data)
\section{Lezione del 10-10-24}

% stili pagina
\thispagestyle{empty}
\pagestyle{fancy}

% testo
\subsection{Introduzione alla programmazione lineare intera}
Iniziamo adesso a studiare un tipo di problemi che finora avevamo menzionato, ma mai risolto.
Si pone il seguente:

\begin{problem}{Caricamento}
	Un ladro è riuscito a scassinare un'importante caveau, e ha di fronte a sé una serie di gemme preziose.
	Tenendo conto del peso e del tipo delle gemme, ricava una tabella col valore e il peso di ogni gemma:
	
	\center \rowcolors{2}{green!10}{green!40!black!20}
	\begin{tabular} { | c || c | c | c | c | c| c |}
		\hline
		Valore & 2 & 4 & 7 & 9 & 13 & 16 \\
		Peso & 6 & 8 & 9 & 11 & 15 & 18 \\
		\hline
	\end{tabular}

	\par\bigskip

	\raggedright
	Amesso che il suo zaino non possa portare più di 30 kg, quali gemme dovrà scegliere per avere il maggiore profitto possibile?
\end{problem}

Questo problema riprende il celebre \textit{knapsack problem}, che risulta essere NP-completo.
Attraverso la \textbf{programmazione lineare intera}, possiamo ricavare soluzioni, almeno approssimate.

Si ha che qualsiasi possibilità delle $2^n$ configurazioni di gemme che il ladro può portare sono rappresentate da un vettore:
$$ 
x = (x_0, ..., x_n), \quad x_i \in \{ 0, 1 \}
$$
e che i il valore e il peso delle stesse sono vettori:
$$
v = \left( 2, 4, 7, 9, 13, 16 \right)
$$
$$
p = \left( 6, 8, 9, 11, 15, 18 \right)
$$

Possiamo quindi formulare il sistema:
\[
	\begin{cases}
		\max \left( 2x_1 + 4x_2 + 7x_3 + 9x_4 + 13x_5 + 16x_6 \right) \\ 
		6x_1 + 8x_2 + 9x_3 + 11x_4 + 15x_5 + 18x_6 \leq 30 \\
		x \in \{ 0, 1 \}^n
	\end{cases}
	\rightarrow
	\begin{cases}
		\max \left( v^\intercal x \right) \\
		p^\intercal x \leq P \\
		x \in \{ 0, 1 \}^n
	\end{cases}
\]
dove $P$ è il vincolo di massimo dei pesi.

Questo problema ricalca la struttura di un problema LP, ma il vincolo finale, che limita i valori possibili delle componenti del vettore soluzione, lo rende IPL.
Decidiamo di trasformare il problema in un LP corrispondente, che chiameremo \textbf{rilassamento continuo}, e risolvere quello.

\begin{definition}{Rilassato continuo}
	Dato un problema ILP in forma:
	\[
		\begin{cases}
			\max c^\intercal \cdot x \\ 
			Ax \leq b \\ 
			x \in \mathbb{Z}^n
		\end{cases}
	\]
	chiamiamo \textbf{rilassato continuo} il problema LP che rimuove il vincolo di interezza:
	\[
		\begin{cases}
			\max c^\intercal \cdot x \\ 
			Ax \leq b 
		\end{cases}	
	\]
\end{definition}

Possiamo ricavare questo rilassato continuo in due modi: il primo prevede di rimuovere semplicemente il vincolo, magari includendo la positività, o ancor meglio l'appartenenza $x \in [0, 1]$:
\[
	\begin{cases}
		\max \left( v^\intercal x \right) \\
		p^\intercal x \leq P \\
		0 \leq x \leq 1
	\end{cases}
\]

Questo metodo è sicuramente funzionante, ma preferiamo calcolare il duale, a partire dal primale con imposta la positività:
\[
	\begin{cases}
		\max \left( 2x_1 + 4x_2 + 7x_3 + 9x_4 + 13x_5 + 16x_6 \right) \\ 
		6x_1 + 8x_2 + 9x_3 + 11x_4 + 15x_5 + 18x_6 \leq 30 \\
		-x_1 \leq 0 \\ 
		-x_2 \leq 0 \\ 
		-x_3 \leq 0 \\ 
		-x_4 \leq 0 \\ 
		-x_5 \leq 0 \\ 
		-x_6 \leq 0
	\end{cases}
	\rightarrow 
	\begin{cases}
		\min \left( 30y_1 \right) \\ 
		6 y_1 - y_2 = 2 \\
		8 y_1 - y_3 = 4 \\ 
		9 y_1 - y_4 = 7 \\ 
		11 y_1 - y_5 = 9 \\ 
		15 y_1 - y_6 = 13 \\ 
		18 y_1 - y_7 = 16 \\ 
		y_i \geq 0
	\end{cases}
\]
da cui notiamo che le disequazioni in forma $a_i x + s_i = b_i$ si possono rendere come $a_i x \geq b_i$: 
\[
	\begin{cases}
		\min \left( 30y_1 \right) \\ 
		6 y_1 \geq 2 \\
		8 y_1 \geq 4 \\ 
		9 y_1 \geq 7 \\ 
		11 y_1 \geq 9 \\ 
		15 y_1 \geq 13 \\ 
		18 y_1 \geq 16 \\ 
		y_1 \geq 0
	\end{cases} \equiv
	\begin{cases}
		\min \left( 30y_1 \right) \\ 
		y_1 \geq \frac{1}{3} \\
		y_1 \geq \frac{1}{2} \\ 
		y_1 \geq \frac{7}{9} \\ 
		y_1 \geq \frac{9}{11} \\ 
		y_1 \geq \frac{13}{15} \\ 
		y_1 \geq \frac{8}{9} \\ 
		y_1 \geq 0
	\end{cases}
\]

Abbiamo che questo sistema è banale: il minimo di $30 y_1$ sottoposto ai vincoli si avrà quando $y_1$ rispetta i vincoli, che sono tutti $\geq$, e rispettare qualsiasi vincolo significa rispettare il vincolo con $b_i$ più grande, quindi in questo caso $\frac{8}{9}$
Si ha quindi che la soluzione ottima del duale è $\frac{8}{9}$, e possiamo trovare con la stessa facilità la soluzione del primale: abbiamo che la disequazione da cui si è ricavato la soluzione del duale ha indice $j = 6$.
Si prende quindi la variabile con lo stesso indice nel primale, e si \textbf{satura}, cioè si porta al livello più alto possibile prima di violare il limite.
Si nota che il valore ottimo trovato nel primale e nel duale (che è uguale dalla dualità forte) è $v = P \bar{y}_j = v_j \bar{x}_j $, con $\bar{x}$ e $\bar{y}$ le soluzioni ottime trovate rispettivamente al primale e al duale. 

Nel nostro caso, $18 x_6 \leq 30$, quindi $x_6 = \frac{30}{18}$, con valore $v = 16 \frac{30}{18} = \frac{80}{3}$.

Questo approccio sarebbe quello che si applicherebbe intuitivamente senza sapere nulla di PL. 
Infatti i rapporti calcolati nel duale:
$$
r_i = \frac{v_i}{p_i}
$$
sono effettivamente i \textbf{rendimenti} di ogni elemento di peso e valore, cioè quanto valore portano in rapporto al loro peso.
Chiaramente, vorremmo massimizzare gli elementi con rendimento maggiore, quindi saturiamo gli elementi con $r_i$ massimo.
Possiamo quindi definire il seguente algoritmo:

\begin{algorithm}[H]
\caption{caricamento per soluzioni ottime del rilassato continuo}
\begin{algorithmic}
	\STATE \textbf{Input:} il rilassato continuo di un problema di caricamento
	\STATE \textbf{Output:} la soluzione ottima $\bar{x}$
	\STATE Considera i $r_i = \frac{v_i}{p_i}$ rendimenti di ogni elemento
	\STATE Scegli l'indice $j$ del rendimento massimo $r_j$
	\STATE Satura la variabile $j$
\end{algorithmic}
\end{algorithm}

Abbiamo che la soluzione trovata è un limite superiore per il problema di partenza, quello ILP: infatti, rilassando i vincoli, otteniamo una regione ammissibile più grande, e quindi aumentiamo il massimo.
In verità anche l'arrotondamento in basso della soluzione $v$ è un limite superiore: nel caso del problema abbiamo $ \lfloor \frac{80}{3} \rfloor = 26$.

Un'idea banale per trovare il punto ottimo di questo problema ILP potrebbe essere quello di arrotondare il punto ottimo del rilassato continuo: nel nostro esempio, troviamo $\bar{x} = \left( 0, 0, 0, 0, 0, 1 \right)$, che dà massimo $v = 16$.

Notiamo un risultato fondamentale: se chiamiamo il valore arrotondato della soluzione del rilassato continuo $v_s$, e il valore ottenuto nell'approssimazione del \textit{punto} di ottimo $v_i$, è vero che:
$$ v_i \leq v_{ILP} \leq v_p $$
dove $v_{ILP}$ è la soluzione del problema ILP che stiamo cercando.
Per la precisione, possiamo definire l'errore:
$$
\varepsilon = \frac{v_s - v_i}{v_i}
$$
che sul problema in esame dà $\varepsilon \approx 60\%$, chiaramente poco utile. 
Sarà necessario adottare un qualche metodo per migliorare la stima data dall'arrotondamento del punto di ottimo del rilassato continuo.

Per adesso, generalizziamo quando trovato finora: bisogna distinguere fra $v_i$ e $v_s$ che si parli di problemi di minimizzazione o di massimizzazione.
Si ha infatti che:
\begin{theorem}{Bound di problemi ILP}
	Per un dato problema ILP con soluzione $v_{ILP}$, si ha che:
	$$
		v_i \leq v_{PLI} \leq v_s
	$$
	\begin{itemize}
		\item Se il problema è di \textbf{massimizzazione}, $v_i$ è il valore calcolato soluzione ammissibile del rilassato continuo arrotondata \textbf{per difetto}, e $v_s$ è la soluzione ottima del rilassato continuo arrotondata \textbf{per difetto};
		\item Se il problema è di \textbf{minimizzazione}, $v_i$ è la soluzione ottima del rilassato continuo arrotondata \textbf{per eccesso}, e $v_i$ è il valore calcolato soluzione ammissibile del rilassato continuo arrotondata \textbf{per eccesso};
	\end{itemize}
\end{theorem}

\subsubsection{Vincoli booleani}
Possiamo chiamare il vincolo introdotto prima, $x \in \{ 0, 1 \}^n$, \textbf{vincolo booleano} (in modo più o meno informale).
Come abbiamo detto, questo vincolo può essere reso come $x_i \geq 0$, o ancor meglio $0 \geq x_i \geq 1$ quando si porta al rilassato continuo (abbiamo usato il primo insieme di vincoli per ricavare il duale del rilassato continuo).

Introduciamo quindi l'algoritmo, equivalente a quello presentato prima per :
\begin{algorithm}[H]
\caption{caricamento per soluzioni ottime del rilassato continuo con vincolo booleano}
\begin{algorithmic}
	\STATE \textbf{Input:} il rilassato continuo di un problema con vincolo booleano
	\STATE \textbf{Output:} la soluzione ottima $\bar{x}$
	\STATE Considera i $r_i = \frac{v_i}{p_i}$ rendimenti di ogni elemento
	\STATE Scegli l'indice $j$ del rendimento massimo $r_j$
	\STATE Satura la variabile $j$
	\STATE Quando finisci lo spazio, satura con il bene rimanente (sic.)
\end{algorithmic}
\end{algorithm}

\end{document}


\documentclass[a4paper,11pt]{article}
\usepackage[a4paper, margin=8em]{geometry}

% usa i pacchetti per la scrittura in italiano
\usepackage[french,italian]{babel}
\usepackage[T1]{fontenc}
\usepackage[utf8]{inputenc}
\frenchspacing 

% usa i pacchetti per la formattazione matematica
\usepackage{amsmath, amssymb, amsthm, amsfonts}

% usa altri pacchetti
\usepackage{gensymb}
\usepackage{hyperref}
\usepackage{standalone}

% imposta il titolo
\title{Appunti Ricerca Operativa}
\author{Luca Seggiani}
\date{2024}

% disegni
\usepackage{pgfplots}
\pgfplotsset{width=10cm,compat=1.9}

% imposta lo stile
% usa helvetica
\usepackage[scaled]{helvet}
% usa palatino
\usepackage{palatino}
% usa un font monospazio guardabile
\usepackage{lmodern}

\renewcommand{\rmdefault}{ppl}
\renewcommand{\sfdefault}{phv}
\renewcommand{\ttdefault}{lmtt}

% disponi il titolo
\makeatletter
\renewcommand{\maketitle} {
	\begin{center} 
		\begin{minipage}[t]{.8\textwidth}
			\textsf{\huge\bfseries \@title} 
		\end{minipage}%
		\begin{minipage}[t]{.2\textwidth}
			\raggedleft \vspace{-1.65em}
			\textsf{\small \@author} \vfill
			\textsf{\small \@date}
		\end{minipage}
		\par
	\end{center}

	\thispagestyle{empty}
	\pagestyle{fancy}
}
\makeatother

% disponi teoremi
\usepackage{tcolorbox}
\newtcolorbox[auto counter, number within=section]{theorem}[2][]{%
	colback=blue!10, 
	colframe=blue!40!black, 
	sharp corners=northwest,
	fonttitle=\sffamily\bfseries, 
	title=Teorema~\thetcbcounter: #2, 
	#1
}

% disponi definizioni
\newtcolorbox[auto counter, number within=section]{definition}[2][]{%
	colback=red!10,
	colframe=red!40!black,
	sharp corners=northwest,
	fonttitle=\sffamily\bfseries,
	title=Definizione~\thetcbcounter: #2,
	#1
}

% disponi problemi
\newtcolorbox[auto counter, number within=section]{problem}[2][]{%
	colback=green!10,
	colframe=green!40!black,
	sharp corners=northwest,
	fonttitle=\sffamily\bfseries,
	title=Problema~\thetcbcounter: #2,
	#1
}

% disponi codice
\usepackage{listings}
\usepackage[table]{xcolor}

\lstdefinestyle{codestyle}{
		backgroundcolor=\color{black!5}, 
		commentstyle=\color{codegreen},
		keywordstyle=\bfseries\color{magenta},
		numberstyle=\sffamily\tiny\color{black!60},
		stringstyle=\color{green!50!black},
		basicstyle=\ttfamily\footnotesize,
		breakatwhitespace=false,         
		breaklines=true,                 
		captionpos=b,                    
		keepspaces=true,                 
		numbers=left,                    
		numbersep=5pt,                  
		showspaces=false,                
		showstringspaces=false,
		showtabs=false,                  
		tabsize=2
}

\lstdefinestyle{shellstyle}{
		backgroundcolor=\color{black!5}, 
		basicstyle=\ttfamily\footnotesize\color{black}, 
		commentstyle=\color{black}, 
		keywordstyle=\color{black},
		numberstyle=\color{black!5},
		stringstyle=\color{black}, 
		showspaces=false,
		showstringspaces=false, 
		showtabs=false, 
		tabsize=2, 
		numbers=none, 
		breaklines=true
}

\lstdefinelanguage{javascript}{
	keywords={typeof, new, true, false, catch, function, return, null, catch, switch, var, if, in, while, do, else, case, break},
	keywordstyle=\color{blue}\bfseries,
	ndkeywords={class, export, boolean, throw, implements, import, this},
	ndkeywordstyle=\color{darkgray}\bfseries,
	identifierstyle=\color{black},
	sensitive=false,
	comment=[l]{//},
	morecomment=[s]{/*}{*/},
	commentstyle=\color{purple}\ttfamily,
	stringstyle=\color{red}\ttfamily,
	morestring=[b]',
	morestring=[b]"
}

% disponi sezioni
\usepackage{titlesec}

\titleformat{\section}
	{\sffamily\Large\bfseries} 
	{\thesection}{1em}{} 
\titleformat{\subsection}
	{\sffamily\large\bfseries}   
	{\thesubsection}{1em}{} 
\titleformat{\subsubsection}
	{\sffamily\normalsize\bfseries} 
	{\thesubsubsection}{1em}{}

% disponi alberi
\usepackage{forest}

\forestset{
	rectstyle/.style={
		for tree={rectangle,draw,font=\large\sffamily}
	},
	roundstyle/.style={
		for tree={circle,draw,font=\large}
	}
}

% disponi algoritmi
\usepackage{algorithm}
\usepackage{algorithmic}
\makeatletter
\renewcommand{\ALG@name}{Algoritmo}
\makeatother

% disponi numeri di pagina
\usepackage{fancyhdr}
\fancyhf{} 
\fancyfoot[L]{\sffamily{\thepage}}

\makeatletter
\fancyhead[L]{\raisebox{1ex}[0pt][0pt]{\sffamily{\@title \ \@date}}} 
\fancyhead[R]{\raisebox{1ex}[0pt][0pt]{\sffamily{\@author}}}
\makeatother

\begin{document}

% sezione (data)
\section{Lezione del 14-10-24}

% stili pagina
\thispagestyle{empty}
\pagestyle{fancy}

% testo
\subsection{Algoritmi dei rendimenti}
Riassumiamo i quattro algoritmi presentati finora per le valutazioni inferiori e superiori di problemi di ILP (con vincolo $x \in \mathbb{Z}^n$). 
Prendiamo in esempio lo "zaino":\begin{table}[h!]
	\center 
	\begin{tabular} { c | c c c c }
		$v$ & $10$ & $17$ & $22$ & $21$ \\
		\hline 
		$p$ & $4$ & $5$ & $6$ & $2$ \\
	\end{tabular}
\end{table}
con $P = 7$.
Il vettore dei rendimenti sarà quindi:
$$
r = \left( \frac{10}{4}, \frac{17}{5}, \frac{22}{6}, \frac{21}{2} \right) \approx (2.5, 3.4, 3.66, 10.5)
$$

\begin{itemize}
	\item \textbf{\textsf{Problema booleano}} \\
	\[
		\begin{cases}
			\max(10 x_1 + 17 x_2 + 22 x_33 + 21 x_4) \\ 
			4 x_1 + 5 x_2 + 6 x_3 + 2 x_4 \leq 7 \\ 
			x \in \{ 0, 1 \}^n
		\end{cases}
	\]
	si hanno gli algoritmi di valutazione:
	\begin{itemize}
		\item \textbf{Valutazione inferiore:} si prendono le variabili con rendimenti migliori, una volta sola, finché non si satura.
			Nel caso la variabile esca da $P$, si prende quella dopo, con il caso limite di non prendere nulla.
			Si ha quindi:
			$$ x = (0, 1, 0, 1), \quad V_I = 21 + 17 = 38 $$
		\item \textbf{Valutazione superiore:} si prende il rilassato continuo:
			\[
				\begin{cases}
					\max (v^\intercal x) \\ 
					p^\intercal x \leq \\ 
					0 \leq x \leq 1
				\end{cases}
			\]
			e si riempie con le variabili dai rendimenti migliori, saturando l'ultima:
			$$ x = \left( 0, 0, \frac{5}{6}, 1 \right), \quad V_\alpha = 21 + \frac{5}{6} 22 = 39.\overline{3}, \quad V_S = \lfloor V_\alpha \rfloor = \lfloor 39.\overline{3} \rfloor = 39 $$
	\end{itemize}
	Si ha quindi $V_I = 38$ e $V_S = 39$, con errore $\epsilon = \frac{39 - 38}{38} = \frac{1}{38} = 2.6\%$.
\item \textbf{\textsf{Problema intero}} \\
	\[
		\begin{cases}
			\max(10 x_1 + 17 x_2 + 22 x_33 + 21 x_4) \\ 
			4 x_1 + 5 x_2 + 6 x_3 + 2 x_4 \leq 7 \\ 
			x \leq 0
		\end{cases}
	\]
	si hanno gli algoritmi di valutazione:
	\begin{itemize}
		\item \textbf{Valutazione inferiore:} si prendono le variabili con rendimenti migliori, con coefficienti maggiori possibili, finché non si satura.
			Nel caso la variabile esca da $P$, si prende quella dopo, con il caso limite di non prendere nulla.
			Si ha quindi:
			$$ x = (0, 0, 0, 3), \quad V_I = 3 \cdot 21 = 63 $$
		\item \textbf{Valutazione superiore:} si prende il rilassato continuo:
			\[
				\begin{cases}
					\max (v^\intercal x) \\ 
					p^\intercal x \leq \\ 
					0 \leq 0
				\end{cases}
			\]
			e si riempie con le variabili dai rendimenti migliori, saturando dalla prima:
			$$ x = \left( 0, 0, 0, \frac{7}{2} \right), \quad V_\alpha = \frac{7}{2} 22 = 73.5, \quad V_S = \lfloor V_\alpha \rfloor = \lfloor 73.5 \rfloor = 73.5 $$
	\end{itemize}
	Si ha quindi $V_I = 63$ e $V_S = 73.5$, con errore $\epsilon = \frac{73.5 - 63}{63} = \frac{1}{6} = 16\%$.
	Notiamo come sul problema intero si accumuli molto più errore.
\end{itemize}

Dobbiamo fare un'ulteriore precisazione: si prendono approssimazioni per \textbf{difetto} quanto si parla di problemi di \textit{massimo}.
Nel caso di problemi di \textit{minimo}, è opportuno prendere approssimazioni per \textbf{eccesso}.

\end{document}


\documentclass[a4paper,11pt]{article}
\usepackage[a4paper, margin=8em]{geometry}

% usa i pacchetti per la scrittura in italiano
\usepackage[french,italian]{babel}
\usepackage[T1]{fontenc}
\usepackage[utf8]{inputenc}
\frenchspacing 

% usa i pacchetti per la formattazione matematica
\usepackage{amsmath, amssymb, amsthm, amsfonts}

% usa altri pacchetti
\usepackage{gensymb}
\usepackage{hyperref}
\usepackage{standalone}

% imposta il titolo
\title{Appunti Ricerca Operativa}
\author{Luca Seggiani}
\date{2024}

% disegni
\usepackage{pgfplots}
\pgfplotsset{width=10cm,compat=1.9}

% imposta lo stile
% usa helvetica
\usepackage[scaled]{helvet}
% usa palatino
\usepackage{palatino}
% usa un font monospazio guardabile
\usepackage{lmodern}

\renewcommand{\rmdefault}{ppl}
\renewcommand{\sfdefault}{phv}
\renewcommand{\ttdefault}{lmtt}

% disponi il titolo
\makeatletter
\renewcommand{\maketitle} {
	\begin{center} 
		\begin{minipage}[t]{.8\textwidth}
			\textsf{\huge\bfseries \@title} 
		\end{minipage}%
		\begin{minipage}[t]{.2\textwidth}
			\raggedleft \vspace{-1.65em}
			\textsf{\small \@author} \vfill
			\textsf{\small \@date}
		\end{minipage}
		\par
	\end{center}

	\thispagestyle{empty}
	\pagestyle{fancy}
}
\makeatother

% disponi teoremi
\usepackage{tcolorbox}
\newtcolorbox[auto counter, number within=section]{theorem}[2][]{%
	colback=blue!10, 
	colframe=blue!40!black, 
	sharp corners=northwest,
	fonttitle=\sffamily\bfseries, 
	title=Teorema~\thetcbcounter: #2, 
	#1
}

% disponi definizioni
\newtcolorbox[auto counter, number within=section]{definition}[2][]{%
	colback=red!10,
	colframe=red!40!black,
	sharp corners=northwest,
	fonttitle=\sffamily\bfseries,
	title=Definizione~\thetcbcounter: #2,
	#1
}

% disponi problemi
\newtcolorbox[auto counter, number within=section]{problem}[2][]{%
	colback=green!10,
	colframe=green!40!black,
	sharp corners=northwest,
	fonttitle=\sffamily\bfseries,
	title=Problema~\thetcbcounter: #2,
	#1
}

% disponi codice
\usepackage{listings}
\usepackage[table]{xcolor}

\lstdefinestyle{codestyle}{
		backgroundcolor=\color{black!5}, 
		commentstyle=\color{codegreen},
		keywordstyle=\bfseries\color{magenta},
		numberstyle=\sffamily\tiny\color{black!60},
		stringstyle=\color{green!50!black},
		basicstyle=\ttfamily\footnotesize,
		breakatwhitespace=false,         
		breaklines=true,                 
		captionpos=b,                    
		keepspaces=true,                 
		numbers=left,                    
		numbersep=5pt,                  
		showspaces=false,                
		showstringspaces=false,
		showtabs=false,                  
		tabsize=2
}

\lstdefinestyle{shellstyle}{
		backgroundcolor=\color{black!5}, 
		basicstyle=\ttfamily\footnotesize\color{black}, 
		commentstyle=\color{black}, 
		keywordstyle=\color{black},
		numberstyle=\color{black!5},
		stringstyle=\color{black}, 
		showspaces=false,
		showstringspaces=false, 
		showtabs=false, 
		tabsize=2, 
		numbers=none, 
		breaklines=true
}

\lstdefinelanguage{javascript}{
	keywords={typeof, new, true, false, catch, function, return, null, catch, switch, var, if, in, while, do, else, case, break},
	keywordstyle=\color{blue}\bfseries,
	ndkeywords={class, export, boolean, throw, implements, import, this},
	ndkeywordstyle=\color{darkgray}\bfseries,
	identifierstyle=\color{black},
	sensitive=false,
	comment=[l]{//},
	morecomment=[s]{/*}{*/},
	commentstyle=\color{purple}\ttfamily,
	stringstyle=\color{red}\ttfamily,
	morestring=[b]',
	morestring=[b]"
}

% disponi sezioni
\usepackage{titlesec}

\titleformat{\section}
	{\sffamily\Large\bfseries} 
	{\thesection}{1em}{} 
\titleformat{\subsection}
	{\sffamily\large\bfseries}   
	{\thesubsection}{1em}{} 
\titleformat{\subsubsection}
	{\sffamily\normalsize\bfseries} 
	{\thesubsubsection}{1em}{}

% disponi alberi
\usepackage{forest}

\forestset{
	rectstyle/.style={
		for tree={rectangle,draw,font=\large\sffamily}
	},
	roundstyle/.style={
		for tree={circle,draw,font=\large}
	}
}

% disponi algoritmi
\usepackage{algorithm}
\usepackage{algorithmic}
\makeatletter
\renewcommand{\ALG@name}{Algoritmo}
\makeatother

% disponi numeri di pagina
\usepackage{fancyhdr}
\fancyhf{} 
\fancyfoot[L]{\sffamily{\thepage}}

\makeatletter
\fancyhead[L]{\raisebox{1ex}[0pt][0pt]{\sffamily{\@title \ \@date}}} 
\fancyhead[R]{\raisebox{1ex}[0pt][0pt]{\sffamily{\@author}}}
\makeatother

\begin{document}

% sezione (data)
\section{Lezione del 15-10-24}

% stili pagina
\thispagestyle{empty}
\pagestyle{fancy}

% testo

\subsection{Relazioni tra LP e ILP}
Vediamo di approfondire il legame fra un problema ILP e i problemi LP che possiamo ricavarne.
Avevamo posto un problema ILP in forma:
\[
	\begin{cases}
			\max (c^\intercal x) \\ 
			Ax \leq b \\ 
			x \in \mathbb{Z}^n
	\end{cases}
\]

E avevamo visto che si può trovare un limite inferiore $V_I$ e un limite superiore $V_S$ a partire dagli algoritmi dei rendimenti, quindi prendendo il rilassato continuo del problema, cioè l'associato che rimuove il vincolo $x \in \mathbb{Z}^n$:
\[
	\begin{cases}
			\max (c^\intercal x) \\ 
			Ax \leq b
	\end{cases}
\]

Chiamiamo $P$ il poliedro del rilassato continuo.
Si ha, in generale, che la soluzione di un problema di ILP è uno dei punti $\in \Omega = P \cap \mathbb{Z}^n$, cioè dei punti $\in \mathbb{Z}^n$ che stanno all'interno del poliedro del rilassato continuo.
Poniamo ad esempio il problema:

\[
	\begin{cases}
		\max(0.3 x_1 + 0.4 x_2) \\
		3 x_1 + 5 x_2 \leq 15 \\ 
		4 x_1 +  4 x_2 \leq 16 \\ 
		x_i \geq 0 \\
		x \in \mathbb{Z}^2
	\end{cases}
\]

Graficamente, si ha:

\begin{center}

\begin{tikzpicture}
\begin{axis}[
    axis lines = middle,
		grid=both,
    xlabel = {$x_1$},
    ylabel = {$x_2$},
    xmin=0, xmax=6.9,
    ymin=0, ymax=3.9,
    domain=0:10,
    samples=100,
    width=15cm, height=9cm,
    legend pos=north east
  ]

% regione ammissibile

\addplot[fill=gray, opacity=0.4, forget plot] 
	coordinates {
		(0,0)
		(0,3)
		(2.5, 1.5)
		(4, 0)
	};

% rette

\addplot[domain=0:2.5, thick, red] {3 - 0.6*x};
\addlegendentry{$ 3 x_1 + 5 x_2 \leq 15 $}

\addplot[domain=2.5:4, thick, blue] {4 - x};
\addlegendentry{$ 4 x_1 + 4 x_2 \leq 16 $}

\addplot[
    only marks,  % Only marks the points without connecting lines
    mark=*,
    color=violet,
    mark size=2pt
] coordinates {
	(0,0)
	(0,1)
	(0,2)
	(0,3)
	(1,0)
	(1,1)
	(1,2)
	(2,0)
	(2,1)
	(3,0)
	(3,1)
	(4,0)
};
\addlegendentry{$\Omega$}

\addplot[
	only marks,
	mark=*,
	mark size=2pt
] coordinates {
	(2.5, 1.5)
};

\node at (axis cs:2.5, 1.5) [above right] {$x_{V_S}$};
 
\end{axis}
\end{tikzpicture}

\end{center}
dove si è riportata la soluzione ottima del rilassato $x_{V_S}$.

Notiamo quindi che esiste un'insieme:
$$
\mathrm{conv} (\Omega) = \left\{ x \in \mathbb{R}^n : \exists x_1, ..., x_p \in \Omega, \quad \lambda_1, ..., \lambda_p \geq 0 \quad \text{t.c.} \ \ \ x = \sum_{i=1}^p  \lambda_i x_i, \quad \sum_{i=1}^p  \lambda_i = 1 \right\}
$$
cioè l'involucro convesso di $\Omega$.
Visto che i punti di $\Omega$ hanno componenti intere, si può dimostrare il seguente teorema:
\begin{theorem}{Caratterizzazione della regione ammissibile di un problema ILP}
	Dato un problema ILP con regione ammissibile $\Omega$, esiste un insieme finito di punti $\left\{ q_l \right\}_{l \in L} = \{ q_1, ..., q_{|L|} \}$ di $\Omega$, e un insieme finito di direzioni di recessione $\{ r_j \}_{j \in J} = \{ r_1, ..., r_|J| \}$ di $P$, tali che:

$$ 
\Omega = \left\{ x \in R^n_+: x = \sum_{l\in L} \alpha_l q_l + \sum_{j\in J} \beta_j r_j, \quad \sum_{l \in L} \alpha_l = 1, \quad \alpha \in \mathbb{Z}^{|L|}_+, \quad \beta \in \mathbb{Z}^{|J|}_+ \right\}
$$
\end{theorem}
cioè si puo ricavare $\Omega$ attraverso una forma simile alla $P = \mathrm{conv}(V) + \mathrm{cono}(E)$ del teorema di Minkowski-Weyl, sugli insiemi $\left\{ q_l \right\}_{l \in L}$ di vertici a componenti intere e $\{ r_j \}_{j \in J}$ di direzioni di recessione.

A partire da questa caratterizzazione di $\Omega$, vogliamo caratterizzare $\mathrm{conv} (\Omega)$:

\begin{theorem}{Caratterizzazione dell'involucro convesso della regione ammissibile di un problema ILP}
	Dato un problema ILP con regione ammissibile $\Omega$, si ha che $\mathrm{conv}(\Omega)$ è un \textbf{poliedro razionale}, cioè esistono due insiemi finiti di vettori, $\left\{ q_l \right\}_{l \in L}$ e $\{ r_j \}_{j \in J}$, a \textbf{componenti razionali}, tali che:

$$
\mathrm{conv}(\Omega) = \mathrm{conv} \left\{ q_l \right\}_{l \in L} + \mathrm{cono} \{ r_j \}_{j \in J}
$$

\end{theorem}

Addirittura, normalizzando si può supporre che gli $r_j$ siano a componenti intere.

Nell'esempio precedente, $\mathrm{conv} (\Omega)$ sarebbe rappresentato dagli insiemi: 
$$
	\left\{ q_l \right\}_{l \in L} = \{ (0,0), (0,3), (3,1), (4,0) \}, \quad \{ r_j \}_{j \in J} = \emptyset 
$$
quindi:
$$
\mathrm{conv}(\Omega) = \mathrm{conv} \left\{ q_l \right\}_{l \in L} + \mathrm{cono} \{ r_j \}_{j \in J} = (x_1, x_2) \in{R^2} \quad  \text{t.c.} \quad
	\begin{cases}
		\frac{2}{3}x_1 + x_2 \leq 3 \\ 
		x_1 +x_2 \leq 4
	\end{cases}
$$

cioè sul grafico:

\begin{center}

\begin{tikzpicture}
\begin{axis}[
    axis lines = middle,
		grid=both,
    xlabel = {$x_1$},
    ylabel = {$x_2$},
    xmin=0, xmax=6.9,
    ymin=0, ymax=3.9,
    domain=0:10,
    samples=100,
    width=15cm, height=9cm,
    legend pos=north east
  ]

% regione ammissibile

\addplot[fill=gray, opacity=0.4, forget plot] 
	coordinates {
		(0,0)
		(0,3)
		(3, 1)
		(4, 0)
	};

% rette

\addplot[domain=0:3, thick, red] {3 - 0.6666666666666666666*x};
\addlegendentry{$ \frac{2}{3} x_1 +  x_2 \leq 3 $}

\addplot[domain=3:4, thick, blue] {4 - x};
\addlegendentry{$ x_1 + 4 x_2 \leq 4 $}

\addplot[
    only marks,  % Only marks the points without connecting lines
    mark=*,
    color=violet,
    mark size=2pt
] coordinates {
	(0,0)
	(0,1)
	(0,2)
	(0,3)
	(1,0)
	(1,1)
	(1,2)
	(2,0)
	(2,1)
	(3,0)
	(3,1)
	(4,0)
};
\addlegendentry{$\Omega$}

\addplot[
	only marks,
	mark=*,
	mark size=2pt
] coordinates {
	(3, 1)
};

\node at (axis cs:3, 1) [above right] {$x_{\mathrm{conv}(\Omega)}$};
 
\end{axis}
\end{tikzpicture}

\end{center}

dove è stata evidenziata la soluzione del primale sull'insieme $\mathrm{conv}(\Omega)$, $x_{\mathrm{conv}(\Omega)}$.

Possiamo quindi dire, visto che $\Omega \subset \mathrm{conv}(\Omega)$ e che $P$ è un'estensione di $\Omega$ in quanto poliedro del rilassato continuo, che è vera la catena di diseguaglianze:
$$
\max_{x \in \Omega} c^\intercal x \leq \max_{x \in \mathrm{conv}(\Omega)} c^\intercal x \leq \max_{x \in P} c^\intercal x
$$
e non solo: si può stringere la diseguaglianza sul lato sinistro, per affermare che:
\begin{theorem}{Equivalenza fra problemi LP e ILP}
	Si prenda un problema di ILP, e il problema di LP associato costruito su:
$$
\mathrm{conv}(\Omega) = \mathrm{conv} \left\{ q_l \right\}_{l \in L} + \mathrm{cono} \{ r_j \}_{j \in J}
$$
con, posto $P$ come il poliedro del rilassato continuo, $q$ ricavato dai vertici $P \cap \mathbb{Z}^n$, e $r$ ricavato dalle direzioni di recessione di $P$. 

Se si prendono le soluzioni:
$$ v_\Omega = \max_{x \in \Omega} c^\intercal x, \quad v_{\mathrm{conv}(\Omega)} = \max_{x \in \mathrm{conv}(\Omega)} c^\intercal x$$
si ha che $v_\Omega = v_{\mathrm{conv}(\Omega)}$, e che se $v_{\mathrm{conv}(\Omega)}$ è finito, allora esiste $x_\Omega \in \Omega$ tale che $c^\intercal x_\Omega = v_\Omega = v_{\mathrm{conv}(\Omega)}$
\end{theorem}

Siamo quindi arrivati a formulare un teorema secondo cui, per ogni problema ILP, possamo costruire un problema LP associato che ha la stessa soluzione, semplicemente riformulando i vincoli in modo che descrivano l'involucro convesso dei punti in $\Omega = P \cap \mathbb{Z}^n$, ed eventuali direzioni di recessione di $P$, dove $P$ è il poliedro del rilassato continuo.
Il problema sorge dal fatto che è \textit{difficile} (nell'accezzione di \textit{difficile} data dalla complessità) ricavare questo problema associato.

\subsubsection{Caratterizzazione dell'involucro convesso}
Esistono tre strade principali per il calcolo di $\mathrm{conv}(\Omega)$:
\begin{itemize}
	\item \textbf{Matrici unimodulari:} in problemi LP dove i vincoli sono espressi da matrici particolari, che chiameremo \textbf{unimodulari}, si ha che $P$ ha vertici a componenti intere, e quindi il $\mathrm{conv}(\Omega) = P \cap \mathbb{Z}^n$ degli $\Omega$ a componenti interi non potrà che coincidere con $P$ che li contiene. Questi problemi sono ad esempio quelli di flusso a costo minimo.
	\item \textbf{Combinatorica poliedrale:} in alcuni problemi LP espressi su grafi, si può stabilire la forma delle disequazioni che esprimono $\mathrm{conv}(\Omega)$. Ad esempio, si ha che i problemi di accoppiamento massimo sui grafi, che richiedono di trovare il maggior numero di archi attraverso il cui si può realizzare un accoppiamento, risulta coincidente in $P$ e $\mathrm{conv}{\Omega}$ quando il grafo di interesse è \textbf{bipartito}, cioè è divisibile in due insiemi di nodi $U$ e $V$ tali che ogni arco porta da $U \rightarrow V$ o viceversa.
	\item \textbf{Piani di taglio:} quando non si possono applicare i metodi riportati sopra, l'unica strada è quella di trovare una serie di disequazioni successive che riducono l'insieme $P$ fino al minimale $\mathrm{conv}(\Omega)$.
		Uno dei metodi che andremo a vedere per realizzare queste riduzioni è quello dei \textbf{piani di taglio di Gomory}.
\end{itemize}

\subsubsection{Matrici unimodulari}
Definiamo innanzitutto:
\begin{definition}{Matrice unimodulare}
	Si chiama \textbf{modulare} ogni matrice quadrata intera con determinante $\det{A_{m}} \in \left\{ 1, -1\right\}$.
\end{definition}
e, sulla base di questo:
\begin{definition}{Matrice totalmente unimodulare}
	Si chiama \textbf{totalmente unimodulare} ogni matrice per cui ogni sottomatrice quadrata invertibile è unimodulare, cioè ogni sottomatrice quadrata ha determinante $\det(A_{m}) \in \left\{0, 1, -1\right\}$.
\end{definition}

Si ha che se una matrice è unimodulare, allora i vertici della regione ammissibile appartengono a $\mathbb{Z}^n$, infatti:
\begin{theorem}{Soluzioni di base di matrici unimodulari}
	Dato $Ax \leq b$, se $A$ e $b$ sono a componenti intere, e $A$ è totalmente unimodulare, allora tutte le soluzioni di base del poliedro $P$:
	$$
		P = \left\{ x \in \mathbb{R}^n : Ax \leq b \right\}
	$$
	sono a componenti intere.
\end{theorem}
dove per componenti intere intendiamo anche razionali sotto normalizzazione.
\par\smallskip
Questo risulato è molto utile: abbiamo che le matrici $A$ dei problemi di \textbf{trasporto} e \textbf{assegnamento di costo minimo} sono totalmente unimodulari, ergo possiamo risolvere le versioni ILP di quei problemi semplicemente rimuovendo il vincolo di interezza $x \in \mathbb{Z}^n$.

\end{document}


\documentclass[a4paper,11pt]{article}
\usepackage[a4paper, margin=8em]{geometry}

% usa i pacchetti per la scrittura in italiano
\usepackage[french,italian]{babel}
\usepackage[T1]{fontenc}
\usepackage[utf8]{inputenc}
\frenchspacing 

% usa i pacchetti per la formattazione matematica
\usepackage{amsmath, amssymb, amsthm, amsfonts}

% usa altri pacchetti
\usepackage{gensymb}
\usepackage{hyperref}
\usepackage{standalone}

% imposta il titolo
\title{Appunti Ricerca Operativa}
\author{Luca Seggiani}
\date{2024}

% disegni
\usepackage{pgfplots}
\pgfplotsset{width=10cm,compat=1.9}

% imposta lo stile
% usa helvetica
\usepackage[scaled]{helvet}
% usa palatino
\usepackage{palatino}
% usa un font monospazio guardabile
\usepackage{lmodern}

\renewcommand{\rmdefault}{ppl}
\renewcommand{\sfdefault}{phv}
\renewcommand{\ttdefault}{lmtt}

% disponi il titolo
\makeatletter
\renewcommand{\maketitle} {
	\begin{center} 
		\begin{minipage}[t]{.8\textwidth}
			\textsf{\huge\bfseries \@title} 
		\end{minipage}%
		\begin{minipage}[t]{.2\textwidth}
			\raggedleft \vspace{-1.65em}
			\textsf{\small \@author} \vfill
			\textsf{\small \@date}
		\end{minipage}
		\par
	\end{center}

	\thispagestyle{empty}
	\pagestyle{fancy}
}
\makeatother

% disponi teoremi
\usepackage{tcolorbox}
\newtcolorbox[auto counter, number within=section]{theorem}[2][]{%
	colback=blue!10, 
	colframe=blue!40!black, 
	sharp corners=northwest,
	fonttitle=\sffamily\bfseries, 
	title=Teorema~\thetcbcounter: #2, 
	#1
}

% disponi definizioni
\newtcolorbox[auto counter, number within=section]{definition}[2][]{%
	colback=red!10,
	colframe=red!40!black,
	sharp corners=northwest,
	fonttitle=\sffamily\bfseries,
	title=Definizione~\thetcbcounter: #2,
	#1
}

% disponi problemi
\newtcolorbox[auto counter, number within=section]{problem}[2][]{%
	colback=green!10,
	colframe=green!40!black,
	sharp corners=northwest,
	fonttitle=\sffamily\bfseries,
	title=Problema~\thetcbcounter: #2,
	#1
}

% disponi codice
\usepackage{listings}
\usepackage[table]{xcolor}

\lstdefinestyle{codestyle}{
		backgroundcolor=\color{black!5}, 
		commentstyle=\color{codegreen},
		keywordstyle=\bfseries\color{magenta},
		numberstyle=\sffamily\tiny\color{black!60},
		stringstyle=\color{green!50!black},
		basicstyle=\ttfamily\footnotesize,
		breakatwhitespace=false,         
		breaklines=true,                 
		captionpos=b,                    
		keepspaces=true,                 
		numbers=left,                    
		numbersep=5pt,                  
		showspaces=false,                
		showstringspaces=false,
		showtabs=false,                  
		tabsize=2
}

\lstdefinestyle{shellstyle}{
		backgroundcolor=\color{black!5}, 
		basicstyle=\ttfamily\footnotesize\color{black}, 
		commentstyle=\color{black}, 
		keywordstyle=\color{black},
		numberstyle=\color{black!5},
		stringstyle=\color{black}, 
		showspaces=false,
		showstringspaces=false, 
		showtabs=false, 
		tabsize=2, 
		numbers=none, 
		breaklines=true
}

\lstdefinelanguage{javascript}{
	keywords={typeof, new, true, false, catch, function, return, null, catch, switch, var, if, in, while, do, else, case, break},
	keywordstyle=\color{blue}\bfseries,
	ndkeywords={class, export, boolean, throw, implements, import, this},
	ndkeywordstyle=\color{darkgray}\bfseries,
	identifierstyle=\color{black},
	sensitive=false,
	comment=[l]{//},
	morecomment=[s]{/*}{*/},
	commentstyle=\color{purple}\ttfamily,
	stringstyle=\color{red}\ttfamily,
	morestring=[b]',
	morestring=[b]"
}

% disponi sezioni
\usepackage{titlesec}

\titleformat{\section}
	{\sffamily\Large\bfseries} 
	{\thesection}{1em}{} 
\titleformat{\subsection}
	{\sffamily\large\bfseries}   
	{\thesubsection}{1em}{} 
\titleformat{\subsubsection}
	{\sffamily\normalsize\bfseries} 
	{\thesubsubsection}{1em}{}

% disponi alberi
\usepackage{forest}

\forestset{
	rectstyle/.style={
		for tree={rectangle,draw,font=\large\sffamily}
	},
	roundstyle/.style={
		for tree={circle,draw,font=\large}
	}
}

% disponi algoritmi
\usepackage{algorithm}
\usepackage{algorithmic}
\makeatletter
\renewcommand{\ALG@name}{Algoritmo}
\makeatother

% disponi numeri di pagina
\usepackage{fancyhdr}
\fancyhf{} 
\fancyfoot[L]{\sffamily{\thepage}}

\makeatletter
\fancyhead[L]{\raisebox{1ex}[0pt][0pt]{\sffamily{\@title \ \@date}}} 
\fancyhead[R]{\raisebox{1ex}[0pt][0pt]{\sffamily{\@author}}}
\makeatother

\begin{document}

% sezione (data)
\section{Lezione del 16-10-24}

% stili pagina
\thispagestyle{empty}
\pagestyle{fancy}

% testo
\subsection{Problema di impacchettamento}
\begin{problem}{Impacchettamento}
	Poniamo di avere 6 file, contenenti registrazioni di tutta la musica di Mozart, con il seguente ingombro in gigabyte:

	\center \rowcolors{2}{green!10}{green!40!black!20}
	\begin{tabular} { | c || c | c | c | c | c | c | }
		\hline 
		& \bfseries Arie & \bfseries Opere & \bfseries Concerti & \bfseries Sinfonie & \bfseries Sonate & \bfseries Messe \\
		\hline
		$p$ & 3 & 6 & 5 & 4 & 4 & 8 \\
		\hline
	\end{tabular}
	
	\par\bigskip

	\raggedright
	Vogliamo trovare il numero minimo di dischi rigidi di dimensione $P = 10$ per archiviare tutti di questi file.

\end{problem}

Chiamiamo questi problemi anche problemi di \textit{bin-packing}.
Rappresentiamo la soluzione come una matrice di adiacenza:
$$
x_{ij} =	
	\begin{cases}
			0 \\ 1
	\end{cases}
$$

dove $i \in \{ 1, ..., p \}$ rappresenta il contenitore e $j \in \{1, ..., 6\}$ l'oggetto che vi inseriamo.
Conviene trovare prima una stima superiore per il numero di contenitori $p$, attraverso un'opportuno algoritmo greedy.

Un'algoritmo banale può essere quello di riempire finché è possibile il primo contenitore, cioè finche si hanno oggetti che entrano nello spazio libero (detto \textit{sfrido}) del contenitore.
Una volta che questa ipotesi è violata, si prende un'altro contenitore, e così via.

Si ricava quindi questa valutazione superiore $V_S$.
A questo punto si può porre:
\[
	\begin{cases}
		x_{11} + x_{21} + x_{31} + x_{41} = 1 \\	
		x_{12} + x_{22} + x_{32} + x_{42} = 1 \\	
		... \\ 
		x_{16} + x_{26} + x_{36} + x_{46} = 1 \\	
	\end{cases}
\]
cioè vogliamo prendere uno e uno solo di tutti gli oggetti, disposti fra i $V_S = 4$ contenitori di valutazione superiore.

Visto che non siamo sicuri di dover prendere tutti i contenitori, dovremo introdurre una variabile $y_i$ per ognuno di essi:
$$
y_i =
	\begin{cases}
			0 \\ 1
	\end{cases}
$$

Infine vogliamo inserire la dimensione di ogni contenitore, $P = 10$, nel problema, sulla base dei pesi $p_i$ di ogni oggetto:
\[
	\begin{cases}
		p_1 x_{11} + p_2 x_{12} + p_3 x_{13} + p_4 x_{14} + p_5 x_{15} + p_6 x_{16} \leq 10 y_1 \\ 
		p_1 x_{21} + p_2 x_{22} + p_3 x_{23} + p_4 x_{24} + p_5 x_{25} + p_6 x_{26} \leq 10 y_2 \\ 
		... \\
		p_1 x_{41} + p_2 x_{42} + p_3 x_{43} + p_4 x_{44} + p_5 x_{45} + p_6 x_{46} \leq 10 y_4 \\ 
	\end{cases}
\]

Combinando quanto posto finora, otteniamo il problema completo:
\[
	\begin{cases}
			\min(y_1 + y_2 + y_3 + y_4) \\

		x_{11} + x_{21} + x_{31} + x_{41} = 1 \\	
		x_{12} + x_{22} + x_{32} + x_{42} = 1 \\	
		... \\ 
		x_{16} + x_{26} + x_{36} + x_{46} = 1 \\	
		
		p_1 x_{11} + p_2 x_{12} + p_3 x_{13} + p_4 x_{14} + p_5 x_{15} + p_6 x_{16} \leq 10 y_1 \\ 
		p_1 x_{21} + p_2 x_{22} + p_3 x_{23} + p_4 x_{24} + p_5 x_{25} + p_6 x_{26} \leq 10 y_2 \\ 
		... \\
		p_1 x_{41} + p_2 x_{42} + p_3 x_{43} + p_4 x_{44} + p_5 x_{45} + p_6 x_{46} \leq 10 y_4 \\ 

		x_i \in \{ 0, 1 \} \\
		y_i \in \{ 0, 1 \} \\
	\end{cases}
\]
dove minimizziamo gli $y_1$ cercando di usare meno contenitori possibile.
Questo è un problema di ILP.
Cerchiamo quindi la valutazione inferiore $V_I$ e la superiore $V_S$.

Possiamo dare una stima inferiore attraverso la formula:
$$
V_I = \left\lceil \frac{\sum_{j=1}^6 p_i}{P} \right\rceil
$$
cioè il peso di tutti gli oggetti fratto le dimensioni dei contenitori, arrotondato per eccesso, che è il minimo numero di contenitori possibile per contenere tutti gli oggetti.

Per il calcolo della stima superiore, invece, avevamo presentato un algoritmo greedy.
In verità sono ci altre (e più intelligenti) strade che possiamo prendere:

\begin{itemize}
	\item \textbf{Next-fit decreasing:} essenzialmente l'algoritmo presentato, dove si ha:
\begin{algorithm}[H]
\caption{next-fit decreasing per impachettamento}
\begin{algorithmic}
	\STATE \textbf{Input:} un problema di impacchettamento % input
	\STATE \textbf{Output:} una soluzione ammissibile % output
	% body
	\WHILE{ci sono ancora oggetti}
		\STATE Prendi il prossimo oggetto
		\IF{entra nel contenitore} 
			\STATE Inseriscilo nel contenitore
		\ELSE 
			\STATE Prendi un'altro contenitore e inseriscici l'oggetto
		\ENDIF
	\ENDWHILE
\end{algorithmic}
\end{algorithm}
	\item \textbf{First-fit decreasing:} analogo al next-fit, ma con la differenza che per ogni oggetto si considerano tutti i contenitori:
\begin{algorithm}[H]
\caption{first-fit decreasing per impachettamento}
\begin{algorithmic}
	\STATE \textbf{Input:} un problema di impacchettamento % input
	\STATE \textbf{Output:} una soluzione ammissibile % output
	% body
	\WHILE{ci sono ancora oggetti}
		\STATE Prendi il prossimo oggetto
		\IF{l'oggetto entra in uno dei contenitori presi finora}
			\STATE Inseriscilo nel contenitore
		\ELSE 
			\STATE Prendi un'altro contenitore e inseriscici l'oggetto
		\ENDIF
	\ENDWHILE
\end{algorithmic}
\end{algorithm}
	\item \textbf{Best-fit decreasing:} è una variante del first-fit che prende sempre i contenitori con sfrido massimo, cioè cerca di riempire i contenitori con meno spazio disponibile (cioè di trovare l'"incastro" migliore per l'oggetto):
\begin{algorithm}[H]
\caption{best-fit decreasing per impachettamento}
\begin{algorithmic}
	\STATE \textbf{Input:} un problema di impacchettamento % input
	\STATE \textbf{Output:} una soluzione ammissibile % output
	% body
	\WHILE{ci sono ancora oggetti}
		\STATE Prendi il prossimo oggetto
		\IF{l'oggetto entra in uno dei contenitori, ordinati per sfrido decrescente, presi finora}
			\STATE Inseriscilo nel contenitore
		\ELSE 
			\STATE Prendi un'altro contenitore e inseriscici l'oggetto
		\ENDIF
	\ENDWHILE
\end{algorithmic}
\end{algorithm}
\end{itemize}

\end{document}


\documentclass[a4paper,11pt]{article}
\usepackage[a4paper, margin=8em]{geometry}

% usa i pacchetti per la scrittura in italiano
\usepackage[french,italian]{babel}
\usepackage[T1]{fontenc}
\usepackage[utf8]{inputenc}
\frenchspacing 

% usa i pacchetti per la formattazione matematica
\usepackage{amsmath, amssymb, amsthm, amsfonts}

% usa altri pacchetti
\usepackage{gensymb}
\usepackage{hyperref}
\usepackage{standalone}

% imposta il titolo
\title{Appunti Ricerca Operativa}
\author{Luca Seggiani}
\date{2024}

% disegni
\usepackage{pgfplots}
\pgfplotsset{width=10cm,compat=1.9}

% imposta lo stile
% usa helvetica
\usepackage[scaled]{helvet}
% usa palatino
\usepackage{palatino}
% usa un font monospazio guardabile
\usepackage{lmodern}

\renewcommand{\rmdefault}{ppl}
\renewcommand{\sfdefault}{phv}
\renewcommand{\ttdefault}{lmtt}

% disponi il titolo
\makeatletter
\renewcommand{\maketitle} {
	\begin{center} 
		\begin{minipage}[t]{.8\textwidth}
			\textsf{\huge\bfseries \@title} 
		\end{minipage}%
		\begin{minipage}[t]{.2\textwidth}
			\raggedleft \vspace{-1.65em}
			\textsf{\small \@author} \vfill
			\textsf{\small \@date}
		\end{minipage}
		\par
	\end{center}

	\thispagestyle{empty}
	\pagestyle{fancy}
}
\makeatother

% disponi teoremi
\usepackage{tcolorbox}
\newtcolorbox[auto counter, number within=section]{theorem}[2][]{%
	colback=blue!10, 
	colframe=blue!40!black, 
	sharp corners=northwest,
	fonttitle=\sffamily\bfseries, 
	title=Teorema~\thetcbcounter: #2, 
	#1
}

% disponi definizioni
\newtcolorbox[auto counter, number within=section]{definition}[2][]{%
	colback=red!10,
	colframe=red!40!black,
	sharp corners=northwest,
	fonttitle=\sffamily\bfseries,
	title=Definizione~\thetcbcounter: #2,
	#1
}

% disponi problemi
\newtcolorbox[auto counter, number within=section]{problem}[2][]{%
	colback=green!10,
	colframe=green!40!black,
	sharp corners=northwest,
	fonttitle=\sffamily\bfseries,
	title=Problema~\thetcbcounter: #2,
	#1
}

% disponi codice
\usepackage{listings}
\usepackage[table]{xcolor}

\lstdefinestyle{codestyle}{
		backgroundcolor=\color{black!5}, 
		commentstyle=\color{codegreen},
		keywordstyle=\bfseries\color{magenta},
		numberstyle=\sffamily\tiny\color{black!60},
		stringstyle=\color{green!50!black},
		basicstyle=\ttfamily\footnotesize,
		breakatwhitespace=false,         
		breaklines=true,                 
		captionpos=b,                    
		keepspaces=true,                 
		numbers=left,                    
		numbersep=5pt,                  
		showspaces=false,                
		showstringspaces=false,
		showtabs=false,                  
		tabsize=2
}

\lstdefinestyle{shellstyle}{
		backgroundcolor=\color{black!5}, 
		basicstyle=\ttfamily\footnotesize\color{black}, 
		commentstyle=\color{black}, 
		keywordstyle=\color{black},
		numberstyle=\color{black!5},
		stringstyle=\color{black}, 
		showspaces=false,
		showstringspaces=false, 
		showtabs=false, 
		tabsize=2, 
		numbers=none, 
		breaklines=true
}

\lstdefinelanguage{javascript}{
	keywords={typeof, new, true, false, catch, function, return, null, catch, switch, var, if, in, while, do, else, case, break},
	keywordstyle=\color{blue}\bfseries,
	ndkeywords={class, export, boolean, throw, implements, import, this},
	ndkeywordstyle=\color{darkgray}\bfseries,
	identifierstyle=\color{black},
	sensitive=false,
	comment=[l]{//},
	morecomment=[s]{/*}{*/},
	commentstyle=\color{purple}\ttfamily,
	stringstyle=\color{red}\ttfamily,
	morestring=[b]',
	morestring=[b]"
}

% disponi sezioni
\usepackage{titlesec}

\titleformat{\section}
	{\sffamily\Large\bfseries} 
	{\thesection}{1em}{} 
\titleformat{\subsection}
	{\sffamily\large\bfseries}   
	{\thesubsection}{1em}{} 
\titleformat{\subsubsection}
	{\sffamily\normalsize\bfseries} 
	{\thesubsubsection}{1em}{}

% disponi alberi
\usepackage{forest}

\forestset{
	rectstyle/.style={
		for tree={rectangle,draw,font=\large\sffamily}
	},
	roundstyle/.style={
		for tree={circle,draw,font=\large}
	}
}

% disponi algoritmi
\usepackage{algorithm}
\usepackage{algorithmic}
\makeatletter
\renewcommand{\ALG@name}{Algoritmo}
\makeatother

% disponi numeri di pagina
\usepackage{fancyhdr}
\fancyhf{} 
\fancyfoot[L]{\sffamily{\thepage}}

\makeatletter
\fancyhead[L]{\raisebox{1ex}[0pt][0pt]{\sffamily{\@title \ \@date}}} 
\fancyhead[R]{\raisebox{1ex}[0pt][0pt]{\sffamily{\@author}}}
\makeatother

\begin{document}

% sezione (data)
\section{Lezione del 17-10-24}

% stili pagina
\thispagestyle{empty}
\pagestyle{fancy}

% testo

\subsection{Piani di taglio}

Abbiamo visto finora il problema:

\[
	\begin{cases}
		\max c^T x \\
		Ax \leq b
		x \in \mathbb{Z}^n_+
	\end{cases}
\]

Dove si può definire il poliedro $P$ del rilassato continuo:
$$
P = \{ x \in \mathbb{R}^n : Ax \leq b \} \rightarrow s_{PL}, \ x_{RC}
$$

L'insieme dei punti $\Omega$:
$$
\Omega = \{ x \in \mathbb{Z}^n : Ax \leq b \} \rightarrow v_{PLI}, \ \bar{x}
$$

E un problema di PL associato, "ristretto" sul convesso dell'insieme $\Omega$, di cui però non sapevamo caratterizzare le disequazioni (salvo alcuni casi specifici in problemi di PL sui grafi): 
$$
\mathrm{conv}(\Omega) \rightarrow v_{\mathrm{conv}(\Omega)}, \ \bar{x} 
$$

Sappiamo valere, fra queste soluzioni, la catena di diseguaglianze:
$$ v_{PLI} = v_{\mathrm{conv}(\Omega)} \leq v_{PL} $$

Decidiamo di ridurre le dimensioni dell'insieme $P$, visto che $\mathrm{conv}(\Omega) \subseteq P$, per approssimare $\mathrm{conv}(\Omega)$ stesso.
Facciamo ciò sfruttando le cosiddette \textbf{disuguaglianze valide} (DV):
\begin{definition}{Diseguaglianza valida}
	Una diseguaglianza è una DV di un'insieme $\Omega \subset \mathbb{Z}^n$ se rispetta la forma:
	$$ \gamma^T x \leq \gamma_0, \quad \forall x \in \Omega $$
\end{definition}

Abbiamo quindi che una diseguaglianza è valida se contiene $\Omega$ completamente nel semispazio che definisce.
Si può poi dimostrare il fatto piuttosto ovvio:

\begin{theorem}{Diseguaglianza valida dell'involucro convesso}	
una diseguaglianza valida per un'insieme $\Omega$ lo è anche per $\mathrm{conv}(\Omega)$.
\end{theorem}

Una tecnica di base per il calcolo delle diseguaglianze valide è quella dell'approssimazione per difetto delle diseguaglianze che definiscono $P$.
Questo però è poco utile nel caso di problemi già definiti con diseguaglianze in componenti intere.
Introduciamo quindi l'idea di un \textbf{piano di taglio}:

\begin{definition}{Piano di taglio}	
Su un problema di $PLI$ con soluzione $x_{RC}$ al rilassato continuo, una DV in forma:
$$ \gamma x \leq \gamma_0, \quad \forall x \in S \ \ : \ \ \gamma x_{RC} > \gamma_0 $$
si dice \textbf{piano di taglio}.
\end{definition}

L'idea dei piani di taglio è quella di rimuovere successivamente soluzioni di $P$ fuori da $\mathrm{conv}(\Omega)$.
Supponiamo di avere il problema:
\[
	\begin{cases}			
		\max c^T x \\ 
		Ax \leq b \\ 
		\gamma x \leq \gamma_0
	\end{cases}
	\rightarrow V_{P_{n}}
\]

Vogliamo ricavare un poliedro $P_n$, che contiene l'ottimo $v_{PLI}$, ma che è più piccolo del rilassato continuo:
$$
v_{PLI} \leq v_{P_n} \leq v_{PL}
$$

Per fare ciò, prendiamo una DV che non contiene $x_{RC}$ soluzione del rilassato continuo. 
A questo punto si avrà che, se il problema di $v_{P_n}$ ha soluzione a componenti intere, allora coinciderà con $v_{\mathrm{conv}(\Omega)}$, altrimenti si dovranno ripetere i passi precedenti.

Abbiamo quindi che qualsiasi problema trovato iterativamente coi piani di taglio è una riduzione della regione ammissibile.
Il sottoinsieme dei piani di taglio che studieremo è quello dei \textbf{piani di taglio di Gomory}.

\subsubsection{Costruzione dei piani di taglio di Gomory}
Prendiamo un problema di ILP in forma:
\[
	\begin{cases}
		\max c^T x \\ 
		Ax = b \\ 
		x \geq 0 \\ 
		x \in \mathb{Z}^n
	\end{cases}
\]
cioè in una forma simile al duale standard.
Possiamo convertire qualsiasi problema in questa forma introducendo eventuali variabili di surplus per portare le disequazioni in equazioni.
Ad esempio, sul problema presentato un paio di lezioni fa:
\[
	\begin{cases}
		\max (0.3x_1 + 0.4x_2) \\ 
		3 x_1 + 5 x_2 \leq 15 \\ 
		4 x_1 + 4 x_2 \leq 16 \\ 
		x_i \geq 0 \\ 
		x \in \mathbb{Z}^2
	\end{cases}
\]
si ricava:
\[
	\begin{cases}
		\max (0.3x_1 + 0.4x_2) \\ 
		3 x_1 + 5 x_2 + x_3 = 15 \\ 
		4 x_1 + 4 x_2 + x_4 = 16 \\ 
		x_i \geq 0 \\ 
		x \in \mathbb{Z}^2
	\end{cases}
\]

Dalla soluzione del rilassato continuo $x_{RC}$, abbiamo una base ottima $B$, tale per cui si può porre:
$$
A = (A_B \  A_N), \quad x_{RC} = \binom{x_B}{x_N}
$$
cioè, $A_B$ sono le colonne di matrice corrispondenti alla base e $A_N$ le altre, e $x_B$ sono le variabili di base e $x_N$ le altre. 

Definiamo allora la matrice $\tilde{A}$:
$$ \tilde{A} = A_B^{-1} A_N $$

Prendiamo quindi un indice $r \in B$, tale che $x_{RC}$ a $r$ ha una componente non intera.
Possiamo a questo punto enunciare il teorema:

\begin{theorem}{Teorema di Gomory}
	Si ha che:
	$$
		\sum_{j \in N} \{ \tilde{a}_{rj} \} x_j \geq \{ ( x_{B} )_r \}
	$$
	è un piano di taglio.
\end{theorem}

Occorre fare qualche chiarimento sulla notazione: le graffe rappresentano l'operatore componente frazionaria, mentre l'$r$ a pedice di $x_{B}$ indica di prendere la componente all'indice $r$, cioè quella non intera.

Applichiamo quindi questo metodo al problema di prima.
Avremo che:
$$
A= \begin{pmatrix}
	3 & 5 & 1 & 0 \\ 4 & 4 & 0 & 1
\end{pmatrix}, \quad 
b= \begin{pmatrix}
15 \\ 16
\end{pmatrix}
$$
e la soluzione ottima $x_{RC} = \left( \frac{5}{2}, \frac{3}{2} \right)$ sulla base ottima $B = \{1,2\}$.
Abbiamo quindi che:
$$
A_B = \begin{pmatrix}
	3 & 5 \\ 4 & 4 
\end{pmatrix}, \quad 
A_N = \begin{pmatrix}
	1 & 0 \\ 0 & 1
\end{pmatrix}, \quad 
x_B = \left( \frac{5}{2}, \frac{3}{2} \right), \quad 
x_N = ( 0, 0 )
$$

e la matrice $\tilde{A}$ è:
$$
\tilde{A} = A_B^{-1} A_N = A_B^{-1} = \begin{pmatrix}
	-\frac{1}{2} & \frac{5}{8} \\ \frac{1}{2} & -\frac{3}{8}
\end{pmatrix} 
$$

Notiamo che in $x_B$ entrambi i componenti sono non interi, ergo possiamo prendere i due casi:
\begin{itemize}
	\item $r=1$, si ha $\left\{ -\frac{1}{2} x_3 \right\} + \left\{ \frac{5}{8} \right\} \geq \left\{ \frac{5}{2} \right\}$. Applicando l'operatore parte frazionaria, e sostituendo a $x_3$ e $x_4$ le loro espressioni in funzione di $x_1$ e $x_2$ (ricavate dalle equazioni del poliedro) si ha:
		$$ 4x_1 + 5x_2 \leq 17 $$
	\item $r=2$, si ha con calcoli simili, lo stesso taglio:
		$$ 4x_1 + 5x_2 \leq 17 $$
\end{itemize}

Quindi entrambe le componenti danno lo stesso taglio (si noti che questa non è la norma).
Possiamo visualizzare sul grafico cosa significa il taglio introdotto:

\begin{center}

\begin{tikzpicture}
\begin{axis}[
    axis lines = middle,
		grid=both,
    xlabel = {$x_1$},
    ylabel = {$x_2$},
    xmin=0, xmax=6.9,
    ymin=0, ymax=3.9,
    domain=0:10,
    samples=100,
    width=15cm, height=9cm,
    legend pos=north east
  ]

% regione ammissibile

\addplot[fill=gray, opacity=0.4, forget plot] 
	coordinates {
		(0,0)
		(0,3)
		(2.5, 1.5)
		(4, 0)
	};

% rette

\addplot[domain=0:2.5, thick, red] {3 - 0.6*x};
\addlegendentry{$ 3 x_1 + 5 x_2 \leq 15 $}

\addplot[domain=2.5:4, thick, blue] {4 - x};
\addlegendentry{$ 4 x_1 + 4 x_2 \leq 16 $}

\addplot[domain=0:10, thick, green] {3.4 - 0.8*x};
\addlegendentry{$ 4 x_1 + 5 x_2 \leq 17 $}

\addplot[
    only marks,  % Only marks the points without connecting lines
    mark=*,
    color=violet,
    mark size=2pt
] coordinates {
	(0,0)
	(0,1)
	(0,2)
	(0,3)
	(1,0)
	(1,1)
	(1,2)
	(2,0)
	(2,1)
	(3,0)
	(3,1)
	(4,0)
};
\addlegendentry{$\Omega$}

\addplot[
	only marks,
	mark=*,
	mark size=2pt
] coordinates {
	(2.5, 1.5)
};

\node at (axis cs:2.5, 1.5) [above right] {$x_{V_S}$};

\addplot[
	only marks,
	mark=*,
	mark size=2pt
] coordinates {
	(3, 1)
};

\node at (axis cs:3, 1) [above right] {$x_{\mathrm{conv}(\Omega)}$};

\end{axis}
\end{tikzpicture}

\end{center}

Notiamo che il taglio, se non ha ridotto completamente il poliedro a $\mathrm{conv}(\Omega)$, ha portato la soluzione ottima a quella di $\mathrm{conv}(\Omega)$.

\end{document}


\documentclass[a4paper,11pt]{article}
\usepackage[a4paper, margin=8em]{geometry}

% usa i pacchetti per la scrittura in italiano
\usepackage[french,italian]{babel}
\usepackage[T1]{fontenc}
\usepackage[utf8]{inputenc}
\frenchspacing 

% usa i pacchetti per la formattazione matematica
\usepackage{amsmath, amssymb, amsthm, amsfonts}

% usa altri pacchetti
\usepackage{gensymb}
\usepackage{hyperref}
\usepackage{standalone}

% imposta il titolo
\title{Appunti Ricerca Operativa}
\author{Luca Seggiani}
\date{2024}

% disegni
\usepackage{pgfplots}
\pgfplotsset{width=10cm,compat=1.9}

% imposta lo stile
% usa helvetica
\usepackage[scaled]{helvet}
% usa palatino
\usepackage{palatino}
% usa un font monospazio guardabile
\usepackage{lmodern}

\renewcommand{\rmdefault}{ppl}
\renewcommand{\sfdefault}{phv}
\renewcommand{\ttdefault}{lmtt}

% disponi il titolo
\makeatletter
\renewcommand{\maketitle} {
	\begin{center} 
		\begin{minipage}[t]{.8\textwidth}
			\textsf{\huge\bfseries \@title} 
		\end{minipage}%
		\begin{minipage}[t]{.2\textwidth}
			\raggedleft \vspace{-1.65em}
			\textsf{\small \@author} \vfill
			\textsf{\small \@date}
		\end{minipage}
		\par
	\end{center}

	\thispagestyle{empty}
	\pagestyle{fancy}
}
\makeatother

% disponi teoremi
\usepackage{tcolorbox}
\newtcolorbox[auto counter, number within=section]{theorem}[2][]{%
	colback=blue!10, 
	colframe=blue!40!black, 
	sharp corners=northwest,
	fonttitle=\sffamily\bfseries, 
	title=Teorema~\thetcbcounter: #2, 
	#1
}

% disponi definizioni
\newtcolorbox[auto counter, number within=section]{definition}[2][]{%
	colback=red!10,
	colframe=red!40!black,
	sharp corners=northwest,
	fonttitle=\sffamily\bfseries,
	title=Definizione~\thetcbcounter: #2,
	#1
}

% disponi problemi
\newtcolorbox[auto counter, number within=section]{problem}[2][]{%
	colback=green!10,
	colframe=green!40!black,
	sharp corners=northwest,
	fonttitle=\sffamily\bfseries,
	title=Problema~\thetcbcounter: #2,
	#1
}

% disponi codice
\usepackage{listings}
\usepackage[table]{xcolor}

\lstdefinestyle{codestyle}{
		backgroundcolor=\color{black!5}, 
		commentstyle=\color{codegreen},
		keywordstyle=\bfseries\color{magenta},
		numberstyle=\sffamily\tiny\color{black!60},
		stringstyle=\color{green!50!black},
		basicstyle=\ttfamily\footnotesize,
		breakatwhitespace=false,         
		breaklines=true,                 
		captionpos=b,                    
		keepspaces=true,                 
		numbers=left,                    
		numbersep=5pt,                  
		showspaces=false,                
		showstringspaces=false,
		showtabs=false,                  
		tabsize=2
}

\lstdefinestyle{shellstyle}{
		backgroundcolor=\color{black!5}, 
		basicstyle=\ttfamily\footnotesize\color{black}, 
		commentstyle=\color{black}, 
		keywordstyle=\color{black},
		numberstyle=\color{black!5},
		stringstyle=\color{black}, 
		showspaces=false,
		showstringspaces=false, 
		showtabs=false, 
		tabsize=2, 
		numbers=none, 
		breaklines=true
}

\lstdefinelanguage{javascript}{
	keywords={typeof, new, true, false, catch, function, return, null, catch, switch, var, if, in, while, do, else, case, break},
	keywordstyle=\color{blue}\bfseries,
	ndkeywords={class, export, boolean, throw, implements, import, this},
	ndkeywordstyle=\color{darkgray}\bfseries,
	identifierstyle=\color{black},
	sensitive=false,
	comment=[l]{//},
	morecomment=[s]{/*}{*/},
	commentstyle=\color{purple}\ttfamily,
	stringstyle=\color{red}\ttfamily,
	morestring=[b]',
	morestring=[b]"
}

% disponi sezioni
\usepackage{titlesec}

\titleformat{\section}
	{\sffamily\Large\bfseries} 
	{\thesection}{1em}{} 
\titleformat{\subsection}
	{\sffamily\large\bfseries}   
	{\thesubsection}{1em}{} 
\titleformat{\subsubsection}
	{\sffamily\normalsize\bfseries} 
	{\thesubsubsection}{1em}{}

% disponi alberi
\usepackage{forest}

\forestset{
	rectstyle/.style={
		for tree={rectangle,draw,font=\large\sffamily}
	},
	roundstyle/.style={
		for tree={circle,draw,font=\large}
	}
}

% disponi algoritmi
\usepackage{algorithm}
\usepackage{algorithmic}
\makeatletter
\renewcommand{\ALG@name}{Algoritmo}
\makeatother

% disponi numeri di pagina
\usepackage{fancyhdr}
\fancyhf{} 
\fancyfoot[L]{\sffamily{\thepage}}

\makeatletter
\fancyhead[L]{\raisebox{1ex}[0pt][0pt]{\sffamily{\@title \ \@date}}} 
\fancyhead[R]{\raisebox{1ex}[0pt][0pt]{\sffamily{\@author}}}
\makeatother

\begin{document}

% sezione (data)
\section{Lezione del 21-10-24}

% stili pagina
\thispagestyle{empty}
\pagestyle{fancy}

% testo
\subsection{Applicazione dell'algoritmo di riduzione del gap}
Applichiamo il metodo dei piani di taglio di Gomory ad alcuni problemi di ILP.

\subsubsection{Zaino intero}
Prendiamo il problema:
\[
	\begin{cases}
		\max 30 x_1 + 36 x_2 + 27 x_3 + 20 x_4 + 24 x_5 + 22 x_6 \\ 
		13 x_1 + 16 x_2 + 14 x_3 + 15 x_4 +17 x_5 + 14 x_6 \leq 57 \\ 
		x \in \mathbb{Z}_+
	\end{cases}
\]
con il vettore dei rendimenti:
$$
r = \left( 2.3 , 2.25 , 1.93 , 1.33 , 1.41 , 1.57 \right)
$$

Saturiamo per ottenere una soluzione $x_{RC}$, e quindi una valutazione superiore:
$$
x_{RC} = \left( \frac{57}{13}, 0, 0, 0 ,0 , 0 \right), \quad v_S = 131  
$$
e saturiamo l'intero per avere una valutazione inferiore:
$$
x_{I} = \left( 4, 0, 0, 0, 0, 0 \right), \quad v_I = 120
$$

Adesso applichiamo Gomory:
\begin{enumerate}
	\item Si porta il rilassato continuo in formato duale standard:
	\[
		\begin{cases}		
		\max 30 x_1 + 36 x_2 + 27 x_3 + 20 x_4 + 24 x_5 + 22 x_6 \\ 
		13 x_1 + 16 x_2 + 14 x_3 + 15 x_4 +17 x_5 + 14 x_6 + x_7 = 57 \\ 
		x_i \geq 0
		\end{cases}
	\]
\item Si individua la base ottima:
	$$ x_{RC} = \left( \frac{57}{13}, 0, 0, 0, 0, 0 \right) \Rightarrow B= \{1\}$$
\item Si ricavano $A_B$, $A_N$, $x_B$, $x_N$ e $r$:
	$$
	A_B = 
	\begin{pmatrix}
		13
	\end{pmatrix}, \quad 
	A_N = 
	\begin{pmatrix}
		16 & 14 & 15 & 17 & 14 & 1
	\end{pmatrix}, \quad 
	x_B =
	\left(
		\frac{57}{13}
	\right), \quad 
	x_N = 
	\left(
		0 , 0 , 0 , 0 , 0
	\right)
	$$
	e chiaramente $r=1$;
\item Si ricava $\tilde{A}$:
	$$
	\tilde{A} = A_B^{-1} A_N = \left( \frac{16}{13}, \frac{14}{13}, \frac{15}{13}, \frac{17}{13}, \frac{14}{13}, \frac{1}{13} \right)
	$$
\item Si trova il nuovo vincolo:
$$
\left\{ \frac{16}{13} \right\} x_2 + \left\{ \frac{14}{13} \right\} x_3 + \left\{ \frac{15}{13} \right\} x_4 + \left\{ \frac{17}{13} \right\} x_5 + \left\{ \frac{14}{13} \right\} x_6 + \left\{ \frac{1}{13} \right\} x_7 \geq \left\{ \frac{57}{13} \right\}
$$
$$
\frac{3}{13} x_2 + \frac{1}{13} x_3 + \frac{2}{13} x_4 + \frac{4}{13} x_5 + \frac{1}{13} x_6 + \frac{1}{13} x_7 \geq \frac{5}{13}
$$
Che espresso sostituendo $x_7 = 57 - 13 x_1 - 16 x_2 -14 x_3 - 15 x_4 - 17 x_5 - 14 x_6$ e unito a quanto già trovato dà il problema:
$$
	\begin{cases}
		\max 30 x_1 + 36 x_2 + 27 x_3 + 20 x_4 + 24 x_5 + 22 x_6 \\ 
		13 x_1 + 16 x_2 + 14 x_3 + 15 x_4 +17 x_5 + 14 x_6 \leq 57 \\ 
		x_1 + x_2 + x_3 + x_4 + x_5 + x_ 6 \leq 4 \\
		x \in \mathbb{Z}_+
	\end{cases}
$$
da cui $x_{RC} = \left( \frac{7}{3}, \frac{5}{3}, 0, 0, 0, 0 \right)$, che è già più vicino all'ottimo $\bar{x} = \left( 3, 1, 0, 0, 0, 0 \right)$.
\end{enumerate}

\subsubsection{Zaino booleano}
Prendiamo a questo punto:
\[
	\begin{cases}
		\max 30 x_1 + 36 x_2 + 27 x_3 + 20 x_4 + 24 x_5 + 22 x_6 \\ 
		13 x_1 + 16 x_2 + 14 x_3 + 15 x_4 +17 x_5 + 15 x_6 \leq 57 \\ 
		x \in \mathbb{Z}_+ \\
		0 \leq x \leq 1
	\end{cases}
\]

Attraverso la saturazione, troviamo:
$$
x_{RC} = \left( 1, 1, 1, 0, 0, \frac{14}{15}, 0 \right), \quad v_S = 115.4
$$
di valutazione superiore, e 
$$x_I = (1, 1, 1, 0, 0, 0), \quad v_I = 93 $$
di valutazione inferiore.

Riapplichiamo Gomory.
\begin{enumerate}
	\item L'ottimo è noto:
$$
x_{RC} = \left( 1, 1, 1, 0, 0, \frac{14}{15}, 0 \right)
$$
	\item Convertiamo in formato duale standard, ricordando i vincoli $0 \leq x_i \leq 1$:
\[
	\begin{cases}	
		\max 30 x_1 + 36 x_2 + 27 x_3 + 20 x_4 + 24 x_5 + 22 x_6 \\ 
		13 x_1 + 16 x_2 + 14 x_3 + 15 x_4 +17 x_5 + 15 x_6 + x_7 = 57 \\ 
		x_1 + x_8 = 1 \\
		x_2 + x_9 = 1 \\
		
		x_3 + x_{10} = 1 \\
		x_4 + x_{11} = 1 \\
		x_5 + x_{12} = 1 \\
		x_6 + x_{13} = 1 \\
		x \geq 0
	\end{cases}
\]
Dobbiamo ricalcolare l'ottimo:
$$ 
x_{RC}' = \left( 1, 1, 1, 0, 0, \frac{14}{15}, 0, 0, 0, 0, 1, 1, \frac{1}{15} \right)
$$
\item Si individua la base ottima:
$$
x_{RC}' = \left( 1, 1, 1, 0, 0, \frac{14}{15}, 0, 0, 0, 0, 1, 1, \frac{1}{15} \right) \Rightarrow B = \{1, 2, 3, 6, 11, 12, 13\}
$$
\item Si ricavano $A_B$, $A_N$, $x_B$, $x_N$ e $r$.
	Conviene prima scrivere $A$:
	\setcounter{MaxMatrixCols}{200}
	$$
	A = 
	\begin{pmatrix}
		13 & 16 & 14 & 15 & 17 & 15 & 1 & 0 & 0 & 0 & 0 & 0 & 0 \\
		1 & 0 & 0 & 0 & 0 & 0 & 0 & 1 & 0 & 0 & 0 & 0 & 0 \\
		0 & 1 & 0 & 0 & 0 & 0 & 0 & 0 & 1 & 0 & 0 & 0 & 0 \\
		0 & 0 & 1 & 0 & 0 & 0 & 0 & 0 & 0 & 1 & 0 & 0 & 0 \\
		0 & 0 & 0 & 1 & 0 & 0 & 0 & 0 & 0 & 0 & 1 & 0 & 0 \\
		0 & 0 & 0 & 0 & 1 & 0 & 0 & 0 & 0 & 0 & 0 & 1 & 0 \\
		0 & 0 & 0 & 0 & 0 & 1 & 0 & 0 & 0 & 0 & 0 & 0 & 1 \\
	\end{pmatrix}
	$$
	e quindi:
	$$
	A_B = 
	\begin{pmatrix}
		13 & 16 & 14 & 15 & 0 & 0 & 0 \\
		1 & 0 & 0 & 0 & 0 & 0 & 0 \\
		0 & 1 & 0 & 0 & 0 & 0 & 0 \\
		0 & 0 & 1 & 0 & 0 & 0 & 0 \\
		0 & 0 & 0 & 0 & 1 & 0 & 0 \\
		0 & 0 & 0 & 0 & 0 & 1 & 0 \\
		0 & 0 & 0 & 1 & 0 & 0 & 1 \\
	\end{pmatrix}, \quad 
	A_N = 
	\begin{pmatrix}
		15 & 17 & 1 & 0 & 0 & 0 \\ 
		0 & 0 & 0 & 1 & 0 & 0 \\ 
		0 & 0 & 0 & 0 & 1 & 0 \\ 
		0 & 0 & 0 & 0 & 0 & 1 \\ 
		1 & 0 & 0 & 0 & 0 & 0 \\ 
		0 & 1 & 0 & 0 & 0 & 0 \\
		0 & 0 & 0 & 0 & 0 & 0
	\end{pmatrix}
	$$
	$$
	x_B =
	\left(
		1, 1, 1, \frac{14}{15}, 1, 1, \frac{1}{15}
	\right), \quad 
	x_N = 
	\left(
		0, 0, 0, 0, 0, 0
	\right)
	$$
	e $r= 4, 7$;
\item Si ricava $\tilde{A}$:
	$$
	\tilde{A} = A_B^{-1} A_N = 
	\begin{pmatrix}
		0 & 0 & 0 & 1 & 0 & 0 \\ 
		0 & 0 & 0 & 0 & 1 & 0 \\ 
		0 & 0 & 0 & 0 & 0 & 1 \\ 
		1 & \frac{17}{15} & \frac{1}{15} & -\frac{13}{15} & -\frac{16}{15} & -\frac{14}{15} \\ 
		1 & 0 & 0 & 0 & 0 & 0 \\ 
		0 & 1 & 0 & 0 & 0 & 0 \\ 
		-1 & -\frac{17}{15} & -\frac{1}{15} & \frac{13}{15} & \frac{16}{15} & \frac{14}{15}
	\end{pmatrix} 
	$$
\item Si trovano due nuovi vincoli, notando che $N = \{ 4, 5, 7, 8, 9, 10 \}$:
$$
r=4 \Rightarrow x_1 + 2x_2 + x_3 + x_4 + x_5 + x_6 \leq 4 
$$
$$
r=7 \Rightarrow 13 x_1 + 15 x_2 + 14 x_3 + 14 x_4 + 15 x_5 + 14 x_6 \leq 55
$$
\end{enumerate}

# da qui in poi insicuro

\subsubsection{Produzione}
Supponiamo di avere il problema:
\[
	\begin{cases}
		\max 5 x_1 + 14 x_2 \\ 
		16 x_1 + 13 x_2 \geq 62 \\ 
		6 x_1 + 15 x_2 \geq 52 \\
		x \in \mathbb{Z}^2_+
	\end{cases}
\]
con valutazioni:
$$ x_{RC} = \left( \frac{26}{3}, 0 \right), \quad v_S = 43 $$
$$ x_I = \left( 8, 0 \right), \quad  v_I = 40 $$

Si applica Gomory:
\begin{enumerate}
	\item Convertiamo in formato duale standard: 
\[
	\begin{cases}	
		\max 5 x_1 + 14 x_2 \\ 
		16 x_1 + 13 x_2 - x_3 = 62 \\ 
		5 x_1 + 15 x_2 - x_4 = 52 \\ 
		x \geq 0 
	\end{cases}
\]
Dobbiamo ricalcolare l'ottimo:
$$ 
x_{RC}' = \left( \frac{26}{3}, 0, \frac{230}{3}, 0 \right)
$$
\item Si individua la base ottima:
$$
x_{RC}' = \left( \frac{26}{3}, 0, \frac{230}{3}, 0 \right) \Rightarrow B = \{1, 3 \}
$$
\item Si ricavano $A_B$, $A_N$, $x_B$, $x_N$ e $r$.
	$$
	A_B = 
	\begin{pmatrix}
		16 & -1  \\ 
		5 & 0 
	\end{pmatrix}, \quad 
	A_N = 
	\begin{pmatrix}
		13 & 0 \\ 
		15 & -1
	\end{pmatrix}, \quad
	x_B =
	\left(
		\frac{26}{3}, \frac{230}{3}
	\right), \quad 
	x_N = 
	\left(
		0, 0
	\right)
	$$
	e $r=1,3$;
\item Si ricava $\tilde{A}$:
	$$
	\tilde{A} = A_B^-1 A_N = \begin{pmatrix}
	\end{pmatrix} 
	$$
\item Si trova il nuovo vincolo:
	$$5 x_1 + 13 x_2 \geq 44$$ 
\end{enumerate}

\end{document}


\documentclass[a4paper,11pt]{article}
\usepackage[a4paper, margin=8em]{geometry}

% usa i pacchetti per la scrittura in italiano
\usepackage[french,italian]{babel}
\usepackage[T1]{fontenc}
\usepackage[utf8]{inputenc}
\frenchspacing 

% usa i pacchetti per la formattazione matematica
\usepackage{amsmath, amssymb, amsthm, amsfonts}

% usa altri pacchetti
\usepackage{gensymb}
\usepackage{hyperref}
\usepackage{standalone}

% imposta il titolo
\title{Appunti Ricerca Operativa}
\author{Luca Seggiani}
\date{2024}

% disegni
\usepackage{pgfplots}
\pgfplotsset{width=10cm,compat=1.9}

% imposta lo stile
% usa helvetica
\usepackage[scaled]{helvet}
% usa palatino
\usepackage{palatino}
% usa un font monospazio guardabile
\usepackage{lmodern}

\renewcommand{\rmdefault}{ppl}
\renewcommand{\sfdefault}{phv}
\renewcommand{\ttdefault}{lmtt}

% disponi il titolo
\makeatletter
\renewcommand{\maketitle} {
	\begin{center} 
		\begin{minipage}[t]{.8\textwidth}
			\textsf{\huge\bfseries \@title} 
		\end{minipage}%
		\begin{minipage}[t]{.2\textwidth}
			\raggedleft \vspace{-1.65em}
			\textsf{\small \@author} \vfill
			\textsf{\small \@date}
		\end{minipage}
		\par
	\end{center}

	\thispagestyle{empty}
	\pagestyle{fancy}
}
\makeatother

% disponi teoremi
\usepackage{tcolorbox}
\newtcolorbox[auto counter, number within=section]{theorem}[2][]{%
	colback=blue!10, 
	colframe=blue!40!black, 
	sharp corners=northwest,
	fonttitle=\sffamily\bfseries, 
	title=Teorema~\thetcbcounter: #2, 
	#1
}

% disponi definizioni
\newtcolorbox[auto counter, number within=section]{definition}[2][]{%
	colback=red!10,
	colframe=red!40!black,
	sharp corners=northwest,
	fonttitle=\sffamily\bfseries,
	title=Definizione~\thetcbcounter: #2,
	#1
}

% disponi problemi
\newtcolorbox[auto counter, number within=section]{problem}[2][]{%
	colback=green!10,
	colframe=green!40!black,
	sharp corners=northwest,
	fonttitle=\sffamily\bfseries,
	title=Problema~\thetcbcounter: #2,
	#1
}

% disponi codice
\usepackage{listings}
\usepackage[table]{xcolor}

\lstdefinestyle{codestyle}{
		backgroundcolor=\color{black!5}, 
		commentstyle=\color{codegreen},
		keywordstyle=\bfseries\color{magenta},
		numberstyle=\sffamily\tiny\color{black!60},
		stringstyle=\color{green!50!black},
		basicstyle=\ttfamily\footnotesize,
		breakatwhitespace=false,         
		breaklines=true,                 
		captionpos=b,                    
		keepspaces=true,                 
		numbers=left,                    
		numbersep=5pt,                  
		showspaces=false,                
		showstringspaces=false,
		showtabs=false,                  
		tabsize=2
}

\lstdefinestyle{shellstyle}{
		backgroundcolor=\color{black!5}, 
		basicstyle=\ttfamily\footnotesize\color{black}, 
		commentstyle=\color{black}, 
		keywordstyle=\color{black},
		numberstyle=\color{black!5},
		stringstyle=\color{black}, 
		showspaces=false,
		showstringspaces=false, 
		showtabs=false, 
		tabsize=2, 
		numbers=none, 
		breaklines=true
}

\lstdefinelanguage{javascript}{
	keywords={typeof, new, true, false, catch, function, return, null, catch, switch, var, if, in, while, do, else, case, break},
	keywordstyle=\color{blue}\bfseries,
	ndkeywords={class, export, boolean, throw, implements, import, this},
	ndkeywordstyle=\color{darkgray}\bfseries,
	identifierstyle=\color{black},
	sensitive=false,
	comment=[l]{//},
	morecomment=[s]{/*}{*/},
	commentstyle=\color{purple}\ttfamily,
	stringstyle=\color{red}\ttfamily,
	morestring=[b]',
	morestring=[b]"
}

% disponi sezioni
\usepackage{titlesec}

\titleformat{\section}
	{\sffamily\Large\bfseries} 
	{\thesection}{1em}{} 
\titleformat{\subsection}
	{\sffamily\large\bfseries}   
	{\thesubsection}{1em}{} 
\titleformat{\subsubsection}
	{\sffamily\normalsize\bfseries} 
	{\thesubsubsection}{1em}{}

% disponi alberi
\usepackage{forest}

\forestset{
	rectstyle/.style={
		for tree={rectangle,draw,font=\large\sffamily}
	},
	roundstyle/.style={
		for tree={circle,draw,font=\large}
	}
}

% disponi algoritmi
\usepackage{algorithm}
\usepackage{algorithmic}
\makeatletter
\renewcommand{\ALG@name}{Algoritmo}
\makeatother

% disponi numeri di pagina
\usepackage{fancyhdr}
\fancyhf{} 
\fancyfoot[L]{\sffamily{\thepage}}

\makeatletter
\fancyhead[L]{\raisebox{1ex}[0pt][0pt]{\sffamily{\@title \ \@date}}} 
\fancyhead[R]{\raisebox{1ex}[0pt][0pt]{\sffamily{\@author}}}
\makeatother

\begin{document}

% sezione (data)
\section{Lezione del 22-10-24}

% stili pagina
\thispagestyle{empty}
\pagestyle{fancy}

% testo
\subsection{Problema del commesso viaggiatore}
\begin{problem}{del commesso viaggiatore}
Supponiamo che un commesso viaggiatore debba fare il giro di 5 città, passando da tutte una e una sola volta.
Si prepara una tabella con le distanze fra le città:
	
\center \rowcolors{2}{green!10}{green!40!black!20}
	\begin{tabular} { c || c | c | c | c | c }
		& \bfseries Roccalbegna & \bfseries Cana & \bfseries Vallerona & \bfseries Santa Caterina & \bfseries Triana \\ 
		\hline 
		\bfseries Roccalbegna & - & 18 & 14 & 17 & 19 \\ 
		\bfseries Cana & 16 & - & 19 & 22 & 23 \\ 
		\bfseries Vallerona & 17 & 14 & - & 18 & 20 \\ 
		\bfseries Santa Caterina & 16 & 19 & 22 & - & 21 \\ 
		\bfseries Triana & 15 & 14 & 13 & 20 & -
	\end{tabular}

\par\bigskip
\raggedright
Quale percorso dovrà seguire il commerciante, in modo da minimizzare la distanza percorsa?
\end{problem}

Il problema del commesso viaggiatore (in inglese \textit{Traveling Salesman Problem}, TSP) è effettivamente quello di trovare un \textbf{ciclo hamiltioniano}, cioè ciclo su un grafo che passa da ogni nodo una e una sola volta.
Possiamo innanzitutto porre la matrice di adiacenza:

$$
C=
\begin{pmatrix}
	- & 18 & 14 & 17 & 19 \\ 
	16 & - & 19 & 22 & 23 \\ 
	17 & 14 & - & 18 & 20 \\ 
	16 & 19 & 22 & - & 21 \\ 
	15 & 14 & 13 & 20 & -
\end{pmatrix}
$$

Notiamo che si può distinguere fra TSP \textbf{simmetrici} e \textbf{asimmetrici}, in base alla simmetria della matrice di adiacenza.
Si ha che i due tipi di problema hanno algoritmi risolutivi molto diversi.
In questo caso, come è chiaro dalla matrice, considereremo il caso \textbf{asimmetrico}.

Inoltre, senza togliere dalla generalità della trattazione, possiamo assumere tutte le connessioni come stabilite, quindi la matrice \textbf{completa}. In questo caso gli archi mancanti del grafo saranno rappresentati da un costo infinito nella matrice di adiacenza (compresi gli archi $x_{ij} \rightarrow x_{ij}$, cioè da un nodo allo stesso nodo).

Abbiamo che tutti i percorsi possibili, per $n$ città partendo dalla prima, sono $(n-1)!$, e che il problema si dimostra NP-completo. 
Diventa inapplicabile un'algoritmo di enumerazione completa: dobbiamo trovare quindi un modello matematico su cui applicare i macchinari della PL.
Chiamiamo $C$ la matrice dei costi, $a_{ij}$ l'arco generico, $C_H$ il ciclo hamiltioniano e introduciamo la variabile binaria:
$$
x_{ij} =
	\begin{cases}
		0, \quad a_{ij} \notin C_H \\
		1, \quad a_{ij} \in C_H
	\end{cases}
$$
che rappresenterà matrice di adiacenza rappresentante il percorso scelto.
Come sempre, linearizzamo le matrici in vettori ordinati lessicograficamente, quindi:

$$ C = \begin{pmatrix}
	x_{11} & x_{12} & ... & x_{1N} \\ 
	x_{21} & x_{22} & ... & x_{2N} \\ 
	... \\ 
	x_{N1} & x_{N2} & ... & x_{NN}
\end{pmatrix} \rightarrow c = (x_{11}, x_{12}, ..., x_{1N}, x_{21}, x_{22}, ..., x_{2N}, ..., x_{N1}, x_{N2}, ..., x_{NN}) $$

A questo punto possiamo impostare il problema di ILP:
\[
	\begin{cases}
		\min c^T x \\ 
		x_{11} + x_{12} + x_{13} + x_{14} + x_{15} = 1 \\ 
		x_{21} + x_{22} + x_{23} + x_{24} + x_{25} = 1 \\ 
		... \\ 
		x_{51} + x_{52} + x_{53} + x_{54} + x_{55} = 1 \\ 
		x_{11} + x_{21} + x_{31} + x_{41} + x_{51} = 1 \\ 
		x_{12} + x_{22} + x_{32} + x_{42} + x_{52} = 1 \\ 
		.... \\
		x_{15} + x_{25} + x_{35} + x_{45} + x_{55} = 1 \\ 
		x \in \{ 0, 1 \}
	\end{cases}
\]

I primi $n=5$ vincoli impongono che da ogni nodo esca uno e uno solo arco, mentre i seguenti $n=5$ impongono che su ogni nodo arrivi uno e un solo arco.
L'ultimo vincolo rappresenta il dominio booleano delle variabili (un arco o è incluso o non è incluso nel ciclo hamiltoniano).

Notiamo che questo non basta a rappresentare cicli hamiltoniani: sono infatti ammessi sottocicli disgiunti fra di loro. 
Aggiungiamo quindi i vincoli, detti \textbf{di connessione}, dato $S \subset N$ sottoinsieme qualsiasi dei nodi:
$$
\sum_{i \in S \, j \notin S} x_{ij} \geq 1, \quad \forall S \subset N, \quad S \neq \emptyset
$$

Questi vincoli rappresentano l'obbligo per ogni sottoinsieme $S$ non vuoto di nodi di avere almeno un'arco uscente, così da evitare cicli isolati.
Si prende sottoinsieme stretto in quanto sarebbe inutile chiedere un'arco uscente dall'insieme completo dei nodi.
Notiamo inoltre che con $|S| = 1$, questo dà i primi vincoli del problema già posto (in forma $x_{11} + x_{12} + x_{13} ...$), che risulta in quanto sarebbero gli archi uscenti da gruppi di cardinalità 1, cioè singoli nodi, e il singolo arco uscente da ogni singolo nodo è già una prerogativa dell'assegnamento.

\par\smallskip

Possiamo quindi dire, in modo più completo, che si prendono i vincoli:
\[
	\begin{cases}
		\sum\limits_{j \in N, \, j \neq i} x_{ij} = 1, \quad \forall i \in N \\		
		\sum\limits_{i \in N, \, i \neq j} x_{ij} = 1, \quad \forall j \in N \\		
		\sum\limits_{i \in S, \, j \notin S} x_{ij} \geq 1, \quad \forall S \subset N, \quad 2 \leq |S| \leq n - 1
	\end{cases}
\]

\subsubsection{Cardinalità dei vincoli}

Si ha che i vincoli iniziali erano $2n$, $n$ per gli archi entranti e $n$ per gli archi uscenti. 
Inoltre, posto $2 \leq |S| \leq n - 1$, abbiamo che il numero di vincoli di connessione è:
$$
|\mathcal{V}_{connessione}| = 2^n - n - 2
$$ 

Ergo si ha un numero di vincoli pari a:
$$
|\mathcal{V}| = 2^n - n - 2 + 2n = 2^n + n - 2
$$

Si ha che i vincoli crescono quindi in maniera esponenziale.
Ad esempio, ecco l'insieme completo dei vincoli preso $n=5$, come dall'esempio precedente, generati attraverso un programma al computer:
$$
\begin{cases}
	x_{1 \, 2} + x_{1 \, 3} + x_{1 \, 4} + x_{1 \, 5} = 1 \\
	x_{2 \, 1} + x_{2 \, 3} + x_{2 \, 4} + x_{2 \, 5} = 1 \\
	x_{3 \, 1} + x_{3 \, 2} + x_{3 \, 4} + x_{3 \, 5} = 1 \\
	x_{4 \, 1} + x_{4 \, 2} + x_{4 \, 3} + x_{4 \, 5} = 1 \\
	x_{5 \, 1} + x_{5 \, 2} + x_{5 \, 3} + x_{5 \, 4} = 1 \\
	x_{2 \, 1} + x_{3 \, 1} + x_{4 \, 1} + x_{5 \, 1} = 1 \\
	x_{1 \, 2} + x_{3 \, 2} + x_{4 \, 2} + x_{5 \, 2} = 1 \\
	x_{1 \, 3} + x_{2 \, 3} + x_{4 \, 3} + x_{5 \, 3} = 1 \\
	x_{1 \, 4} + x_{2 \, 4} + x_{3 \, 4} + x_{5 \, 4} = 1 \\
	x_{1 \, 5} + x_{2 \, 5} + x_{3 \, 5} + x_{4 \, 5} = 1 \\
	x_{1 \, 2} + x_{3 \, 2} + x_{4 \, 2} + x_{5 \, 2} \geq 1 \\
	x_{1 \, 3} + x_{2 \, 3} + x_{4 \, 3} + x_{5 \, 3} \geq 1 \\
	x_{1 \, 4} + x_{2 \, 4} + x_{3 \, 4} + x_{5 \, 4} \geq 1 \\
	x_{1 \, 5} + x_{2 \, 5} + x_{3 \, 5} + x_{4 \, 5} \geq 1 \\
	x_{2 \, 1} + x_{3 \, 1} + x_{4 \, 1} + x_{5 \, 1} \geq 1 \\
	x_{1 \, 4} + x_{1 \, 5} + x_{2 \, 4} + x_{2 \, 5} + x_{3 \, 4} + x_{3 \, 5} \geq 1 \\
	x_{1 \, 3} + x_{1 \, 5} + x_{2 \, 3} + x_{2 \, 5} + x_{4 \, 3} + x_{4 \, 5} \geq 1 \\
	x_{1 \, 3} + x_{1 \, 4} + x_{2 \, 3} + x_{2 \, 4} + x_{5 \, 3} + x_{5 \, 4} \geq 1 \\
	x_{1 \, 3} + x_{1 \, 4} + x_{1 \, 5} + x_{2 \, 3} + x_{2 \, 4} + x_{2 \, 5} \geq 1 \\
	x_{1 \, 2} + x_{1 \, 5} + x_{3 \, 2} + x_{3 \, 5} + x_{4 \, 2} + x_{4 \, 5} \geq 1 \\
	x_{1 \, 2} + x_{1 \, 4} + x_{3 \, 2} + x_{3 \, 4} + x_{5 \, 2} + x_{5 \, 4} \geq 1 \\
	x_{1 \, 2} + x_{1 \, 4} + x_{1 \, 5} + x_{3 \, 2} + x_{3 \, 4} + x_{3 \, 5} \geq 1 \\
	x_{1 \, 2} + x_{1 \, 3} + x_{4 \, 2} + x_{4 \, 3} + x_{5 \, 2} + x_{5 \, 3} \geq 1 \\
	x_{1 \, 2} + x_{1 \, 3} + x_{1 \, 5} + x_{4 \, 2} + x_{4 \, 3} + x_{4 \, 5} \geq 1 \\
	x_{1 \, 2} + x_{1 \, 3} + x_{1 \, 4} + x_{5 \, 2} + x_{5 \, 3} + x_{5 \, 4} \geq 1 \\
	x_{2 \, 1} + x_{2 \, 5} + x_{3 \, 1} + x_{3 \, 5} + x_{4 \, 1} + x_{4 \, 5} \geq 1 \\
	x_{2 \, 1} + x_{2 \, 4} + x_{3 \, 1} + x_{3 \, 4} + x_{5 \, 1} + x_{5 \, 4} \geq 1 \\
	x_{2 \, 1} + x_{2 \, 4} + x_{2 \, 5} + x_{3 \, 1} + x_{3 \, 4} + x_{3 \, 5} \geq 1 \\
	x_{2 \, 1} + x_{2 \, 3} + x_{4 \, 1} + x_{4 \, 3} + x_{5 \, 1} + x_{5 \, 3} \geq 1 \\
	x_{2 \, 1} + x_{2 \, 3} + x_{2 \, 5} + x_{4 \, 1} + x_{4 \, 3} + x_{4 \, 5} \geq 1 \\
	x_{2 \, 1} + x_{2 \, 3} + x_{2 \, 4} + x_{5 \, 1} + x_{5 \, 3} + x_{5 \, 4} \geq 1 \\
	x_{3 \, 1} + x_{3 \, 2} + x_{4 \, 1} + x_{4 \, 2} + x_{5 \, 1} + x_{5 \, 2} \geq 1 \\
	x_{3 \, 1} + x_{3 \, 2} + x_{3 \, 5} + x_{4 \, 1} + x_{4 \, 2} + x_{4 \, 5} \geq 1 \\
	x_{3 \, 1} + x_{3 \, 2} + x_{3 \, 4} + x_{5 \, 1} + x_{5 \, 2} + x_{5 \, 4} \geq 1 \\
	x_{4 \, 1} + x_{4 \, 2} + x_{4 \, 3} + x_{5 \, 1} + x_{5 \, 2} + x_{5 \, 3} \geq 1 \\
\end{cases}
$$

Dobbiamo poi considerare il vincolo di interezza $x \in \mathbb{Z}$, e nel caso si prenda il rilassato continuo (che vedremo in questo caso è il problema di assegnamento corrispondente), i $2n$ vincoli in forma $x_{ij} \leq 1$, $x_{ij} \geq 0$.

\subsubsection{Valutazioni inferiori e superiori}
Vediamo come trovare buone valutazioni inferiori e superiori delle soluzioni $v_{TSP}$ di problemi TSP asimmetrici.

\begin{itemize}
	\item \textbf{Valutazione inferiore:} per fare una valutazione inferiore $v_I$, si rimuovono i vincoli di connessione, trasformando il problema in un problema di assegnamento (infatti si indica anche $v_I = v_{ASS}$) e concedendo quindi cicli disgiunti.
		Questo approccio è solitamente molto efficace per i problemi TSP \textit{asimmetrici}, restituendo rapporti $\epsilon = \frac{v_{TSP} - v_{ASS}}{v_{ASS}}$ nell'ordine di poche frazioni di punto percentuale.
		Nel caso di TSP simmetrici, invece, il rilassamento restituisce errori fino al $60\%$, e non è quindi particolarmente indicato.
	\item \textbf{Valutazione superiore:} per fare una valutazione superiore $v_S$, invece, si usa il cosiddetto \textbf{algoritmo delle toppe}, o \textit{algoritmo di fusione dei cicli disgiunti} (ma tu vedi gli americani).
Questo algoritmo prevede di calcolare il $v_I$, quindi la soluzione che concede cicli disgiunti, e di selezionare da due di questi un arco ciascuno, $(i, j)$ e $(k, l)$.
Si eliminano quindi questi archi, incrociandoli, cioè rimuovendo $(i, j)$ e $(k, l)$ e introducendo $(i,l)$ e $(k, j)$.
Avremo variazione di $v_{ASS}$:
$$ 
v_S' = v_{ASS} - c_{ij} - c_{kl} + c_{il} + c_{kj}, \quad \Delta v = -c_{ij} - c_{kl} + c_{il} + c_{kj}
$$
Si ripete iterativamente il processo, riducendo di volta in volta il numero di vincoli, finche non si arriva ad un ciclo hamiltoniano.
L'algoritmo completo è quindi:

\begin{algorithm}
\caption{delle toppe}
\begin{algorithmic}
	\STATE \textbf{Input:} una $v_{ASS}$ soluzione dall'assegnamento ricavato da un TSP % input
	\STATE \textbf{Output:} una valutazione inferiore $v_I$ del TSP % output
	% body
	\STATE \textsf{ciclo:}
	\STATE Chiama $C = \{ C_1, ..., C_p \}$ l'insieme dei cicli (sottocicli) dati da $v_{ASS}$
	\STATE Per ogni coppia di cicli $(C_h, C_k)$ valuta l'incremento di costo $\Delta v$ dovuto alla fusione di $C_h$ e $C_k$ nel modo più conveniente possibile
	\STATE Effettua la fusione dei cicli $C_h$ e $C_k$ con $\Delta v = \min\limits_{C_i, \, C_j} \Delta v, \quad i, j \in [1, p], \quad i \neq j$
	\STATE $p \leftarrow p - 1$
	\IF{$p > 1$}
		\STATE Vai a \textsf{ciclo}
	\ENDIF
\end{algorithmic}
\end{algorithm}

\end{itemize}

\end{document}


\documentclass[a4paper,11pt]{article}
\usepackage[a4paper, margin=8em]{geometry}

% usa i pacchetti per la scrittura in italiano
\usepackage[french,italian]{babel}
\usepackage[T1]{fontenc}
\usepackage[utf8]{inputenc}
\frenchspacing 

% usa i pacchetti per la formattazione matematica
\usepackage{amsmath, amssymb, amsthm, amsfonts}

% usa altri pacchetti
\usepackage{gensymb}
\usepackage{hyperref}
\usepackage{standalone}

% imposta il titolo
\title{Appunti Ricerca Operativa}
\author{Luca Seggiani}
\date{2024}

% disegni
\usepackage{pgfplots}
\pgfplotsset{width=10cm,compat=1.9}

% imposta lo stile
% usa helvetica
\usepackage[scaled]{helvet}
% usa palatino
\usepackage{palatino}
% usa un font monospazio guardabile
\usepackage{lmodern}

\renewcommand{\rmdefault}{ppl}
\renewcommand{\sfdefault}{phv}
\renewcommand{\ttdefault}{lmtt}

% disponi il titolo
\makeatletter
\renewcommand{\maketitle} {
	\begin{center} 
		\begin{minipage}[t]{.8\textwidth}
			\textsf{\huge\bfseries \@title} 
		\end{minipage}%
		\begin{minipage}[t]{.2\textwidth}
			\raggedleft \vspace{-1.65em}
			\textsf{\small \@author} \vfill
			\textsf{\small \@date}
		\end{minipage}
		\par
	\end{center}

	\thispagestyle{empty}
	\pagestyle{fancy}
}
\makeatother

% disponi teoremi
\usepackage{tcolorbox}
\newtcolorbox[auto counter, number within=section]{theorem}[2][]{%
	colback=blue!10, 
	colframe=blue!40!black, 
	sharp corners=northwest,
	fonttitle=\sffamily\bfseries, 
	title=Teorema~\thetcbcounter: #2, 
	#1
}

% disponi definizioni
\newtcolorbox[auto counter, number within=section]{definition}[2][]{%
	colback=red!10,
	colframe=red!40!black,
	sharp corners=northwest,
	fonttitle=\sffamily\bfseries,
	title=Definizione~\thetcbcounter: #2,
	#1
}

% disponi problemi
\newtcolorbox[auto counter, number within=section]{problem}[2][]{%
	colback=green!10,
	colframe=green!40!black,
	sharp corners=northwest,
	fonttitle=\sffamily\bfseries,
	title=Problema~\thetcbcounter: #2,
	#1
}

% disponi codice
\usepackage{listings}
\usepackage[table]{xcolor}

\lstdefinestyle{codestyle}{
		backgroundcolor=\color{black!5}, 
		commentstyle=\color{codegreen},
		keywordstyle=\bfseries\color{magenta},
		numberstyle=\sffamily\tiny\color{black!60},
		stringstyle=\color{green!50!black},
		basicstyle=\ttfamily\footnotesize,
		breakatwhitespace=false,         
		breaklines=true,                 
		captionpos=b,                    
		keepspaces=true,                 
		numbers=left,                    
		numbersep=5pt,                  
		showspaces=false,                
		showstringspaces=false,
		showtabs=false,                  
		tabsize=2
}

\lstdefinestyle{shellstyle}{
		backgroundcolor=\color{black!5}, 
		basicstyle=\ttfamily\footnotesize\color{black}, 
		commentstyle=\color{black}, 
		keywordstyle=\color{black},
		numberstyle=\color{black!5},
		stringstyle=\color{black}, 
		showspaces=false,
		showstringspaces=false, 
		showtabs=false, 
		tabsize=2, 
		numbers=none, 
		breaklines=true
}

\lstdefinelanguage{javascript}{
	keywords={typeof, new, true, false, catch, function, return, null, catch, switch, var, if, in, while, do, else, case, break},
	keywordstyle=\color{blue}\bfseries,
	ndkeywords={class, export, boolean, throw, implements, import, this},
	ndkeywordstyle=\color{darkgray}\bfseries,
	identifierstyle=\color{black},
	sensitive=false,
	comment=[l]{//},
	morecomment=[s]{/*}{*/},
	commentstyle=\color{purple}\ttfamily,
	stringstyle=\color{red}\ttfamily,
	morestring=[b]',
	morestring=[b]"
}

% disponi sezioni
\usepackage{titlesec}

\titleformat{\section}
	{\sffamily\Large\bfseries} 
	{\thesection}{1em}{} 
\titleformat{\subsection}
	{\sffamily\large\bfseries}   
	{\thesubsection}{1em}{} 
\titleformat{\subsubsection}
	{\sffamily\normalsize\bfseries} 
	{\thesubsubsection}{1em}{}

% disponi alberi
\usepackage{forest}

\forestset{
	rectstyle/.style={
		for tree={rectangle,draw,font=\large\sffamily}
	},
	roundstyle/.style={
		for tree={circle,draw,font=\large}
	}
}

% disponi algoritmi
\usepackage{algorithm}
\usepackage{algorithmic}
\makeatletter
\renewcommand{\ALG@name}{Algoritmo}
\makeatother

% disponi numeri di pagina
\usepackage{fancyhdr}
\fancyhf{} 
\fancyfoot[L]{\sffamily{\thepage}}

\makeatletter
\fancyhead[L]{\raisebox{1ex}[0pt][0pt]{\sffamily{\@title \ \@date}}} 
\fancyhead[R]{\raisebox{1ex}[0pt][0pt]{\sffamily{\@author}}}
\makeatother

\begin{document}

% sezione (data)
\section{Lezione del 23-10-24}

% stili pagina
\thispagestyle{empty}
\pagestyle{fancy}

% testo
\subsection{Multizaino}
Abbiamo visto in precedente i cosiddetti problemi di \textbf{caricamento}, o di \textit{zaino}, cioè problemi in forma:
\[
	\begin{cases}
		\max \sum_{i=1}^n c_i x_i \\ 
		\sum_{i=1}^n a_i x_i \leq b \\ 
		x_i \in \{ 0, 1 \}
	\end{cases}
\]

dove gli $c_i$ definiscono i \textbf{valori} e $a_i$ i \textbf{pesi} di $n$ oggetti, in questo caso presi una volta sola.
Avevamo ammesso anche il caso con un numero arbitrario di oggetti:
\[
	\begin{cases}
		\max \sum_{i=1}^n c_i x_i \\ 
		\sum_{i=1}^n a_i x_i \leq b \\ 
		x_i \geq 0
	\end{cases}
\]

Entrambi questi tipi di problemi si risolvono con valutazioni e algoritmi euristici, tenendo conto del vettore dei \textbf{rendimenti} $r_i = \frac{c_i}{a_i}$.

Esiste una variante del problema dove abbiamo più di un vincolo di peso: questa si dice \textbf{multizaino}.
In forma binaria, si esprime come:
\[
	\begin{cases}	
		\max \sum_{i=1}^n c_i x_i \\ 
		\sum_{i=1}^n a_i x_i \leq b \\ 
		\sum_{i=1}^n p_i x_i \leq p \\
		... \\ 
		x_i \in \{ 0, 1 \}
	\end{cases}
\]

In forma non binaria, si sostituisce come prima il vincolo $x_i \in \{ 0, 1 \}$ con $x_i \geq 0$.

In generale, questo tipo di problema non è semplice da risolvere, anche attraverso metodi euristici.
In questo corso ci si limita a riportarne l'esistenza, e non vengono dati algoritmi risolutivi.

\subsection{Riassunto sugli algoritmi di valutazione}

Prendiamo in considerazione il poliedro generico di ILP:

\[
	\begin{cases}
		Ax \leq b \\ 
		x \in \mathbb{Z}^n
	\end{cases}
\]

Facciamo un riassunto sulle tecniche greedy per il ricavo di una soluzione ammissibile (che danno un limite inferiore $v_I$) e sul ricavo del rilassato (che dà un limite superiore $v_S$), con riferimento alla ILP.
Bisogna notare che un algoritmo greedy da vincolo inferiore su problemi di massimo, e superiore altrimenti.
Allo stesso modo, il rilassato dà vincolo superiore su problemi di massimo, e inferiore altrimenti.

\subsubsection{Algoritmi per valutazioni inferiori}
\begin{itemize}
	\item \textbf{Zaino:} abbiamo visto gli \textbf{algoritmi dei rendimenti} per il caricamento di problemi di zaino in ILP, cioè per uno zaino di vettore valore $v$ e peso $p$, con peso massimo $P$, si calcolano i rendimenti:
		$$
		r_i = \frac{v_i}{p_i}
		$$
		e si selezionano uno o più elementi di rendimento massimo (dipende anche dal tipo booleano o meno di problema), saturando fino a frazione se si è nel rilassato, e fino a intero in caso contrario;
	\item \textbf{Bin-packing:} per la ricerca di soluzioni ammissibili dei problemi di bin-packing abbiamo visto tre algoritmi di ricerca, \textbf{next-fit decreasing}, \textbf{first-fit decreasing} e \textbf{best-fit decreasing};
	\item \textbf{Algoritmo delle toppe:} l'algoritmo delle toppe per i problemi TSP asimmetrici rappresenta ancora un metodo euristico per il calcolo di cicli hamiltoniani a costo minimo. 
\end{itemize}

\subsubsection{Algoritmi per valutazioni superiori}
Abbiamo introdotto il concetto di rilassamento di vincoli, in modo da trovare un sovrainsieme della regione ammissibile di ILP da dove è più facile ricavare una soluzione ammissibile.
In particolare, la tecnica più immediata è quella di rimuovere il vincolo di interezza ($x \in \mathbb{Z}^n$) o booleano ($x \in \{0, 1\}^n$).


\subsection{TSP simmetrico}
Veniamo quindi ai problemi di TSP simmetrico, cioè dove la tabella dei costi è simmetrica, ovvero in forma:
$$
c =
\begin{pmatrix}
	27 & 23 & 24 & 26 \\ 
	- & 21 & 32 & 33 \\ 
	- & - & 41 & 42 \\ 
	- & - & - & 47
\end{pmatrix}
$$
dove notiamo si mettono in relazione i nodi da $1$ a $n-1$ sulle righe con i nodi da $2$ a $n$ sulle colonne (per cui la diagonale è riempita).

Siccome questo è un caso particolare del TSP, possiamo effettivamente usare tutti i metodi studiati per il TSP asimmetrico.
Vediamo però che ci sono delle semplificazioni particolari che possiamo fare sul problema.

Innanzitutto, visto che possiamo prendere gli archi come non orientati, le variabili possibili sono dimezzano.
Se nello scorso problema avevamo $5 \cdot 5 = 25$ variabili meno $5$ della diagonale, ergo $n^2 - n$, adesso ne abbiamo soltanto $\frac{n^2 - n}{2}$. 

Abbiamo poi che i vincoli si presentano in forma:
\[
	\begin{cases}
		\min c^T x \\ 
		\sum\limits_{h,i \in A} x_{hi} + \sum\limits_{i,k \in A} = 2, \quad \forall i \in N \\
		\sum\limits_{i \in S, \, j \notin S} x_{ij} + \sum\limits_{i \notin S, \, j \in S} x_{ij} \geq 1, \quad \forall S \subset N, \quad 2 \leq |S| \leq \left\lceil \frac{|N|}{2} \right\rceil
	\end{cases}
\]
La prima serie di vincoli rappresenta il fatto che ogni nodo è parte di al massimo due archi, ergo uno uscente e entrante (anche se questa definizione non ha molto significato per grafi non orientati). 
La seconda serie rappresenta invece i vincoli di connessione, che in questo caso prendiamo in entrambe le direzioni, e che limitiamo in dimensioni di sottoinsieme alla cardinalità $|N|$ fratto 2, al limite approssimata all'eccesso, in quanto il vincolo per $S$ vale anche per $N \setminus S$. 
Anche in questo caso si evitano gli insiemi singoletti con $|S| = 1$, in quanto rappresentano i vincoli già espressi nella seconda riga.

\subsubsection{Valutazioni inferiori e superiori}

Un'approccio greedy alla soluzione, che fornisce un vincolo superiore $v_S$ (siamo in minimo), è quello di scegliere un nodo ad arbitrio e proseguire da lì in poi scegliendo il nodo con $\min(c_{xj})$ di adiacenza, ovvero:

\begin{algorithm}[H]
\caption{del nodo vicino}
\begin{algorithmic}
	\STATE \textbf{Input:} un insieme $N$ di nodi % input
	\STATE \textbf{Output:} una valutazione superiore $v_S$ del TSP simmetrico % output
	% body
	\STATE Parti da un nodo ad arbitrio;
	\STATE \textsf{ciclo:}
	\STATE Scegli il nodo adiacente più vicino e unifica
	\IF{ci sono altri nodi}
	\STATE Torna a \textsf{ciclo}
	\ENDIF
\end{algorithmic}
\end{algorithm}

Per il calcolo di una valutazione inferiore avremo bisogno di un rilassamento.
Scegliamo di rilassare i vincoli sul grado di tutti i nodi tranne uno, detto $k$: 
\[
	\begin{cases}
		\min c^T x \\ 
		\sum\limits_{h,r \in A} x_{hi} + \sum\limits_{r,k \in A} = 2, \quad \forall i \in N \\
		\sum\limits_{i \in S, \, j \notin S} x_{ij} + \sum\limits_{i \notin S, \, j \in S} x_{ij} \geq 1, \quad \forall S \subset N \setminus \{r\}, \quad 2 \leq |S| \leq \left\lceil \frac{|N|}{2} \right\rceil
	\end{cases}
\]

Diamo quindi la seguente definizione:
\begin{definition}{K-albero}
	Scelto un nodo $k$, chiamiamo $k$-albero un insieme di $n$ archi dove $n-2$ archi formano un albero di copertura sul sottografo formato dai nodi $N \setminus \{k\}$, e 2 archi incidono sul nodo $k$.	
\end{definition}
In sostanza, un $k$-albero è un \textbf{albero di copertura} con un ciclo.
Bisogna notare che questa definizione di $k$-albero differisce da quella comunemente usata sia in informatica (albero con ramificazione $k$) che in teoria dei grafi (un altro tipo di grafo non diretto).
Inoltre, si trova anche sotto altri nomi, tra cui $r$-albero, che però è ancora conflitto con gli $r$-grafi informatici (che sono un tipo di albero per l'organizzazione di informazione spaziale).

Tolta quest'ultima precisazione, abbiamo che ciò che vogliamo è un $k$-albero minimo.
Ricordiamo quindi il seguente algoritmo:
\subsubsection{Algoritmo di Kruskal}
Possiamo usare l'\textbf{algoritmo di Kruskal} per trovare l'albero di copertura minimale per un insieme $N$ di nodi.
Assumendo grafi connessi, si può formulare come:

\begin{algorithm}[H]
\caption{di Kruskal}
\begin{algorithmic}
	\STATE \textbf{Input:} un insieme $N$ di nodi % input
	\STATE \textbf{Output:} un albero di copertura minimale di $N$% output
	% body
	\STATE \textsf{ciclo:}
	\STATE Ordina gli archi in maniera crescente rispetto al peso
	\STATE Scegli il primo arco. 
	\IF{aggiungere l'arco non crea cicli} 
		\STATE Aggiungi l'arco e unifica le componenti
	\ELSE		
		\STATE Scegli il prossimo arco e riprova
	\ENDIF
	\IF{non hai finito gli archi} 
		\STATE Vai a \textsf{ciclo}
	\ENDIF
\end{algorithmic}
\end{algorithm}
Questo algoritmo trova sempre una soluzione, cioè un albero di copertura minimale, e visto che sceglie sempre archi a costo minimo, trova sempre la soluzione ottima.
\par\medskip
Ritornando al problema dei $k$-alberi minimi, possiamo finalmente presentare il seguente algoritmo per il calcolo di un $k$-albero di costo minimo su qualsiasi nodo $k$:
\begin{algorithm}
\caption{del $k$-albero}
\begin{algorithmic}
	\STATE \textbf{Input:} un insieme $N$ nodi, di cui $k \in N$  % input
	\STATE \textbf{Output:} un $k$-albero di costo minimo % output
	% body
	\STATE Trova l'albero di copertura minimale $C_H$ con Kruskal del sottografo $N \setminus \{k\}$
	\STATE Connetti $k$ a $C_H$ attraverso i due nodi a costo minimo
\end{algorithmic}
\end{algorithm}

Possiamo quindi dire perchè abbiamo introdotto il concetto di $k$-albero: il problema rilassato posto prima, dove si erano rilassate le cardinalità degli archi su tutti i nodi tranne $k$, ha come soluzione il $k$-albero a costo minimo.
Trovando questo $k$-albero, abbiamo una valutazione superiore del problema di TSP simmetrico di partenza.

\end{document}


\documentclass[a4paper,11pt]{article}
\usepackage[a4paper, margin=8em]{geometry}

% usa i pacchetti per la scrittura in italiano
\usepackage[french,italian]{babel}
\usepackage[T1]{fontenc}
\usepackage[utf8]{inputenc}
\frenchspacing 

% usa i pacchetti per la formattazione matematica
\usepackage{amsmath, amssymb, amsthm, amsfonts}

% usa altri pacchetti
\usepackage{gensymb}
\usepackage{hyperref}
\usepackage{standalone}

% imposta il titolo
\title{Appunti Ricerca Operativa}
\author{Luca Seggiani}
\date{2024}

% disegni
\usepackage{pgfplots}
\pgfplotsset{width=10cm,compat=1.9}

% imposta lo stile
% usa helvetica
\usepackage[scaled]{helvet}
% usa palatino
\usepackage{palatino}
% usa un font monospazio guardabile
\usepackage{lmodern}

\renewcommand{\rmdefault}{ppl}
\renewcommand{\sfdefault}{phv}
\renewcommand{\ttdefault}{lmtt}

% disponi il titolo
\makeatletter
\renewcommand{\maketitle} {
	\begin{center} 
		\begin{minipage}[t]{.8\textwidth}
			\textsf{\huge\bfseries \@title} 
		\end{minipage}%
		\begin{minipage}[t]{.2\textwidth}
			\raggedleft \vspace{-1.65em}
			\textsf{\small \@author} \vfill
			\textsf{\small \@date}
		\end{minipage}
		\par
	\end{center}

	\thispagestyle{empty}
	\pagestyle{fancy}
}
\makeatother

% disponi teoremi
\usepackage{tcolorbox}
\newtcolorbox[auto counter, number within=section]{theorem}[2][]{%
	colback=blue!10, 
	colframe=blue!40!black, 
	sharp corners=northwest,
	fonttitle=\sffamily\bfseries, 
	title=Teorema~\thetcbcounter: #2, 
	#1
}

% disponi definizioni
\newtcolorbox[auto counter, number within=section]{definition}[2][]{%
	colback=red!10,
	colframe=red!40!black,
	sharp corners=northwest,
	fonttitle=\sffamily\bfseries,
	title=Definizione~\thetcbcounter: #2,
	#1
}

% disponi problemi
\newtcolorbox[auto counter, number within=section]{problem}[2][]{%
	colback=green!10,
	colframe=green!40!black,
	sharp corners=northwest,
	fonttitle=\sffamily\bfseries,
	title=Problema~\thetcbcounter: #2,
	#1
}

% disponi codice
\usepackage{listings}
\usepackage[table]{xcolor}

\lstdefinestyle{codestyle}{
		backgroundcolor=\color{black!5}, 
		commentstyle=\color{codegreen},
		keywordstyle=\bfseries\color{magenta},
		numberstyle=\sffamily\tiny\color{black!60},
		stringstyle=\color{green!50!black},
		basicstyle=\ttfamily\footnotesize,
		breakatwhitespace=false,         
		breaklines=true,                 
		captionpos=b,                    
		keepspaces=true,                 
		numbers=left,                    
		numbersep=5pt,                  
		showspaces=false,                
		showstringspaces=false,
		showtabs=false,                  
		tabsize=2
}

\lstdefinestyle{shellstyle}{
		backgroundcolor=\color{black!5}, 
		basicstyle=\ttfamily\footnotesize\color{black}, 
		commentstyle=\color{black}, 
		keywordstyle=\color{black},
		numberstyle=\color{black!5},
		stringstyle=\color{black}, 
		showspaces=false,
		showstringspaces=false, 
		showtabs=false, 
		tabsize=2, 
		numbers=none, 
		breaklines=true
}

\lstdefinelanguage{javascript}{
	keywords={typeof, new, true, false, catch, function, return, null, catch, switch, var, if, in, while, do, else, case, break},
	keywordstyle=\color{blue}\bfseries,
	ndkeywords={class, export, boolean, throw, implements, import, this},
	ndkeywordstyle=\color{darkgray}\bfseries,
	identifierstyle=\color{black},
	sensitive=false,
	comment=[l]{//},
	morecomment=[s]{/*}{*/},
	commentstyle=\color{purple}\ttfamily,
	stringstyle=\color{red}\ttfamily,
	morestring=[b]',
	morestring=[b]"
}

% disponi sezioni
\usepackage{titlesec}

\titleformat{\section}
	{\sffamily\Large\bfseries} 
	{\thesection}{1em}{} 
\titleformat{\subsection}
	{\sffamily\large\bfseries}   
	{\thesubsection}{1em}{} 
\titleformat{\subsubsection}
	{\sffamily\normalsize\bfseries} 
	{\thesubsubsection}{1em}{}

% disponi alberi
\usepackage{forest}

\forestset{
	rectstyle/.style={
		for tree={rectangle,draw,font=\large\sffamily}
	},
	roundstyle/.style={
		for tree={circle,draw,font=\large}
	}
}

% disponi algoritmi
\usepackage{algorithm}
\usepackage{algorithmic}
\makeatletter
\renewcommand{\ALG@name}{Algoritmo}
\makeatother

% disponi numeri di pagina
\usepackage{fancyhdr}
\fancyhf{} 
\fancyfoot[L]{\sffamily{\thepage}}

\makeatletter
\fancyhead[L]{\raisebox{1ex}[0pt][0pt]{\sffamily{\@title \ \@date}}} 
\fancyhead[R]{\raisebox{1ex}[0pt][0pt]{\sffamily{\@author}}}
\makeatother

\begin{document}

% sezione (data)
\section{Lezione del 24-10-24}

% stili pagina
\thispagestyle{empty}
\pagestyle{fancy}

% testo
\subsubsection{Riassunto TSP simmetrico}
Avevamo quindi posto problemi con vincoli  in forma:
\[
	\begin{cases}
			\min c^T \cdot x \\ 
			\sum\limits_{x < j} x_{ij} + \sum\limits_{j < y} x_{ij} = 2, \quad \forall j \\ 
			\sum\limits_{i \in S \, j \notin S} x_{ij} + \sum\limits_{i \notin S \, j \in S} x_{ij} \geq 1, \quad \forall S \subset N, \quad 2 \leq |S| \leq \left\lceil \frac{|N|}{2} \right\rceil  
	\end{cases}
\]

cioè dove si poneva la somma dei nodi entranti e uscenti (i vincoli \textit{di grado}) da $j$ come $=2$.
Visto che $j$ era l'unico nodo su cui era imposto il vincolo, si aveva che la soluzione del problema era il $j$-albero a costo minimo.

Inoltre, si aveva che i vincoli sulla terza erano quelli \textit{di connessione}, che si cercavano solo su cardinalità dei sottoinsiemi $= \left\lceil \frac{|N|}{2} \right\rceil $, in quanto i vincoli per $S$ valevano anche per $N \setminus S$.

\subsection{Branch and bound}
Avevamo visto l'algoritmo dei piani di taglio di Gomory per il calcolo di soluzioni approssimate di problemi ILP.
Presentiamo adesso un altro metodo, detto \textbf{branch and bound}.
Il metodo più naive che possiamo adottare per risolvere un problema in forma:
\[
	\begin{cases}
		\max c^T x \\ 
		A x\leq b \\ 
		x \in \{0, 1\}^n
	\end{cases}
\]
è quello di enumerare tutti le possibili soluzioni ammissibili $\in \{0,1\}^n$, costruendo il cosiddetto \textbf{albero di enumerazione}.
Scegliamo quindi una variabile, $x_1$, e costruiamo l'albero:

\begin{center}
	\begin{forest}
		[$x_1$, roundstyle 
			[$x_2$, name=1, label={$x_0 = 0$}
				[,phantom
				[$x_n$, name=1a 
						[{$\scriptstyle (0,..., 0,0)$}, rectstyle]
						[{$\scriptstyle (0,..., 0,1)$}, rectstyle]
					]
					[$x_n$, name=1b
						[{$\scriptstyle (0,,..., 1,0)$}, rectstyle]
						[{$\scriptstyle (0,..., 1,1)$}, rectstyle]
					]
				]
			]
			[$x_2$, name=2, label={$x_0 = 1$}
				[,phantom
					[$x_n$, name=2a
						[{$\scriptstyle (1,,..., 0,0)$}, rectstyle]
						[{$\scriptstyle (1,,..., 0,1)$}, rectstyle]
					]
					[$x_n$,name=2b
						[{$\scriptstyle (1,,..., 1,0)$}, rectstyle]
						[{$\scriptstyle (1,..., 1,1)$}, rectstyle]
					]
				]
			]
		]	
		\path(1)--node[sloped]{\Large\dots}(1a);
		\path(1)--node[sloped]{\Large\dots}(1b);
		\path(2)--node[sloped]{\Large\dots}(2a);
		\path(2)--node[sloped]{\Large\dots}(2b);
	\end{forest}
\end{center}

Abbiamo che il calcolo di ogni nodo dell'albero non è effettivamente necessario.
Chiamiamo quindi \textbf{problemi} $P_{ij}$ ogni nodo dell albero, con $i$ che seleziona il livello e $j$ il fratello, da sinstra verso destra.

Stabilita una valutazione superiore e inferiore di ogni problema, che chiameremo $v_S(P)$ e $v_I(P)$, abbiamo:
$$
v_I(P) \leq v(P) \leq v_S(P) 
$$

Possiamo usare queste valutazioni per stabilire \textbf{regole di taglio} che ci permettano di tagliare (si dice anche \textit{visitare implicitamente}) un'intero sottoalbero a partire da un certo nodo $P_{ij}$.
Queste regole di taglio assicurano, essenzialmente, che ogni sottoproblema $P_{{i+k}, j'}$ figlio di $P_{ij}$ non contiene l'ottimo, e quindi si può saltare.

Vediamo quindi le regole di taglio che possiamo adottare.
Mantendendo un'ottimo corrente $x$, abbiamo che calcolato un nuovo $P_{ij}$ per enumerazione:
\begin{enumerate}
	\item $P_{ij} = \emptyset$ significa che per un certo nodo $P_{ij}$ possiamo tagliare tutti i problemi che istanziano le $i$ variabili di $P_{ij}$, ergo tutti i suoi nodi figli;
	\item Calcolo di $v_S(P_{ij})$. Se $v_S(P_{ij}) < v_I(P)$ del problema, allora posso scartare il sottoalbero: non troverò soluzioni migliori scendendovi;
	\item $v_S(P_{ij}) > v_I(P)$, e l'$\bar{x}$ dove si ha tale $v_S$ è ammissibile per $P$, allora si prende $\bar{x}$ come nuovo $x$ e si fa una visita implicità di $P_{ij}$: questo perchè un suo sottoproblema non potrà darci di meglio (più scendiamo nell'albero, più stringiamo i vincoli, ergo $P_{i+k, j'}$ figlio di $P_{ij}$ ha $v_S(P_{i+k, j'}) \leq v_S(P_{ij})$).
\end{enumerate}

Ricordiamo che quanto detto finora vale su problemi \textbf{massimizzanti}: su problemi \textbf{minimizzanti} dovremo invertire l'ordine delle disugaglianze, e prendere vincoli superiori anziché inferiori e viceversa.

\end{document}


\documentclass[a4paper,11pt]{article}
\usepackage[a4paper, margin=8em]{geometry}

% usa i pacchetti per la scrittura in italiano
\usepackage[french,italian]{babel}
\usepackage[T1]{fontenc}
\usepackage[utf8]{inputenc}
\frenchspacing 

% usa i pacchetti per la formattazione matematica
\usepackage{amsmath, amssymb, amsthm, amsfonts}

% usa altri pacchetti
\usepackage{gensymb}
\usepackage{hyperref}
\usepackage{standalone}

% imposta il titolo
\title{Appunti Ricerca Operativa}
\author{Luca Seggiani}
\date{2024}

% disegni
\usepackage{pgfplots}
\pgfplotsset{width=10cm,compat=1.9}

% imposta lo stile
% usa helvetica
\usepackage[scaled]{helvet}
% usa palatino
\usepackage{palatino}
% usa un font monospazio guardabile
\usepackage{lmodern}

\renewcommand{\rmdefault}{ppl}
\renewcommand{\sfdefault}{phv}
\renewcommand{\ttdefault}{lmtt}

% disponi il titolo
\makeatletter
\renewcommand{\maketitle} {
	\begin{center} 
		\begin{minipage}[t]{.8\textwidth}
			\textsf{\huge\bfseries \@title} 
		\end{minipage}%
		\begin{minipage}[t]{.2\textwidth}
			\raggedleft \vspace{-1.65em}
			\textsf{\small \@author} \vfill
			\textsf{\small \@date}
		\end{minipage}
		\par
	\end{center}

	\thispagestyle{empty}
	\pagestyle{fancy}
}
\makeatother

% disponi teoremi
\usepackage{tcolorbox}
\newtcolorbox[auto counter, number within=section]{theorem}[2][]{%
	colback=blue!10, 
	colframe=blue!40!black, 
	sharp corners=northwest,
	fonttitle=\sffamily\bfseries, 
	title=Teorema~\thetcbcounter: #2, 
	#1
}

% disponi definizioni
\newtcolorbox[auto counter, number within=section]{definition}[2][]{%
	colback=red!10,
	colframe=red!40!black,
	sharp corners=northwest,
	fonttitle=\sffamily\bfseries,
	title=Definizione~\thetcbcounter: #2,
	#1
}

% disponi problemi
\newtcolorbox[auto counter, number within=section]{problem}[2][]{%
	colback=green!10,
	colframe=green!40!black,
	sharp corners=northwest,
	fonttitle=\sffamily\bfseries,
	title=Problema~\thetcbcounter: #2,
	#1
}

% disponi codice
\usepackage{listings}
\usepackage[table]{xcolor}

\lstdefinestyle{codestyle}{
		backgroundcolor=\color{black!5}, 
		commentstyle=\color{codegreen},
		keywordstyle=\bfseries\color{magenta},
		numberstyle=\sffamily\tiny\color{black!60},
		stringstyle=\color{green!50!black},
		basicstyle=\ttfamily\footnotesize,
		breakatwhitespace=false,         
		breaklines=true,                 
		captionpos=b,                    
		keepspaces=true,                 
		numbers=left,                    
		numbersep=5pt,                  
		showspaces=false,                
		showstringspaces=false,
		showtabs=false,                  
		tabsize=2
}

\lstdefinestyle{shellstyle}{
		backgroundcolor=\color{black!5}, 
		basicstyle=\ttfamily\footnotesize\color{black}, 
		commentstyle=\color{black}, 
		keywordstyle=\color{black},
		numberstyle=\color{black!5},
		stringstyle=\color{black}, 
		showspaces=false,
		showstringspaces=false, 
		showtabs=false, 
		tabsize=2, 
		numbers=none, 
		breaklines=true
}

\lstdefinelanguage{javascript}{
	keywords={typeof, new, true, false, catch, function, return, null, catch, switch, var, if, in, while, do, else, case, break},
	keywordstyle=\color{blue}\bfseries,
	ndkeywords={class, export, boolean, throw, implements, import, this},
	ndkeywordstyle=\color{darkgray}\bfseries,
	identifierstyle=\color{black},
	sensitive=false,
	comment=[l]{//},
	morecomment=[s]{/*}{*/},
	commentstyle=\color{purple}\ttfamily,
	stringstyle=\color{red}\ttfamily,
	morestring=[b]',
	morestring=[b]"
}

% disponi sezioni
\usepackage{titlesec}

\titleformat{\section}
	{\sffamily\Large\bfseries} 
	{\thesection}{1em}{} 
\titleformat{\subsection}
	{\sffamily\large\bfseries}   
	{\thesubsection}{1em}{} 
\titleformat{\subsubsection}
	{\sffamily\normalsize\bfseries} 
	{\thesubsubsection}{1em}{}

% disponi alberi
\usepackage{forest}

\forestset{
	rectstyle/.style={
		for tree={rectangle,draw,font=\large\sffamily}
	},
	roundstyle/.style={
		for tree={circle,draw,font=\large}
	}
}

% disponi algoritmi
\usepackage{algorithm}
\usepackage{algorithmic}
\makeatletter
\renewcommand{\ALG@name}{Algoritmo}
\makeatother

% disponi numeri di pagina
\usepackage{fancyhdr}
\fancyhf{} 
\fancyfoot[L]{\sffamily{\thepage}}

\makeatletter
\fancyhead[L]{\raisebox{1ex}[0pt][0pt]{\sffamily{\@title \ \@date}}} 
\fancyhead[R]{\raisebox{1ex}[0pt][0pt]{\sffamily{\@author}}}
\makeatother

\begin{document}

% sezione (data)
\section{Lezione del 29-10-24}

% stili pagina
\thispagestyle{empty}
\pagestyle{fancy}

% testo
\subsection{Regole di branch}
Vediamo quindi nel dettaglio le regole di ramificazione da usare nell'applicazione del branch and bound, su problemi di minimo e di massimo.
Consideriamo un problema TSP simmetrico minimizzante:
\subsubsection{Regole di branch minimizzanti}
Partiamo dal problema $P$, e istanziamo variabili per ottenere una serie di $P_{i,j}$, dove la $i$ rappresenta il livello di profondità nell'albero e $j$ il sottoproblema corrente a quel livello.

Ogni volta che si istanzia la prima cosa da controllare è l'ammissibilità delle soluzioni ottenute istanziando le variabili.
Se si ha infatti che se $P_{i,j} = \emptyset$, quel sottoalbero può essere tagliato.

Si controlla poi se vale $v_I(P_{i,j}) \geq v_S(P)$, cioè se il minimo possibile ottenuto dal sottoproblema $P_{i,j}$ è più piccolo della valutazione superiore del problema $P$. In questo caso si taglia il sottoalbero.

Si controlla infine se vale $v_I(P_{i,j}) < v_S(P)$ e questa $v_I$ è ammissibile per $P$. Se sì, allora si aggionra il $v_S$ corrente di $P$ a $v_I$.
Inoltre, si può applicare anche la regola precedente, e quindi tagliare il sottoalbero.

Riassumendo si hanno quindi le regole:
\begin{itemize}
	\item $P_{i,j} = \emptyset \implies $ taglio;
	\item $v_I(P_{i,j}) \geq v_S(P) \implies $ taglio;
	\item $v_I(P_{i,j}) < v_S(P)$ e $v_I \in P \implies $ $v_S(P) \leftarrow v_I(P_{i,j})$, taglio.
\end{itemize}

\par\smallskip 
Consideriamo poi un problema di zaino booleano massimizzante:
\subsubsection{Regole di branch massimizzanti}
In un problema massimizzante si segue un approccio simile.
Dal problema $P$ si istanzia successivamente per ottenere $P_{i,j}$ con le stesse caratteristiche di prima.

Istanziando, se si va a svuotare la regione ammissbilie, cioè si ottiene $P_{i,j} = \emptyset$, si scarta quel sottoalbero.

Si controlla poi se vale $v_S(P_{i, j}) \leq v_I (P)$, cioè se il massimo possibile ottenuto dal sottoproblema $P_{i,j}$ è più grande della valutazione inferiore del problema $P$. In questo caso, chiaramente, si taglia il sottoalbero.

Infine si controlla se vale $v_S(P_{i,j}) > v_I(P)$, e questa $v_S$ è ammissibile per $P$ (cioè, in questo caso, è a componenti intere). Se sì, allora si aggiorna il $v_I$ corrente di $P$ a $v_S$. 
Inoltre, si taglia il sottoalbero.

Anche qui, riassumendo, si hanno le regole:
\begin{itemize}
	\item $P_{i,j} = \emptyset \implies $ taglio;
	\item $v_S(P_{i,j}) \leq v_I(P) \implies $ taglio;
	\item $v_S(P_{i,j}) > v_I(P)$ e $v_S \in P \implies $ $v_I(P) \leftarrow v_S(P_{i,j})$, taglio.
\end{itemize}


\end{document}


\documentclass[a4paper,11pt]{article}
\usepackage[a4paper, margin=8em]{geometry}

% usa i pacchetti per la scrittura in italiano
\usepackage[french,italian]{babel}
\usepackage[T1]{fontenc}
\usepackage[utf8]{inputenc}
\frenchspacing 

% usa i pacchetti per la formattazione matematica
\usepackage{amsmath, amssymb, amsthm, amsfonts}

% usa altri pacchetti
\usepackage{gensymb}
\usepackage{hyperref}
\usepackage{standalone}

% imposta il titolo
\title{Appunti Ricerca Operativa}
\author{Luca Seggiani}
\date{2024}

% disegni
\usepackage{pgfplots}
\pgfplotsset{width=10cm,compat=1.9}

% imposta lo stile
% usa helvetica
\usepackage[scaled]{helvet}
% usa palatino
\usepackage{palatino}
% usa un font monospazio guardabile
\usepackage{lmodern}

\renewcommand{\rmdefault}{ppl}
\renewcommand{\sfdefault}{phv}
\renewcommand{\ttdefault}{lmtt}

% disponi il titolo
\makeatletter
\renewcommand{\maketitle} {
	\begin{center} 
		\begin{minipage}[t]{.8\textwidth}
			\textsf{\huge\bfseries \@title} 
		\end{minipage}%
		\begin{minipage}[t]{.2\textwidth}
			\raggedleft \vspace{-1.65em}
			\textsf{\small \@author} \vfill
			\textsf{\small \@date}
		\end{minipage}
		\par
	\end{center}

	\thispagestyle{empty}
	\pagestyle{fancy}
}
\makeatother

% disponi teoremi
\usepackage{tcolorbox}
\newtcolorbox[auto counter, number within=section]{theorem}[2][]{%
	colback=blue!10, 
	colframe=blue!40!black, 
	sharp corners=northwest,
	fonttitle=\sffamily\bfseries, 
	title=Teorema~\thetcbcounter: #2, 
	#1
}

% disponi definizioni
\newtcolorbox[auto counter, number within=section]{definition}[2][]{%
	colback=red!10,
	colframe=red!40!black,
	sharp corners=northwest,
	fonttitle=\sffamily\bfseries,
	title=Definizione~\thetcbcounter: #2,
	#1
}

% disponi problemi
\newtcolorbox[auto counter, number within=section]{problem}[2][]{%
	colback=green!10,
	colframe=green!40!black,
	sharp corners=northwest,
	fonttitle=\sffamily\bfseries,
	title=Problema~\thetcbcounter: #2,
	#1
}

% disponi codice
\usepackage{listings}
\usepackage[table]{xcolor}

\lstdefinestyle{codestyle}{
		backgroundcolor=\color{black!5}, 
		commentstyle=\color{codegreen},
		keywordstyle=\bfseries\color{magenta},
		numberstyle=\sffamily\tiny\color{black!60},
		stringstyle=\color{green!50!black},
		basicstyle=\ttfamily\footnotesize,
		breakatwhitespace=false,         
		breaklines=true,                 
		captionpos=b,                    
		keepspaces=true,                 
		numbers=left,                    
		numbersep=5pt,                  
		showspaces=false,                
		showstringspaces=false,
		showtabs=false,                  
		tabsize=2
}

\lstdefinestyle{shellstyle}{
		backgroundcolor=\color{black!5}, 
		basicstyle=\ttfamily\footnotesize\color{black}, 
		commentstyle=\color{black}, 
		keywordstyle=\color{black},
		numberstyle=\color{black!5},
		stringstyle=\color{black}, 
		showspaces=false,
		showstringspaces=false, 
		showtabs=false, 
		tabsize=2, 
		numbers=none, 
		breaklines=true
}

\lstdefinelanguage{javascript}{
	keywords={typeof, new, true, false, catch, function, return, null, catch, switch, var, if, in, while, do, else, case, break},
	keywordstyle=\color{blue}\bfseries,
	ndkeywords={class, export, boolean, throw, implements, import, this},
	ndkeywordstyle=\color{darkgray}\bfseries,
	identifierstyle=\color{black},
	sensitive=false,
	comment=[l]{//},
	morecomment=[s]{/*}{*/},
	commentstyle=\color{purple}\ttfamily,
	stringstyle=\color{red}\ttfamily,
	morestring=[b]',
	morestring=[b]"
}

% disponi sezioni
\usepackage{titlesec}

\titleformat{\section}
	{\sffamily\Large\bfseries} 
	{\thesection}{1em}{} 
\titleformat{\subsection}
	{\sffamily\large\bfseries}   
	{\thesubsection}{1em}{} 
\titleformat{\subsubsection}
	{\sffamily\normalsize\bfseries} 
	{\thesubsubsection}{1em}{}

% disponi alberi
\usepackage{forest}

\forestset{
	rectstyle/.style={
		for tree={rectangle,draw,font=\large\sffamily}
	},
	roundstyle/.style={
		for tree={circle,draw,font=\large}
	}
}

% disponi algoritmi
\usepackage{algorithm}
\usepackage{algorithmic}
\makeatletter
\renewcommand{\ALG@name}{Algoritmo}
\makeatother

% disponi numeri di pagina
\usepackage{fancyhdr}
\fancyhf{} 
\fancyfoot[L]{\sffamily{\thepage}}

\makeatletter
\fancyhead[L]{\raisebox{1ex}[0pt][0pt]{\sffamily{\@title \ \@date}}} 
\fancyhead[R]{\raisebox{1ex}[0pt][0pt]{\sffamily{\@author}}}
\makeatother

\begin{document}

% sezione (data)
\section{Lezione del 30-10-24}

% stili pagina
\thispagestyle{empty}
\pagestyle{fancy}

% testo
\subsection{Problema di copertura}
\begin{problem}{di copertura}
	Supponiamo che l'ASL debba dislocare ambulanze su 5 sedi distribuite sul territorio, in 9 diverse località.
	La matrice di adiacenza fra sedi e località è la seguente:

	\center \rowcolors{2}{green!10}{green!40!black!20}
	\begin{tabular} { | c | c | c | c | c | c | }
		\hline
		& \bfseries Sede 1 & \bfseries Sede 2 & \bfseries Sede 3 & \bfseries Sede 4 & \bfseries Sede 5\\
		\hline 
		\bfseries Località 1 & 1 & 0 & 1 & 0 & 1 \\
		\bfseries Località 2 & 0 & 1 & 0 & 1 & 0 \\
		\bfseries Località 3 & 1 & 1 & 0 & 0 & 0 \\
		\bfseries Località 4 & 0 & 1 & 1 & 0 & 1 \\
		\bfseries Località 5 & 0 & 0 & 0 & 0 & 1 \\
		\bfseries Località 6 & 1 & 1 & 1 & 1 & 1 \\
		\bfseries Località 7 & 1 & 1 & 0 & 0 & 1 \\
		\bfseries Località 8 & 0 & 1 & 1 & 1 & 0 \\
		\bfseries Località 9 & 0 & 0 & 1 & 0 & 1 \\
		\hline
	\end{tabular}	
	\par\bigskip
	\raggedright
	Vogliamo capire in quali sedi dovremo dislocare le ambulanze in modo da coprire tutte le località.
\end{problem}
Il problema è risolvibile attraverso la ILP.
Possiamo assumere che la matrice di adiacenza sia stata ricavata da qualche matrice dei tempi (rappresentante il tempo necessario a raggiungere una località), e quindi tagliata su un tempo massimo di, ad esempio, 20 minuti, con il risultato che le località a meno di 20 minuti di distanza risultano connesse e quelle a più di 20 minuti disconnesse.

Possiamo quindi definire alcune \textbf{regole di riduzione} per semplificare la  matrice di adiacenza:
\begin{enumerate}
	\item Le righe di soli zeri ($0, ..., 0$) possono essere eliminate in quanto rappresentano una soluzione impossibile del problema;
	\item Le righe di soli uni ($1, ..., 1$) possono essere eliminate in quanto possono essere risolte da qualsiasi servizio; 
	\item Se una riga contiene un solo 1, apro un "servizio" per quella riga (cioè gli dedico una delle colonne), la elimino, e elimino tutte le righe che hanno 1 sulla stessa colonna.
	\item Se per due colonne $r$ e $k$, $r_{ij} \geq k_{kj}$ per ogni riga, allora si diche che $r$ \textbf{domina} $k$, e quindi che $k$ può essere scartata.
		Questo vale solo se l'apertura è a costo costante su tutte le colonne: in caso contrario potremmo scartare opzioni viabili di ottimizzazione. 
\end{enumerate}

	Vogliamo capire quali località dovranno essere servite da quale sede in modo da coprirle tutte. 
Si ricavano quindi i vincoli su ogni riga:
\[
	\begin{cases}
		x_1 + x_3 + x_5 \geq 1 \\
		x_2 + x_4 \geq 1 \\ 
		... \\
		x_3+x_5 \geq 1 \\ 
		x_i \in \{ 0, 1 \}
	\end{cases}
\]

ergo si ricava un problema di ILP in forma:
\[
	\begin{cases}
		\min c^\intercal \cdot x \\ 
		Ax \geq 1 \\ 
		x \in \{ 0, 1 \}
	\end{cases}
\]

dove $c$ rappresenta un vettore costo, nel problema riportato assunto come costante.

Quello che facciamo effettivamente è scegliere fra $n$ sottoinsiemi, scegliendo l'insieme minimo di questi per coprire l'interezza degli elementi (detti anche nodi).

\subsubsection{Valutazioni inferiori e superiori}
Vediamo quali valutazioni possiamo usare per trovare soluzioni approssimate. 
\begin{itemize}
	\item \textbf{Valutazione inferiore:} una valutazione inferiore $v_I$ può essere agevolmente calcolata dal rilassato continuo del problema ILP:
		\[
			\begin{cases}
				\min c^\intercal \cdot x \\ 
				Ax \geq 1 \\ 
				0 \leq x \leq 1
			\end{cases}
		\]
	\item \textbf{Valutazione superiore:} si può applicare un algoritmo greedy per il calcolo di una valutazione superiore, assunti costi costanti:
\begin{algorithm}[H]
\caption{di copertura a costi costanti}
\begin{algorithmic}
	\STATE \textbf{Input:} un problema di copertura % input
	\STATE \textbf{Output:} una valutazione superiore $v_S$ % output
	% body
	\STATE \textsf{ciclo:}
	\STATE $r_i \leftarrow$ la somma dei valori su ogni colonna
	\STATE Rimuovi la colonna con $r_i$ maggiore, scegli quella colonna nella soluzione, e rimuovi le righe servite da quella colonna
	\STATE Torna a \textsf{ciclo}
\end{algorithmic}
\end{algorithm}
Notiamo che $r_i$ rappresenta effettivamente un rendimento, come avevamo visto nei problemi di zaino.
Possiamo infatti applicare un'altro algoritmo, più sofisticato, nel caso dei costi variabili:
	\begin{algorithm}[H]
\caption{di Chvatal}
\begin{algorithmic}
	\STATE \textbf{Input:} un problema di copertura % input
	\STATE \textbf{Output:} una valutazione superiore $v_S$% output
	% body
	\STATE \textsf{ciclo:}
	\STATE $r_i \leftarrow$ il rapporto fra il costo di ogni colonna e la somma dei valori su quella colonna
	\STATE Rimuovi la colonna con $r_i$ maggiore, scegli quella colonna nella soluzione, e rimuovi le righe servite da quella colonna
	\STATE Torna a \textsf{ciclo}
\end{algorithmic}
\end{algorithm}
\end{itemize}

\subsubsection{Problemi di massima copertura}
Supponiamo di avere una stima del valore (che possiamo assimilare alla domanda, o al guadagno associato all'apertura di un servizio su un certo nodo) su ogni riga (nell'esempio precedente di ogni località), e di voler prediligere soluzioni che coprono righe con valore maggiore, supponendo di avere un numero limitato di $k$ risorse (in questo caso ambulanze) a disposizione.
Prendiamo ad esempio la tabella:

\begin{table}[h!]
	\center \rowcolors{2}{white}{black!10}
	\begin{tabular} {  c || c || c | c | c | c | c  }
		& \bfseries Abitanti & \bfseries Sede 1 & \bfseries Sede 2 & \bfseries Sede 3 & \bfseries Sede 4 & \bfseries Sede 5\\
		\hline 
		\bfseries Località 1 & 2000 & 1 & 0 & 1 & 0 & 1 \\
		\bfseries Località 2 & 1000 & 0 & 1 & 0 & 1 & 0 \\
		\bfseries Località 3 & 800 & 1 & 1 & 0 & 0 & 0 \\
		\bfseries Località 4 & 3000 & 0 & 1 & 1 & 0 & 1 \\
		\bfseries Località 5 & 2500 & 0 & 0 & 0 & 0 & 1 \\
		\bfseries Località 6 & 1200 & 1 & 1 & 1 & 1 & 1 \\
		\bfseries Località 7 & 1700 & 1 & 1 & 0 & 0 & 1 \\
		\bfseries Località 8 & 400 & 0 & 1 & 1 & 1 & 0 \\
		\bfseries Località 9 & 150 & 0 & 0 & 1 & 0 & 1 \\
	\end{tabular}	
\end{table}

Il problema potrebbe essere di assegnare ambulanze alle sedi che offrono copertura della maggior popolazione possibile. 
Formuleremo allora il problema di ILP:
\[
	\begin{cases}
		\max h_1 z_1 + ... + h_9 z_9 \\
		x_1 + x_3 + x_5 \geq z_1 \\
		x_2 + x_4 \geq z_2 \\ 
		... \\ 
		x_3 + x_5 \geq z_9 \\
		x_1 + x_2 + x_3 + x_4 = k \\ 
		x \in \{ 0, 1 \} \\ 
		z \in \{ 0, 1 \}
	\end{cases}
\] 

Questo rappresenta un \textbf{problema di massima copertura}.
I coefficienti $h_1, ..., h_n$ sono i valori associati ad ogni nodo (qui ad ogni località): la funzione obiettivo risponde meglio all'apertura dei servizi ai nodi con valore alto.
I successivi $n$ vincoli rappresentano il fatto che, presa un nodo $z$, cioè una riga, dobbiamo rendere uguale a 1 almeno una fra le colonne che la servono, cioè dobbiamo aprirgli almeno un servizio.
Infine, il vincolo $x_1 + ... + x_n =k$ rappresenta il fatto che vogliamo prendere $k$ servizi, cioè colonne (nel nostro caso distribuire $k$ ambulanze).

Quello che facciamo, effettivamente, è scegliere fra i sottoinsiemi $x$ per massimizzare la copertura sugli elementi $z$, massimizzando il guadagno ottenuto in base alla scelta degli $z$ (scelti gli $z$ i successivi vincoli determinano quali $x$ dobbiamo aprire per servirli).

\end{document}


\documentclass[a4paper,11pt]{article}
\usepackage[a4paper, margin=8em]{geometry}

% usa i pacchetti per la scrittura in italiano
\usepackage[french,italian]{babel}
\usepackage[T1]{fontenc}
\usepackage[utf8]{inputenc}
\frenchspacing 

% usa i pacchetti per la formattazione matematica
\usepackage{amsmath, amssymb, amsthm, amsfonts}

% usa altri pacchetti
\usepackage{gensymb}
\usepackage{hyperref}
\usepackage{standalone}

% imposta il titolo
\title{Appunti Ricerca Operativa}
\author{Luca Seggiani}
\date{2024}

% disegni
\usepackage{pgfplots}
\pgfplotsset{width=10cm,compat=1.9}

% imposta lo stile
% usa helvetica
\usepackage[scaled]{helvet}
% usa palatino
\usepackage{palatino}
% usa un font monospazio guardabile
\usepackage{lmodern}

\renewcommand{\rmdefault}{ppl}
\renewcommand{\sfdefault}{phv}
\renewcommand{\ttdefault}{lmtt}

% disponi il titolo
\makeatletter
\renewcommand{\maketitle} {
	\begin{center} 
		\begin{minipage}[t]{.8\textwidth}
			\textsf{\huge\bfseries \@title} 
		\end{minipage}%
		\begin{minipage}[t]{.2\textwidth}
			\raggedleft \vspace{-1.65em}
			\textsf{\small \@author} \vfill
			\textsf{\small \@date}
		\end{minipage}
		\par
	\end{center}

	\thispagestyle{empty}
	\pagestyle{fancy}
}
\makeatother

% disponi teoremi
\usepackage{tcolorbox}
\newtcolorbox[auto counter, number within=section]{theorem}[2][]{%
	colback=blue!10, 
	colframe=blue!40!black, 
	sharp corners=northwest,
	fonttitle=\sffamily\bfseries, 
	title=Teorema~\thetcbcounter: #2, 
	#1
}

% disponi definizioni
\newtcolorbox[auto counter, number within=section]{definition}[2][]{%
	colback=red!10,
	colframe=red!40!black,
	sharp corners=northwest,
	fonttitle=\sffamily\bfseries,
	title=Definizione~\thetcbcounter: #2,
	#1
}

% disponi problemi
\newtcolorbox[auto counter, number within=section]{problem}[2][]{%
	colback=green!10,
	colframe=green!40!black,
	sharp corners=northwest,
	fonttitle=\sffamily\bfseries,
	title=Problema~\thetcbcounter: #2,
	#1
}

% disponi codice
\usepackage{listings}
\usepackage[table]{xcolor}

\lstdefinestyle{codestyle}{
		backgroundcolor=\color{black!5}, 
		commentstyle=\color{codegreen},
		keywordstyle=\bfseries\color{magenta},
		numberstyle=\sffamily\tiny\color{black!60},
		stringstyle=\color{green!50!black},
		basicstyle=\ttfamily\footnotesize,
		breakatwhitespace=false,         
		breaklines=true,                 
		captionpos=b,                    
		keepspaces=true,                 
		numbers=left,                    
		numbersep=5pt,                  
		showspaces=false,                
		showstringspaces=false,
		showtabs=false,                  
		tabsize=2
}

\lstdefinestyle{shellstyle}{
		backgroundcolor=\color{black!5}, 
		basicstyle=\ttfamily\footnotesize\color{black}, 
		commentstyle=\color{black}, 
		keywordstyle=\color{black},
		numberstyle=\color{black!5},
		stringstyle=\color{black}, 
		showspaces=false,
		showstringspaces=false, 
		showtabs=false, 
		tabsize=2, 
		numbers=none, 
		breaklines=true
}

\lstdefinelanguage{javascript}{
	keywords={typeof, new, true, false, catch, function, return, null, catch, switch, var, if, in, while, do, else, case, break},
	keywordstyle=\color{blue}\bfseries,
	ndkeywords={class, export, boolean, throw, implements, import, this},
	ndkeywordstyle=\color{darkgray}\bfseries,
	identifierstyle=\color{black},
	sensitive=false,
	comment=[l]{//},
	morecomment=[s]{/*}{*/},
	commentstyle=\color{purple}\ttfamily,
	stringstyle=\color{red}\ttfamily,
	morestring=[b]',
	morestring=[b]"
}

% disponi sezioni
\usepackage{titlesec}

\titleformat{\section}
	{\sffamily\Large\bfseries} 
	{\thesection}{1em}{} 
\titleformat{\subsection}
	{\sffamily\large\bfseries}   
	{\thesubsection}{1em}{} 
\titleformat{\subsubsection}
	{\sffamily\normalsize\bfseries} 
	{\thesubsubsection}{1em}{}

% disponi alberi
\usepackage{forest}

\forestset{
	rectstyle/.style={
		for tree={rectangle,draw,font=\large\sffamily}
	},
	roundstyle/.style={
		for tree={circle,draw,font=\large}
	}
}

% disponi algoritmi
\usepackage{algorithm}
\usepackage{algorithmic}
\makeatletter
\renewcommand{\ALG@name}{Algoritmo}
\makeatother

% disponi numeri di pagina
\usepackage{fancyhdr}
\fancyhf{} 
\fancyfoot[L]{\sffamily{\thepage}}

\makeatletter
\fancyhead[L]{\raisebox{1ex}[0pt][0pt]{\sffamily{\@title \ \@date}}} 
\fancyhead[R]{\raisebox{1ex}[0pt][0pt]{\sffamily{\@author}}}
\makeatother

\begin{document}

% sezione (data)
\section{Lezione del 05-11-24}

% stili pagina
\thispagestyle{empty}
\pagestyle{fancy}

% testo
\subsection{Ottimizzazione sui grafi}
Veniamo adesso all'\textbf{ottimizzazione su grafi}, cioè la massimizzazione o minimizzazione del flusso su grafi definiti come insiemi $G = (N,A)$, di $\mathrm{card}(N) = n$ nodi e $\mathrm{card}(A) = m$ archi, dove $A \subseteq N \times N$ (dalla commutatività del prodotto cartesiano si parla di grafi \textbf{orientati}).
Si potrà avere ad esempio:

\begin{center}
	\begin{tikzpicture}
		\node[circle, draw=black] (1) at (0,0) {1};
		\node[circle, draw=black] (2) at (2,1) {2};
		\node[circle, draw=black] (3) at (2,-1) {3};
		\node[circle, draw=black] (4) at (4,1) {4};
		\node[circle, draw=black] (5) at (4,-1) {5};
		\draw[->, to path={-| (\tikztotarget)}] (1) -- (2);
		\draw[->, to path={-| (\tikztotarget)}] (1) -- (3);
		\draw[->, to path={-| (\tikztotarget)}] (3) -- (2);
		\draw[->, to path={-| (\tikztotarget)}] (2) -- (4);
		\draw[->, to path={-| (\tikztotarget)}] (3) -- (5);
		\draw[->, to path={-| (\tikztotarget)}] (5) -- (4);
		\draw[->, to path={-| (\tikztotarget)}] (2) -- (5);


		\node at (0,0.6) {-5};
		\node at (2,1.6) {-6};
		\node at (4,1.6) {4};
		\node at (2,-1.6) {3};
		\node at (4,-1.6) {4};
	\end{tikzpicture}
\end{center}

Etichettiamo con il \textbf{bilancio} di flusso ogni nodo, adottando la convenzione dei \textbf{segni positivi} per i \textit{pozzi} (i nodi che assorbono flusso) e \textbf{segni negativi} per le \textit{sorgenti} (i nodi che erogano flusso).
A questo punto, il vettore soluzione $x_{ij}$ rappresenterà le unità di flusso su ogni arco ($i$ e $j$ sono indici rispettivamente dei nodi di partenza e arrivo).
Come sempre, il vettore soluzione è ordinato \textbf{lessicograficamente}.

Quello che vogliamo fare è quindi esprimere i \textbf{vincoli} a cui è sottoposto flusso sugli archi del grafo.
Conviene quindi guardare sempre al bilancio in termini di flusso di ogni nodo, prendendo gli entranti (cioè i nodi che \textit{approvigionano} il nodo) come positivi e gli uscenti (cioè i nodi che si \textit{riforniscono} dal nodo) come negativi (da qui il segno negativo alle sorgenti).
Ad esempio, avremo per l'esempio precedente:
\[
	\begin{cases}
		-x_{12} - x_{13} = -5 \\ 
		x_{12} + x_{32} - x_{24} - x_{25} = -6 \\ 
		x_{13} - x_{32} - x_{35} = 3 \\ 
		x_{24} + x_{54} = 4 \\ 
		x_{25 }+ x_{35} - x_{54} = -4 \\ 
		x_{ij} \geq 0
	\end{cases}
\]
in formato duale standard.

La situazione diventa un problema di LP quando si introduce un \textbf{costo di movimentazione} su ogni arco: a questo punto può essere interessante calcolare, se ogni unità di flusso "paga" per ogni unità il costo dell'arco su cui scorre, il \textbf{flusso di costo minimo} sul grafo.
Questo si traduce nella funzione costo:
$$
\min{2 x_{12} + 3 x_{13} + 4 x_{24} + 6 x_{25} + 7 x_{32} + 5 x_{35} + 8 x_{54} \equiv c^T \cdot x}
$$
e quindi dà il sistema completo:
\[
	\begin{cases}
		\min{2 x_{12} + 3 x_{13} + 4 x_{24} + 6 x_{25} + 7 x_{32} + 5 x_{35} + 8 x_{54}} \\ 
		-x_{12} - x_{13} = -5 \\ 
		x_{12} + x_{32} - x_{24} - x_{25} = -6 \\ 
		x_{13} - x_{32} - x_{35} = 3 \\ 
		x_{24} + x_{54} = 4 \\ 
		x_{25 }+ x_{35} - x_{54} = -4 \\ 
		x_{ij} \geq 0
	\end{cases}
\]

\subsubsection{Matrici di incidenza}
Diamo una definizione più generale dell'insieme di vincoli necessari a definire un problema di ottimizzazione su un grafo.
Avevamo che vogliamo, preso il vettore di bilanci $b_i$, eguagliare il bilancio del flusso effettivo su un nodo in funzione degli archi entranti ed uscenti con questo vettore.
Questo si esprime nella forma:
$$
\sum_{(e, i) \in A} x_{ei} - \sum_{(i,u) \in A} x_{iu} = b_i
$$
dove si è preso con $A$ l'insieme di tutti gli archi, con $(e,i) \in A$ gli indici dei nodi che formano (\textit{incidono su}) archi entranti al nodo, e con $(i,q) \in A$ gli indici dei nodi che formano archi uscenti dal nodo, cioè la definizione operativa di bilancio del flusso che avevamo usato nell'esempio.
Più formalmente, abbiamo che i vincoli di un problema di ottimizzazione su un grafo rispettano la forma:
$$
E x = b 
$$
dove $E$ è la \textbf{matrice di incidenza} del grafo.
\begin{definition}{Matrice di incidenza}
	Per un grafo di $\mathrm{card}(N) = n$ nodi e $\mathrm{card}(A) = m$ archi, la matrice di incidenza $E$ è la matrice $n \times m$ che mette in relazione ogni nodo con ogni arco, con i valori $e_{ij}$:
\begin{itemize}
	\item $e_{ij} = 0$: il nodo non partecipa all'arco;
	\item $e_{ij} = -1$: il nodo è la partenza dell'arco;
	\item $e_{ij} = 1$: il nodo è l'arrivo dell'arco.
\end{itemize}
\end{definition}

Riassumiamo la differenza fra \textbf{matrice di incidenza} e \textbf{matrice di adiacenza} (che abbiamo già usato, ad esempio nei problemi di assegnamento di costo minimo e di trasporto).
La prima mette in relazione nodi con archi, la seconda nodi con nodi.
Entrambe sono utili alla modellizzazione di problemi di tipo diverso.
A scopo esplicativo, si riportano le matrici di incidenza $E$ e di adiacenza $A$ dell'esempio precedente:
$$
E =
\begin{array}{c | c c c c c c c}
	& (1,2) & (1,3) & (2,3) & (2,4) & (2,5) & (3,5) & (4,5) \\
	\hline
	1 & -1 & -1 & 0 & 0 & 0 & 0 & 0 \\
	2 & 1 & 0 & 1 & -1 & -1 & 0 & 0 \\ 
	3 & 0 & 1 & -1 & 0 & 0 & -1 & 0 \\ 
	4 & 0 & 0 & 0 & 1 & 0 & 0 & 1 \\ 
	5 & 0 & 0 & 0 & 0 & 1 & 1 & -1 
\end{array}
A = 
\begin{pmatrix}
	0 & 1 & 1 & 0 & 0 \\ 
	0 & 0 & 0 & 0 & 1 \\ 
	0 & 1 & 0 & 0 & 1 \\ 
	0 & 0 & 0 & 0 & 0 \\ 
	0 & 0 & 0 & 1 & 0
\end{pmatrix}
$$
 
\par\smallskip 

Abbiamo quindi che la forma più generale di un problema di ottimizzazione su un grafo è:
\[
	\begin{cases}
		\min c^T \cdot x \\ 
		Ex = b \\ 
		x \geq 0 
	\end{cases}
\]
dove segnaliamo è importante il grafo sia \textbf{connesso}:
\begin{definition}{Grafo connesso}
	Un grafo si dice connesso quando l'insieme degli archi $A$ contiene un albero di copertura $T \subseteq A$.
\end{definition}
Questa definizione (che in verità si potrebbe intendere come caratterizzazione) si fonda sull'idea di textbf{albero di copertura} (che in verità abbiamo già usato nei problemi del commesso viaggiatore in ILP), che definiamo come segue:
\begin{definition}{Albero di copertura}
	Dato un grafo $G = (N, A)$, un albero di copertura è un sottoinsieme di archi $T \subseteq A$ tale che il sottografo $(N, T)$ è connesso e non presenta cicli. Un nodo di questo grafo su cui incide solo un arco si dice \textbf{foglia}.
\end{definition}

Notiamo che su ogni colonna della matrice di adiacenza si ha esattamente un 1 e un -1, e tutti gli altri elementi valgono 0 (in altre parole, un arco è formato da un nodo di partenza e un nodo di arrivo).
Questo significa che la somma delle righe della matrice è nulla, ergo per definizione queste sono fra di loro linearmente dipendenti.
Si ha quindi che:
\begin{theorem}{Rango della matrice di incidenza}
	La matrice di incidenza di un problema di ottimizzazione su un grafo con $n$ nodi ha rango $n-1$.
\end{theorem}

La proposizione precedente rispetto alla dipendenza lineare delle righe della matrice di adiacenza è solo metà delle ipotesi necessarie a dimostrare quanto sopra: il resto viene dalla definizione di albero di copertura. 
Dimostriamo quindi che il rango vale $n-1$, mostrando come si possono prendere come minori dela matrice gli alberi di copertura costruiti sul grafo in questione.
Per formare un albero di copertura si prendono i minimi $n-1$ archi necessari a collegare fra di loro tutti i nodi, ergo si prendono le $n-1$ colonne della matrice di adiacenza del grafo corrispondenti agli archi che formano l'albero per formare un minore $E_T$.
Da queste, permutando appositamente righe e colonne, si può sempre ottenere una forma del tipo:
$$
E_T =
\left(\begin{array}{ c | c  }
		\pm 1 & 0 \, ... \,  0 \\
	\hline
		* & E_S
\end{array}\right)
$$
dove la prima riga rappresenta una foglia $z$, e la matrice $E_S$ la matrice di adiacenza del sottoalbero di copertura ottenuto su $N \setminus z$.
Questa forma, per ipotesi induttiva, ci dice che la matrice di incidenza dell'albero è \textbf{triangolare inferiore}, da cui ricaviamo due cose:
\begin{itemize}
	\item La matrice di adiacenza dell'albero di copertura, ergo il minore, è invertibile, da cui il rango $n-1$;
	\item La matrice di adiacenza dell'albero ha determinante $\pm 1$, ergo è \textbf{unimodulare}.
		Questo ha dei risvolti importanti per quanto riguarda l'ILP, in quanto vedremo che una soluzione di LP per un problema di flusso minimo corrisponde con la soluzione del problema di ILP corrispondente (a patto di bilanci $b_i$ interi).
\end{itemize}

\subsubsection{Calcolo delle basi}
Abbiamo appurato che la matrice di incidenza di un problema di ottimizzazione su un grafo ha rango $n-1$ e presenta minori invertibili, rappresentanti matrici di incidenza di alberi di copertura.
Visto che il problema stesso era stato espresso in forma duale standard, e che quello che facciamo nel prendere minori di rango $n-1$ è effettivamente scegliere $n-1$ colonne della matrice di incidenza, si ha che effettivamente ogni albero di copertura rappresenta una \textbf{base} del poliedro.

\begin{theorem}{Caratterizzazione di base di matrici di incidenza}
	ei ha che, per una matrice di incidenza $E$ di un problema di ottimizzazione su grafi, un albero di copertura coincide con una data base $T$ ottenuta scegliendo le $n-1$ colonne degli archi che formano l'albero, ovvero:
	$$
	T \text{ è un albero di copertura} \Leftrightarrow T \text{ è una base}
	$$
\end{theorem}

Vediamo un modo per calcolare una soluzione di base data una base $T$.
Iniziamo con l'impostare, come era sempre valso per il duale standard, a 0 le variabili non di base.
Adesso si può fare una visita positicipata per foglie (considerando la base come un albero).
Vorremo che questa foglia rispetti:
$$
E_T x_T = b
$$

Partendo dalle foglie, si può mettere sull'arco corrispondente (e quindi sulla $x_{ij}$ corrispondente) la capacità di quella foglia, e poi tagliarla dall'albero.
Il procedimento si ripete sulle foglie rimaste, fino ad arrivare alla radice dell'albero.

Così facendo troveremo un vettore soluzione $x_{ij}$, su cui notiamo se almeno una componente è $< 0$, la soluzione di base non è ammissibile (significa che un nodo sorgente chiede più di quanto può erogare).

\subsubsection{Reti sbilanciate}
Notiamo che finora abbiamo assunto l'ipotesi:
$$
\sum_{i=1}^n b_i = 0
$$

Potremmo avere invece che questa sommatoria è \textit{negativa} o \textit{positiva}, ergo il grafo \textbf{sbilanciato} in \textbf{positivo} o in \textbf{negativo}.
Nel primo caso, si ha più richeista che produzione, e il problema è quindi vuoto.
Nel secondo, si ha puù produzione che richiesta, e bisogna quindi introdurre un \textbf{nodo fittizio}, di capacità pari all'\textbf{eccedente} $\sum_{i=1}^n b_i$ e costo nullo, dove  verrà a riversarsi (in condizioni di ottimo) il flusso in eccesso. 

\end{document}


\documentclass[a4paper,11pt]{article}
\usepackage[a4paper, margin=8em]{geometry}

% usa i pacchetti per la scrittura in italiano
\usepackage[french,italian]{babel}
\usepackage[T1]{fontenc}
\usepackage[utf8]{inputenc}
\frenchspacing 

% usa i pacchetti per la formattazione matematica
\usepackage{amsmath, amssymb, amsthm, amsfonts}

% usa altri pacchetti
\usepackage{gensymb}
\usepackage{hyperref}
\usepackage{standalone}

% imposta il titolo
\title{Appunti Ricerca Operativa}
\author{Luca Seggiani}
\date{2024}

% disegni
\usepackage{pgfplots}
\pgfplotsset{width=10cm,compat=1.9}

% imposta lo stile
% usa helvetica
\usepackage[scaled]{helvet}
% usa palatino
\usepackage{palatino}
% usa un font monospazio guardabile
\usepackage{lmodern}

\renewcommand{\rmdefault}{ppl}
\renewcommand{\sfdefault}{phv}
\renewcommand{\ttdefault}{lmtt}

% disponi il titolo
\makeatletter
\renewcommand{\maketitle} {
	\begin{center} 
		\begin{minipage}[t]{.8\textwidth}
			\textsf{\huge\bfseries \@title} 
		\end{minipage}%
		\begin{minipage}[t]{.2\textwidth}
			\raggedleft \vspace{-1.65em}
			\textsf{\small \@author} \vfill
			\textsf{\small \@date}
		\end{minipage}
		\par
	\end{center}

	\thispagestyle{empty}
	\pagestyle{fancy}
}
\makeatother

% disponi teoremi
\usepackage{tcolorbox}
\newtcolorbox[auto counter, number within=section]{theorem}[2][]{%
	colback=blue!10, 
	colframe=blue!40!black, 
	sharp corners=northwest,
	fonttitle=\sffamily\bfseries, 
	title=Teorema~\thetcbcounter: #2, 
	#1
}

% disponi definizioni
\newtcolorbox[auto counter, number within=section]{definition}[2][]{%
	colback=red!10,
	colframe=red!40!black,
	sharp corners=northwest,
	fonttitle=\sffamily\bfseries,
	title=Definizione~\thetcbcounter: #2,
	#1
}

% disponi problemi
\newtcolorbox[auto counter, number within=section]{problem}[2][]{%
	colback=green!10,
	colframe=green!40!black,
	sharp corners=northwest,
	fonttitle=\sffamily\bfseries,
	title=Problema~\thetcbcounter: #2,
	#1
}

% disponi codice
\usepackage{listings}
\usepackage[table]{xcolor}

\lstdefinestyle{codestyle}{
		backgroundcolor=\color{black!5}, 
		commentstyle=\color{codegreen},
		keywordstyle=\bfseries\color{magenta},
		numberstyle=\sffamily\tiny\color{black!60},
		stringstyle=\color{green!50!black},
		basicstyle=\ttfamily\footnotesize,
		breakatwhitespace=false,         
		breaklines=true,                 
		captionpos=b,                    
		keepspaces=true,                 
		numbers=left,                    
		numbersep=5pt,                  
		showspaces=false,                
		showstringspaces=false,
		showtabs=false,                  
		tabsize=2
}

\lstdefinestyle{shellstyle}{
		backgroundcolor=\color{black!5}, 
		basicstyle=\ttfamily\footnotesize\color{black}, 
		commentstyle=\color{black}, 
		keywordstyle=\color{black},
		numberstyle=\color{black!5},
		stringstyle=\color{black}, 
		showspaces=false,
		showstringspaces=false, 
		showtabs=false, 
		tabsize=2, 
		numbers=none, 
		breaklines=true
}

\lstdefinelanguage{javascript}{
	keywords={typeof, new, true, false, catch, function, return, null, catch, switch, var, if, in, while, do, else, case, break},
	keywordstyle=\color{blue}\bfseries,
	ndkeywords={class, export, boolean, throw, implements, import, this},
	ndkeywordstyle=\color{darkgray}\bfseries,
	identifierstyle=\color{black},
	sensitive=false,
	comment=[l]{//},
	morecomment=[s]{/*}{*/},
	commentstyle=\color{purple}\ttfamily,
	stringstyle=\color{red}\ttfamily,
	morestring=[b]',
	morestring=[b]"
}

% disponi sezioni
\usepackage{titlesec}

\titleformat{\section}
	{\sffamily\Large\bfseries} 
	{\thesection}{1em}{} 
\titleformat{\subsection}
	{\sffamily\large\bfseries}   
	{\thesubsection}{1em}{} 
\titleformat{\subsubsection}
	{\sffamily\normalsize\bfseries} 
	{\thesubsubsection}{1em}{}

% disponi alberi
\usepackage{forest}

\forestset{
	rectstyle/.style={
		for tree={rectangle,draw,font=\large\sffamily}
	},
	roundstyle/.style={
		for tree={circle,draw,font=\large}
	}
}

% disponi algoritmi
\usepackage{algorithm}
\usepackage{algorithmic}
\makeatletter
\renewcommand{\ALG@name}{Algoritmo}
\makeatother

% disponi numeri di pagina
\usepackage{fancyhdr}
\fancyhf{} 
\fancyfoot[L]{\sffamily{\thepage}}

\makeatletter
\fancyhead[L]{\raisebox{1ex}[0pt][0pt]{\sffamily{\@title \ \@date}}} 
\fancyhead[R]{\raisebox{1ex}[0pt][0pt]{\sffamily{\@author}}}
\makeatother

\begin{document}

% sezione (data)
\section{Lezione del 06-11-24}

% stili pagina
\thispagestyle{empty}
\pagestyle{fancy}

% testo
\subsection{Dualità nei problemi di ottimizzazione su grafi}
Abbiamo visto come l'algoritmo del simplesso si basava sulla dualità fra un problema, espresso in \textit{forma primale standard}, e un suo \textbf{duale}, espresso in \textit{forma duale standard}.
Abbiamo inoltre detto che l'operazione duale è \textbf{involutoria}: il duale del duale è nuovamente il primale, e così via.
Si ha che questo è vero anche sui problemi di ottimizzazione su grafi, con l'unica differenza che il problema da cui partiamo è solitamente espresso in \textit{forma duale standard} (come abbiamo visto negli esempi precedenti), e che vi applicheremo l'operazione duale per ricavarne, sì un duale, ma che verrà espresso in \textit{forma primale standard}. 
Chiameremo comunque \textbf{primale} il primo problema, e \textbf{duale} il secondo.

\subsubsection{Forma della matrice dei vincoli}
Facciamo innanzitutto una precisazione sulla forma in cui si esprime la matrice $A$ (o nei problemi su grafi, $E$ di incidenza).
Sono equivalenti le forme:
$$
Ax = b \Leftrightarrow x^\intercal A^\intercal = b
$$
e quindi sui grafi:
$$
Ex = b \Leftrightarrow x^\intercal E^\intercal = b
$$

La seconda forma ha un vantaggio: la matrice così espressa risulta identica sia nel primale che nel duale, in quanto avevamo visto i vincoli duali si esprimono, almeno nei problemi di ottimizzazione visti finora, come:
$$
A^\intercal y = c
$$

\subsection{Duali di problemi di ottimizzazione sui grafi}
Veniamo quindi al calcolo vero e proprio dei problemi associati di problemi di ottimizzazione sui grafi.
A partire dal problema:
\[
	\begin{cases}
		\min c^\intercal \cdot x \\ 
		Ex = b \\ 
		x \geq 0
	\end{cases}
\]
potremo ricavare:
\[
	\begin{cases}
			\max b^\intercal \cdot \pi \\ 
			\pi^\intercal E \leq c \Leftrightarrow E^\intercal\pi \leq c
	\end{cases}
\]
dove si nota, come prima, l'equivalenza delle due forme di espressione dei vincoli.

Chiamiamo questo problema \textbf{problema dei potenziali} ai nodi del grafo.
:qa!
:qa!
Notiamo che, se nel duale avevamo rimosso una riga, dal teorema del rango $n-1$, qui potremo rimuovere una \textbf{variabile}, cioè una colonna della matrice $E^\intercal$.

Presa quindi una base $T$ (cioè un albero di copertura), avremo che il \textbf{potenziale di base} relativo a $T$ è la soluzione del sistema $ \pi^\intercal E_T =c_T^\intercal $, cioè:
$$
\pi^\intercal = c^\intercal E_T^{-1}
$$

Il potenziale di base, come il flusso di base, ha un algoritmo greedy di calcolo piuttosto intuitivo: partendo dalla \textbf{radice} dell'albero $T$, si calcola il costo complessivo sugli archi necessario a spostarsi da questa al nodo di cui vogliamo trovare il potenziale.

Dopo il calcolo dei potenziali, potremo calcolare i cosiddetti \textbf{costi ridotti}, come la differenza fra il costo di ogni arco e la \textbf{differenza di potenziale} dei nodi che collega:
$$
c^\pi_{ij} = c_{ij} + \pi_i - \pi_j
$$

Notiamo come l'unico valore che ci interessa riguardo ai potenziali non sono i potenziali stessi, ma le loro differenze. 

Una volta stabilito il vettore dei potenziali, e quindi dei costi ridotti (sul duale), si può dare la seguente caratterizzazione di vertici ammissibili:

\begin{definition}{Caratterizzazione dei potenziali di base ammissibili}
	Un potenziale di base, posto $T$ albero di copertura, è ammissibile se $\forall (i, j) \in L$ si ha $c^\pi_{ij} \geq 0$. 
\end{definition}

e degeneri, ricordando l'ortogonalità di queste due caratteristiche:

\begin{definition}{Caratterizzazione dei potenziali di base degeneri}
	Un potenziale di base, posto $T$ albero di copertura, è degenere se $\exists (i, j) \in L$ tale per cui $c^\pi_{ij} = 0$.
\end{definition}

Infine, possiamo riformulare il concetto di \textbf{scarti complementari} per soluzioni di base complementari nei problemi di ottimizzazione su grafi:
\begin{itemize}
	\item Un flusso ammissibile $x$ è ottimo se e solo se esiste un potenziale $\pi$ tale che:
		\[
			\begin{cases}
				x_{ij} = 0 \implies c^\pi_{ij} \geq 0 \\ 	
				x_{ij} > 0 \implies c^\pi_{ij} = 0 \\ 	
			\end{cases}
		\]
	\item Un potenziale ammissibile $\pi$ è ottimo se e solo se esiste un flusso $x$ tale che:
		\[
			\begin{cases}
				c^\pi_{ij} = 0 \implies x_{ij} \geq 0 \\ 
				c^\pi_{ij} > 0 \implies x_{ij} = 0
			\end{cases}
		\]
\end{itemize}

Applicando la dualità forte e le condizioni di ammissibilità riportate prima, possiamo formulare il seguente teorema:
\begin{theorem}{di Bellman}
	Una partizione $(T, L)$ di archi con $T$ albero di copertura che genera un flusso di base $x$ ammissibile è ottima se:
	$$c^\pi_{i,j} \geq 0 \quad \forall (i, j) \in L$$
\end{theorem}

Questo teorema non rivela altro che un risultato già noto: se una soluzione è ammissibile sia nel primale che nel duale, allora è ottima.

\end{document}


\documentclass[a4paper,11pt]{article}
\usepackage[a4paper, margin=8em]{geometry}

% usa i pacchetti per la scrittura in italiano
\usepackage[french,italian]{babel}
\usepackage[T1]{fontenc}
\usepackage[utf8]{inputenc}
\frenchspacing 

% usa i pacchetti per la formattazione matematica
\usepackage{amsmath, amssymb, amsthm, amsfonts}

% usa altri pacchetti
\usepackage{gensymb}
\usepackage{hyperref}
\usepackage{standalone}

% imposta il titolo
\title{Appunti Ricerca Operativa}
\author{Luca Seggiani}
\date{2024}

% disegni
\usepackage{pgfplots}
\pgfplotsset{width=10cm,compat=1.9}

% imposta lo stile
% usa helvetica
\usepackage[scaled]{helvet}
% usa palatino
\usepackage{palatino}
% usa un font monospazio guardabile
\usepackage{lmodern}

\renewcommand{\rmdefault}{ppl}
\renewcommand{\sfdefault}{phv}
\renewcommand{\ttdefault}{lmtt}

% disponi il titolo
\makeatletter
\renewcommand{\maketitle} {
	\begin{center} 
		\begin{minipage}[t]{.8\textwidth}
			\textsf{\huge\bfseries \@title} 
		\end{minipage}%
		\begin{minipage}[t]{.2\textwidth}
			\raggedleft \vspace{-1.65em}
			\textsf{\small \@author} \vfill
			\textsf{\small \@date}
		\end{minipage}
		\par
	\end{center}

	\thispagestyle{empty}
	\pagestyle{fancy}
}
\makeatother

% disponi teoremi
\usepackage{tcolorbox}
\newtcolorbox[auto counter, number within=section]{theorem}[2][]{%
	colback=blue!10, 
	colframe=blue!40!black, 
	sharp corners=northwest,
	fonttitle=\sffamily\bfseries, 
	title=Teorema~\thetcbcounter: #2, 
	#1
}

% disponi definizioni
\newtcolorbox[auto counter, number within=section]{definition}[2][]{%
	colback=red!10,
	colframe=red!40!black,
	sharp corners=northwest,
	fonttitle=\sffamily\bfseries,
	title=Definizione~\thetcbcounter: #2,
	#1
}

% disponi problemi
\newtcolorbox[auto counter, number within=section]{problem}[2][]{%
	colback=green!10,
	colframe=green!40!black,
	sharp corners=northwest,
	fonttitle=\sffamily\bfseries,
	title=Problema~\thetcbcounter: #2,
	#1
}

% disponi codice
\usepackage{listings}
\usepackage[table]{xcolor}

\lstdefinestyle{codestyle}{
		backgroundcolor=\color{black!5}, 
		commentstyle=\color{codegreen},
		keywordstyle=\bfseries\color{magenta},
		numberstyle=\sffamily\tiny\color{black!60},
		stringstyle=\color{green!50!black},
		basicstyle=\ttfamily\footnotesize,
		breakatwhitespace=false,         
		breaklines=true,                 
		captionpos=b,                    
		keepspaces=true,                 
		numbers=left,                    
		numbersep=5pt,                  
		showspaces=false,                
		showstringspaces=false,
		showtabs=false,                  
		tabsize=2
}

\lstdefinestyle{shellstyle}{
		backgroundcolor=\color{black!5}, 
		basicstyle=\ttfamily\footnotesize\color{black}, 
		commentstyle=\color{black}, 
		keywordstyle=\color{black},
		numberstyle=\color{black!5},
		stringstyle=\color{black}, 
		showspaces=false,
		showstringspaces=false, 
		showtabs=false, 
		tabsize=2, 
		numbers=none, 
		breaklines=true
}

\lstdefinelanguage{javascript}{
	keywords={typeof, new, true, false, catch, function, return, null, catch, switch, var, if, in, while, do, else, case, break},
	keywordstyle=\color{blue}\bfseries,
	ndkeywords={class, export, boolean, throw, implements, import, this},
	ndkeywordstyle=\color{darkgray}\bfseries,
	identifierstyle=\color{black},
	sensitive=false,
	comment=[l]{//},
	morecomment=[s]{/*}{*/},
	commentstyle=\color{purple}\ttfamily,
	stringstyle=\color{red}\ttfamily,
	morestring=[b]',
	morestring=[b]"
}

% disponi sezioni
\usepackage{titlesec}

\titleformat{\section}
	{\sffamily\Large\bfseries} 
	{\thesection}{1em}{} 
\titleformat{\subsection}
	{\sffamily\large\bfseries}   
	{\thesubsection}{1em}{} 
\titleformat{\subsubsection}
	{\sffamily\normalsize\bfseries} 
	{\thesubsubsection}{1em}{}

% disponi alberi
\usepackage{forest}

\forestset{
	rectstyle/.style={
		for tree={rectangle,draw,font=\large\sffamily}
	},
	roundstyle/.style={
		for tree={circle,draw,font=\large}
	}
}

% disponi algoritmi
\usepackage{algorithm}
\usepackage{algorithmic}
\makeatletter
\renewcommand{\ALG@name}{Algoritmo}
\makeatother

% disponi numeri di pagina
\usepackage{fancyhdr}
\fancyhf{} 
\fancyfoot[L]{\sffamily{\thepage}}

\makeatletter
\fancyhead[L]{\raisebox{1ex}[0pt][0pt]{\sffamily{\@title \ \@date}}} 
\fancyhead[R]{\raisebox{1ex}[0pt][0pt]{\sffamily{\@author}}}
\makeatother

\begin{document}

% sezione (data)
\section{Lezione del 07-11-24}

% stili pagina
\thispagestyle{empty}
\pagestyle{fancy}

% testo
\subsection{Simplesso per flussi}
Vediamo come applicare l'algoritmo del simplesso ai problemi di ottimizzazione sui grafi, in particolare per risolvere problemi di flusso minimo. 

Partiamo dall'assunto fondamentale che una soluzione ammissibile è rappresentata da un albero di copertura che genera flusso ammissibile, che chiameremo \textbf{albero di copertura ammissibile}.
Potremo quindi partizionare l'insieme $A$ degli archi in due sottoinsiemi:
\begin{itemize}
	\item $T$, formato dagli archi che formano l'albero (e che denota quindi l'albero stesso);
	\item $L$, formato dagli archi rimanenti.
\end{itemize}

L'insieme di possibili partizioni $(T, L)$ rappresenta le basi dei vertici di un poliedro, cioè qualsiasi albero di copertura ammissibile $T$ rappresenta una base del poliedro.

Si possono quindi calcolare, dato $T$, i costi ridotti $c_{ij}^\pi$ su tutti gli archi che comprende:
$$
c_{ij}^\pi = c_{ij} + \pi_i - \pi_j
$$
Si ha che, dal teorema di Bellman, se $\forall (i, j) \in L : c_{ij}^\pi \geq 0$, allora la base duale è ammissibile e siamo all'ottimo.
Altrimenti, dovrà essere che $\exists (i,j) \in L : c_{ij}^\pi < 0$. Scegliamo questo $(i, j)$ come \textbf{arco entrante}.

Si ha che l'arco entrante $(i, j)$ forma un ciclo con gli archi dell'albero $T$: la selezione di un arco uscente coinciderà nel simplesso su grafi ad un processo di \textit{eliminazione di cicli}.
Si sceglie allora una direzione di percorrenza del ciclo concorde a $(i, j)$, e si partizionano gli archi del ciclo in $\mathcal{C}^+$ per gli archi concordi a questa direzione, e $\mathcal{C}^-$ per gli archi discordi.

L'obiettivo è quello di "spingere" più unità di flusso $\vartheta$ possibili nella direzione del ciclo, sfruttando l'arco appena introdotto.
Riguardo ai flussi, questo significherà aggiungere $\vartheta$ unità di flusso agli archi in $\mathcal{C}^+$, e rimuovere $\vartheta$ unità di flusso dagli archi in $\mathcal{C}^-$.
Se $\mathcal{C}^-$ è vuoto, non c'è limite al flusso che possiamo spingere, e l'ottimo è $-\infty$. 
Altrimenti, il valore massimo di $\vartheta$ sarà quello che non viola i vincoli, portando un arco in $\mathcal{C}^-$ a valori $<0$. 
Questo valore non sarà altro che il flusso minimo già impegnato sugli archi $x_{ij}$ con $(i, j) \in C^-$, cioè il flusso dell'arco che si "svuoterà" per primo spingendo flusso nella direzione opposta.
Chiamiamo questo arco \textbf{arco uscente}.

Si aggiorna quindi la base rimuovendo l'arco uscente e introducendo l'arco entrante, e si ripete fino all'ottimo.

Possiamo quindi formulare l'algoritmo per intero:

\begin{algorithm}[H]
\caption{del simplesso per flussi}
\begin{algorithmic}
	\STATE \textbf{Input:} un problema di flusso di costo minimo 
	\STATE \textbf{Output:} la soluzione ottima 
	\STATE Trova una partizione degli archi $(T,L)$ con $T$ albero di copertura che genera un flusso ammissibile
	\STATE \textsf{ciclo:}
	\STATE Calcola il flusso di base $\bar{x}_T = E_T^{-1} b$, posti gli $\bar{x}_L = 0$, e il potenziale di base $\bar{\pi}_T = c_T^T E_T^{-1}$
	\STATE Calcola i costi ridotti $c_{ij}^{\bar{\pi}} = c_{ij} + \bar{\pi}_{ij} - \bar{\pi}_{ij}$ per ogni arco
	\IF{$c_{ij}^{\overline{\pi}} \geq 0 \ \ \forall (i, j) \in L$}
	\STATE Fermati, $\bar{x}$ è un flusso ottimo e $\bar{\pi}$ è un potenziale ottimo
\ELSE
		\STATE Calcola l'arco entrante: 
		$$
		(p, q) = \min \left\{ (i, j) \in L : c_{ij}^{\bar{\pi}} < 0 \right\}
		$$
		stabilito l'ordinamento lessicografico degli archi
		\STATE Chiama $\mathcal{C}$ il ciclo che l'arco $(p, q)$ forma con gli archi in $T$
		\STATE Fissa un orientamento concorde a $(p,q)$ su $\mathcal{C}$ e partiziona $\mathcal{C}$ in $\mathcal{C^+}$ archi concordi e $\mathcal{C^-}$ archi discordi a tale orientamento
	\ENDIF
	\IF{$\mathcal{C}^- = \emptyset$}
		\STATE Fermati, il flusso di costo minimo ha valore $-\infty$ 
	\ELSE
		\STATE Calcola:
		$$
		\vartheta = \min\{ \bar{x}_{ij} : (i, j) \in \mathcal{C}^- \}
		$$
		e trova l'arco uscente: 
		$$ 
		(r, s) = \min\{ (i, j) \in \mathcal{C}^- : \bar{x}_{ij} = \vartheta \} 
		$$
		stabilito un ordinamento lessicografico degli archi
	\ENDIF
	\STATE Aggiorna le partizioni come:
	$$
	T = T \setminus \left\{ (r,s) \right\} \cup \left\{ (p, q) \right\} \quad L = L \setminus \left\{ (p, q) \right\} \cup \left\{ (r, s) \right\}
	$$
	\STATE Torna a \textsf{ciclo}
\end{algorithmic}
\end{algorithm}

\subsubsection{Ottimizzazioni del simplesso per i flussi}
Possiamo fare alcune ottimizzazioni considerevoli sull'algoritmo del simplesso per i flussi:
\begin{itemize}
	\item Non è necessario calcolare i costi ridotti $c^\pi_{ij}$ per ogni arco $\in A$, ma soltanto in $L$, in quanto sappiamo gli altri (quelli $\in T$) varranno 0;
	\item Non è necessario calcolare l'insieme $\mathcal{C}^*$: le rimozioni potranno essere effettuate solo da $\mathcal{C}^-$;
	\item Il calcolo dei potenziali di base dopo l'aggiornamento delle partizioni può essere effettuato senza inversioni di matrici.
		Si ha che, presi l'arco entrante $(p, q)$ e l'arco uscente $(r, s)$, la rimozione di $(r,s)$ (prima dell'introduzione di $(p,q)$) partiziona effettivamente i nodi del grafo in due sottoinsiemi $N_p$ e $N_q$, con $p \in N_p$ e $q \in N_q$.
		Avremo allora che i nuovi potenziali di base $\pi'_i$ saranno:
		\[
			\pi'_i =
			\begin{cases}
				\pi_i, \quad i \in N_p \\ 
				\pi_i'+ c_{pq}^\pi, \quad i \in N_q
			\end{cases}
		\]
		cioè la partizione $p$ resta invariata, e la partizione $q$ sale di potenziale pari al costo ridotto $c^\pi_{pq}$.
	\item Una volta variati i potenziali, si possono ricalcolare anche i costi ridotti $c_{ij}^{\pi'}$ con una regola simile:
		\[
			c_{ij}^{\pi'} =
			\begin{cases}
				c_{ij}^{\pi'}, \quad (i,j) \in N_p \vee (i,j) \in N_q \\ 
				c_{ij}^{\pi'} - c_{pq}^\pi, \quad (i,j) \in N_p \wedge (i,j) \in N_q \\ 
				c_{ij}^{\pi'} + c_{pq}^\pi, \quad (i,j) \in N_q \wedge (i,j) \in N_p
			\end{cases}
		\]
		cioè, se l'arco è interamente contenuto in $N_p$ o $N_q$, il costo ridotto resta invariato.
		Altrimenti, se l'arco "salta" dalla partizione $p$ alla partizione $q$, si aggiunge o si sottrae, a seconda della direzione, il costo ridotto di $c_pq^\pi$.
	\end{itemize}

\subsubsection{Ricavare le variazioni dei cambi di partizione}
Potremmo voler ricavare le variazioni in termini di valore ottimo che si ottengono dopo un passo del simplesso sui flussi, quindi dopo l'aggiornamento delle partizioni.
Abbiamo che la variazione del valore ottimo è quindi il flusso introdotto sul ciclo, cioè $\vartheta$, moltiplicato per il costo ridotto dell'arco introdotto, cioè il valore si evolve di $v$ in $v'$ come:
$$
v' = V - \vartheta \cdot c_{ij}^\pi
$$

\end{document}


\documentclass[a4paper,11pt]{article}
\usepackage[a4paper, margin=8em]{geometry}

% usa i pacchetti per la scrittura in italiano
\usepackage[french,italian]{babel}
\usepackage[T1]{fontenc}
\usepackage[utf8]{inputenc}
\frenchspacing 

% usa i pacchetti per la formattazione matematica
\usepackage{amsmath, amssymb, amsthm, amsfonts}

% usa altri pacchetti
\usepackage{gensymb}
\usepackage{hyperref}
\usepackage{standalone}

% imposta il titolo
\title{Appunti Ricerca Operativa}
\author{Luca Seggiani}
\date{2024}

% disegni
\usepackage{pgfplots}
\pgfplotsset{width=10cm,compat=1.9}

% imposta lo stile
% usa helvetica
\usepackage[scaled]{helvet}
% usa palatino
\usepackage{palatino}
% usa un font monospazio guardabile
\usepackage{lmodern}

\renewcommand{\rmdefault}{ppl}
\renewcommand{\sfdefault}{phv}
\renewcommand{\ttdefault}{lmtt}

% disponi il titolo
\makeatletter
\renewcommand{\maketitle} {
	\begin{center} 
		\begin{minipage}[t]{.8\textwidth}
			\textsf{\huge\bfseries \@title} 
		\end{minipage}%
		\begin{minipage}[t]{.2\textwidth}
			\raggedleft \vspace{-1.65em}
			\textsf{\small \@author} \vfill
			\textsf{\small \@date}
		\end{minipage}
		\par
	\end{center}

	\thispagestyle{empty}
	\pagestyle{fancy}
}
\makeatother

% disponi teoremi
\usepackage{tcolorbox}
\newtcolorbox[auto counter, number within=section]{theorem}[2][]{%
	colback=blue!10, 
	colframe=blue!40!black, 
	sharp corners=northwest,
	fonttitle=\sffamily\bfseries, 
	title=Teorema~\thetcbcounter: #2, 
	#1
}

% disponi definizioni
\newtcolorbox[auto counter, number within=section]{definition}[2][]{%
	colback=red!10,
	colframe=red!40!black,
	sharp corners=northwest,
	fonttitle=\sffamily\bfseries,
	title=Definizione~\thetcbcounter: #2,
	#1
}

% disponi problemi
\newtcolorbox[auto counter, number within=section]{problem}[2][]{%
	colback=green!10,
	colframe=green!40!black,
	sharp corners=northwest,
	fonttitle=\sffamily\bfseries,
	title=Problema~\thetcbcounter: #2,
	#1
}

% disponi codice
\usepackage{listings}
\usepackage[table]{xcolor}

\lstdefinestyle{codestyle}{
		backgroundcolor=\color{black!5}, 
		commentstyle=\color{codegreen},
		keywordstyle=\bfseries\color{magenta},
		numberstyle=\sffamily\tiny\color{black!60},
		stringstyle=\color{green!50!black},
		basicstyle=\ttfamily\footnotesize,
		breakatwhitespace=false,         
		breaklines=true,                 
		captionpos=b,                    
		keepspaces=true,                 
		numbers=left,                    
		numbersep=5pt,                  
		showspaces=false,                
		showstringspaces=false,
		showtabs=false,                  
		tabsize=2
}

\lstdefinestyle{shellstyle}{
		backgroundcolor=\color{black!5}, 
		basicstyle=\ttfamily\footnotesize\color{black}, 
		commentstyle=\color{black}, 
		keywordstyle=\color{black},
		numberstyle=\color{black!5},
		stringstyle=\color{black}, 
		showspaces=false,
		showstringspaces=false, 
		showtabs=false, 
		tabsize=2, 
		numbers=none, 
		breaklines=true
}

\lstdefinelanguage{javascript}{
	keywords={typeof, new, true, false, catch, function, return, null, catch, switch, var, if, in, while, do, else, case, break},
	keywordstyle=\color{blue}\bfseries,
	ndkeywords={class, export, boolean, throw, implements, import, this},
	ndkeywordstyle=\color{darkgray}\bfseries,
	identifierstyle=\color{black},
	sensitive=false,
	comment=[l]{//},
	morecomment=[s]{/*}{*/},
	commentstyle=\color{purple}\ttfamily,
	stringstyle=\color{red}\ttfamily,
	morestring=[b]',
	morestring=[b]"
}

% disponi sezioni
\usepackage{titlesec}

\titleformat{\section}
	{\sffamily\Large\bfseries} 
	{\thesection}{1em}{} 
\titleformat{\subsection}
	{\sffamily\large\bfseries}   
	{\thesubsection}{1em}{} 
\titleformat{\subsubsection}
	{\sffamily\normalsize\bfseries} 
	{\thesubsubsection}{1em}{}

% disponi alberi
\usepackage{forest}

\forestset{
	rectstyle/.style={
		for tree={rectangle,draw,font=\large\sffamily}
	},
	roundstyle/.style={
		for tree={circle,draw,font=\large}
	}
}

% disponi algoritmi
\usepackage{algorithm}
\usepackage{algorithmic}
\makeatletter
\renewcommand{\ALG@name}{Algoritmo}
\makeatother

% disponi numeri di pagina
\usepackage{fancyhdr}
\fancyhf{} 
\fancyfoot[L]{\sffamily{\thepage}}

\makeatletter
\fancyhead[L]{\raisebox{1ex}[0pt][0pt]{\sffamily{\@title \ \@date}}} 
\fancyhead[R]{\raisebox{1ex}[0pt][0pt]{\sffamily{\@author}}}
\makeatother

\begin{document}

% sezione (data)
\section{Lezione del 11-11-24}

% stili pagina
\thispagestyle{empty}
\pagestyle{fancy}

% testo
\subsection{Problemi esprimibili come flussi minimi}
Diversi problemi che abbiamo già visto possono essere formulati come problemi di flusso minimo su grafi.
In particolare, vediamo l'\textbf{assegnamento di costo minimo} e il problema di \textbf{trasporto}.

\subsubsection{Trasporto}
Avevamo definito un problema di trasporto come un problema LP in forma:
\[
	\begin{cases}
		\min \sum_{i=1}^m \sum_{j=1}^n c_{ij} x_{ij} \\ 
		\sum_{i=1}^m x_{ij} \geq d_{j}, \quad \forall i = 1, ..., n \\ 
		\sum_{j=1}^n x_{ij} \leq o_{ij}, \quad \forall j = 1, ..., m \\
		x_{ij} \geq 0
	\end{cases}
\]

Possiamo concettualizzare un problema di questo tipo come un problema di flusso minimo su un \textbf{grafo bipartito}:
\begin{definition}{Grafo bipartito}
	Si dice \textbf{bipartito} un grafo dove i nodi $N$ possono essere divisi in due insiemi disgiunti $U$ e $V$, dove ogni arco collega un nodo $U$ a un nodo $V$.
\end{definition}

Graficamente, il grafo bipartito di un problema di trasporto ha la forma:
\begin{center}
	\begin{tikzpicture}
		\node[circle, draw=black] (1) at (0,1) {$O_1$};
		\node[circle, draw=black] (2) at (0,-1) {$O_2$};
		\node[circle, draw=black] (3) at (2,2) {$D_1$};
		\node[circle, draw=black] (4) at (2,0) {$D_2$};
		\node[circle, draw=black] (5) at (2,-2) {$D_3$};

		\draw[->, to path={-| (\tikztotarget)}] (1) -- (3);
		\draw[->, to path={-| (\tikztotarget)}] (1) -- (4);
		\draw[->, to path={-| (\tikztotarget)}] (1) -- (5);

		\draw[->, to path={-| (\tikztotarget)}] (2) -- (3);
		\draw[->, to path={-| (\tikztotarget)}] (2) -- (4);
		\draw[->, to path={-| (\tikztotarget)}] (2) -- (5);

		\node at (-1, 1) {$-o_1$};
		\node at (-1, -1) {$-o_2$};

		\node at (3, 2) {$d_1$};
		\node at (3, 0) {$d_2$};
		\node at (3, -2) {$d_3$};
	\end{tikzpicture}
\end{center}
dove ogni nodo di \textbf{origine} $O_i$ eroga $o_i$ unità di flusso, e ogni nodo di domanda $D_i$ richiede $d_i$ unità di flusso.


\subsubsection{Assegnamento di costo minimo}
L'assegnamento di costo minimo può intedersi come un trasporto a \textbf{volume unitario}, cioè ogni nodo di origine eroga una singola unità di flusso, e ogni nodo di domanda richiede una singola unità di flusso.
Chiaramente, per $n$ nodi di domanda dovremo avere $n$ nodi di origine (cioè, per ogni "spazio" assegnabile abbiamo bisogno di un possibile assegnamento).
Questo si traduce sempre in un grafo bipartito, nella forma:
\begin{center}
	\begin{tikzpicture}
		\node[circle, draw=black] (1) at (0,2) {$1$};
		\node[circle, draw=black] (2) at (0,0) {$i$};
		\node[circle, draw=black] (3) at (0,-2) {$n$};
		\node[circle, draw=black] (4) at (2,2) {$1$};
		\node[circle, draw=black] (5) at (2,0) {$i$};
		\node[circle, draw=black] (6) at (2,-2) {$n$};

		\draw[->, to path={-| (\tikztotarget)}] (1) -- (4);
		\draw[->, to path={-| (\tikztotarget)}] (1) -- (5);
		\draw[->, to path={-| (\tikztotarget)}] (1) -- (6);

		\draw[->, to path={-| (\tikztotarget)}] (2) -- (4);
		\draw[->, to path={-| (\tikztotarget)}] (2) -- (5);
		\draw[->, to path={-| (\tikztotarget)}] (2) -- (6);

		\draw[->, to path={-| (\tikztotarget)}] (3) -- (4);
		\draw[->, to path={-| (\tikztotarget)}] (3) -- (5);
		\draw[->, to path={-| (\tikztotarget)}] (3) -- (6);

		\node at (-1, 2) {$-1$};
		\node at (-1, 0) {$-1$};
		\node at (-1, -2) {$-1$};

		\node at (3, 2) {$1$};
		\node at (3, 0) {$1$};
		\node at (3, -2) {$1$};
	\end{tikzpicture}
\end{center}

\subsection{Cammini minimi}
Abbiamo visto come l'ottimizzazione sui grafi può essere usata per risolvere problemi di flusso di costo minimo.
Vediamo un'altro tipo di problema che può essere risolto attraverso una versione modificata del simplesso sui grafi: quello dei \textbf{cammini a costo minimo}.

In generale, un problema dei cammini a costo minimo è un problema di ottimizzazione sui grafi in forma:
\[
	\begin{cases}
		\min c^\intercal \cdot x \\
		Ex = b \\ 
		x \geq 0
	\end{cases}
\]
che dà i cammini minimi da un nodo $r$ a tutti gli altri nodi $i \neq r$.
Questo si ottiene scegliendo il vettore $b$ appositamente:
\[
	b_i =
	\begin{cases}
		-(n - 1), \quad i = r \\ 
		1, \quad i \neq r
	\end{cases}
\]
cioè impostando il nodo $r$ come una sorgente da $n-1$ unità, e ogni nodo di arrivo $i \neq r$ come un pozzo di $1$ unità.

Notiamo che un albero di copertura $T_r$ radicato in $r$ rappresenta una base ammissibile del problema, in quanto porterà le $-(n-1)$ unità di flusso sugli $n-1$ nodi.
Inoltre, il flusso di base asssociato a $T_r$ sarà \textbf{non degenere}, cioè nessuno flusso sarà uguale a 0 (ogni nodo avrà in arrivo almeno 1 unità di flusso).
Possiamo quindi dire che il teorema di Bellman da solo basta a dimostrare l'ottimalià, e non occorrono regole anticiclo di Bland (non esistono soluzioni di base ammissibili degeneri).
Anzi, possiamo dire che conviene prendere l'arco con costo ridotto minimo, in quanto comporterà l'abbattimento maggiore del valore ottimo.

Infine, possiamo fare un'ottimizzazione significativa per quanto riguarda la rimozione degli archi in $\mathcal{C}^-$ (cioè quelli discordi all'arco introdotto $(p,q)$) nella fase di \textbf{rimozione dei cicli}.
Si ha che a ogni passso dell'algoritmo la soluzione ammissibile considerata è un albero di copertura, ergo ogni nodo sarà raggiunto.
L'algoritmo di Bellman restituirà quindi un modo \textit{più efficiente} di quello previsto dall'albero di arrivare ad un dato nodo $q$, partendo da un nodo $p$.
Avremo quindi che esiste sempre un nodo $(i,q)$ \textbf{entrante} in $q$, cioè il modo di arrivare in $q$ che era originariamente previsto dall'albero che abbiamo appena considerato.
Inoltre, si avrà che questo arco sarà l'ultimo a portare a $q$, o almeno sarà a costo minore dei suoi precedenti (ogni precedente dovrà portare a $q$ e a tutti gli eventuali nodi a cui arriva $q$).
Ergo, $(i, q)$ è a \textbf{flusso minimo}.
Possiamo quindi semplificare la regola dell'arco uscente per la rimozione di cicli.

Formuliamo allora il \textbf{simplesso per cammini}:
\begin{algorithm}[H]
\caption{del simplesso per cammini}
\begin{algorithmic}
	\STATE \textbf{Input:} un problema di cammini di costo minimo 
	\STATE \textbf{Output:} la soluzione ottima 
	\STATE Trova un albero $T$ di radice $r$ 
	\STATE \textsf{ciclo:}
	\STATE Calcola il potenziale di base $\bar{\pi}^\intercal = c_T^\intercal E_T^{-1}$
	\STATE Calcola i costi ridotti $c_{ij}^{\bar{\pi}} = c_{ij} + \bar{\pi}_{ij} - \bar{\pi}_{ij}$ per ogni arco
	\IF{$c_{ij}^{\overline{\pi}} \geq 0 \ \ \forall (i, j) \in L$}
	\STATE Fermati, $\bar{x}$ è un albero dei cammini minimi 
\ELSE
		\STATE Calcola l'arco entrante: 
		$$
		(p, q) = (i, j) \in L : c_{ij}^{\bar{\pi}} = \min \{ c_{ij}^{\bar{\pi}} \} 
		$$
		\STATE Chiama $\mathcal{C}$ il ciclo che l'arco $(p, q)$ forma con gli archi in $T$
		\STATE Fissa un orientamento concorde a $(p,q)$ su $\mathcal{C}$ e partiziona $\mathcal{C}$ in $\mathcal{C^+}$ archi concordi e $\mathcal{C^-}$ archi discordi a tale orientamento
	\ENDIF
	\IF{$\mathcal{C}^- = \emptyset$}
		\STATE Fermati, non esiste un albero dei cammini minimi 
	\ELSE
		\STATE Scegli come arco uscente l'unico arco $(i, q) \in T$ che entra in $q$ 
	\ENDIF
	\STATE Aggiorna l'albero come:
	$$
	T = T \setminus \left\{ (r,s) \right\} \cup \left\{ (p, q) \right\} 
	$$
	\STATE Torna a \textsf{ciclo}
\end{algorithmic}
\end{algorithm}

\subsubsection{Cammini minimi multiobiettivo}
\begin{problem}{Cammini minimi a due obiettivi}
	Vogliamo progettare un'applicazione di navigazione per dispositivi cellulari.
	L'applicazione usa una struttura dati a grafo per rappresentare località e le strade che collegano.
	Per ogni strada, si tiene conto di una distanza $d_{ij}$ in KM, e di un costo $c_{ij}$ al pedaggio.
	Si vuole trovare un algoritmo che trovi il percorso ottimo fra due località $i$ e $j$.
\end{problem}

Il problema presentato è un di ottimizzazione \textbf{multiobiettivo}: vogliamo \textit{minimizzare} sia il costo che la distanza.
Un'approccio è quello di impostare il problema di costo minimo sulle distanze, cioè come:
\[
	\begin{cases}
		\min d^\intercal \cdot x \\ 
		Ex = b \\ 
		x \geq 0 \\ 
	\end{cases}
\]
e di introdurre termine di \textit{budget}, cioè un valore massimo $C$ di costo che siamo disposti a pagare:
$$
c^\intercal x \leq C
$$

Questo approccio però non è propriamente ottimale, in quanto introducendo un nuovo vincolo, che potrebbe violare le condizoni di \textbf{interezza} rispettate dalla matrice di adiacenza, che sappiamo essere \textbf{unimodulare}.
Per continuare a prendere archi interi, dovremmo obbligatoriamente introdurre un vincolo $x_{ij} \in \{0, 1\}$, ergo trasformare il problema in un problema di ILP.
Vedremo quindi metodi più efficienti per ottimizzare problemi multiobiettivo di questo tipo.

\end{document}


\documentclass[a4paper,11pt]{article}
\usepackage[a4paper, margin=8em]{geometry}

% usa i pacchetti per la scrittura in italiano
\usepackage[french,italian]{babel}
\usepackage[T1]{fontenc}
\usepackage[utf8]{inputenc}
\frenchspacing 

% usa i pacchetti per la formattazione matematica
\usepackage{amsmath, amssymb, amsthm, amsfonts}

% usa altri pacchetti
\usepackage{gensymb}
\usepackage{hyperref}
\usepackage{standalone}

% imposta il titolo
\title{Appunti Ricerca Operativa}
\author{Luca Seggiani}
\date{2024}

% disegni
\usepackage{pgfplots}
\pgfplotsset{width=10cm,compat=1.9}

% imposta lo stile
% usa helvetica
\usepackage[scaled]{helvet}
% usa palatino
\usepackage{palatino}
% usa un font monospazio guardabile
\usepackage{lmodern}

\renewcommand{\rmdefault}{ppl}
\renewcommand{\sfdefault}{phv}
\renewcommand{\ttdefault}{lmtt}

% disponi il titolo
\makeatletter
\renewcommand{\maketitle} {
	\begin{center} 
		\begin{minipage}[t]{.8\textwidth}
			\textsf{\huge\bfseries \@title} 
		\end{minipage}%
		\begin{minipage}[t]{.2\textwidth}
			\raggedleft \vspace{-1.65em}
			\textsf{\small \@author} \vfill
			\textsf{\small \@date}
		\end{minipage}
		\par
	\end{center}

	\thispagestyle{empty}
	\pagestyle{fancy}
}
\makeatother

% disponi teoremi
\usepackage{tcolorbox}
\newtcolorbox[auto counter, number within=section]{theorem}[2][]{%
	colback=blue!10, 
	colframe=blue!40!black, 
	sharp corners=northwest,
	fonttitle=\sffamily\bfseries, 
	title=Teorema~\thetcbcounter: #2, 
	#1
}

% disponi definizioni
\newtcolorbox[auto counter, number within=section]{definition}[2][]{%
	colback=red!10,
	colframe=red!40!black,
	sharp corners=northwest,
	fonttitle=\sffamily\bfseries,
	title=Definizione~\thetcbcounter: #2,
	#1
}

% disponi problemi
\newtcolorbox[auto counter, number within=section]{problem}[2][]{%
	colback=green!10,
	colframe=green!40!black,
	sharp corners=northwest,
	fonttitle=\sffamily\bfseries,
	title=Problema~\thetcbcounter: #2,
	#1
}

% disponi codice
\usepackage{listings}
\usepackage[table]{xcolor}

\lstdefinestyle{codestyle}{
		backgroundcolor=\color{black!5}, 
		commentstyle=\color{codegreen},
		keywordstyle=\bfseries\color{magenta},
		numberstyle=\sffamily\tiny\color{black!60},
		stringstyle=\color{green!50!black},
		basicstyle=\ttfamily\footnotesize,
		breakatwhitespace=false,         
		breaklines=true,                 
		captionpos=b,                    
		keepspaces=true,                 
		numbers=left,                    
		numbersep=5pt,                  
		showspaces=false,                
		showstringspaces=false,
		showtabs=false,                  
		tabsize=2
}

\lstdefinestyle{shellstyle}{
		backgroundcolor=\color{black!5}, 
		basicstyle=\ttfamily\footnotesize\color{black}, 
		commentstyle=\color{black}, 
		keywordstyle=\color{black},
		numberstyle=\color{black!5},
		stringstyle=\color{black}, 
		showspaces=false,
		showstringspaces=false, 
		showtabs=false, 
		tabsize=2, 
		numbers=none, 
		breaklines=true
}

\lstdefinelanguage{javascript}{
	keywords={typeof, new, true, false, catch, function, return, null, catch, switch, var, if, in, while, do, else, case, break},
	keywordstyle=\color{blue}\bfseries,
	ndkeywords={class, export, boolean, throw, implements, import, this},
	ndkeywordstyle=\color{darkgray}\bfseries,
	identifierstyle=\color{black},
	sensitive=false,
	comment=[l]{//},
	morecomment=[s]{/*}{*/},
	commentstyle=\color{purple}\ttfamily,
	stringstyle=\color{red}\ttfamily,
	morestring=[b]',
	morestring=[b]"
}

% disponi sezioni
\usepackage{titlesec}

\titleformat{\section}
	{\sffamily\Large\bfseries} 
	{\thesection}{1em}{} 
\titleformat{\subsection}
	{\sffamily\large\bfseries}   
	{\thesubsection}{1em}{} 
\titleformat{\subsubsection}
	{\sffamily\normalsize\bfseries} 
	{\thesubsubsection}{1em}{}

% disponi alberi
\usepackage{forest}

\forestset{
	rectstyle/.style={
		for tree={rectangle,draw,font=\large\sffamily}
	},
	roundstyle/.style={
		for tree={circle,draw,font=\large}
	}
}

% disponi algoritmi
\usepackage{algorithm}
\usepackage{algorithmic}
\makeatletter
\renewcommand{\ALG@name}{Algoritmo}
\makeatother

% disponi numeri di pagina
\usepackage{fancyhdr}
\fancyhf{} 
\fancyfoot[L]{\sffamily{\thepage}}

\makeatletter
\fancyhead[L]{\raisebox{1ex}[0pt][0pt]{\sffamily{\@title \ \@date}}} 
\fancyhead[R]{\raisebox{1ex}[0pt][0pt]{\sffamily{\@author}}}
\makeatother

\begin{document}

% sezione (data)
\section{Lezione del 12-11-24}

% stili pagina
\thispagestyle{empty}
\pagestyle{fancy}

% testo
\subsection{Flusso di costo minimo capacitato}
In realtà, il modello di flusso di costo minimo prevede, su ogni arco, un'altra quantità: la \textbf{portata} (o \textbf{capacità}) massima dell'arco.
Avremo quindi che, oltre oltre agli $n$ bilanci $b_i$ e agli $m$ costi $c$ per $n$ nodi e $m$ archi, dovremo introdurre un nuovo vettore $u$ di dimensione $m$ per le \textbf{capacità superiori} di ogni arco.

Porremo quindi il nostro sistema come:
\[
	\begin{cases}
		\max c^T x \\ 
		Ex = b \\ 
		0 \leq x \leq u
	\end{cases}
\]
dove l'ultimo vincolo è l'espressione in forma vettoriale di $0 \leq x_{ij} \leq u_{ij} \ \forall (i, j) \in A$.

Avevamo notato che un problema di flusso di costo minimo \textit{non capacitato} si esprime agilmente in forma duale standard. 
Convverrà quindi esprimere il vincolo delle capacità superiori come uguaglianza, per ricondursi allo stesso formato:
\[
	\begin{cases}
		\max c^T x \\ 
		Ex = b \\
		x + w = u \\ 
		x \geq 0 \\ 
		w \geq 0
	\end{cases}
\]

Questa forma, con $w \in \mathbb{R^m}$, intende $x_ {ij} + w_{ij} = u_{ij}$, cioè gli $w_{ij}$ di ogni arco sono gli \textbf{scarti} dal valore di capacità massima. A $w$ negativi si avranno $x > u$, ergo saremo oltre la portata massima, mentre a $w = 0$ avremo "fissato" la capacità al massimo per la $x$ corrispondente.

Se volessimo restare nella forma matriciale, potremmo riscrivere il vincolo di capacità come:
$$
Ix + Iw = u
$$
sulla matrice identità $I$, e quindi esprimere a blocchi:
$$
\begin{pmatrix}
	E & 0 \\ 
	I & I
\end{pmatrix}
\begin{pmatrix}
	x \\ w
\end{pmatrix}
=
\begin{pmatrix}
	b \\ u
\end{pmatrix}
$$
dove la prima matrice ha dimensioni $(m + n) \times 2n$.

Se avevamo dimostrato che il rango di $Ex = b$ è $n - 1$, cioè quello delle matrici di base (che danno alberi di copertura), sarà abbastanza intuitivo che il rango di questa nuova matrice sarà $m + n - 1$.
In particolare, si avrà la seguente caratterizzazione delle basi:
\begin{theorem}{Caratterizzazione di base di matrici di incidenza capacitate}
	Presa la matrice di vincoli data da un problema di flusso di costo minimo capacitato, espressa in forma:
$$
\begin{pmatrix}
	E & 0 \\ 
	I & I
\end{pmatrix}
$$
si suppone di avere una \textbf{tripartizione} $T, L, U$ degli archi del grafo (cioè delle colonne della matrice), dove $T$ è un \textit{albero di copertura}. 
Si chiamano poi $T', L', U'$ le colonne corrispondenti alle variabili di scarto $w$.

A questo punto, $B = T \cup U \cup T' \cup L'$ sarà un base. 

\end{theorem}

Abbiamo che la tripartizione $T, L, U, T', L', U'$ è effettivamente una \textit{esapartizione} negli insiemi di colonne $T, L, U$ e le colonne con le $w$ associate.
Notiamo che, da $T$ di dimensione $n-1$ e $T'$ + $L'$ di dimensione pari a $m$, si ha che la dimensione della base è $m + n - 1$, come volevamo.

Vediamo quindi come calcolare il flusso di base corrispondente a una base:
$$
(x, w) = \left( E_T^{-1} (b - E_U u_U), 0, u_U, u_T - x_T, u_L, 0 \right)
$$

# qua sopra dubbi

\end{document}


\documentclass[a4paper,11pt]{article}
\usepackage[a4paper, margin=8em]{geometry}

% usa i pacchetti per la scrittura in italiano
\usepackage[french,italian]{babel}
\usepackage[T1]{fontenc}
\usepackage[utf8]{inputenc}
\frenchspacing 

% usa i pacchetti per la formattazione matematica
\usepackage{amsmath, amssymb, amsthm, amsfonts}

% usa altri pacchetti
\usepackage{gensymb}
\usepackage{hyperref}
\usepackage{standalone}

% imposta il titolo
\title{Appunti Ricerca Operativa}
\author{Luca Seggiani}
\date{2024}

% disegni
\usepackage{pgfplots}
\pgfplotsset{width=10cm,compat=1.9}

% imposta lo stile
% usa helvetica
\usepackage[scaled]{helvet}
% usa palatino
\usepackage{palatino}
% usa un font monospazio guardabile
\usepackage{lmodern}

\renewcommand{\rmdefault}{ppl}
\renewcommand{\sfdefault}{phv}
\renewcommand{\ttdefault}{lmtt}

% disponi il titolo
\makeatletter
\renewcommand{\maketitle} {
	\begin{center} 
		\begin{minipage}[t]{.8\textwidth}
			\textsf{\huge\bfseries \@title} 
		\end{minipage}%
		\begin{minipage}[t]{.2\textwidth}
			\raggedleft \vspace{-1.65em}
			\textsf{\small \@author} \vfill
			\textsf{\small \@date}
		\end{minipage}
		\par
	\end{center}

	\thispagestyle{empty}
	\pagestyle{fancy}
}
\makeatother

% disponi teoremi
\usepackage{tcolorbox}
\newtcolorbox[auto counter, number within=section]{theorem}[2][]{%
	colback=blue!10, 
	colframe=blue!40!black, 
	sharp corners=northwest,
	fonttitle=\sffamily\bfseries, 
	title=Teorema~\thetcbcounter: #2, 
	#1
}

% disponi definizioni
\newtcolorbox[auto counter, number within=section]{definition}[2][]{%
	colback=red!10,
	colframe=red!40!black,
	sharp corners=northwest,
	fonttitle=\sffamily\bfseries,
	title=Definizione~\thetcbcounter: #2,
	#1
}

% disponi problemi
\newtcolorbox[auto counter, number within=section]{problem}[2][]{%
	colback=green!10,
	colframe=green!40!black,
	sharp corners=northwest,
	fonttitle=\sffamily\bfseries,
	title=Problema~\thetcbcounter: #2,
	#1
}

% disponi codice
\usepackage{listings}
\usepackage[table]{xcolor}

\lstdefinestyle{codestyle}{
		backgroundcolor=\color{black!5}, 
		commentstyle=\color{codegreen},
		keywordstyle=\bfseries\color{magenta},
		numberstyle=\sffamily\tiny\color{black!60},
		stringstyle=\color{green!50!black},
		basicstyle=\ttfamily\footnotesize,
		breakatwhitespace=false,         
		breaklines=true,                 
		captionpos=b,                    
		keepspaces=true,                 
		numbers=left,                    
		numbersep=5pt,                  
		showspaces=false,                
		showstringspaces=false,
		showtabs=false,                  
		tabsize=2
}

\lstdefinestyle{shellstyle}{
		backgroundcolor=\color{black!5}, 
		basicstyle=\ttfamily\footnotesize\color{black}, 
		commentstyle=\color{black}, 
		keywordstyle=\color{black},
		numberstyle=\color{black!5},
		stringstyle=\color{black}, 
		showspaces=false,
		showstringspaces=false, 
		showtabs=false, 
		tabsize=2, 
		numbers=none, 
		breaklines=true
}

\lstdefinelanguage{javascript}{
	keywords={typeof, new, true, false, catch, function, return, null, catch, switch, var, if, in, while, do, else, case, break},
	keywordstyle=\color{blue}\bfseries,
	ndkeywords={class, export, boolean, throw, implements, import, this},
	ndkeywordstyle=\color{darkgray}\bfseries,
	identifierstyle=\color{black},
	sensitive=false,
	comment=[l]{//},
	morecomment=[s]{/*}{*/},
	commentstyle=\color{purple}\ttfamily,
	stringstyle=\color{red}\ttfamily,
	morestring=[b]',
	morestring=[b]"
}

% disponi sezioni
\usepackage{titlesec}

\titleformat{\section}
	{\sffamily\Large\bfseries} 
	{\thesection}{1em}{} 
\titleformat{\subsection}
	{\sffamily\large\bfseries}   
	{\thesubsection}{1em}{} 
\titleformat{\subsubsection}
	{\sffamily\normalsize\bfseries} 
	{\thesubsubsection}{1em}{}

% disponi alberi
\usepackage{forest}

\forestset{
	rectstyle/.style={
		for tree={rectangle,draw,font=\large\sffamily}
	},
	roundstyle/.style={
		for tree={circle,draw,font=\large}
	}
}

% disponi algoritmi
\usepackage{algorithm}
\usepackage{algorithmic}
\makeatletter
\renewcommand{\ALG@name}{Algoritmo}
\makeatother

% disponi numeri di pagina
\usepackage{fancyhdr}
\fancyhf{} 
\fancyfoot[L]{\sffamily{\thepage}}

\makeatletter
\fancyhead[L]{\raisebox{1ex}[0pt][0pt]{\sffamily{\@title \ \@date}}} 
\fancyhead[R]{\raisebox{1ex}[0pt][0pt]{\sffamily{\@author}}}
\makeatother

\begin{document}

% sezione (data)
\section{Lezione del 13-11-24}

% stili pagina
\thispagestyle{empty}
\pagestyle{fancy}

% testo
\subsection{Potenziali di base capacitati}
Abbiamo visto come calcolare i flussi di base, data un'opportuna tripartizione, sulla base $T, U, T', L'$ di un problema di flusso minimo capacitato.
Vediamo adesso come calcolare i potenziali di base.
Si imposta innanzitutto il duale, cioè il \textbf{problema dei potenziali}: 
\[
	\begin{cases}
		\max b^T \pi + u^T \mu \\ 
		E^T \pi + \mu \leq c \\ 
		\mu \leq 0
	\end{cases}
\]
dove $\mu$ rappresenta gli \textbf{scarti} ai potenziali. 

Notiamo che avremmo potuto usare la stessa matrice dei vincoli nel primale seguendo le stesse uguaglianze già viste sulle forme di matrici primali e duali:
\[
	\begin{cases}
		\min
		\begin{pmatrix}
			x & w
		\end{pmatrix}^T
		\begin{pmatrix}
			c \\ 0
		\end{pmatrix} \\ 
		\begin{pmatrix}
			x & w
		\end{pmatrix}^T
		\begin{pmatrix}
			E^T & I \\ 
			0 & I 
		\end{pmatrix}
		=
		\begin{pmatrix}
			b & u
		\end{pmatrix}^T \\ 
		\begin{pmatrix}
			x & w
		\end{pmatrix}
		\geq 0
	\end{cases}
\]

Lo stesso problema potrebbe essere stato espresso nella forma a blocchi, come avevamo visto sul primale:
\[
	\begin{cases}
		\max 
		\begin{pmatrix}
			b^T & u^T
		\end{pmatrix}
		\begin{pmatrix}
			\pi \\ \mu
		\end{pmatrix}\\
		\begin{pmatrix}
			E^T & I \\ 
			0 & I
		\end{pmatrix}
		\begin{pmatrix}
			\pi \\ \mu
		\end{pmatrix}
			\geq
		\begin{pmatrix}
			c \\ 0
		\end{pmatrix}
	\end{cases}
\]

Vediamo quindi il calcolo vero e proprio.
Si tratta il problema come qualsiasi altro problema in formato primale, cioè si rendono valide le equazioni date dalla base $T, U, T', L'$.
Avremo quindi che $\mu_T = 0$ e $\mu_L = 0$.
A questo punto troviamo i $\pi$ dati da $E_T^T \mu = c_T$, su $T$, $L$ e $U$.
Infine, abbiamo i $\mu$ su $U$, dati da $\pi_T E_U + \mu_U^T = c^T$.
Cioè, riassumendo, secondo la stessa notazione usata per i flussi di base (si noti che il flusso in $T$ è soluzione di un unico sistema):
$$
(\pi(T), \mu(T'. L', U')) = \left(c_T^T E_T^{-1}, 0, 0, c_U^T - \pi^T E_U \right)
$$

Dai potenziali possiamo calcolare i costi ridotti, analogamente a come avevamo fatto sui flussi non capacitati:
$$
c^\pi_{ij} = c_{ij} - \pi_i + \pi_j
$$
e dimostrare una variante del teorema di Bellman:

\begin{theorem}{di Bellman capacitato}
	Supponiamo di avere una tripartizione $T, L, U$ che generi un flusso di base ammissibile.
	Se la soluzione è anche ottima, vale riguardo ai costi ridotti:
	\[
		\begin{cases}
			c_ {ij}^T, \geq 0 \quad \forall (i,j) \in L \\ 
			c_ {ij}^T \leq 0, \quad \forall (i,j) \in U
		\end{cases}
	\]	
\end{theorem}

\end{document}


\documentclass[a4paper,11pt]{article}
\usepackage[a4paper, margin=8em]{geometry}

% usa i pacchetti per la scrittura in italiano
\usepackage[french,italian]{babel}
\usepackage[T1]{fontenc}
\usepackage[utf8]{inputenc}
\frenchspacing 

% usa i pacchetti per la formattazione matematica
\usepackage{amsmath, amssymb, amsthm, amsfonts}

% usa altri pacchetti
\usepackage{gensymb}
\usepackage{hyperref}
\usepackage{standalone}

% imposta il titolo
\title{Appunti Ricerca Operativa}
\author{Luca Seggiani}
\date{2024}

% disegni
\usepackage{pgfplots}
\pgfplotsset{width=10cm,compat=1.9}

% imposta lo stile
% usa helvetica
\usepackage[scaled]{helvet}
% usa palatino
\usepackage{palatino}
% usa un font monospazio guardabile
\usepackage{lmodern}

\renewcommand{\rmdefault}{ppl}
\renewcommand{\sfdefault}{phv}
\renewcommand{\ttdefault}{lmtt}

% disponi il titolo
\makeatletter
\renewcommand{\maketitle} {
	\begin{center} 
		\begin{minipage}[t]{.8\textwidth}
			\textsf{\huge\bfseries \@title} 
		\end{minipage}%
		\begin{minipage}[t]{.2\textwidth}
			\raggedleft \vspace{-1.65em}
			\textsf{\small \@author} \vfill
			\textsf{\small \@date}
		\end{minipage}
		\par
	\end{center}

	\thispagestyle{empty}
	\pagestyle{fancy}
}
\makeatother

% disponi teoremi
\usepackage{tcolorbox}
\newtcolorbox[auto counter, number within=section]{theorem}[2][]{%
	colback=blue!10, 
	colframe=blue!40!black, 
	sharp corners=northwest,
	fonttitle=\sffamily\bfseries, 
	title=Teorema~\thetcbcounter: #2, 
	#1
}

% disponi definizioni
\newtcolorbox[auto counter, number within=section]{definition}[2][]{%
	colback=red!10,
	colframe=red!40!black,
	sharp corners=northwest,
	fonttitle=\sffamily\bfseries,
	title=Definizione~\thetcbcounter: #2,
	#1
}

% disponi problemi
\newtcolorbox[auto counter, number within=section]{problem}[2][]{%
	colback=green!10,
	colframe=green!40!black,
	sharp corners=northwest,
	fonttitle=\sffamily\bfseries,
	title=Problema~\thetcbcounter: #2,
	#1
}

% disponi codice
\usepackage{listings}
\usepackage[table]{xcolor}

\lstdefinestyle{codestyle}{
		backgroundcolor=\color{black!5}, 
		commentstyle=\color{codegreen},
		keywordstyle=\bfseries\color{magenta},
		numberstyle=\sffamily\tiny\color{black!60},
		stringstyle=\color{green!50!black},
		basicstyle=\ttfamily\footnotesize,
		breakatwhitespace=false,         
		breaklines=true,                 
		captionpos=b,                    
		keepspaces=true,                 
		numbers=left,                    
		numbersep=5pt,                  
		showspaces=false,                
		showstringspaces=false,
		showtabs=false,                  
		tabsize=2
}

\lstdefinestyle{shellstyle}{
		backgroundcolor=\color{black!5}, 
		basicstyle=\ttfamily\footnotesize\color{black}, 
		commentstyle=\color{black}, 
		keywordstyle=\color{black},
		numberstyle=\color{black!5},
		stringstyle=\color{black}, 
		showspaces=false,
		showstringspaces=false, 
		showtabs=false, 
		tabsize=2, 
		numbers=none, 
		breaklines=true
}

\lstdefinelanguage{javascript}{
	keywords={typeof, new, true, false, catch, function, return, null, catch, switch, var, if, in, while, do, else, case, break},
	keywordstyle=\color{blue}\bfseries,
	ndkeywords={class, export, boolean, throw, implements, import, this},
	ndkeywordstyle=\color{darkgray}\bfseries,
	identifierstyle=\color{black},
	sensitive=false,
	comment=[l]{//},
	morecomment=[s]{/*}{*/},
	commentstyle=\color{purple}\ttfamily,
	stringstyle=\color{red}\ttfamily,
	morestring=[b]',
	morestring=[b]"
}

% disponi sezioni
\usepackage{titlesec}

\titleformat{\section}
	{\sffamily\Large\bfseries} 
	{\thesection}{1em}{} 
\titleformat{\subsection}
	{\sffamily\large\bfseries}   
	{\thesubsection}{1em}{} 
\titleformat{\subsubsection}
	{\sffamily\normalsize\bfseries} 
	{\thesubsubsection}{1em}{}

% disponi alberi
\usepackage{forest}

\forestset{
	rectstyle/.style={
		for tree={rectangle,draw,font=\large\sffamily}
	},
	roundstyle/.style={
		for tree={circle,draw,font=\large}
	}
}

% disponi algoritmi
\usepackage{algorithm}
\usepackage{algorithmic}
\makeatletter
\renewcommand{\ALG@name}{Algoritmo}
\makeatother

% disponi numeri di pagina
\usepackage{fancyhdr}
\fancyhf{} 
\fancyfoot[L]{\sffamily{\thepage}}

\makeatletter
\fancyhead[L]{\raisebox{1ex}[0pt][0pt]{\sffamily{\@title \ \@date}}} 
\fancyhead[R]{\raisebox{1ex}[0pt][0pt]{\sffamily{\@author}}}
\makeatother

\begin{document}

% sezione (data)
\section{Lezione del 14-11-24}

% stili pagina
\thispagestyle{empty}
\pagestyle{fancy}

% testo
\subsection{Simplesso per flussi capacitati}
Vediamo quindi come applicare l'algoritmo del simplesso ai problemi di flusso minimo capacitato.
Si ha che, dal teorema di Bellman, gli archi entranti saranno quelli che violano i vincoli:
\[
	\begin{cases}
		c_{ij}^\pi \geq 0, \quad \forall (i, j) \in L \\ 
		c_{ij}^\pi \leq 0, \quad \forall (i, j) \in U
	\end{cases}
\]

Il fatto che un arco entrante può appartenere sia a $L$ che a $U$ significherà che dovremo fare delle considerazioni diverse in fase  di elminazione dei cicli e in fase di inserzione dell'arco fra gli insiemi $T, L, U$ 

Troviamo quindi questo arco entrante $(p, q) \in L \cup U$, discriminando attraverso la regola anticiclo di Bland sull'ordine lessicografico degli archi. 
Vorremo eliminare il ciclo formato introducendovi $\vartheta \geq 0$ unità di flusso.

\begin{itemize}
	\item 
Nel caso $(p,q) \in L$, introdurremo le unità $\vartheta$ nella direzione \textbf{concorde} al flusso per cercare di \textit{riempire} l'arco.
	\item 
In caso contrario, con $(p,q) \in U$, introdurremo le unità $\vartheta$ nella direzione \textbf{discorde} al flusso, per cercare di \textit{svuotare} l'arco (ricordiamo che gli archi in $U$ sono quelli forzati pieni).
\end{itemize}

Scelta quindi una direzione per le unità $\vartheta$, avremo che i flussi sul ciclo si evolvono come:
$$
x_\vartheta =
	\begin{cases}
		\overline{x}_{ij} + \vartheta, \quad \forall (i, j) \in \mathcal{C}^+ \\ 
		\overline{x}_{ij} - \vartheta, \quad \forall (i, j) \in \mathcal{C}^- \\
		\overline{x}_{ij}, \quad \forall (i, j) \notin C
	\end{cases}
$$
stabilite le partizioni concordi e discordi sul ciclo $\mathcal{C}^+$ e $\mathcal{C}^-$.

Notiamo che il flusso trovato aggiungendo o rimuovendo $\vartheta$ unità di flusso, che chiamiamo $x_\vartheta$, rispetta Bellman, in quanto per un nodo posto in qualsiasi posizione fra archi concordi e discordi del ciclo, modificando i flussi secondo la regola riportata sopra non modificheremo mai il bilancio complessivo.

Questo si può dimostrare per enumerazione completa: preso un nodo $n_c$ ad arbitrio su $\mathcal{C}$, si hanno i casi:
\begin{itemize}
	\item $n_c$ si trova fra due archi discordi: uno sarà discorde e l'altro concorde alla direzione di percorrenza, ergo avremo un unità di flusso in meno da un lato e una in più dall'altro.
	\item $n_c$ si trova fra due archi concordi: questi cresceranno o diminuiranno di uno, ma in entrambi i casi l'unità in più (o in meno) di un arco dovrà essere fornita (ottenuta) dall'altro arco.
\end{itemize}

Ciò che cambierà invece saranno i costi, in quanto assunto che $c_{pq}^\pi$ è uguale al costo del ciclo, si avrà che il percorso fatto sul ciclo senza $(p,q)$ costerà la differenza di potenziale $c_{cic} = \pi_{p} - \pi_{q}$, e da $c_{pq} < c_{cic}$ sarà chiaramente più vantaggioso passare per $(p,q)$.

Avremo quindi due condizioni di arresto per la variabile $\vartheta$:
\begin{itemize}
	\item Il primo caso è che un arco in $\mathcal{C}^-$ si annulli, cioè quello che avevamo visto sul flusso non capacitato, che si applica all'arco $(r,s) \in \mathcal{C}^-$ con $x_{rs} = \vartheta$
	\item Il secondo caso è che un arco in $\mathcal{C}^+$ arrivi a capienza totale, e si applica all'arco $(r,s) \in \mathcal{C}^-$ con $u_{rs} - x_{rs} = \vartheta$.
\end{itemize}

Notiamo che dai due casi non è possibile dare una definizione univoca di $\vartheta$. Diciamo allora:
$$
\vartheta^+ = \min\{ u_{ij} - x_{ij}, \, (i, j) \in \mathcal{C}^+ \}
$$
$$
\vartheta^- = \min\{ x_{\ij}, \, (i, j) \in \mathcal{C}^- \}
$$
da cui:
$$
\vartheta = \min\{\vartheta^+, \vartheta^-\}
$$
cioè definiamo separatamente il $\vartheta^+$ del primo arco saturo su $\mathcal{C}^+$, e il $\vartheta^-$ del primo arco vuoto su $\mathcal{C}^-$.
Di conseguenza, $\vartheta$ sarà il minimo fra $\vartheta^+$ e $\vartheta^-$, cioè il primo che porta a una violazione dei vincoli.

Notiamo infine come $\vartheta$ può tendere a $\infty$: questo è il caso già visto dei cicli a costo negativo, e porta la soluzione del problema a $-\infty$.

Decideremo quindi di scegliere come arco uscente il primo arco $(r,s)$, dalla regola anticiclo di Bland sull'ordine lessicografico degli archi, che ha $u_{rs} - x_{rs} = \vartheta$ per $(r,s) \in \mathcal{C}^+$ o $x_{rs} = \vartheta$ per $(r,s) \in \mathcal{C}^-$.

Resta quindi il problema di aggiornare le partizioni $T, L, U$ sulla base degli archi $(p,q)$ entrante e $(r,s)$ uscente scelti.
Dovremo effettivamente porci due domande:
\begin{itemize}
	\item L'arco entrante $(p,q)$ è vuoto ($\in L$) o saturo ($\in U$)?
	\item L'arco uscente $(r,s)$ esce dopo essere svuotato ($\in \mathcal{C}^-$) o saturato ($\in \mathcal{C}^+$)?
\end{itemize}

Iniziamo a discriminare dalla prima domanda.
\begin{itemize}
	\item $(p, q) \in L$: 
		\begin{itemize}
			\item $(r,s) \in \mathcal{C}^-$: l'arco entrante è vuoto e l'uscente esce svuotato. Vorremo spostare l'arco entrante in $T$ e l'arco uscente in in $L$, cioè:
				$$ T = T \setminus (r,s) \cup (p,q), \quad L = L \setminus (p, q) \cup (r,s) $$
			\item $(r,s) \in \mathcal{C}^+$: l'arco entrante è vuoto e l'uscente esce saturato. Vorremo spostare l'arco entrante in $T$ e l'arco uscente in in $U$, cioè:
				$$ T = T \setminus (r,s) \cup (p,q), \quad L = L \setminus (p, q), \quad U = U \cup (r, s) $$
				Notiamo che in questa situazione può presentarsi il caso dove $(p,q) = (r,s)$, cioè l'arco $(p,q)$ entra da vuoto e esce saturato, cioè semplicemente passa da $L$ a $U$.
		\end{itemize}
	\item $(p,q) \in U$:
		\begin{itemize}
			\item $(r,s) \in \mathcal{C}^-$: l'arco entrante è pieno e l'uscente esce svuotato. Vorremo spostare l'arco entrante in $T$ e l'arco uscente in in $L$, cioè:
				$$ T = T \setminus (r,s) \cup (p,q), \quad L = L \cup (r,s), \quad U = U \setminus (p,q) $$
				Notiamo che in questa situazione può presentarsi il caso dove $(p,q) = (r,s)$, cioè l'arco $(p,q)$ entra da pieno e esce svuotato, cioè semplicemente passa da $U$ a $l$.
			\item $(r,s) \in \mathcal{C}^+$: l'arco entrante è pieno e l'uscente esce saturato. Vorremo spostare l'arco entrante in $T$ e l'arco uscente in in $U$, cioè:
				$$ T = T \setminus (r,s) \cup (p,q), \quad U = U \setminus (p, q) \cup (r,s) $$
		\end{itemize}
\end{itemize}

Possiamo quindi formulare l'algoritmo:
\begin{algorithm}[H]
\caption{del simplesso per flussi capacitati}
\begin{algorithmic}
	\STATE \textbf{Input:} un problema di flusso di costo minimo capacitato
	\STATE \textbf{Output:} la soluzione ottima 
	\STATE Trova una partizione degli archi $(T,L,U)$ con $T$ albero di copertura che genera un flusso ammissibile
	\STATE \textsf{ciclo:}
	\STATE Calcola il flusso di base capacitato:
	$$
	\bar{x} = \left(E_T^{-1} (b - E_U u_U), 0, u_U, u_T - x_T, u_L, 0\right) $$
	e il potenziale di base capacitato:
	$$
	\bar{\pi} = \left( c_T^\intercal E_T^{-1}, 0, 0, c_U^\intercal - \pi^\intercal E_U \right)
	$$
	\STATE Calcola i costi ridotti $c_{ij}^{\bar{\pi}} = c_{ij} + \bar{\pi}_{ij} - \bar{\pi}_{ij}$ per ogni arco
	\IF{Bellman è soddisfatto, cioè:
	$$
c_{ij}^{\overline{\pi}} \geq 0 \ \ \forall (i, j) \in L
$$
$$
c_{ij}^{\overline{\pi}} \leq 0 \ \ \forall (i, j) \in U
$$}
	\STATE Fermati, $\bar{x}$ è un flusso ottimo e $\bar{\pi}$ è un potenziale ottimo
\ELSE
		\STATE Calcola l'arco entrante: 
		$$
		(p, q) = \min \left\{ \{ (i, j) \in L : c_{ij}^{\bar{\pi}} < 0 \} \cup \{ (i, j) \in U : c_{ij}^{\bar{\pi}} > 0 \} \right\}
		$$
		stabilito l'ordinamento lessicografico degli archi
		\STATE Chiama $\mathcal{C}$ il ciclo che l'arco $(p, q)$ forma con gli archi in $T$
		\STATE Fissa un orientamento concorde a $(p,q)$ su $\mathcal{C}$ se $(p,q) \in L$, e discorde se $(p,q) \in U$ e partiziona $\mathcal{C}$ in $\mathcal{C^+}$ archi concordi e $\mathcal{C^-}$ archi discordi a tale orientamento
	\ENDIF
		\STATE Calcola:
		$$
		\vartheta^+ = \min\{ u_{ij} - \bar{x}_{ij} : (i, j) \in \mathcal{C}^+ \}
		$$
		$$
		\vartheta^- = \min\{ \bar{x}_{ij} : (i, j) \in \mathcal{C}^- \}
		$$
		$$
		\vartheta = \min\{\vartheta^+, \vartheta^-\}
		$$

\end{algorithmic}
\end{algorithm}
\begin{algorithm}
\begin{algorithmic}
	\IF{$\vartheta = \infty$}
		\STATE Fermati, il flusso di costo minimo ha valore $-\infty$ 
	\ELSE
		\STATE Trova l'arco uscente: 
		$$ 
		(r, s) = \min \left\{ \{(i, j) \in \mathcal{C}^+ : u_{ij} - \bar{x}_{ij} = \vartheta \} \cup \{(i, j) \in \mathcal{C}^- : \bar{x}_{ij} = \vartheta \} \right\} 
		$$
		stabilito un ordinamento lessicografico degli archi
	\ENDIF
	\STATE Aggiorna le partizioni come:
	
	\IF{$(p,q) \in L$}
		\IF{$(r,s) \in \mathcal{C}^-$}
			\STATE $ T = T \setminus (r,s) \cup (p,q), \, L = L \setminus (p, q) \cup (r,s) $
		\ELSE
			\IF{$(p,q) = (r,s)$}
				\STATE $ L = L \setminus (p, q), \, U = U \cup (p, q) $
			\ELSE
				\STATE $ T = T \setminus (r,s) \cup (p,q), \, L = L \setminus (p, q), \, U = U \cup (r, s) $
			\ENDIF
		\ENDIF
	\ELSE
		\IF{$(r,s) \in \mathcal{C}^-$}
			\IF{$(p,q) = (r,s)$}
				\STATE $ L = L \cup (p, q), \, U = U \setminus (p, q) $
			\ELSE
				\STATE $ T = T \setminus (r,s) \cup (p,q), \, L = L \cup (r,s), \, U = U \setminus (p,q) $
			\ENDIF
		\ELSE
				\STATE $ T = T \setminus (r,s) \cup (p,q), \, U = U \setminus (p, q) \cup (r,s) $
		\ENDIF
	\ENDIF
	
	\STATE Torna a \textsf{ciclo}
\end{algorithmic}
\end{algorithm}

Notiamo che le stesse ottimizzazioni che avevamo visto sul simplesso per i flussi valgono per il simplesso per i flussi capacitati.


\end{document}


\documentclass[a4paper,11pt]{article}
\usepackage[a4paper, margin=8em]{geometry}

% usa i pacchetti per la scrittura in italiano
\usepackage[french,italian]{babel}
\usepackage[T1]{fontenc}
\usepackage[utf8]{inputenc}
\frenchspacing 

% usa i pacchetti per la formattazione matematica
\usepackage{amsmath, amssymb, amsthm, amsfonts}

% usa altri pacchetti
\usepackage{gensymb}
\usepackage{hyperref}
\usepackage{standalone}

% imposta il titolo
\title{Appunti Ricerca Operativa}
\author{Luca Seggiani}
\date{2024}

% disegni
\usepackage{pgfplots}
\pgfplotsset{width=10cm,compat=1.9}

% imposta lo stile
% usa helvetica
\usepackage[scaled]{helvet}
% usa palatino
\usepackage{palatino}
% usa un font monospazio guardabile
\usepackage{lmodern}

\renewcommand{\rmdefault}{ppl}
\renewcommand{\sfdefault}{phv}
\renewcommand{\ttdefault}{lmtt}

% disponi il titolo
\makeatletter
\renewcommand{\maketitle} {
	\begin{center} 
		\begin{minipage}[t]{.8\textwidth}
			\textsf{\huge\bfseries \@title} 
		\end{minipage}%
		\begin{minipage}[t]{.2\textwidth}
			\raggedleft \vspace{-1.65em}
			\textsf{\small \@author} \vfill
			\textsf{\small \@date}
		\end{minipage}
		\par
	\end{center}

	\thispagestyle{empty}
	\pagestyle{fancy}
}
\makeatother

% disponi teoremi
\usepackage{tcolorbox}
\newtcolorbox[auto counter, number within=section]{theorem}[2][]{%
	colback=blue!10, 
	colframe=blue!40!black, 
	sharp corners=northwest,
	fonttitle=\sffamily\bfseries, 
	title=Teorema~\thetcbcounter: #2, 
	#1
}

% disponi definizioni
\newtcolorbox[auto counter, number within=section]{definition}[2][]{%
	colback=red!10,
	colframe=red!40!black,
	sharp corners=northwest,
	fonttitle=\sffamily\bfseries,
	title=Definizione~\thetcbcounter: #2,
	#1
}

% disponi problemi
\newtcolorbox[auto counter, number within=section]{problem}[2][]{%
	colback=green!10,
	colframe=green!40!black,
	sharp corners=northwest,
	fonttitle=\sffamily\bfseries,
	title=Problema~\thetcbcounter: #2,
	#1
}

% disponi codice
\usepackage{listings}
\usepackage[table]{xcolor}

\lstdefinestyle{codestyle}{
		backgroundcolor=\color{black!5}, 
		commentstyle=\color{codegreen},
		keywordstyle=\bfseries\color{magenta},
		numberstyle=\sffamily\tiny\color{black!60},
		stringstyle=\color{green!50!black},
		basicstyle=\ttfamily\footnotesize,
		breakatwhitespace=false,         
		breaklines=true,                 
		captionpos=b,                    
		keepspaces=true,                 
		numbers=left,                    
		numbersep=5pt,                  
		showspaces=false,                
		showstringspaces=false,
		showtabs=false,                  
		tabsize=2
}

\lstdefinestyle{shellstyle}{
		backgroundcolor=\color{black!5}, 
		basicstyle=\ttfamily\footnotesize\color{black}, 
		commentstyle=\color{black}, 
		keywordstyle=\color{black},
		numberstyle=\color{black!5},
		stringstyle=\color{black}, 
		showspaces=false,
		showstringspaces=false, 
		showtabs=false, 
		tabsize=2, 
		numbers=none, 
		breaklines=true
}

\lstdefinelanguage{javascript}{
	keywords={typeof, new, true, false, catch, function, return, null, catch, switch, var, if, in, while, do, else, case, break},
	keywordstyle=\color{blue}\bfseries,
	ndkeywords={class, export, boolean, throw, implements, import, this},
	ndkeywordstyle=\color{darkgray}\bfseries,
	identifierstyle=\color{black},
	sensitive=false,
	comment=[l]{//},
	morecomment=[s]{/*}{*/},
	commentstyle=\color{purple}\ttfamily,
	stringstyle=\color{red}\ttfamily,
	morestring=[b]',
	morestring=[b]"
}

% disponi sezioni
\usepackage{titlesec}

\titleformat{\section}
	{\sffamily\Large\bfseries} 
	{\thesection}{1em}{} 
\titleformat{\subsection}
	{\sffamily\large\bfseries}   
	{\thesubsection}{1em}{} 
\titleformat{\subsubsection}
	{\sffamily\normalsize\bfseries} 
	{\thesubsubsection}{1em}{}

% disponi alberi
\usepackage{forest}

\forestset{
	rectstyle/.style={
		for tree={rectangle,draw,font=\large\sffamily}
	},
	roundstyle/.style={
		for tree={circle,draw,font=\large}
	}
}

% disponi algoritmi
\usepackage{algorithm}
\usepackage{algorithmic}
\makeatletter
\renewcommand{\ALG@name}{Algoritmo}
\makeatother

% disponi numeri di pagina
\usepackage{fancyhdr}
\fancyhf{} 
\fancyfoot[L]{\sffamily{\thepage}}

\makeatletter
\fancyhead[L]{\raisebox{1ex}[0pt][0pt]{\sffamily{\@title \ \@date}}} 
\fancyhead[R]{\raisebox{1ex}[0pt][0pt]{\sffamily{\@author}}}
\makeatother

\begin{document}

% sezione (data)
\section{Lezione del 18-11-24}

% stili pagina
\thispagestyle{empty}
\pagestyle{fancy}

% testo
\subsection{Problema di flusso massimo}
Poniamo di avere un grafo su cui riportiamo solamente le capacità superiori $u_{ij}$ sui singoli archi: 

\begin{center}
	\begin{tikzpicture}
		\node[circle, draw=black] (1) at (0,0) {1};
		\node[circle, draw=black] (2) at (2,1) {2};
		\node[circle, draw=black] (3) at (2,-1) {3};
		\node[circle, draw=black] (4) at (4,1) {4};
		\node[circle, draw=black] (5) at (4,-1) {5};
		\node[circle, draw=black] (6) at (6, 0) {6};
		\draw[->, to path={-| (\tikztotarget)}] (1) -- (2);
		\draw[->, to path={-| (\tikztotarget)}] (1) -- (3);
		\draw[->, to path={-| (\tikztotarget)}] (2) -- (3);
		\draw[->, to path={-| (\tikztotarget)}] (2) -- (4);
		\draw[->, to path={-| (\tikztotarget)}] (3) -- (5);
		\draw[->, to path={-| (\tikztotarget)}] (5) -- (4);
		\draw[->, to path={-| (\tikztotarget)}] (3) -- (4);
		\draw[->, to path={-| (\tikztotarget)}] (4) -- (6);
		\draw[->, to path={-| (\tikztotarget)}] (5) -- (6);

		\node at (1, 1)  {$8$};
		\node at (1, -1)  {$9$};

		\node at (3, 1.5)  {$2$};
		\node at (3, -1.5)  {$7$};
		\node at (3, 0.5)  {$8$};

		\node at (1.5, 0)  {$3$};
		\node at (4.5, 0)  {$6$};

		\node at (5, 1)  {$10$};
		\node at (5, -1)  {$4$};
	\end{tikzpicture}
\end{center}

Prendiamo due nodi $s$ e $t$, e cerchiamo di \textbf{massimizzare} il flusso da $s$ a $t$.
L'idea fondamentale che il flusso che parte da $s$ dovrà essere uguale al flusso che arriva in $t$.
Potremo partire da un flusso ammissibile qualsiasi, cioè che si limita a rispettare le capacità:
$$
x = \left( 2, 4, 0, 2, 0, 4, 2, 0, 4 \right)
$$
Notando che le capacità sugli \textit{ultimi} archi influenzano quelle sugli archi precedenti (ad esempio, $(3,5)$ è limitato a 4 da $(5,6)$ con $u_{56} = 4$).
Poniamo allora, più intelligentemente, un problema di PL:
\[
	\begin{cases}
		\max v \\ 
		Ex = b \\ 
		0 \leq x \leq u
	\end{cases}
\]
dove $E$ è la matrice di incidenza della rete, e i bilanci $b_i$ stessi dipendono dalla variabile $v$: 
\[
	b_i = 
	\begin{cases}
		-v, \quad i = s \\ 
		0, \quad i \neq s \wedge i \neq t \\ 
		v, \quad i = t
	\end{cases}
\]

Potremmo avere dubbi sul fatto che la  variabile $v$ compare nei bilanci $b$.
Scriviamo per esteso le uguaglianze dei vincoli:
\[
	\begin{cases}
		-x_{12} - x_{13} = -v \\ 
		x_{12} - x_{23} - x_{24} = 0 \\ 
		x_{13} + x_{23} - x_{34} - x_{35} = 0 \\ 
		x_{24} + x_{34} + x_{54} - x_{46} = 0 \\ 
		x_{35} - x_{54} - x_{56} = 0 \\ 
		x_{46} + x_{56} = v
	\end{cases}
\]

Notiamo che questo problema è effettivamente un problema di flusso minimo (massimo) su $n+1$ variabili per $n$ nodi, dove la $n$-esima variabile è proprio il flusso $v$.
Inoltre, con capacità massime intere, anche $v$ sarà necessariamente intero (dato da somma di interi) e quindi la matrice \textbf{unimodulare}, con la conseguenza già vista che $\text{PL} = \text{ILP}$.

Inoltre, possiamo portare i termini $v$ di $b$ a sinistra delle rispettive equazioni, ottenendo effettivamente la matrice:
$$
\begin{pmatrix}
 & 1 \\ 
E & 0 \\ 
 & -1
\end{pmatrix}
\begin{pmatrix}
x \\ v
\end{pmatrix}
= 0
$$
cioè dove si è introdotto la un \textbf{arco fittizio}, quello che parte da $t$ e arriva in $s$.
Capovolgendo la funzione obiettivo (e riportando in vista i termini di costo nullo su ogni arco $x_{ij}$), otteniamo quindi:
\[
	\begin{cases}
		\min 0 \cdot x - v \\ 	
\begin{pmatrix}
 & 1 \\ 
E & 0 \\ 
 & -1
\end{pmatrix}
\begin{pmatrix}
x \\ v
\end{pmatrix}
= 0 \\ 
0 \leq x \leq u
	\end{cases}
\]
dove per l'ultimo arco fittizio la capacità massima $u$ è un $M$ molto grande (o $\infty$).

Questo è un problema di flusso di costo minimo capacitato che include l'arco fittizio come unico arco a costo diverso da 0 (per giunta negativo), cioè che è "costretto" a imporre flusso massimo da $t$ ad $s$, e visto che tutti i nodi sono a bilancio 0, a riportarlo in direzione opposta da $s$ a $t$ lungo i nodi della rete vera e propria.

A questo punto possiamo proporre la soluzione ammissibile:
$$
(x,v) = \left( 2, 9, 0, 2, 5, 4, 7, 0, 4, 11 \right)
$$
da cui si ricavano le partizioni, controllando quali archi si svuotano e quali saturano:
$$
T = \left\{ (1,2), (3,4), (3,5), (4,6), (6,1) \right\}, \quad L = \left\{ (2,3), (5,4) \right\}, \quad U = \left\{ (1,3), (2,4), (5,6) \right\}
$$
Possiamo quindi ricavare il potenziale dell'albero di copertura, notando che dividerà necessariamente i nodi in partizione di potenziale 0 e potenziale 1:
$$
\pi = \left( 0, 0, 1, 1 ,1 \right)
$$
e calcolare i costi ridotti, che troviamo negativi su $(2,3) \in L$, quindi arco entrante:
$$
c^\pi_{23} = 0 + 0 - 1 = -1
$$
che è quanto ci aspettavamo, in quanto chiaramente $v = 14$ (dagli archi entranti nel nodo 6).

Procediamo quindi con l'eliminazione del ciclo, distinguendo innanzitutto le pratizioni $\mathcal{C}^+$ e $\mathcal{C}^-$:
$$
\mathcal{C}^+ = \left\{ (1,2), (2,3), (3,4), (4,6), (6, 1) \right\}, \quad \mathcal{C}^- = \emptyset
$$
Su $\mathcal{C}^-$ mettiamo $\vartheta^- = \infty$ (essenzialmente vogliamo considerare solo $\vartheta^+$).
Calcoliamo allora $\vartheta^+$ dagli $u_{ij} - x_{ij}$, tenendo conto che la capacità $u$ di $(6,1)$ (arco fittizio) è $\infty$:
$$
\vartheta^+ = \max \left\{ 6, 3, 3, 3, \infty \right\}
$$
da cui prendiamo $\vartheta^+ = 3$ e l'arco uscente $(2,3)$ per l'ordinamento lessicografico.
Notiamo di essere nel caso particolare dove l'arco \textit{entra} ed \textit{esce} (in questo caso si sposta da L a U).

Aggiungendo quindi il $\vartheta$ agli archi in $\mathcal{C}^+$ otteniamo il flusso:
$$
(x, v) = \left( 5, 9, 3, 2, 8, 4, 10, 0, 4, 14 \right)
$$
da cui $v=14$, come ci aspettavamo.
Vedremo in seguito un algoritmo per il calcolo del flusso massimo su reti più efficiente dell'applicazione diretta del simplesso.

\end{document}


\documentclass[a4paper,11pt]{article}
\usepackage[a4paper, margin=8em]{geometry}

% usa i pacchetti per la scrittura in italiano
\usepackage[french,italian]{babel}
\usepackage[T1]{fontenc}
\usepackage[utf8]{inputenc}
\frenchspacing 

% usa i pacchetti per la formattazione matematica
\usepackage{amsmath, amssymb, amsthm, amsfonts}

% usa altri pacchetti
\usepackage{gensymb}
\usepackage{hyperref}
\usepackage{standalone}

% imposta il titolo
\title{Appunti Ricerca Operativa}
\author{Luca Seggiani}
\date{2024}

% disegni
\usepackage{pgfplots}
\pgfplotsset{width=10cm,compat=1.9}

% imposta lo stile
% usa helvetica
\usepackage[scaled]{helvet}
% usa palatino
\usepackage{palatino}
% usa un font monospazio guardabile
\usepackage{lmodern}

\renewcommand{\rmdefault}{ppl}
\renewcommand{\sfdefault}{phv}
\renewcommand{\ttdefault}{lmtt}

% disponi il titolo
\makeatletter
\renewcommand{\maketitle} {
	\begin{center} 
		\begin{minipage}[t]{.8\textwidth}
			\textsf{\huge\bfseries \@title} 
		\end{minipage}%
		\begin{minipage}[t]{.2\textwidth}
			\raggedleft \vspace{-1.65em}
			\textsf{\small \@author} \vfill
			\textsf{\small \@date}
		\end{minipage}
		\par
	\end{center}

	\thispagestyle{empty}
	\pagestyle{fancy}
}
\makeatother

% disponi teoremi
\usepackage{tcolorbox}
\newtcolorbox[auto counter, number within=section]{theorem}[2][]{%
	colback=blue!10, 
	colframe=blue!40!black, 
	sharp corners=northwest,
	fonttitle=\sffamily\bfseries, 
	title=Teorema~\thetcbcounter: #2, 
	#1
}

% disponi definizioni
\newtcolorbox[auto counter, number within=section]{definition}[2][]{%
	colback=red!10,
	colframe=red!40!black,
	sharp corners=northwest,
	fonttitle=\sffamily\bfseries,
	title=Definizione~\thetcbcounter: #2,
	#1
}

% disponi problemi
\newtcolorbox[auto counter, number within=section]{problem}[2][]{%
	colback=green!10,
	colframe=green!40!black,
	sharp corners=northwest,
	fonttitle=\sffamily\bfseries,
	title=Problema~\thetcbcounter: #2,
	#1
}

% disponi codice
\usepackage{listings}
\usepackage[table]{xcolor}

\lstdefinestyle{codestyle}{
		backgroundcolor=\color{black!5}, 
		commentstyle=\color{codegreen},
		keywordstyle=\bfseries\color{magenta},
		numberstyle=\sffamily\tiny\color{black!60},
		stringstyle=\color{green!50!black},
		basicstyle=\ttfamily\footnotesize,
		breakatwhitespace=false,         
		breaklines=true,                 
		captionpos=b,                    
		keepspaces=true,                 
		numbers=left,                    
		numbersep=5pt,                  
		showspaces=false,                
		showstringspaces=false,
		showtabs=false,                  
		tabsize=2
}

\lstdefinestyle{shellstyle}{
		backgroundcolor=\color{black!5}, 
		basicstyle=\ttfamily\footnotesize\color{black}, 
		commentstyle=\color{black}, 
		keywordstyle=\color{black},
		numberstyle=\color{black!5},
		stringstyle=\color{black}, 
		showspaces=false,
		showstringspaces=false, 
		showtabs=false, 
		tabsize=2, 
		numbers=none, 
		breaklines=true
}

\lstdefinelanguage{javascript}{
	keywords={typeof, new, true, false, catch, function, return, null, catch, switch, var, if, in, while, do, else, case, break},
	keywordstyle=\color{blue}\bfseries,
	ndkeywords={class, export, boolean, throw, implements, import, this},
	ndkeywordstyle=\color{darkgray}\bfseries,
	identifierstyle=\color{black},
	sensitive=false,
	comment=[l]{//},
	morecomment=[s]{/*}{*/},
	commentstyle=\color{purple}\ttfamily,
	stringstyle=\color{red}\ttfamily,
	morestring=[b]',
	morestring=[b]"
}

% disponi sezioni
\usepackage{titlesec}

\titleformat{\section}
	{\sffamily\Large\bfseries} 
	{\thesection}{1em}{} 
\titleformat{\subsection}
	{\sffamily\large\bfseries}   
	{\thesubsection}{1em}{} 
\titleformat{\subsubsection}
	{\sffamily\normalsize\bfseries} 
	{\thesubsubsection}{1em}{}

% disponi alberi
\usepackage{forest}

\forestset{
	rectstyle/.style={
		for tree={rectangle,draw,font=\large\sffamily}
	},
	roundstyle/.style={
		for tree={circle,draw,font=\large}
	}
}

% disponi algoritmi
\usepackage{algorithm}
\usepackage{algorithmic}
\makeatletter
\renewcommand{\ALG@name}{Algoritmo}
\makeatother

% disponi numeri di pagina
\usepackage{fancyhdr}
\fancyhf{} 
\fancyfoot[L]{\sffamily{\thepage}}

\makeatletter
\fancyhead[L]{\raisebox{1ex}[0pt][0pt]{\sffamily{\@title \ \@date}}} 
\fancyhead[R]{\raisebox{1ex}[0pt][0pt]{\sffamily{\@author}}}
\makeatother

\begin{document}

% sezione (data)
\section{Lezione del 19-11-24}

% stili pagina
\thispagestyle{empty}
\pagestyle{fancy}

% testo
\subsubsection{Algoritmo di Ford-Fulkerson}
L'algoritmo di \textbf{Ford-Fulkerson} (precisamente nella variante di \textit{Edmonds-Karp}) è un algoritmo per il calcolo del flusso massimo su reti.

Iniziamo con alcune definizioni, prima fra tutte quella di \textbf{taglio di rete}:
\begin{definition}{Taglio di rete}
	Dato un problema di flusso massimo, un taglio di rete è una partizione dei nodi in due sottoinsiemi $N_s$ e $N_t$, quindi con $N_s \cup N_t = N$ e $N_s \cap N_t = \emptyset$, con il vincolo aggiunto che la sorgente di flusso $s$ deve trovarsi $\in N_s$, e la destinazione del flusso $t$ deve trovarsi $\in N_t$.
\end{definition}

Possiamo quindi dare la definizione di \textbf{arco diretto del taglio}:
\begin{definition}{Arco diretto del taglio}
	Dato un taglio, si dicono archi diretti del taglio tutti gli archi $(i,j)$ che hanno $i \in N_s$ e $j \in N_t$.
\end{definition}
con enfasi sul fatto che un arco diretto del taglio va da $N_s$ a $N_t$, e \textbf{non} viceversa.

Infine, su questi archi diretti del taglio definiamo la \textbf{portata}:
\begin{definition}{Portata del taglio}
	Dato un taglio $(N_s, N_t)$, la portata $\mu(N_s, N_t)$ viene definita come:
	$$
		\mu(N_s, N_t) = \sum_{(i,j) \in A^\intercal} u_{ij}
	$$
\end{definition}

Notiamo come ogni taglio di rete rappresenta un limite superiore sul flusso massimo:
	$$
		\overline{x} \leq \mu(N_s, N_t)
	$$
Possiamo quindi dimostrare il teorema:
\begin{theorem}{Massimo flussimo - minimo taglio}
	Il flusso massimo di un problema di flusso massimo con soluzione $\neq -\infty$ è uguale al taglio di portata minima:
	$$
		\overline{v} = \overline{u}(N_s, N_t) = \min_{s, t} \mu(N_s, N_t)
	$$
\end{theorem}

Notiamo come le ultime due equazioni ricalcano rispettivamente la \textbf{dualità debole} e la \textbf{dualità forte}: si può effettivamente dimostrare che il problema del taglio minimo è il \textbf{duale} del problema di flusso massimo (effettivamente vogliamo trovare un \textit{limite superiore} del flusso, che è lo stesso procedimento che avevamo adottato,, in quel caso adattando le diseguaglianze, nel ricavare il duale nei problemi di LP).

Chiaramente è improbabile calcolare il taglio minimo per enumerazione completa: togliendo il fatto che le partizioni dei tagli lavorano su $n-2$ anziché $n$ nodi (in quanto ci sono i vincoli $s \in N_s$ e $t \in N_t$), il numero di partizioni va comunque come $\sim 2^n$, che è quindi a complessità \textbf{esponenziale}.

Vediamo allora quale algoritmo possiamo usare per ricavare i tagli minimi (e quindi i flussi massimi) in maniera efficiente.

\subsubsection{Definizione dell'algoritmo}
Prendiamo in considerazione un grafo e il suo \textbf{grafo residuo}, cioè il grafo che complementa tutti gli archi (diretti) con un arco (diretto) identico ma di verso opposto. 
Chiameremo \textit{archi reali} gli archi che fanno effettivamente parte del grafo, e \textit{archi fittizi} gli archi che introduciamo sul grafo residuo. 

Per ogni arco $(i,j)$, reale o fittizio, definiamo la \textbf{capacità residua} come:
\begin{itemize}
	\item Archi reali: $r_{ij} = u_{ij} - \overline{x}_{ij}$
	\item Archi fittizi: $r_{ji} = \overline{x}_{ij}$
\end{itemize}
assunto nelle definizioni che $(i, j)$ sia l'arco reale e di conseguenza $(j, i)$ l'arco fittizio.

A parole, la capacità residua dell'arco \textit{reale} è \textit{quanto posso ancora spedire sull'arco}, mentre la capacità residua dell'arco \textit{fittizio} è \textit{quanto ho già spedito sull'arco vero}, da cui:
$$
r_{ij} + r_{ji} = u_{ij} - \overline{x}_{ij} + \overline{x}_{ij} = u_{ij}
$$
cioè la somma delle capacità residue su entrambi gli archi nelle due direzioni del grafo residuo dà sempre la capacità massima sull'arco reale nella stessa posizione del grafo originale.

Definiamo allora il concetto di \textbf{cammino aumentante} sul grafo:
\begin{definition}{Cammino aumentante}
Dato un problema di flusso massimo, chiamiamo cammino aumentante $C_{aum}$ sul grafo residuo un qualsiasi cammino orientato da $s$ a $t$ formato da archi con capacità residue $r_{ij} > 0$.
\end{definition}

Di un dato cammino aumentante $C_{aum}$, il dato interessante è la \textbf{portata} di $C_{aum}$:
$$
\delta = \min_{i,j \in C_{aum}} \{ r_{ij} \}
$$

Il primo passo dell'algoritmo di Ford-Fulkerson sarà allora selezionare un cammino aumentante e calcolarne la portata $\delta$.
In seguito vorremo spedire queste $\delta$ unità di flusso lungo il cammino, quindi applicando la regola:
\[
	\overline{x}' = 	
	\begin{cases}
		\overline{x}_{ij} + \delta, \quad \forall (i, j) \in C_{aum} \\ 
		\overline{x}_{ij}, \quad \forall (i, j) \notin C_{aum}
	\end{cases}
\]

A questo punto basta ricalcolare le varie capacità residue $r_{ij}$ e ripetere il passaggio.
Chiaramente, quando non si riuscirà più a trovare cammini aumentanti, cioè archi con capacità residue $> 0$, significherà che avremo trovato la soluzione ottima. 

\begin{algorithm}[H]
\caption{di Ford-Fulkerson}
\begin{algorithmic}
	\STATE \textbf{Input:} un problema di flusso massimo
	\STATE \textbf{Output:} la soluzione ottima
	% body
	\STATE \textsf{ciclo:}
	\STATE Stabilisci un cammino aumentante $C_{aum}$ e calcola la portata $\delta$.
	\IF{$C_{aum} = \emptyset$} 
		\STATE Fermati, sei all'ottimo
	\ENDIF
	\STATE Aggiorna il flusso $\overline{x}$ secondo la regola:
	\[
		\overline{x}' = 	
		\begin{cases}
			\overline{x}_{ij} + \delta, \quad \forall (i, j) \in C_{aum} \\ 
			\overline{x}_{ij}, \quad \forall (i, j) \notin C_{aum}
		\end{cases}
	\]
	\STATE Ricalcola le capacità residue $r_{ij}$
	\STATE Torna a \textsf{ciclo}
\end{algorithmic}
\end{algorithm}

Si presenta un algoritmo per il calcolo dei cammini aumentanti, detto \textbf{algoritmo della croce}:
\begin{algorithm}[H]
\caption{della croce}
\begin{algorithmic}
	\STATE \textbf{Input:} il grafo residuo di un problema di flusso massimo % input
	\STATE \textbf{Output:} un cammino aumentante $C_{aum}$ % output
	% body
	\STATE Inizializza due vettori, \lstinline|closed| e \lstinline|open|
	\STATE Inserisci $s$, il nodo di partenza, in \lstinline|closed|
	\STATE Inserisci tutti i nodi raggiungibili da \lstinline|closed|, cioè quelli che sono collegati con un arco diretto dal nodo 1 con costo residuo $> 0$, in \lstinline|open|
	\STATE \textsf{ciclo:}
	\STATE Prendi il primo elemento di \lstinline|open| e mettilo in \lstinline|closed|
	\STATE Inserisci tutti i nodi raggiungibili da \lstinline|closed|, cioè quelli che sono collegati con un arco diretto dal nodo in \lstinline|closed| con costo residuo $> 0$
	\IF{non hai raggiunto $t$; cioè il nodo di arrivo}
		\STATE Torna a \textsf{ciclo}
	\ENDIF
	\STATE Ripercorrendo i nodi in \lstinline|closed| al contrario, partendo da quello di arrivo, hai il cammino aumentate $C_{aum}$
\end{algorithmic}
\end{algorithm}

Notiamo come, nell'applicazione dell'algoritmo della croce, è necessaria la presenza degli archi residui in entrambe le direzioni: potrebbe infatti succedere che un arco abbia flusso aggiuntivo $\delta$ nella direzione opposta a quella canonica.
Questo significherà che il flusso che spingiamo nella direzione opposta andrà a sottrarsi al flusso già presente sull'arco (in altre parole, l'algoritmo di Ford-Fulkerson può "cambiare idea" nell'esplorazione dello spazio delle soluzioni).

\end{document}


\documentclass[a4paper,11pt]{article}
\usepackage[a4paper, margin=8em]{geometry}

% usa i pacchetti per la scrittura in italiano
\usepackage[french,italian]{babel}
\usepackage[T1]{fontenc}
\usepackage[utf8]{inputenc}
\frenchspacing 

% usa i pacchetti per la formattazione matematica
\usepackage{amsmath, amssymb, amsthm, amsfonts}

% usa altri pacchetti
\usepackage{gensymb}
\usepackage{hyperref}
\usepackage{standalone}

% imposta il titolo
\title{Appunti Ricerca Operativa}
\author{Luca Seggiani}
\date{2024}

% disegni
\usepackage{pgfplots}
\pgfplotsset{width=10cm,compat=1.9}

% imposta lo stile
% usa helvetica
\usepackage[scaled]{helvet}
% usa palatino
\usepackage{palatino}
% usa un font monospazio guardabile
\usepackage{lmodern}

\renewcommand{\rmdefault}{ppl}
\renewcommand{\sfdefault}{phv}
\renewcommand{\ttdefault}{lmtt}

% disponi il titolo
\makeatletter
\renewcommand{\maketitle} {
	\begin{center} 
		\begin{minipage}[t]{.8\textwidth}
			\textsf{\huge\bfseries \@title} 
		\end{minipage}%
		\begin{minipage}[t]{.2\textwidth}
			\raggedleft \vspace{-1.65em}
			\textsf{\small \@author} \vfill
			\textsf{\small \@date}
		\end{minipage}
		\par
	\end{center}

	\thispagestyle{empty}
	\pagestyle{fancy}
}
\makeatother

% disponi teoremi
\usepackage{tcolorbox}
\newtcolorbox[auto counter, number within=section]{theorem}[2][]{%
	colback=blue!10, 
	colframe=blue!40!black, 
	sharp corners=northwest,
	fonttitle=\sffamily\bfseries, 
	title=Teorema~\thetcbcounter: #2, 
	#1
}

% disponi definizioni
\newtcolorbox[auto counter, number within=section]{definition}[2][]{%
	colback=red!10,
	colframe=red!40!black,
	sharp corners=northwest,
	fonttitle=\sffamily\bfseries,
	title=Definizione~\thetcbcounter: #2,
	#1
}

% disponi problemi
\newtcolorbox[auto counter, number within=section]{problem}[2][]{%
	colback=green!10,
	colframe=green!40!black,
	sharp corners=northwest,
	fonttitle=\sffamily\bfseries,
	title=Problema~\thetcbcounter: #2,
	#1
}

% disponi codice
\usepackage{listings}
\usepackage[table]{xcolor}

\lstdefinestyle{codestyle}{
		backgroundcolor=\color{black!5}, 
		commentstyle=\color{codegreen},
		keywordstyle=\bfseries\color{magenta},
		numberstyle=\sffamily\tiny\color{black!60},
		stringstyle=\color{green!50!black},
		basicstyle=\ttfamily\footnotesize,
		breakatwhitespace=false,         
		breaklines=true,                 
		captionpos=b,                    
		keepspaces=true,                 
		numbers=left,                    
		numbersep=5pt,                  
		showspaces=false,                
		showstringspaces=false,
		showtabs=false,                  
		tabsize=2
}

\lstdefinestyle{shellstyle}{
		backgroundcolor=\color{black!5}, 
		basicstyle=\ttfamily\footnotesize\color{black}, 
		commentstyle=\color{black}, 
		keywordstyle=\color{black},
		numberstyle=\color{black!5},
		stringstyle=\color{black}, 
		showspaces=false,
		showstringspaces=false, 
		showtabs=false, 
		tabsize=2, 
		numbers=none, 
		breaklines=true
}

\lstdefinelanguage{javascript}{
	keywords={typeof, new, true, false, catch, function, return, null, catch, switch, var, if, in, while, do, else, case, break},
	keywordstyle=\color{blue}\bfseries,
	ndkeywords={class, export, boolean, throw, implements, import, this},
	ndkeywordstyle=\color{darkgray}\bfseries,
	identifierstyle=\color{black},
	sensitive=false,
	comment=[l]{//},
	morecomment=[s]{/*}{*/},
	commentstyle=\color{purple}\ttfamily,
	stringstyle=\color{red}\ttfamily,
	morestring=[b]',
	morestring=[b]"
}

% disponi sezioni
\usepackage{titlesec}

\titleformat{\section}
	{\sffamily\Large\bfseries} 
	{\thesection}{1em}{} 
\titleformat{\subsection}
	{\sffamily\large\bfseries}   
	{\thesubsection}{1em}{} 
\titleformat{\subsubsection}
	{\sffamily\normalsize\bfseries} 
	{\thesubsubsection}{1em}{}

% disponi alberi
\usepackage{forest}

\forestset{
	rectstyle/.style={
		for tree={rectangle,draw,font=\large\sffamily}
	},
	roundstyle/.style={
		for tree={circle,draw,font=\large}
	}
}

% disponi algoritmi
\usepackage{algorithm}
\usepackage{algorithmic}
\makeatletter
\renewcommand{\ALG@name}{Algoritmo}
\makeatother

% disponi numeri di pagina
\usepackage{fancyhdr}
\fancyhf{} 
\fancyfoot[L]{\sffamily{\thepage}}

\makeatletter
\fancyhead[L]{\raisebox{1ex}[0pt][0pt]{\sffamily{\@title \ \@date}}} 
\fancyhead[R]{\raisebox{1ex}[0pt][0pt]{\sffamily{\@author}}}
\makeatother

\begin{document}

% sezione (data)
\section{Lezione del 20-11-24}

% stili pagina
\thispagestyle{empty}
\pagestyle{fancy}

% testo
\subsection{Algoritmo di Dijkstra}
Torniamo sul problema dei \textbf{cammini di costo minimo}.
Prendiamo il grafo:
\begin{center}
	\begin{tikzpicture}
		\node[circle, draw=black] (1) at (0,0) {1};
		\node[circle, draw=black] (2) at (2,1) {2};
		\node[circle, draw=black] (3) at (2,-1) {3};
		\node[circle, draw=black] (4) at (4,1) {4};
		\node[circle, draw=black] (5) at (4,-1) {5};
		\node[circle, draw=black] (6) at (6, 0) {6};
		\draw[->, to path={-| (\tikztotarget)}] (1) -- (2);
		\draw[->, to path={-| (\tikztotarget)}] (1) -- (3);
		\draw[->, to path={-| (\tikztotarget)}] (2) -- (3);
		\draw[->, to path={-| (\tikztotarget)}] (2) -- (4);
		\draw[->, to path={-| (\tikztotarget)}] (3) -- (5);
		\draw[->, to path={-| (\tikztotarget)}] (5) -- (4);
		\draw[->, to path={-| (\tikztotarget)}] (3) -- (4);
		\draw[->, to path={-| (\tikztotarget)}] (4) -- (6);
		\draw[->, to path={-| (\tikztotarget)}] (5) -- (6);

		\node at (1, 1)  {$9$};
		\node at (1, -1)  {$3$};

		\node at (3, 1.5)  {$7$};
		\node at (3, -1.5)  {$6$};
		\node at (3, 0.5)  {$2$};

		\node at (1.5, 0)  {$4$};
		\node at (4.5, 0)  {$8$};

		\node at (5, 1)  {$1$};
		\node at (5, -1)  {$9$};
	\end{tikzpicture}
\end{center}

Un (celebre) algoritmo per il problema è quello di \textbf{Dijkstra}.
\begin{algorithm}
\caption{di Dijkstra}
\begin{algorithmic}
	\STATE \textbf{Input:} un problema di cammini minimi con costi $c_{ij}>0$ % input
	\STATE \textbf{Output:} l'albero dei cammini minimi % output
	% body
	\STATE Etichetta ogni nodo come $\pi = \left( 0, +\infty, ..., +\infty \right)$, assunto il primo nodo come partenza
	\STATE Associa un predecessore ad ogni nodo come $P = (1, 0, ..., 0)$
	\STATE Poni $U = N$
	\WHILE{$U \neq \emptyset$}
		\STATE Prendi il nodo $n \in U$ di etichetta $\pi$ minima, rimuovilo da $U$, e prendi la stella uscente $\mathrm{FS}(n)$ di $n$
		\FOR{$\forall j \in \mathrm{FS}(n)$}
			\IF{$\pi_j \geq \pi_i + c_{ij}$}
				\STATE Poni il predecessore $P_j = i$
				\STATE Poni l'etichetta $\pi_j = \pi_i + c_{ij}$
			\ENDIF
		\ENDFOR
	\ENDWHILE
\end{algorithmic}
\end{algorithm}

Le etichette $\pi$ rappresenteranno la \textbf{distanza minima} dal nodo di partenza a ogni nodo.
Ad algoritmo compiuto (quando $N$ è vuoto) la catena dei predecessori da ogni nodo a quello di partenza forma i cammini minimi.

Sull'esempio riportato sopra, il primo passaggio è:
$$ i = 1, \quad N = \{ 2, 3, 4, 5, 6 \} $$
Da cui si calcola quindi la stella uscente $FS(1) = \{ 2, 3\}$ 
Prendiamo entrambi gli indici:
\begin{itemize}
	\item $j=2$: $\pi_2 \geq \pi_1 + c_{12} \rightarrow
		\begin{cases}
			\pi_2 = 8 \\ 
			p_2 = 1
		\end{cases}
		$
	\item $j=3$: $\pi_3 \geq \pi_1 + c_{13} \rightarrow
		\begin{cases}
			\pi_3 = 3 \\ 
			p_2 = 1
		\end{cases}
		$
\end{itemize}

Da cui, all'inizio del secondo passaggio, etichette e predecessori sono:
$$
\pi = (0, 8, 3, +\infty, + \infty, +\infty)
$$
$$
p = (1, 1, 1, 0, 0, 0)
$$

L'algoritmo, con complessità linearitmica $O(n \log{n} + m \log{n})$ su $n$ nodi a ramificazione $m$, risulta più efficiente della risoluzione diretta dei problemi dei cammini minimi come problemi di flusso.

\end{document}


\documentclass[a4paper,11pt]{article}
\usepackage[a4paper, margin=8em]{geometry}

% usa i pacchetti per la scrittura in italiano
\usepackage[french,italian]{babel}
\usepackage[T1]{fontenc}
\usepackage[utf8]{inputenc}
\frenchspacing 

% usa i pacchetti per la formattazione matematica
\usepackage{amsmath, amssymb, amsthm, amsfonts}

% usa altri pacchetti
\usepackage{gensymb}
\usepackage{hyperref}
\usepackage{standalone}

% imposta il titolo
\title{Appunti Ricerca Operativa}
\author{Luca Seggiani}
\date{2024}

% disegni
\usepackage{pgfplots}
\pgfplotsset{width=10cm,compat=1.9}

% imposta lo stile
% usa helvetica
\usepackage[scaled]{helvet}
% usa palatino
\usepackage{palatino}
% usa un font monospazio guardabile
\usepackage{lmodern}

\renewcommand{\rmdefault}{ppl}
\renewcommand{\sfdefault}{phv}
\renewcommand{\ttdefault}{lmtt}

% disponi il titolo
\makeatletter
\renewcommand{\maketitle} {
	\begin{center} 
		\begin{minipage}[t]{.8\textwidth}
			\textsf{\huge\bfseries \@title} 
		\end{minipage}%
		\begin{minipage}[t]{.2\textwidth}
			\raggedleft \vspace{-1.65em}
			\textsf{\small \@author} \vfill
			\textsf{\small \@date}
		\end{minipage}
		\par
	\end{center}

	\thispagestyle{empty}
	\pagestyle{fancy}
}
\makeatother

% disponi teoremi
\usepackage{tcolorbox}
\newtcolorbox[auto counter, number within=section]{theorem}[2][]{%
	colback=blue!10, 
	colframe=blue!40!black, 
	sharp corners=northwest,
	fonttitle=\sffamily\bfseries, 
	title=Teorema~\thetcbcounter: #2, 
	#1
}

% disponi definizioni
\newtcolorbox[auto counter, number within=section]{definition}[2][]{%
	colback=red!10,
	colframe=red!40!black,
	sharp corners=northwest,
	fonttitle=\sffamily\bfseries,
	title=Definizione~\thetcbcounter: #2,
	#1
}

% disponi problemi
\newtcolorbox[auto counter, number within=section]{problem}[2][]{%
	colback=green!10,
	colframe=green!40!black,
	sharp corners=northwest,
	fonttitle=\sffamily\bfseries,
	title=Problema~\thetcbcounter: #2,
	#1
}

% disponi codice
\usepackage{listings}
\usepackage[table]{xcolor}

\lstdefinestyle{codestyle}{
		backgroundcolor=\color{black!5}, 
		commentstyle=\color{codegreen},
		keywordstyle=\bfseries\color{magenta},
		numberstyle=\sffamily\tiny\color{black!60},
		stringstyle=\color{green!50!black},
		basicstyle=\ttfamily\footnotesize,
		breakatwhitespace=false,         
		breaklines=true,                 
		captionpos=b,                    
		keepspaces=true,                 
		numbers=left,                    
		numbersep=5pt,                  
		showspaces=false,                
		showstringspaces=false,
		showtabs=false,                  
		tabsize=2
}

\lstdefinestyle{shellstyle}{
		backgroundcolor=\color{black!5}, 
		basicstyle=\ttfamily\footnotesize\color{black}, 
		commentstyle=\color{black}, 
		keywordstyle=\color{black},
		numberstyle=\color{black!5},
		stringstyle=\color{black}, 
		showspaces=false,
		showstringspaces=false, 
		showtabs=false, 
		tabsize=2, 
		numbers=none, 
		breaklines=true
}

\lstdefinelanguage{javascript}{
	keywords={typeof, new, true, false, catch, function, return, null, catch, switch, var, if, in, while, do, else, case, break},
	keywordstyle=\color{blue}\bfseries,
	ndkeywords={class, export, boolean, throw, implements, import, this},
	ndkeywordstyle=\color{darkgray}\bfseries,
	identifierstyle=\color{black},
	sensitive=false,
	comment=[l]{//},
	morecomment=[s]{/*}{*/},
	commentstyle=\color{purple}\ttfamily,
	stringstyle=\color{red}\ttfamily,
	morestring=[b]',
	morestring=[b]"
}

% disponi sezioni
\usepackage{titlesec}

\titleformat{\section}
	{\sffamily\Large\bfseries} 
	{\thesection}{1em}{} 
\titleformat{\subsection}
	{\sffamily\large\bfseries}   
	{\thesubsection}{1em}{} 
\titleformat{\subsubsection}
	{\sffamily\normalsize\bfseries} 
	{\thesubsubsection}{1em}{}

% disponi alberi
\usepackage{forest}

\forestset{
	rectstyle/.style={
		for tree={rectangle,draw,font=\large\sffamily}
	},
	roundstyle/.style={
		for tree={circle,draw,font=\large}
	}
}

% disponi algoritmi
\usepackage{algorithm}
\usepackage{algorithmic}
\makeatletter
\renewcommand{\ALG@name}{Algoritmo}
\makeatother

% disponi numeri di pagina
\usepackage{fancyhdr}
\fancyhf{} 
\fancyfoot[L]{\sffamily{\thepage}}

\makeatletter
\fancyhead[L]{\raisebox{1ex}[0pt][0pt]{\sffamily{\@title \ \@date}}} 
\fancyhead[R]{\raisebox{1ex}[0pt][0pt]{\sffamily{\@author}}}
\makeatother

\begin{document}

% sezione (data)
\section{Lezione del 26-11-24}

% stili pagina
\thispagestyle{empty}
\pagestyle{fancy}

\subsection{Introduzione alla programmazione non lineare}
% testo
\begin{problem}{di programmazione non lineare}
	In una nuova linea di produzione sono integrati 6 robot, ciascuno dei quali può rotare di 360 gradi attorno all'asse verticale.
	Le aree di lavoro di ciascun robot non devono sovrapporsi, in modo da evitare collisioni fra di essi.
	Inoltre, i robot devono essere collegati fra di loro attraverso cavi in fibra ottica, e poichè questi sono costosi, è necessario minimizzare la distanza da coprire.
\end{problem}

I raggi di lavoro dei robot sono, poniamo, i seguenti:
\begin{table}[h!]
	\center \rowcolors{2}{white}{black!10}
	\begin{tabular} { c | c  }
		\bfseries Robot & \bfseries Raggio  \\
		\hline 
		 1 & 120 \\ 
		 2 & 80 \\ 
		 3 & 100 \\ 
		 4 & 70 \\ 
		 5 & 45 \\ 
		 6 & 120
	\end{tabular}
\end{table}

Il problema sarà quindi quello di individuare i 6 punti sul piano $(x_1, y_1), (x_2, y_2)$, $...$, $(x_6, y_6)$ che corrispondono alle posizioni degli assi verticali di ogni robot, imponendo il vincolo di distanza fra due robot adiacenti $i$ e $j$ di $r_i + r_j$.
La funzione obiettivo sarà quindi la quantità di cavo in fibra ottica, cioè la distanza fra ogni coppia $i$, $j$ di robot.
Possiamo quindi disporre un modello:
\[
	\begin{cases}
		\max \sqrt{(x_1 - x_2)^2 + (y_1 - y_2)^2} + \sqrt{(x_2 - x_3)^2 + (y_2 - y_3)^2} + ... + \sqrt{(x_5 - x_6)^2 + (y_5 - y_6)^2} \\ 
		d_{ij} \geq r_j + r_j, \quad \forall i = 1, 2, ..., 6, \quad \forall j = 1, 2, ..., 6, \quad i < j
	\end{cases}
\]
su 12 variabili e $\frac{n(n-1)}{2} \big|_{n=6} = 15$ vincoli.
Nelle prossime lezioni vedremo come risolvere questo tipo di problemi, notando per ora che tutte le funzioni che abbiamo generato finora sono \textbf{quadratiche} tolte le radici (che possiamo fare conservando massimi e minimi).

\par\smallskip

In generale, nella programmazione non lineare, vorremo avere le nostre funzioni $\in C^1$ in modo da poter derivare, cioè ricavare il gradiente e stabilire punti di massimo e di minimo, attraverso il teorema di Fermat:
\begin{theorem}{di Fermat}
	Sia data una funzione $f: \mathbb{R}^n \rightarrow \mathbb{R}$, con adeguate condizioni di continuità.
	Si ha che in ogni punto di massimo o minimo locale $\bar{x}$ vale:
	$$
		\nabla f (\bar{x}) = 0
	$$
\end{theorem}

Addirittura, fosse la funzione $\in C^2$ potremmo ricavare l'Hessiana come ulteriore discriminante, ricordando che \textit{Hessiana definita positiva} significa \textbf{punto di minimo}, e viceversa \textit{Hessiana definita negativa} significa \textbf{punto di massimo} (con le opportune considerazioni di definizione positiva/negativa stretta e non).

Vediamo quindi di fare una classificazione di funzioni utili ai fini dell'ottimizzazione.

\subsubsection{Funzioni quadratiche}
Diamo quidi una forma generale di polinomi di secondo grado su cui fare ottimizzazione.
Una funzione tipo potrebbe essere:
$$
f(x) = \frac{1}{2}x^\intercal Qx + c^\intercal x
$$
imponendo, vedremo poi come mai, la matrice $Q$ definita positiva.

Di cui potrebbe essere un esempio:
$$
f(x) = ax_1^2 + bx_1x_2 + cx_2^2 + dx_1 + ex_2
$$

Vediamo come trovare la matrice $Q$ e il vettore $c$. 
Nel caso in due variabili, che continuiamo per la convenzione della ricerca operativa a chiamare $x_1$ e $x_2$, avremo per calcolo diretto, spezzando le componenti in $Q$ e in $C$:
\begin{itemize}
	\item $Q$:
		$$
		\frac{1}{2}x^\intercal Q x = 
		\frac{1}{2}
		\begin{pmatrix}
			x_1 & x_2
		\end{pmatrix}
		\begin{pmatrix}
			q_{11} & q_{12} \\ 
			q_{21} & q_{22}
		\end{pmatrix}
		\begin{pmatrix}
			x_1 & x_2
		\end{pmatrix}
		=
		\frac{1}{2}
		\begin{pmatrix}
			x & y
		\end{pmatrix}
		\begin{pmatrix}
			q_{11} x_1 & q_{12} x_2 \\ 
			q_{21} x_1 & q_{22} x_2
		\end{pmatrix}
		$$
		$$
		= \frac{1}{2}\left( q_{11} x_1^2 + (q_{12} + q_{21}) x_1 x_2 + q_{22} x_2^2 \right)
		$$
		sarà quindi:
		\[
			\begin{cases}
				q_{11} = 2a \\ 
				q_{12} = b \\
				q_{21} = b \\
				q_{22} = 2c
			\end{cases}
		\]
	\item $c$: valgono le stesse regole viste sulle lineari, cioè nel nostro esempio:
		\[
			\begin{cases}
				c_1 = d \\ 
				c_2 = e
			\end{cases}
		\]
\end{itemize}

Abbiamo quindi stabilito che esiste una corrispondenza fra polinomi di secondo grado e matrici quadrate simmetriche di dimensione 2, dove la definizione positiva indica la presenza di un solo minimo globale della quadratica.

Si può poi dimostrare che in generale esiste una corrispondenza fra \textbf{funzioni polinomiali di secondo grado} e \textbf{matrici quadrate simmetriche} di qualsiasi dimensione.

Notiamo che, stabilita la $Q$ e la $c$, il calcolo di gradiente risulta immediato in quanto:
\[
	\begin{cases}
		\nabla f = Qx + c \\ 
		Hf = Q
	\end{cases}
\]
da cui si spiega l'$\frac{1}{2}$ accanto a $Q$, che "normalizza" il modulo dell'Hessiana.

\subsubsection{Funzioni convesse}
Diamo innanzitutto una definizione:
\begin{definition}{Funzione convessa}
	Una funzione convessa è qualsiasi funzione che rispetta:
	$$
		f(\lambda x + (1 - \lambda) y) \leq \lambda f(x) + (1 - \lambda) f(y)
	$$
	con $\forall \lambda \in [0, 1]$, $\forall x$, $\forall y \in \mathbb{R}^n$.
\end{definition}
la definizione di funzione convessa stabilisce essenzialmente per una qualsiasi retta intercettante la funzione $f$, in qualsiasi punto compreso fra le intercette la funzione raggiunge un valore minore o uguale al valore della retta calcolato nello stesso punto.

Una proprietà importante delle funzioni convesse è che il gradiente è dato da una funzione \textbf{monotona crescente}, cioè $f(x) < f(y)$ $\forall x < y$, e quindi che l'Hessiana è \textbf{semidefinita positiva} ($\geq 0$ nel caso scalare).

Una conseguenza immediata di quest'ultima considerazione è che in una funzione convessa, ogni punto dove $\nabla f(x) = 0$ è necessariamente \textbf{minimo}.

Notiamo alcune altre proprietà delle funzioni convesse:
\begin{itemize}
	\item Una funzione convessa può non avere minimo: si pensi all'esponenziale, che tende a 0 senza raggiungerlo mai, e quindi ha solo un \textit{limite inferiore};
	\item Il minimo locale di una funzione convessa è anche minimo globale;
	\item Le funzioni convesse non hanno selle.
\end{itemize}

\subsubsection{Funzioni coercive}
Definiamo infine un'ultima classe, data da:
\begin{definition}{Funzione coerciva}
	Una funzione coerciva è una funzione che rispetta il limite:
	$$
		\lim_{|x| \rightarrow +\infty} f(x) = +\infty
	$$
\end{definition}
	
Possiamo quindi essere sicuri che una funzione coerciva non ha massimo globale, ma al massimo solo massimi locali.

\end{document}


\documentclass[a4paper,11pt]{article}
\usepackage[a4paper, margin=8em]{geometry}

% usa i pacchetti per la scrittura in italiano
\usepackage[french,italian]{babel}
\usepackage[T1]{fontenc}
\usepackage[utf8]{inputenc}
\frenchspacing 

% usa i pacchetti per la formattazione matematica
\usepackage{amsmath, amssymb, amsthm, amsfonts}

% usa altri pacchetti
\usepackage{gensymb}
\usepackage{hyperref}
\usepackage{standalone}

% imposta il titolo
\title{Appunti Ricerca Operativa}
\author{Luca Seggiani}
\date{2024}

% disegni
\usepackage{pgfplots}
\pgfplotsset{width=10cm,compat=1.9}

% imposta lo stile
% usa helvetica
\usepackage[scaled]{helvet}
% usa palatino
\usepackage{palatino}
% usa un font monospazio guardabile
\usepackage{lmodern}

\renewcommand{\rmdefault}{ppl}
\renewcommand{\sfdefault}{phv}
\renewcommand{\ttdefault}{lmtt}

% disponi il titolo
\makeatletter
\renewcommand{\maketitle} {
	\begin{center} 
		\begin{minipage}[t]{.8\textwidth}
			\textsf{\huge\bfseries \@title} 
		\end{minipage}%
		\begin{minipage}[t]{.2\textwidth}
			\raggedleft \vspace{-1.65em}
			\textsf{\small \@author} \vfill
			\textsf{\small \@date}
		\end{minipage}
		\par
	\end{center}

	\thispagestyle{empty}
	\pagestyle{fancy}
}
\makeatother

% disponi teoremi
\usepackage{tcolorbox}
\newtcolorbox[auto counter, number within=section]{theorem}[2][]{%
	colback=blue!10, 
	colframe=blue!40!black, 
	sharp corners=northwest,
	fonttitle=\sffamily\bfseries, 
	title=Teorema~\thetcbcounter: #2, 
	#1
}

% disponi definizioni
\newtcolorbox[auto counter, number within=section]{definition}[2][]{%
	colback=red!10,
	colframe=red!40!black,
	sharp corners=northwest,
	fonttitle=\sffamily\bfseries,
	title=Definizione~\thetcbcounter: #2,
	#1
}

% disponi problemi
\newtcolorbox[auto counter, number within=section]{problem}[2][]{%
	colback=green!10,
	colframe=green!40!black,
	sharp corners=northwest,
	fonttitle=\sffamily\bfseries,
	title=Problema~\thetcbcounter: #2,
	#1
}

% disponi codice
\usepackage{listings}
\usepackage[table]{xcolor}

\lstdefinestyle{codestyle}{
		backgroundcolor=\color{black!5}, 
		commentstyle=\color{codegreen},
		keywordstyle=\bfseries\color{magenta},
		numberstyle=\sffamily\tiny\color{black!60},
		stringstyle=\color{green!50!black},
		basicstyle=\ttfamily\footnotesize,
		breakatwhitespace=false,         
		breaklines=true,                 
		captionpos=b,                    
		keepspaces=true,                 
		numbers=left,                    
		numbersep=5pt,                  
		showspaces=false,                
		showstringspaces=false,
		showtabs=false,                  
		tabsize=2
}

\lstdefinestyle{shellstyle}{
		backgroundcolor=\color{black!5}, 
		basicstyle=\ttfamily\footnotesize\color{black}, 
		commentstyle=\color{black}, 
		keywordstyle=\color{black},
		numberstyle=\color{black!5},
		stringstyle=\color{black}, 
		showspaces=false,
		showstringspaces=false, 
		showtabs=false, 
		tabsize=2, 
		numbers=none, 
		breaklines=true
}

\lstdefinelanguage{javascript}{
	keywords={typeof, new, true, false, catch, function, return, null, catch, switch, var, if, in, while, do, else, case, break},
	keywordstyle=\color{blue}\bfseries,
	ndkeywords={class, export, boolean, throw, implements, import, this},
	ndkeywordstyle=\color{darkgray}\bfseries,
	identifierstyle=\color{black},
	sensitive=false,
	comment=[l]{//},
	morecomment=[s]{/*}{*/},
	commentstyle=\color{purple}\ttfamily,
	stringstyle=\color{red}\ttfamily,
	morestring=[b]',
	morestring=[b]"
}

% disponi sezioni
\usepackage{titlesec}

\titleformat{\section}
	{\sffamily\Large\bfseries} 
	{\thesection}{1em}{} 
\titleformat{\subsection}
	{\sffamily\large\bfseries}   
	{\thesubsection}{1em}{} 
\titleformat{\subsubsection}
	{\sffamily\normalsize\bfseries} 
	{\thesubsubsection}{1em}{}

% disponi alberi
\usepackage{forest}

\forestset{
	rectstyle/.style={
		for tree={rectangle,draw,font=\large\sffamily}
	},
	roundstyle/.style={
		for tree={circle,draw,font=\large}
	}
}

% disponi algoritmi
\usepackage{algorithm}
\usepackage{algorithmic}
\makeatletter
\renewcommand{\ALG@name}{Algoritmo}
\makeatother

% disponi numeri di pagina
\usepackage{fancyhdr}
\fancyhf{} 
\fancyfoot[L]{\sffamily{\thepage}}

\makeatletter
\fancyhead[L]{\raisebox{1ex}[0pt][0pt]{\sffamily{\@title \ \@date}}} 
\fancyhead[R]{\raisebox{1ex}[0pt][0pt]{\sffamily{\@author}}}
\makeatother

\begin{document}

% sezione (data)
\section{Lezione del 27-11-24}

% stili pagina
\thispagestyle{empty}
\pagestyle{fancy}

% testo
\subsection{Domini di problemi di programmazione non lineare}
Abbiamo studiato finora funzioni non lineari nell'ottica dell'ottimizzazione non vincolata.
Vediamo adesso una delle forme dei domini su cui possiamo limitare tali funzioni nel caso dell'ottimizzazione vincolata:
$$
D = \left\{ x \in \mathbb{R^n}: \ g_1(x) \leq 0, \ g_2(x) \leq 0, \ ..., \ g_m(x) \leq 0, \
							h_1(x) = 0, \ h_2(x) = 0, \ ..., \ h_p(x) = 0 \right\}
$$
dove
\[
	\begin{cases}
		g_i: \mathbb{R}^n \rightarrow \mathbb{R}, \quad i = 1, ..., m \\ 
		h_j: \mathbb{R}^n \rightarrow \mathbb{R}, \quad j = 1, ..., p
	\end{cases}
\]
o in maniera più compatta:
$$
D = \left\{ x \in \mathbb{R}^n: \ g(x) \leq 0, \ h(x) = 0 \right\}
$$
dove
\[
	\begin{cases}
		g(x) = \left( g_1(x), \ g_2(x), \ ..., \ g_m(x) \right) \\	
		h(x) = \left( h_1(x), \ h_2(x), \ ..., \ h_p(x) \right)	
	\end{cases}
\]

Notiamo che i poliedri studiati finora non sono che un caso particolare di insiemi in questa forma.
Possiamo quindi dire che da qui in poi formuleremo i problemi di NLP come:
\[
	\begin{cases}
		\max f(x) \\ 
		g(x) \leq 0 \\ 
		h(x) = 0
	\end{cases}
\]

dove $f(x)$ è una funzione in $n$ variabili, $g(x)$ è un vettore di $m$ funzioni, e $h(x)$ è un vettore di $p$ funzioni.
Vediamo alcune propietà che ci aspettiamo dal dominio:
\begin{itemize}
	\item Assumeremo ognuna delle funzioni $g_i(x)$ e $h_j(x)$ come almeno $\in C^2$, in modo da poter calcolare gradiente ed Hessiana;
	\item Un vantaggio sarà dato dal fatto che l'insieme è \textbf{chiuso}: avevamo visto sulla PL come la soluzione stava \textit{sempre} sulla frontiera. Qui, avremo che la soluzione sta \textit{spesso} sulla frontiera, e dovremo quindi includerla in modo da non perdere soluzioni valide;
	\item L'insieme sarà idealmente \textbf{limitato}, così da permetterci di confermare attraverso il teorema di Weierstrass l'esistenza di minimi e massimi.
Questo tra l'altro è necessario in quanto computazionalmente porremo come limite ai domini un iperrettangolo $-M \leq x_i \leq M$ con $M >> 0$.
Essendo l'iperrettangolo limitato, e il dominio contenuto interamente al suo interno, risulterà chiaramente che anche il dominio stesso sarà (o dovrà essere) limitato;
\item Vorremo poi che l'insieme sia \textbf{convesso}, visto che questo facilita la sua esplorazione (spostandoci verso vettori \textit{interni} all'insieme da ogni suo punto saremo sicuri di uscire dall'insieme stesso una e una sola volta).
	Possiamo dare la seguente caratterizzazione:
	\begin{theorem}{caratterizzazione di insieme convesso}
		Se le $g_i$ sono convesse e le $h_j$ sono lineari, allora $D$ è convesso.
	\end{theorem}

\subsubsection{Domini regolari}
Vediamo quindi una classificazione per domini \textbf{regolari}, # definizione

\begin{enumerate}
	\item Poliedri
	\item $g_i$ convesse, $h_j$ lineari, $\exists \bar{x} : g_i(\bar{x}) < 0 \ \ \forall i$
	\item $
		\begin{cases}
			\nabla g_i(\bar{x}), \quad i \in A(\bar{x})	\\ 
			\nabla h_j(x)
		\end{cases}
		\text{sono linearmente indipendenti}
	$ \\
	dove $A(\bar{x})$ dà i \textbf{vincoli attivi} $A(\bar{x}) = \left\{ i : g_i(\bar{x}) = 0 \right\}$.

	A parole, il gradiente dei dei vincoli di diseguaglianza attivi è linearmente indipendente dal gradiente dei vincoli di uguaglianza.
\end{enumerate}

\end{itemize}



\end{document}

\end{document}